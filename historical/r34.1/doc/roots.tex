\documentstyle[11pt,reduce]{article}
\title{The REDUCE Root Finding Package \\ Mod 1.91, 16 May 1990}
\date{}
\author {Stanley L. Kameny \\ E-mail: valley!stan@rand.org}
\begin{document}
\maketitle
\index{root finding} \index{ROOTS package}

\section{Introduction}

The Root Finding package is designed so that it can be used as an
independent package, or it can be integrated with and called by {\tt SOLVE}.
\index{SOLVE package ! with ROOTS package}
This document describes the package in its independent use.  It can be
used to find some or all of the roots of polynomials with real or
complex coefficients, to the accuracy specified by the user.

\section{Top Level Functions}

The top level functions can be called either as symbolic operators from
algebraic mode, or they can be called directly from symbolic mode with
symbolic mode arguments.  Outputs are expressed in forms that print out
correctly in algebraic mode.


\subsection{Functions which refer to real roots only}

Three top level functions refer only to real roots.  Each of these
functions can receive 1, 2 or 3 arguments.

The first argument is the polynomial p, which can be complex and can
have multiple or zero roots.  If arg2 and arg3 are not present, all real
roots are found.  If the additional arguments are present, they restrict
the region of consideration.
                                                    
\begin{itemize}
\item If arguments are (p,arg2) then
(Arg2 must be  POSITIVE or NEGATIVE)  If arg2=NEGATIVE then only
negative roots of p are included; if arg2=POSITIVE then only positive
roots of p are included. Zero roots are excluded.

\item If arguments are (p,arg2,arg3)
\ttindex{EXCLUDE} \ttindex{POSITIVE} \ttindex{NEGATIVE} \ttindex{INFINITY}
(Arg2 and Arg3 must be r (a real number) or  EXCLUDE r  or a member of
the list POSITIVE, NEGATIVE, INFINITY, -INFINITY.  EXCLUDE r causes the
value r to be excluded from the region.  The order of the sequence
arg2, arg3 is unimportant.  Assuming that arg2 $\leq$ arg3 if both are
numeric, then

\begin{tabular}{l c l}
\{-INFINITY,INFINITY\} & is equivalent to & \{\} represents all roots; \\
\{arg2,NEGATIVE\} & represents & $-\infty < r < arg2$; \\
\{arg2,POSITIVE\} & represents & $arg2 < r < \infty$;
\end{tabular}

In each of the following, replacing an {\em arg} with EXCLUDE {\em arg}
converts the corresponding inclusive $\leq$ to the exclusive $<$

\begin{tabular}{l c l}
\{arg2,-INFINITY\} & represents & $-\infty < r \leq arg2$; \\
\{arg2,INFINITY\} & represents & $arg2 \leq r < \infty$; \\
\{arg2,arg3\} & represents & $arg2 \leq r \leq arg3$;
\end{tabular}

\item If zero is in the interval, zero root is included.
\end{itemize}

\begin{description}
\ttindex{REALROOTS} \index{Sturm Sequences}
\item[REALROOTS] This function finds the real roots of the polynomial p,
using the REALROOT package to isolate real roots by the method of Sturm
sequences, then polishing the root to the desired accuracy.  Precision
of computation is guaranteed to be sufficient to separate all real roots
in the specified region.  (cf. MULTIROOT for treatment of multiple
roots.)

\ttindex{ISOLATER}
\item[ISOLATER] This function produces a list of rational intervals, each
containing a single real root of the polynomial p, within the specified
region, but does not find the roots.

\ttindex{RLROOTNO}
\item[RLROOTNO] This function computes the number of real roots of p in
the specified region, but does not find the roots.
\end{description}

\subsection{Functions which return both real and complex roots}

\begin{description}
\ttindex{ROOTS}
\item[ROOTS p;] This is the main top level function of the roots package.
It will find all roots, real and complex, of the polynomial p to an
accuracy sufficient to separate them.  The value returned by ROOTS is a
list of equations for all roots.  In addition, ROOTS stores separate lists
of real roots and complex roots in the global variables ROOTSREAL and
ROOTSCOMPLEX. \ttindex{ROOTSREAL} \ttindex{ROOTSCOMPLEX}

\ttindex{NEARESTROOT}
\item[NEARESTROOT(p,s);] This top level function uses an iterative method
to find the root to which the method converges given the initial starting
origin s, which can be complex.  If there are several roots in the
vicinity of s and s is not significantly closer to one root than it is to
all others, the convergence could arrive at a root which is not truly the
nearest root.  This function should therefore be used only when the user
is certain that there is only one root in the immediate vicinity of the
starting point s.

\ttindex{FIRSTROOT}
\item[FIRSTROOT p;]   Equivalent to NEARESTROOT(p,0).
\end{description}


\subsection{Other top level function}

\begin{description}
\ttindex{CSIZE}
\item[CSIZE p;] This function will determine the maximum coefficient size of
the polynomial p.  The initial precision used in root finding is at
least 2+CSIZE p (in some cases significantly greater, as determined by
the heuristic function CALCPREC.)

\ttindex{GETROOT} \ttindex{ROOTS} \ttindex{REALROOTS} \ttindex{NEARESTROOTS}
\item[GETROOT(n,rr);] If rr has the form of the output of ROOTS, REALROOTS,
or NEARESTROOTS; GETROOT returns the rational, real, or complex value of the
root equation.  Error occurs if $n<1$ or $n>$ the number of roots in rr.

\ttindex{MKPOLY}
\item[MKPOLY rr;] This function can be used to reconstruct a polynomial
whose root equation list is rr and whose denominator is 1.  Thus one can
verify that $if rr := ROOTS p, and rr1 := ROOTS MKPOLY rr, then 
rr1 = rr$.
(This will be true if MULTIROOT and RATROOT are ON,  and  BIGFLOAT  and
FLOAT are off.)
However, $MKPOLY rr - NUM p = 0$ will be true iff all roots of p 
have been computed exactly.
\end{description}

\subsection{Functions available for diagnostic or instructional use only}

\begin{description}
\ttindex{GFNEWT}
\item[GFNEWT(p,r,cpx);] This function will do a single pass through the
function GFNEWTON for polynomial p and root r.  If cpx=T, then any
complex part of the root will be kept, no matter how small.

\ttindex{GFROOT}
\item[GFROOT(p,r,cpx);] This function will do a single pass through the
function GFROOTFIND for polynomial p and root r.  If cpx=T, then any
complex part of the root will be kept, no matter how small.

\ttindex{ROOTS2}
\item[ROOTS2 p;] The same as ROOTS p, except that if an abort occurs, the
roots already found will be printed and then ROOTS2 will be applied to
the polynomial which exists at that point.  (Note:  there is no
known polynomial on which ROOTS aborts.)
\end{description}

\section{Switches Used in Input}

The input of polynomials in algebraic mode is sensitive to the switches {\tt
COMPLEX}, {\tt FLOAT} and {\tt BIGFLOAT}.  The correct choice of input method
is important since incorrect choices will result in undesirable truncation or
rounding of the input coefficients.

Truncation or rounding will occur if {\tt FLOAT} or {\tt BIGFLOAT} is on and
one of the following is true:

\begin{enumerate}
\item a coefficient is entered in floating point form or rational form.
\item {\tt COMPLEX} is on and a coefficient is imaginary or complex.
\end{enumerate}

Therefore, to avoid undesirable truncation or rounding, then:

\begin{enumerate}
\item both {\tt FLOAT} and {\tt BIGFLOAT} should be off and input should be
in integer or rational form; or
\item {\tt FLOAT} can be on if it is acceptable to truncate or round input to
the machine-dependent precision limit, which may be quite small; or
\item {\tt BIGFLOAT} can be on if {\tt PRECISION} is set to a value large
enough to prevent undesired rounding. \end{enumerate}

\begin{description}
\item[integer and complex modes] (off {\tt FLOAT, BIGFLOAT}) any real
polynomial can be input using integer coefficients of any size; integer or
rational coefficients can be used to input any real or complex polynomial,
independent of the setting of the switch {\tt COMPLEX}.  These are the most
versatile input modes, since any real or complex polynomial can be input
exactly.

\item[modes float and complex-float] (on {\tt FLOAT}) polynomials can be input using
integer coefficients of any size.  Floating point coefficients will be
truncated or rounded, to a size dependent upon the system.  If complex
is on, real coefficients can be input to any precision using integer
form, but coefficients of imaginary parts of complex coefficients will
be rounded or truncated.

\item[modes bigfloat and big-complex] (on {\tt BIGFLOAT}) the setting of
precision determines the precision of all coefficients except for real
coefficients input in integer form.  Floating point coefficients will be
truncated by the system to a size dependent upon the system, the same as
floating point coefficients in float mode.  If precision is set high enough,
any real or complex polynomial can be input exactly provided that
coefficients are input in integer or rational form.
\end{description}

\section{Internal and Output Use of Switches}

REDUCE arithmetic mode switches {\tt BIGFLOAT, FLOAT}, and {\tt COMPLEX}.
These switches are returned in the same state in which they were set
initially, (barring catastrophic error).

\begin{description}
\ttindex{COMPLEX}
\item[COMPLEX] The Root Finding Package controls the switch {\tt COMPLEX}
internally, turning the switch on if it is processing a complex
polynomial. (However, if {\tt COMPLEX} is on, algebraic mode input may not
work correctly in modes {\tt COMPLEX\_FLOAT} or {\tt BIG\_COMPLEX}, so it is
best to
use integer or rational input only.  See example 62 of {\tt roots.tst} for a
way to get this to work.) For a polynomial with real coefficients, the
\ttindex{NEARESTROOT}
starting point argument for NEARESTROOT can be given in algebraic mode
in complex form as rl + im * I  and will be handled correctly,
independent of the setting of the switch {\tt COMPLEX.} Complex roots will be
computed and printed correctly regardless of the setting of the switch
{\tt COMPLEX}.  However, if {\tt COMPLEX} is off, the imaginary part will
print out ahead of the real part, while the reverse order will be obtained if
COMPLEX is on.

\ttindex{FLOAT} \ttindex{BIGFLOAT}
\item[FLOAT, BIGFLOAT] If the switch {\tt AUTOMODE} (Default ON) is ON, the
Root Finding package performs computations using the arithmetic mode that is
required at the time, which may be integer, Gaussian integer, float,
bigfloat, complex float or complex bigfloat.  Switch BFTAG is used internally
to govern the mode of computation and :PREC: is adjusted whenever necessary.
The initial position of switches {\tt FLOAT} and {\tt BIGFLOAT} are ignored.
At output, these switches will emerge in their initial positions.  Outputs
will be printed out in float format only if the float format of the Lisp
system will properly print out quantities of the required accuracy.
Otherwise, the printout will be in bigfloat format. (See also the paragraph
describing {\tt AUTOMODE.)} \ttindex{AUTOMODE}
\end{description}

\section{Root Package Switches}

Note: switches ISOROOT and ACCROOT, present in earlier versions, have been
eliminated.

\begin{description}
\ttindex{RATROOT}
\item[RATROOT] (Default OFF) If {\tt RATROOT} is on all root equations are
output in rational form.  Assuming that the mode is {\tt COMPLEX} (i.e. {\tt
FLOAT} and {\tt BIGFLOAT} are both off,) the root equations are guaranteed to
be able to be input into REDUCE without truncation or rounding errors. (Cf.
the function MKPOLY described above.)

\ttindex{MULTIROOT}
\item[MULTIROOT] (Default ON) Whenever the polynomial has complex
coefficients or has real coefficients and has multiple roots, as
\ttindex{SQFRF} determined by the Sturm function, the function {\tt SQFRF} is
called automatically to factor the polynomial into square-free factors.  If
{\tt MULTIROOT} is on, the multiplicity of the roots will be indicated in the
output of ROOTS or REALROOTS by printing the root output repeatedly,
according to its multiplicity.  If {\tt MULTIROOT} is off, each root will be
printed once, and all roots should be normally be distinct. (Two identical
roots should not appear.  If the initial precision of the computation or the
accuracy of the output was insufficient to separate two closely-spaced roots,
the program attempts to increase accuracy and/or precision if it detects
equal roots.  If however, if the initial accuracy specified was too low, and
it was possible to separate the roots, the program will abort.)

\index{tracing ! ROOTS package}
\ttindex{TRROOT}
\item[TRROOT] (Default OFF)  If switch {\tt TRROOT} is on, trace messages are
printed out during the course of root determination, to show the
progress of solution.
                  
\ttindex{ROOTMSG}
\item[ROOTMSG] (Default OFF) If switch {\tt ROOTMSG} is on in addition to
switch {\tt TRROOT,} additional messages are printed out to aid in following
the progress of Laguerre and Newton complex iteration.  These messages are
intended for debugging use primarily.


NOTE: the switch {\tt AUTOMODE} is included mainly for diagnostic purposes.
If it is changed from its default setting, the automatic determination
of computation modes is bypassed, and correct root determination may not
be achieved!

\ttindex{AUTOMODE}
\item[AUTOMODE] (Default ON) If switch {\tt AUTOMODE} is on, then,
independent of the user setting of the switch {\tt BIGFLOAT}, all floating
point computations are carried out in floating point mode (rather than
bigfloat) if the system floating point mode has sufficient precision at that
point in the computation.  If {\tt AUTOMODE} is off and the user setting of
{\tt BIGFLOAT} is on, bigfloat computations are used for all floating point
computations.  The default setting of {\tt AUTOMODE} is {\tt ON}, in order to
speed up computations and guarantee that the exact input polynomial is
evaluated. \end{description}


\section{Operational Parameters and Parameter Setting.}

\begin{description}                 
\ttindex{ROOTACC\#}
\item[ROOTACC\#] (Default 6) This parameter can be set using the function
ROOTACC n; which causes {\tt ROOTACC\#} to be set to MAX(n,6).  If {\tt
ACCROOT} is on, roots will be determined to a minimum of {\tt ROOT\-ACC\#}
significant places. (If roots are closely spaced, a higher number of
significant places is computed where needed.)

\ttindex{:PREC:}
\item[:PREC:] (Default 8) This REDUCE parameter is used to determine the
precision of bigfloat computations.  The function PRECISION n; causes
:PREC: to be set to the value n+2 but returns the value n.  The roots
package, during its operation, will change the value of :PREC: but will
restore the original value of :PREC: at termination except that the
value of :PREC: is increased if necessary to allow the full output to be
printed.

\ttindex{ROOTPREC}
\item[ROOTPREC n;] The roots package normally sets the computation mode and
precision automatically if {\tt AUTOMODE} is on.  However, if ROOTPREC n; is
called and $n>!!NFPD$  (where !!NFPD is the number of floating point
digits in the Lisp system,) then all root computation will be done
initially in bigfloat mode of minimum precision n.  Automatic operation
can be restored by input of ROOTPREC 0;.
\ttindex{"!"!NFPD}
\end{description}


\section{Avoiding truncation of polynomials on input}

The roots package will not internally truncate polynomials provided that the
switch {\tt AUTOMODE} is on (or, if {\tt AUTOMODE} is off, provided that
{\tt ROOTPREC} is not set to some value smaller than the number of
significant figures needed to represent the polynomial precisely.) However,
it is possible that a polynomial can be truncated by input reading functions
of the embedding lisp system, particularly when input is given in floating
point or bigfloat formats. (Some lisp systems use the floating point input
routines to input bigfloats.)

To avoid any difficulties, input can be done in integer or Gaussian
integer format, or mixed, with integers or rationals used to represent
quantities of high precision. There are many examples of this in the
test package. Note that use of bigfloat of high precision will not
necessarily avoid truncation of coefficients if floating point input
format is used.  It is usually sufficient to let the roots package
determine the precision needed to compute roots.

The number of digits that can be safely represented in floating point in the
lisp system are contained in the global variable {\tt !!NFPD}.  Similarly,
the maximum number of significant figures in floating point output are
contained in the global variable {\tt !!FLIM}.  The roots package computes
these values, which are needed to control the logic of the program.
\ttindex{"!"!FLIM} \ttindex{"!"!NFPD}

The values of intermediate root iterations (that are printed when
{\tt TRROOT} is on) are given in bigfloat format even when the actual values
are computed in floating point.  This avoids intrusive rounding of root
printout.

\end{document}
