\documentclass[a4paper]{article}

\usepackage[utf8]{inputenc}
\usepackage{tikz}

\title{Old and new memory layout for PSL}
\author{Rainer Schöpf}
\date{09 Aug 2025}

\begin{document}

\section{Old memory layout}

PSL was developed back in the 1980s and has mostly kept the memory
layout of the times: linear memory with text (code), data (initialized
data), and bss (uninitialized data) areas, plus heap. On Unix and
similar operating systems, the heap area was traditionally allocated
either statically at link time or dynamically using the sbrk system call.

The code area consists of simple C Code for the most basic operating
system interfaces: I/O and very simple memory management plus a layer
of assembler code t hat is translated from Lisp. All the rest of PSL
is loaded at runtime into the so-called binary program space (BPS)
residing in the bss area.

A running PSL system can be dumped to a so-called image file. This
image file can be loaded later at startup to restart the previously
running system. Originally the whole memory image of the process was
written to disc as an executable to be started again. In the course of
time this was replaced with code that dumps (and loads) only the
relevant data. In order for this method to work, the important memory
areas like symbol table, BPS, and heap must be allocated at the very
same memory address. Since more recent operating systems cannot always
guarantee the memory address of a dynamically allocated heap, all
memory references within the heap need to be checked and adjusted when
the image file is loaded. However, this is not possible for the BPS:
The compiled lisp code area includes constant lisp objects whose
addresses are referenced in the actual code.

\section{Problems of the old model}

Current operating systems employ several security techniques that
comflict with the simple old model:
\begin{itemize}
\item Address space layout randomization: a specific address for the
  various areas cannot be guaranteed\footnote{In addition, the lisp
    compiler for the x86\_64 platform generates memory references as
    32 bit, not 64 bit offsets. This could be changed, of course.}.

\item W\textasciicircum X protection: a memory area way be written to,
  or the code in it may be executed but not both. Code loaded at
  runtime can therefore not easily be run. There are ways around this
  restriction, but so far they work only with dynamically allocated
  memory areas - which conflicts with the requirement that the
  addresses in the BPS do not change.
  
\end{itemize}

\section{The new memory model}

To get around the problems of the old model the BPS area must be made
relocatable. This means:
\begin{enumerate}
\item The constant lisp objects need to be stored separately. To this
  end a new static memory area is allocated in the bss segment. This
  addresses of the lisp objects stored there are not changed across
  program starts.

\item The code must be position-independent. There must be no explicit
  memory references, instead all accesses to other memory areas, i.e.
  the symbol table and the static lisp area must be relativ to their
  base addresses\footnote{This is most easily done by keeping the base
  addresses of the symbol table and the static lisp area in processsor
  registers.}.

\item Since code and constant lisp objects are now separate the fast
  loadable (FASL) file format needs to be changed.
  
\end{enumerate}




\end{document}
