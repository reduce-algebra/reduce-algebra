
\chapter{Compiler}


\section{Introduction}

  The evaluation of an expression which has been compiled may be
different  from interpretation. Some of the more important
differences are listed here. Subsequent  sections  of  this
chapter will discuss these differences in more detail.

  The  application  of  a  macro  is  replaced by its expansion.
Therefore, the definition of the macro  must  preceed  its  use.
Since the call on the macro has been replaced, a redefinition of
the  macro  will  have  not  have  an  effect on the compiled
expression.

  A constant expression  is  replaced  by  its value (see  the
section Constant Declaration for more information)

  There are some functions which are defined in two ways. One of
these  versions  will reduce the execution time by not verifying
that the arguments are of  correct  type.  Which function is
called  from  the  compiled  code  is  based upon the value of a
switch.  If the  switch  is  non-nil  then  the  faster  version
replaces  the  slower  one.  Note that an instance of the faster
version is not replaced by the slower one when the switch is set
to nil (see the section Switches that Control the  Compiler  for
more information).

  An  attempt is made to convert recursive control structures to
iterative ones.

\section{Compiling Files}

It is best to compile files using {\tt pslcomp}, a minimal version
of PSL capable of compiling files. The successful compilation of
a  file  should not depend upon optional modules which have been
loaded.  On some computer systems the names of the files  to  be
compiled  can  be listed as command line arguments. When PSLCOMP
is brought up the following banner is displayed.

Portable Standard Lisp Compiler\\
Usage: PSLCOMP source-file ...

If an error occurs, the message
\begin{verbatim}
***** Error during compilation of NAME
\end{verbatim}
is printed and the compilation is aborted.


\de{compile-file}{(compile-file FILE:string): \{t, nil, abort\}}{expr}
{    This function is used to compile files.  A  binary  file  is
    created  with  the  same  name  except that the file type is
    changed to {\bf b}. Beware, it is possible to overwrite the source
    file if the operating system truncates file names (this  can
    be  avoided  by using compile-file-aux, described below). If
    the file name is without a suffix then {\bf sl} is assumed. This
    function  will  return  t if the compilation was successful,
    abort if after an error was signalled the user  aborted  the
    compilation,  or  nil if the source file was not found. When
    the source file cannot be found the message
}
\begin{verbatim}
    ***** Unable to find source file for: FILE
\end{verbatim}
    will be printed. Prior to the compilation the message

\begin{verbatim}
    ----- Compiling SOURCE-FILE to BINARY-FILE
\end{verbatim}
    is displayed. In addition, one of the following two messages
    will be printed when compilation has stopped.

\begin{verbatim}
    ----- Compilation of NAME completed

    ----- Compilation of NAME aborted
\end{verbatim}

\de{compile-file-aux}{(compile-file-aux SOURCE:string BINARY:string):\\
\{t, nil, abort\}}{expr}
{    
    Since this function  is  used  by  compile-file  to  compile
    files, most of the comments concerning compile-file apply to
    compile-file-aux.   Although one application of compile-file
    may be used to compile a number of files, with this function
    only one file can be  compiled.    The  advantage  of  using
    compile-file-aux  is  that  the  name  of the binary file is
    given explicitly.
}

  During the compilation of a file, if an atomic  expression  is
read at the top level the

\begin{verbatim}
*** Warning: non-list form ignored: ATOM
\end{verbatim}
will  be  displayed  and  the expression will be ignored. A list
whose first element is not atomic is saved  for  insertion  into
the an initialization function which is added to the binary file
by   the   compiler  (its  name  is  **fasl**initcode**).    The
processing of a list whose first element is  an  id  depends  in
large part upon values found on the property list of the id. The
EVAL  flag  will cause the expression to be evaluated, using the
current binding of toploopeval*, and compiled. The expression is
only evaluated, again using the current binding of toploopeval*,
if the id is flagged IGNORE. All other expressions are compiled.
Compilation will modify  the  environment,  in  particular,  all
macro definitions are installed. Note that although defflavor is
a  macro  it does not create a macro definition. It will however
modify the property list of the flavor name.

  As expressions are processed a list of expressions  is  built.
When  a  function  definition  is made an application of putd is
added to the list. Other expressions which are  intended  to  be
evaluated  at  load  time  were  also  added.  This  sequence of
instructions, which preserves the  order  in  which  expressions
were  read,  is  compiled  as  the  body  of  the initialization
function (named **fasl**initcode**).

\subsection{Order of Functions for Compilation}

  Functions whose type is not expr must be defined before  their
use in a compiled function, since the type of a function affects
how  it's  application  will  be  treated  by  the compiler. The
application of a macro is  replaced  by  the  expansion  of  the
macro.  For  functions  whose  type  is fexpr it is necessary to
gather the unevaluated arguments into a list. Calling a function
which is an nexpr requires that the arguments be  evaluated  and
gathered  into  a  list.  When the compiler cannot determine the
type of a function, because it has not been defined, it  assumes
that  the  type  of  the  function  is expr. Note that it is not
possible to define a recursive function, whose type is not expr,
without first defining a dummy version.

\section{Compiling Functions into Memory}

  Functions can be compiled directly into memory by  turning  on
the  switch  comp,  by evaluating (on comp). This compilation is
accomplished in part by a call on the function compd from within
the function putd.


\de{compd}{(compd NAME:id TYPE:ftype BODY:lambda): NAME:id}{expr}
{    The function compd is analogous to the  function  putd.  }

  Once  comp  is  set  to  t subsequent function definitions are
automatically compiled and a message of the form

\begin{verbatim}
*** (NAME): base <number>, length <number> bytes
\end{verbatim}
is displayed. Unfortunately this information is  of  little  use
unless  one  happens  to  be  doing something like debugging the
compiler.


\de{compile}{(compile NAMES:id-list): any}{expr}
{    This function is used to compile resident functions.  It  is
    possible  to compile a function which is being traced. There
    are two warning messages which may be printed.
}
\begin{verbatim}
    *** NAME already compiled
\end{verbatim}
    This  indicates  that  the  function  associated  with   the
    function  name is already compiled.  If there is no function
    definition associated  with  the  name  then  the  following
    message will be displayed.

\begin{verbatim}
    *** No definition for NAME
\end{verbatim}

\section{Compiler Errors and Warnings}

  In  addition  to the error messages mentioned in this chapter,
the following error messages may be displayed. In  general,  the
five  star  prefix  is  used  for error messages, the three star
prefix for warnings.  The display of  warning  messages  can  be
suppressed by evaluating (off msg).

\begin{verbatim}
***** Too many arguments N
\end{verbatim}
The  restriction  on the number of arguments is based in part on
the fact that arguments to  compiled  functions  are  passed  in
general  purpose  registers rather than directly on a stack. The
compiler  will  generate  code  for  the  offending  definition.
Although the code appears to be correct, attempting to apply the
function will result in error.

\begin{verbatim}
***** Attempt to compile non-lambda expression FORM
\end{verbatim}
This  error  occurs  within the context of a call on compd, when
the expression is not a valid lambda expression.

\begin{verbatim}
*** NAME has not been defined, because it is flagged LOSE
\end{verbatim}
If a function name is flagged LOSE then it cannot be  redefined.
Attempting  to  compile an interpretive definition is considered
redefinition in this context.

\begin{verbatim}
***** Ill-formed function expression FORM
\end{verbatim}

The offending expression is an instance of function application.
This message will appear if the function is atomic  and  not  an
id,  or  if  the  function is a list which is not a valid lambda
expression.

\begin{verbatim}
***** Odd number of arguments to SETQ FORM
\end{verbatim}

The number of arguments passed to a SETQ must be even.

\begin{verbatim}
***** Incorrect number of arguments to NAME
\end{verbatim}

The  number  of  arguments  passed  to  either  apply,  idapply,
codeapply, cons, go, or return was not correct.

\section{Differences between Compiled and Interpreted Code}

  In  the process of compilation, many functions are Open-Coded,
and hence cannot be redefined or traced in  the  compiled  code.
Such  functions  are  noted  to be Open-Coded in the manual. The
call on the function is replaced by a sequence  of  instructions
which  make up the body of the function. This trick gives faster
execution, but it requires more space. A  call  on  a  macro  is
another  instance of open coding. A PSL program usually consists
of a large number  of  relatively  small  functions.  Calling  a
function requires manipulation of a parameter stack for variable
binding  and  a  control  stack for return information. The time
spent inside a function may be small compared to the time  spent
maintaining  the stacks.  Some commonly used PSL functions which
are open-coded are and, or, apply, idapply,  codeapply,  return,
go, cons, cond, case, prog, progn, and prog2.

  Unless  variables  are  declared,  or detected, to be fluid or
global they are compiled as local variables. This  causes  their
names  to  disappear, and so they are not visible on the binding
stack. In addition these variables  will  not  be  available  to
functions called in the dynamic scope of the function containing
their binding.

  The compiler attempts to do the following conversions\\

%\TABELLE***
\begin{tabular}{lp{9.0cm}}

    (function  (lambda  ...))    $->$ &   the lambda expression is
     compiled into a gensymed name.
    If the functional form of  one  of  the  mapping  functions
     (map,   mapc,  maplist,  mapcar,  mapcon,  and  mapcan)  is
     explicitly tagged function then the call is open coded.\\

    (setq na va nb vb ...)  $->$ & (setq na va), (setq nb vb) ...\\

 (nth sequence n)  $->$ & (car sequence)  if  n = 1,\\
&                         (cadr sequence)  if  n = 2,\\
&                       (caddr sequence)  if  n = 3,\\
&                       (cadddr sequence)  if  n = 4\\

    (pnth sequence n)  $->$ &  (identity sequence)  if  n = 1,\\
&                           (cdr sequence)  if  n = 2,\\
&                           (cddr sequence)  if  n = 3,\\
&                           (cdddr sequence)  if  n = 4,\\
&                           (cddddr sequence)  if  n = 5\\
\end{tabular}

    The  following  transformations  are  made  when  the  list
     function  is applied to less than six arguments. When there
     are more that five arguments the  compiler  uses  the  cons
     expansion for additional arguments.\\

\begin{tabular}{lp{6.0cm}}
     (list a)  $->$  & (ncons a)\\
     (list a b)  $->$ &  (list2 a b)\\
     (list a b c)  $->$ & (list3 a b c)\\
     (list a b c d)  $->$ &  (list4 a b c d)\\
     (list a b c d e)  $->$ &  (list5 a b c d e)\\

    (apply (function foo) ...)  $->$ &  (foo ...)\\

    (assoc ...)  $->$  & (atsoc ...)\\
\end{tabular}

    If  the  last clause within a cond expression is not of the

     form (t ...) then  the  compiler  will  add  the  additinal
     clause (t nil).\\

\begin{tabular}{lp{6.0cm}}
    (difference n 1)  $->$ &  (sub1 n)\\

    (equal ...)  $->$ &  (eq ...)\\

    (geq ...)  $->$ & (not (lessp ...))\\

    (intern (compress ...))  $->$ &  (implode ...)\\

    (intern (gensym ...))  $->$ &  (interngensym ...)\\

    (leq ...)  $->$ &  (not (greaterp ...))\\

    (lessp n 0)  $->$ &  (minusp n)\\

    (member ...)  $->$ &  (memq ...)\\

    (neq ...)  $->$ &  (not (equal ...))\\

    (not ...)  $->$ &  (null ...)\\

    (plus2 n 1)  $->$ &  (add1 n)\\

    (null (eq ...))  $->$ &  (neq ...)\\

    (null (atom ...))  $->$ &  (pairp ...)\\
%\end TABELLE***
\end{tabular}

\vspace{0.5cm}
  Prior  to  the  application of some functions, such as car and
cdr, there is no verification that the types  of  the  arguments
are correct.

  Since  compiled  calls  on  macros,  fexprs,  and  nexprs  are
different from  the  default  exprs,  these  functions  must  be
defined,  or declared, before compiling the code that uses them.
While fexprs and nexprs may be subsequently  redefined,  as  new
functions  of the same type, macros are executed by the compiler
to get  the  replacement  form,  which  is  then  compiled.  The
interpreter  would  pick  up  the  most recent definition of any
function, and so functions can switch type as well as body.

  Constants which appear in code that is to be compiled  may  be
collapsed.  For  example,  the  following  code  references  two
distinct strings which happen to contain  the  same  characters.
The compiled code will only reference one string.

\begin{verbatim}
  (de fubar ()
    (let ((this "foo")
          (that "foo"))
    (eq this that))

1 lisp> (foo)
nil
2 lisp> (compile '(foo))
nil
3 lisp> (foo)
t
\end{verbatim}
  As  noted  below,  the  switches  *r2i  and *nolinke may cause
function calls to be replaced by  jumps.  This  means  that  the
backtrace   of  compiled  functions  may  differ  from  that  of
interpreted functions. In addition  a  sequence  of  interpreted
function calls could consume stack space while compiled function
calls might not.


\section{Constant Declaration}


\de{define-constant}{(define-constant NAME:id VALUE:any): any}{macro}
{    The identifier NAME is declared a constant and its valuecell
    is set to VALUE.  This provides a means  for  informing  the
    compiler that the value of an id is constant.  If a constant
    id  is  used  in such a way that during evaluation its value
    would be accessed the compiler can then simply  replace  the
    id  by its value. It is possible to use an id which has been
    declared a constant as a local variable within a prog or  as
    a parameter name.  It is possible to redefine the value of a
    constant.
}

    Request to set constant NAME to a different value.


\de{constant?}{(constant? NAME:id): boolean}{expr}
{    Returns  t  if  NAME  has  been  declared  a  constant  with
    define-constant, otherwise nil.
}

  The compiler attempts to replace function applications by  the
corresponding  value  whenever  each  of the arguments is either
constant or an application which may be replaced by  its  value.
The  following  is a list of all functions whose application may
be evaluated during compilation.  Clearly  some  functions  like
cons cannot be evaluated at compile time.

\begin{verbatim}
// /= < << <= = > >= >> ^ | ~ ~=
\end{verbatim}

abs aconc acos acosd acot acotd acsc acscd add1
alphanumbericp alphap asec asecd asin asind ass
assoc atan atan2 atan2d atand atom atsoc\\
 
bldmsg bothcasep\\

car cdr ceiling char-downcase char-equal char-font char-
greaterp char-int char-lessp char-upcase char$<$ char= char$>$ 
constantp cos cosd cot cotd csc cscd\\

degreestodms degreestoradians difference digit digit-char
digitp divide dmstodegrees dmstoradians\\

eq eqcar eqn eqstr equal expt\\

factorial fix fixp float floatp floor\\

geq getv\\

land lastcar lastpair length leq lessp list lnot log log10
log2 lor lowercasep lshift lxor\\

max2 member memq min2 minus minusp mkquote mod\\

ne neq not nth null\\

onep\\

pairp plus2 pnth\\

radianstodegrees radianstodms recip remainder rest reserse 
round\\

sec secd sin sind size sqrt string-not-greaterp string-not-
lessp string-repeat string-search string-search-equal string-
search-from-string-search-from-equal string-size string-trim 
string-upcase string-upper-bound string$<$ string$<$=
string$<$$>$  string= string$>$ string$>$= stringp sub sub1
substring-equal  substring=\\

tan tand times2\\

upbv uppercasep\\

vector-empty? vector-fetch vector-size vector-upper-bound
vector2l ist vector 2string vectorp\\

There is another facility in PSL for defining constants.  You
are  encouraged  to  use  the  function  define-constant. The
following two functions are documented here for completeness.


\de{defconst}{(defconst [U:id V:any]): undefined}{macro}
{    Defconst  provides  for  the  definition and use of symbolic
    constants.  Each U is defined to represent the corresponding
    value V.  The value V is stored on the property list  of  U,
    not  in  the valuecell.  The macro const is used to retrieve
    the value.
}

\de{const}{(const U:id): any}{macro}
{    Const gives access to the  value  defined  by  the  function
    defconst.  The value will be evaluated.
}
\begin{verbatim}
    1 lisp> (defconst foo (add1 2))
    (add1 2)
    2 lisp> (const foo)
    3
\end{verbatim}
\section{Fluid and Global Declarations}

  Fluid  and  global  declarations  must  be  used  to  indicate
variables that are to be used as non-local variables in compiled
code. The values associated with local variables are stored on a
stack. If a variable has been declared global or fluid then  the
compiled  code  will  instead  access  the  value  cell  of that
variable. In addition the content of the value cells of  special
variables  being  used  as  formal parameters must be preserved.
The compiler currently defaults variables bound in a  particular
function  to local. Note that local variables exist as anonymous
stack locations, therefore called functions cannot see them.  An
undeclared  non-local variable is declared fluid by the compiler
with the warning:

\begin{verbatim}
*** NAME declared fluid
\end{verbatim}
The appearance of this message may indicate an additional error.
If a previous function used this name in the parameter list then
it may have  to  be  recompiled.   The  local  variable  in  the
previous function will not refer to the same variable as the one
recently  declared  fluid, even though the names are the same in
the source code listing.

  Previous documentation has given users the impression that  it
is  illegal  to bind an id which has been declared global.  This
is not true, an attempt to bind a global variable will cause the
following message to be displayed.

\begin{verbatim}
*** Illegal to bind global VARIABLE but binding anyway
\end{verbatim}
The lack of distinction between global and fluid in PSL  appears
to  be  due to the fact that shallow binding is employed in PSL.
In a deep bound system one has to search down a binding stack to
locate the value of a fluid id, which  could  get  expensive.  A
global  declaration  within  a deep bound system would allow the
interpreter to assume that the value of an  id  is  in  a  fixed
location.  Of course in a shallow bound system the value of a id
is always found in the value cell of that id.

\section{Control Over the Time When Something is Done}

  Which expressions are evaluated during compilation only, which
are written to the file to be evaluated at load time, and  which
do  both  can be controlled by the EVAL and IGNORE properties of
function names. In addition, the following functions may also be
used.


\de{commentoutcode}{(commentoutcode U:form): nil}{macro}
{        Comment out a single expression.
}

\de{compiletime}{(compiletime U:form): nil}{expr}
{    Evaluate the expression U at compile time only.
}

\de{bothtimes}{(bothtimes U:form): U:form}{expr}
{    Evaluate the expression U at  both  compile  time  and  load
    time.
}

\de{loadtime}{(loadtime U:form): U:form}{expr}
{    Evaluate the expression at load time only.
}
\section{Switches That Control the Compiler}

  The compilation process is controlled by a number of switches.
In addition to comp there are the following switches.

\variable{*fast-vectors}{[Initially: t]}{switch}
{}

\variable{*fast-evectors}{[Initially: t]}{switch}
{}

\variable{*fast-strings}{[Initially: nil]}{switch}
{}

\variable{*fast-integers}{[Initially: nil]}{switch}
{
    There  are some functions which are defined in two ways. One
    of these versions will reduce  the  execution  time  by  not
    verifying  that  the arguments are of correct type. The name
    of the slower version will have  two  properties,  FAST-FLAG
    and  FAST-FUNCTION.  The  FAST-FLAG  property  references  a
    switch whose value determines which function will be  called
    from  compiled  code.  An instance of the slower function is
    replaced by the faster one when the switch is set to  t.  Of
    course, to make this conversion the compiler has to know the
    name of the faster function.  This name is referenced by the
    FAST-FUNCTION property.
}

    The property list of the function vector-fetch looks something like

\begin{verbatim}
(... (fast-flag . *fast-vectors) (fast-function . igetv) ...)
\end{verbatim}
    The expression (vector-fetch vector index) will be replaced by the
    expression (igetv vector index) if the switch *fast-vectors is set
    to t.

\variable{*r2i}{[Initially: t]}{switch}
{
    If  non-nil, recursive calls are converted to jumps whenever
    possible. The effect of this is that tracing  this  function
    will  not  show  the  recursive  calls  since they have been
    eliminated.  The  function  RECURSIVE-FACTORIAL  contains  a
    recursive  call which can be converted to a jump. The result
    of compiling RECURSIVE-FACTORIAL is almost identical to that
    of compiling ITERATIVE-FACTORIAL when r2i is non-nil.
}
\begin{verbatim}
    (de factorial (n)  (iterative-factorial (1 n)))
\end{verbatim}
\begin{verbatim}
    (de recursive-factorial (accumulator count)
      (if (<= count 1)
         accumulator
         (recursive-factorial (* accumulator count)
                              (sub1 count))))
\end{verbatim}
\begin{verbatim}
     (de iterative-factorial (accumulator count)
      (prog ()
       loop
         (when (<= count 1) (return accumulator))
         (setf accumulator (* accumulator count))
         (decr count)
         (go loop)))
\end{verbatim}
\variable{*nolinke}{[Initially: nil]}{switch}
{
    If nil, when the last form to be  evaluated  is  a  function
    call  then  the  compiler will generate an instruction which
    will perform  two  actions  during  evaluation.    Prior  to
    jumping to the entry point of the function being called, the
    space  allocated  on the stack for local variable storage is
    reclaimed. Although the amount of  space  reclaimed  may  be
    small, this does provide the called function with more stack
    space.  In  contrast,  when this switch is non-nil a pair of
    instructions would be used, a call on the function  followed
    by  an  exit.  In this case the stack space is not reclaimed
    until the exit.
}

\variable{*ord}{[Initially: nil]}{switch}
{
    If non-nil, the compiler is forced to  generate  code  which
    evaluates  arguments  of  a function call in a left to right
    order. It  is  possible  to  generate  code  which  is  more
    effecient  when  this  switch  is  set to nil. Arguments are
    currently passed  in  registers.  As  the  calling  function
    computes  each  argument  the result is stored on the stack.
    References to constants  need  not  be  generated  and  then
    pushed  onto  the stack, the other arguments can be compiled
    first,  then  just  before  the  function  is   called   the
    appropriate  register is loaded with the constant. A similar
    argument can  be  made  for  variable  references.  However,
    allowing for side effects we must be sure that the postponed
    value would be the same as that fetched at the proper time.
}

\variable{*plap}{[Initially: nil]}{switch}
{
    If  non-nil,  the portable intermediate code produced by the
    compiler is printed. This is useful for  examining  compiler
    output prior to assembly, in particular it can be helpful in
    debugging macros.
}

\variable{*pgwd}{[Initially: nil]}{switch}
{
    If  non-nil,  the  actual  assembly  language  mnemonics are
    displayed. This flag is useful for examining the code  which
    is     generated    for    the    initialization    function
    **fasl**initcode**.
}

\variable{*pcmac}{[Initially: nil]}{switch}
{
    If non-nil, both the  portable  intermediate  code  and  the
    assembly  mnemonics  are displayed.  After each intermediate
    instruction the assembly mnemonic is shown.
}

\variable{*pwrds}{[Initially: t]}{switch}
{
    If non-nil, the address and size of compiled  functions  are
    displayed as they are defined.
}

\section{Conditional Compilation}


\de{if\_system}{(if\_system SYS-NAME:id TRUE-CASE:any\\ 
FALSE-CASE:any): any}{macro}
{    This  is  a  conditional macro for choosing system dependent
    code during compilation.  The expression for the false  case
    defaults to nil if it is ommited. The id must be a member of
    the  list whose name is system\_list*. An example of its use
    follows.
}

\begin{verbatim}
    (if_system  tops20  (setq buildfileformat* "PL:%W"))
\end{verbatim}
\section{Implementation Details}

  The output from the compiler is a list of instructions in  the
order  which  they will be executed. Each instruction is a list,
an operation followed by as many items as are required  by  that
operation.  The  compiled code may be executed by simulating the
actions of the machine on each element of the sequence.  In  the
interest  of efficiency, the compiler ouput is translated into a
sequence of actual machine instructions. An area of memory
(which  is  not  garbage collected), is allocated to  receive
the processed compiler  output.  This  area  is   called  the 
Binary Program  Space  or  BPS.  The function  lap translates
the output from the compiler into machine  instructions.

  The functions faslout, faslend,  and  faslabort  are  used  by
Compile-file-aux  to  compile files.  It should not be necessary
for  you  to  used  these  functions,  they  are  included   for
completeness.  Evaluation of

\begin{verbatim}
(faslout name)
\end{verbatim}
will cause the message

FASLOUT: (DSKIN files) or type in expressions\\
When all done execute (FASLEND)

to  be displayed.  Subsequent expressions, typed in or read from
files, are not processed in the normal  fashion.  The  following
piece of code demonstrates how this feature is implemented.

\begin{verbatim}
  (de toploop
      ...
    (cond ((not *defn) (top-loop-eval-and-print input-value))
          (dfprint* (funcall dfprint* input-value))
          (t (prettyprint input-value)))
      ... )
\end{verbatim}
  When  the  switch  *defn is nil the input is evaluated and the
result is  printed.   Setting  *defn  to  t  and  associating  a
function with the global dfprint* allows for the redefinition of
what is meant by evaluating the input and printing the result of
that  evaluation.  It  the  function  bound to dfprint* which is
responsible for the compilation  of  expressions.    Faslend  is
provided  to discontinue compilation, faslabort is used to abort
the compilation.  The only context in which the  application  of
these  functions  makes  sense  is when a previous invocation of
faslout has not been terminated.

\begin{verbatim}
***** FASLEND not within FASLOUT
\end{verbatim}
