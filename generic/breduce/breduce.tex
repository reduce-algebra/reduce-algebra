\documentclass[a4paper]{article}
\def\co{,}\def\<{\langle}\def\>{\rangle}\def\d{\hbox{\rm d}}
\usepackage{longtable,enumerate}
\begin{document}
\title{\textsc{BREDUCE} User Manual}
\author{Thomas Sturm, FIM, Universit\"at Passau,
  Germany\footnote{\texttt{sturm@redlog.eu}}}
\maketitle

\begin{abstract}
  This document serves as a user guide for \textsc{breduce}. This is a
  shell script plus a supporting \textsc{reduce} module which together
  admit batch processing of \textsc{reduce} jobs under Unix.
  \textsc{breduce} takes as input several \textsc{reduce} files
  containing input data plus a configuration file. It produces as
  output a \LaTeX document with the formatted results of all
  computations specified in the configuration file. Important features
  include the systematic application of all combinations of a choice
  of \textsc{reduce} switches and a mechanism for limiting the runtime
  of the single computations.
\end{abstract}

\section{Introduction}
\cite{Hearn:04}

\section{Installation of \textsc{breduce}}
The \textsc{breduce} distribution comprises three files:
\begin{enumerate}
\item An executable program \texttt{breduce},
\item a supporting \textsc{reduce} module \texttt{breduce.red},
\item this manual \texttt{breduce.tex}.
\end{enumerate}
Make sure that \texttt{breduce} is executable, and put it into a
directory of your choice, which you probably want to have within your
search path. Put \texttt{breduce.red} into the same directory as
\texttt{breduce}.

\section{A Simple First Example}\label{SE:first}
We are now going to describe an extremely simple example application
of \textsc{breduce}. This will help the users to verify that their installation
is correct. Furthermore it will give good first idea about what
\textsc{breduce} does. Finally it will explain a major part of the files
maintained by \textsc{breduce}.

\textsc{breduce} is based on the concept of a \emph{series} of computations to
be performed. A series consists of one or several \emph{instances}.
The user chooses a name \texttt{<name>} for the series. The central
file describing the series is the \textsc{breduce} \emph{configuration file}
\texttt{<name>.breduce}.

For our first example here, all relevant information is contained in
our configuration file \texttt{fac.breduce}. For convenience, we
create a new directory \texttt{bredex1} to work in. Our
\texttt{fac.breduce} reads as follows:
\begin{verbatim}
REDUCE='reduce'
seriesinstances='1 2 3 4 5 6 7 8 9 10'
command='factorize'
\end{verbatim}
The reader might have to adapt the choice for \texttt{REDUCE}, which
is the name of the executable \textsc{reduce}. The choice
\texttt{reduce} above is also the default used if \texttt{REDUCE} is
not specified at all. There are rather rigorous formal restrictions on
the entries in configuration files:
\begin{itemize}
\item A \emph{valid} line either starts with a \textsc{breduce}
  \emph{keyword}, such as the 3 lines above, or contains exclusively
  whitespace, or starts with the \emph{comment sign} '\texttt{\%}'.
\item In keyword lines, there must not be any whitespace at the
  beginning or at the end or around '\texttt{=}'. The right hand side
  of '\texttt{=}' must be quoted with either single or double
  quotes.\footnote{Technically, \textsc{breduce} passes the keyword
    lines to \texttt{eval} in a Bash. Readers who do not understand
    what this means or feel they do not fully understand the possible
    consequences of this are strongly recommended to exclusively use
    single quotes! For advanced users the careful use of double quotes
    provides some hook to add information from the OS level.}
\end{itemize}
We now start our first \textsc{breduce} jobs as follows:
\begin{verbatim}
breduce fac.breduce
\end{verbatim}
where the extension is optional in the style of \LaTeX.

Within a few milliseconds we obtain a file \LaTeX{} \texttt{fac.tex}
containing the table in Figure~\ref{FI:fac1}.
\begin{figure}[t]
  \begin{center}
    \fbox{\begin{minipage}{0.95\textwidth}
        \small
        \noindent sturm using reduce on lennier.local (i386 Darwin)
        \smallskip

        \noindent start of job on  Tue Jan 1 10:32:03 CET 2008
        \noindent\begin{longtable}{lrr}
          \hline
          instance & result & time (ms)\\
          \hline
          \endfirsthead
          \hline
          instance & result & time (ms)\\
          \hline
          \endhead
          1 & (list 1 1) & 0\\
          \hline
          2 & (list 2 1) & 0\\
          \hline
          3 & (list 3 1) & 0\\
          \hline
          4 & (list 2 2) & 0\\
          \hline
          5 & (list 5 1) & 0\\
          \hline
          6 & (list 2 1)(list 3 1) & 0\\
          \hline
          7 & (list 7 1) & 0\\
          \hline
          8 & (list 2 3) & 0\\
          \hline
          9 & (list 3 2) & 0\\
          \hline
          10 & (list 2 1)(list 5 1) & 0\\
          \hline
        \end{longtable}
        \noindent end of job on  Tue Jan 1 10:32:03 CET 2008
      \end{minipage}}
    \caption{\texttt{fac.tex}\label{FI:fac1}}
  \end{center}
\end{figure}
In addition, \texttt{fac.tex} contains the name of the working
directory (\texttt{bredex1} for our example) followed by a dump of
\texttt{fac.breduce}. Since \textsc{breduce} uses the \LaTeX{} package
``longtable,'' which allows have tables spread over several pages, it
may be necessary to process \texttt{fac.tex} several times until the
table is properly arranged. The same holds for this manual. We are
going to discuss and considerably improve the content of the result
column of the table in Figure~\ref{FI:fac1} in the next section.

In addition to \texttt{fac.tex} \textsc{breduce} creates a directory
\texttt{fac/}, which contains log files \texttt{instance1-0.log},
\dots,~\texttt{instance10-0.log} of the 10 \textsc{reduce} runs. We
are going to discuss the purpose of the appended '\texttt{-0}' later
on in Section~\ref{SE:switches}.

\section{Processing Results}\label{SE:processresults}
The result column in Figure~\label{FI:fac1} certainly looks a bit
disappointing: It contains the raw Lisp representations of the
results, which have actually been obtained in the algebraic mode.
Moreover, these Lisp representations are processed with \LaTeX, which
would cause problems whenever they happen to contain special \LaTeX{}
symbols. Finally, it is clear that the output of the full result would
soon exceed the space available in a table when factorizing only
slightly larger numbers.

To address all these issues, \textsc{breduce} provides an optional
keyword \texttt{process\-results}. For instance
\begin{verbatim}
processresults='breduce_verbatim'
\end{verbatim}
Puts the result into a \LaTeX{} \texttt{verbatim} environment such
that it may contain arbitrary characters. Another nice choice for
small mathematical results is \texttt{breduce\_tex}, which uses the
\texttt{reduce} package \texttt{tri} \cite{ASW:89} for producing
\LaTeX{}-formatted output. For factorizing larger numbers in our
examples the user might want to use \texttt{length} for obtaining the
number of different factors.

Generally, with large results one can provide a self-written procedure
for analyzing the results and outputting suitable \LaTeX{} source
code. The specification for such a procedure is as follows:
\begin{itemize}
\item It is called in the algebraic mode with the result of the
  application of the specified \texttt{command} as its only argument.
\item It returns a either string, which is suitable to be processed by
  \LaTeX{}, or an algebraic mode list of such strings. With the second
  variant all strings in the list are successively output.
\end{itemize}
We are going to see in Sections~\ref{SE:initcommands} and
\ref{SE:files} where such self-written procedure can possibly be
placed.

\textsc{breduce} itself provides one such function:
\texttt{breduce\_process\-formula} analyzes results of \texttt{redlog}
\cite{DolzmannSturm:97a} quantifier elimination \texttt{rlqe}. If the
result is ``true'' or ``false,'' then this is printed. Otherwise, if
all quantifiers have been successfully eliminated there is ``$\top$''
output, else ``$\bot$.'' These symbols are followed by ``($\langle
k\rangle$q/$\langle n\rangle$at)'' giving the numbers $\langle
k\rangle$ of quantifiers and of atomic formulas $\langle n\rangle$
contained in the result.

\section{Switches}\label{SE:switches}
The systematic evaluation of switches was one major motivation for the
development of \textsc{breduce}: With computer algebra software
developers are often faced with design alternatives where it is not at
all clear which choice is the best one. It is a helpful technique to
introduce at least temporarily \textsc{reduce} switches which allow to
interactively choose between the options. This allows to experiment
with the options and to experimentally adjust the switches to default
settings that deliver satisfactory performance and quality of results
for most input data. At a later stage, there are usually some switches
completely removed while others remain in order to allow to adapt the
software to certain special input data which benefits from alternative
settings.

The idea is now to use \textsc{breduce} in this development process
for systematically evaluating the effect of all possible combinations
of certain switches.

For illustration, we use two switches. The switch \texttt{nopowers} is
off by default. When on, multiple factors are collected in one flat
list rather than building a pair from each factor and its
multiplicity. The switch \texttt{rounded} toggles the use of
fixed-precision floating point numbers, which does not have any
relevant effect on the computation considered here but illustrates the
treatment of multiple switches. The following \texttt{fac2.breduce}
serves also as an example for the result processing option
\texttt{breduce\_tex} discussed in the previous section:
\begin{verbatim}
REDUCE='reduce'
seriesinstances='7 8 9 10'
switches='nopowers rounded'
command='factorize'
processresult='breduce_tex'
\end{verbatim}
\begin{figure}[t]
  \begin{center}
    \fbox{\begin{minipage}{0.95\textwidth}
        \small
        \noindent sturm using reduce on lennier.local (i386 Darwin)
        \smallskip

        \noindent start of job on  Thu Jan 3 18:14:47 CET 2008
        \noindent\begin{longtable}{lccrr}
          \hline
          instance & nopowers & rounded & result & time (ms)\\
          \hline
          \endfirsthead
          \hline
          instance & nopowers & rounded & result & time (ms)\\
          \hline
          \endhead
          \hline
          7 &  $\circ$  &  $\circ$  & $\{\{7\co 1\}\}$ & 0\\
          7 &  $\circ$  & $\bullet$ & $\{\{7\co 1\}\}$ & 0\\
          7 & $\bullet$ &  $\circ$  & $\{7\}$ & 0\\
          7 & $\bullet$ & $\bullet$ & $\{7\}$ & 0\\
          \hline
          8 &  $\circ$  &  $\circ$  & $\{\{2\co 3\}\}$ & 0\\
          8 &  $\circ$  & $\bullet$ & $\{\{2\co 3\}\}$ & 0\\
          8 & $\bullet$ &  $\circ$  & $\{2\co 2\co 2\}$ & 0\\
          8 & $\bullet$ & $\bullet$ & $\{2\co 2\co 2\}$ & 0\\
          \hline
          9 &  $\circ$  &  $\circ$  & $\{\{3\co 2\}\}$ & 0\\
          9 &  $\circ$  & $\bullet$ & $\{\{3\co 2\}\}$ & 0\\
          9 & $\bullet$ &  $\circ$  & $\{3\co 3\}$ & 0\\
          9 & $\bullet$ & $\bullet$ & $\{3\co 3\}$ & 0\\
          \hline
          10 &  $\circ$  &  $\circ$  & $\{\{2\co 1\}\co \{5\co 1\}\}$ & 0\\
          10 &  $\circ$  & $\bullet$ & $\{\{2\co 1\}\co \{5\co 1\}\}$ & 0\\
          10 & $\bullet$ &  $\circ$  & $\{2\co 5\}$ & 0\\
          10 & $\bullet$ & $\bullet$ & $\{2\co 5\}$ & 0\\
          \hline
        \end{longtable}
        \noindent end of job on  Thu Jan 3 18:14:49 CET 2008
      \end{minipage}}
    \caption{\texttt{fac2.tex}\label{FI:fac2}}
  \end{center}
\end{figure}
The table obtained in \texttt{fac2.tex} is displayed in
Figure~\ref{FI:fac2}.

In the directory \texttt{fac2/} we obtain \textsc{reduce} log files
named \texttt{instance1-0}, \dots, \texttt{instance1-3},
\texttt{instance2-0}, \dots, \texttt{instance5-3}. Notice that switch
settings are arranged in such a way that they correspond to the binary
expansion of the appended number ``\texttt{-<n>}''.

\section{Packages and Initcommands}
\label{SE:initcommands}
The keyword \texttt{packages} contains a whitespace-separated list of
packages to be loaded. Here is an example:
\begin{verbatim}
packages='groebner groebnr2'
\end{verbatim}
The keyword \texttt{initcommands} takes \textsc{reduce} commands,
which are executed right after the packages are loaded, as for
instance
\begin{verbatim}
initcommands='on gc,gsugar; in "my.red"; torder lex;'
\end{verbatim}
Notice that here the switches are simply switched on in contrast to
being systematically tested as with the \texttt{switch} keyword
discussed in the previous section. It may in fact make sense to switch
on \texttt{gc} in order to obtain corresponding information from the
log files. The file \texttt{my.red} could provide suitable procedures
for \texttt{processresult} as discussed in
Section~\ref{SE:processresults}.

In contrast to all other keywords, the content of
\texttt{initcommands} is not a white\-space-sep\-a\-rated list.
Instead it has to be specified in such a way that it can be literally
pasted into \textsc{reduce}.

\section{Commands}\label{SE:commands}
In this section we want to discuss in more detail the keyword
\texttt{command}, which we had introduced already in
Section~\ref{SE:first}.

To start with, we can specify a list of procedure names as for
instance
\begin{verbatim}
commands='factorize factorial'
\end{verbatim}
which is interpreted as a nested function call
\texttt{factorize(factorial(\dots))}. The same specification holds for
\texttt{processresult}. There is a subtle point about whether to
include a procedure into \texttt{command} or into
\texttt{processresult}: The time measurement given in the last column
of the generated \LaTeX{} include all procedures in \texttt{command}
but not those in \texttt{processresult}. In order to obtain precise
timings for the computations, the procedures in command are executed
in \textsc{reduce} with a trailing '\texttt{\$}' such that the timings
do not include any printing times. Users explicitly interested in
including printing times would add \texttt{write} as the leftmost
procedure in \texttt{command}.

There is one apparent limitation of the approach: The rightmost
procedure in \texttt{commands} and thus the entire chain of nested
procedure calls specified by \texttt{commands} must establish a
\emph{unary} function. This can, however, easily be worked around as
follows: Imagine we actually want to work with a binary
\texttt{myproc(x,y)}. Then we would provide an additional procedure
\begin{verbatim}
algebraic procedure myproc_nospread(l);
   myproc(first l,second l);
\end{verbatim}
In the configuration file we would use this as follows:
\begin{verbatim}
commands='myproc myproc_nospread'
\end{verbatim}
Consequently, for each instance of a series the two arguments for
\texttt{myproc} have to be provided in two-element algebraic mode
lists. This approach straightforwardly generalizes to arbitrary
numbers of arguments. \textsc{reduce} provides \texttt{nth(<l>,<n>)}
for obtaining the \texttt{<n>}-th element of a list \texttt{<l>}.

Recall the discussion on the systematic evaluation of switches at the
beginning of Section~\ref{SE:switches}. One would obviously like to
similarly compare and evaluate alternative implementations
\texttt{myfun1} and \texttt{myfun2} of some algorithm. For this one
would introduce, say in a file \texttt{myfun.red}, corresponding
switches and a procedure testing these switches:
\begin{verbatim}
switch use_myfun2;

algebraic procedure myfun0(x);
   if lisp !*use_myfun2 then myfun2 x else myfun1 x;
\end{verbatim}
A suitable configuration file \texttt{myfun.breduce} would then
contain
\begin{verbatim}
initcommands='in "myfun.red";'
switches='use_myfun2'
command='myfun0'
\end{verbatim}
\section{Random}
Let us determine the number of (different) factors of some not too
small random numbers. For this purpose we write the following
configuration file \texttt{fac3a.breduce}:
\begin{verbatim}
switches='nopowers'
seriesinstances='random(10^37) random(10^38) random(10^39)'
command='factorize'
processresult='length'
\end{verbatim}
The \texttt{reduce} procedure \texttt{random(<n>)} returns a random
number between $0$ and \texttt{<n>}. It is clear that for \texttt{off
  nopowers} the length of the list obtained from \texttt{factorize} is
the number of different factors while for \texttt{on nopowers} we
obtain the number of factors counting multiplicities. Since the output
of \texttt{length} is simply a number we need not any
\texttt{processresult} command.
\begin{figure}[p]
  \begin{center}
    \fbox{\begin{minipage}{0.95\textwidth}
        \small
        \noindent sturm using reduce on lennier.local (i386 Darwin)
        \smallskip

        \noindent start of job on  Fri Jan 4 18:47:16 CET 2008
        \noindent\begin{longtable}{lcrr}
          \hline
          instance & nopowers & result & time (ms)\\
          \hline
          \endfirsthead
          \hline
          instance & nopowers & result & time (ms)\\
          \hline
          \endhead
          1870895791191171734284970352061524775 &  $\circ$  & 2 & 20\\
          1870895791191171734284970352061524775 & $\bullet$ & 3 & 20\\
          \hline
          91870895791191171734284970352061524775 &  $\circ$  & 4 & 80\\
          91870895791191171734284970352061524775 & $\bullet$ & 5 & 80\\
          \hline
          891870895791191171734284970352061524775 &  $\circ$  & 6 & 20\\
          891870895791191171734284970352061524775 & $\bullet$ & 7 & 20\\
          \hline
        \end{longtable}
        \noindent end of job on  Fri Jan 4 18:47:16 CET 2008
      \end{minipage}}
    \caption{\texttt{fac3.tex}\label{FI:fac3}}
  \end{center}
\end{figure}
Looking at the obtained \texttt{fac3.tex} in Figure~\ref{FI:fac3} we
observe two things about our randomly generated numbers:
\begin{enumerate}
\item In each instance we obtain for both switch settings the same
  random number although these have been generated in different
  \textsc{reduce} sessions.
\item Comparing the random numbers of different sizes obtained in the
  various instances they turn out to be all very similar.
\end{enumerate}
We are going to explain the situation without going into technical
details: Since every single line in out table has been computed in a
fresh \texttt{reduce} session, we have started every time with a
random generator that has been freshly initialized in the very same
way. To avoid identities between random numbers of the same size and
the observed similarities for different sizes, we must explicitly
initialize the random generator. \textsc{breduce} offers four
alternative styles of initialization. They can be specified via the
keyword \texttt{random}:
\begin{enumerate}[(a)]
\item \verb!random='breduce_instance'!
\item \verb!random='breduce_every'!
\item \verb!random='breduce_instance_abs'!
\item \verb!random='breduce_every_abs'!
\end{enumerate}

\texttt{breduce\_instance} uses the number of the current instance for
initialization. Recall that instances are by definition the entries in
\texttt{seriesinstances}. When trying all possible combinations of
switches, these belong to the same instance, and would thus generate
the same random numbers. This is a good choice for our example
considered here.
\begin{figure}[p]
  \begin{center}
    \fbox{\begin{minipage}{0.95\textwidth}
        \small
        \noindent sturm using reduce on lennier.local (i386 Darwin)

        \noindent start of job on  Sat Jan 5 09:30:14 CET 2008
        \noindent\begin{longtable}{lcrr}
          \hline
          instance & nopowers & result & time (ms)\\
          \hline
          \endfirsthead
          \hline
          instance & nopowers & result & time (ms)\\
          \hline
          \endhead
          1870895791191171734284970352061524775 &  $\circ$  & 2 & 20\\
          1870895791191171734284970352061524775 & $\bullet$ & 3 & 20\\
          \hline
          80834363874062020393350577869448966607 &  $\circ$  & 2 & 870\\
          80834363874062020393350577869448966607 & $\bullet$ & 2 & 870\\
          \hline
          469797831956933869052426185386836408439 &  $\circ$  & 4 & 110\\
          469797831956933869052426185386836408439 & $\bullet$ & 6 & 110\\
          \hline
        \end{longtable}
        \noindent end of job on  Sat Jan 5 09:30:18 CET 2008
      \end{minipage}}
    \caption{\texttt{fac3a.tex}\label{FI:fac3a}}
  \end{center}
\end{figure}
\unskip The result is displayed in Figure~\ref{FI:fac3a}.

\texttt{breduce\_every} initializes with the sum of the instance
number and the decimal representation of the current switch setting.
\begin{figure}[p]
  \begin{center}
    \fbox{\begin{minipage}{0.95\textwidth}
        \small
        \noindent sturm using reduce on lennier.local (i386 Darwin)
        \smallskip

        \noindent start of job on  Sat Jan 5 09:33:09 CET 2008
        \noindent\begin{longtable}{lcrr}
          \hline
          instance & nopowers & result & time (ms)\\
          \hline
          \endfirsthead
          \hline
          instance & nopowers & result & time (ms)\\
          \hline
          \endhead
          1870895791191171734284970352061524775 &  $\circ$  & 2 & 20\\
          834363874062020393350577869448966607 & $\bullet$ & 6 & 12080\\
          \hline
          80834363874062020393350577869448966607 &  $\circ$  & 2 & 870\\
          69797831956933869052426185386836408439 & $\bullet$ & 5 & 30\\
          \hline
          469797831956933869052426185386836408439 &  $\circ$  & 4 & 110\\
          258761300039804717711491792904223850271 & $\bullet$ & 5 & 5180\\
          \hline
        \end{longtable}
        \noindent end of job on  Sat Jan 5 09:33:28 CET 2008
      \end{minipage}}
    \caption{\texttt{fac3b.tex}\label{FI:fac3b}}
  \end{center}
\end{figure}
This results in a different initialization for every single row of the
table as displayed in Figure~\ref{FI:fac3b}. Note, however, that we
are then not testing the switch combination for an instance on the
same input data, which we probably want in most situations.

With both \texttt{breduce\_instance} and \texttt{breduce\_every} we
still observe that there is the essentially same sequence of random
numbers used every time. Since this might not really feel like random
to some users there are \texttt{breduce\_instance\_all} and
\texttt{breduce\_every\_all}. These are variants of the discussed
options, which add for initialization also the wall clock time and
date. Technically, it uses the number of seconds since the epoch in
the sense of Unix (00:00:00 UTC, January 1, 1970).

In order to make also such \textsc{breduce} jobs reproducible, the
number of seconds used is dumped into the \LaTeX{} output after the
configuration file using the keyword \texttt{epoch}. When including
this dumped \texttt{epoch} line into the configuration file, its value
will be used instead of the real time and date and thus reproduce the
result of the considered \textsc{breduce} job.

\section{Limiting the Computation Time}

\section{Instance Files}\label{SE:files}

\section{Signal Handling}

\section{Keyword Summary}
\begin{description}
\item[\texttt{command}]
\item[\texttt{headline}]
\item[\texttt{initcommands}]
\item[\texttt{killtime}]
\item[\texttt{packages}]
\item[\texttt{processresult}]
\item[\texttt{random}]
\item[\texttt{REDUCE}]
\item[\texttt{seriesinstances}]
\item[\texttt{seriesfilebasename}]
\item[\texttt{seriesprintname}]
\item[\texttt{switches}]
\end{description}

\bibliographystyle{alpha}
\bibliography{adts}

\end{document}

\section{Preparing the Input}
For each instance of a series there must be one \textsc{reduce} input file
present. The last evaluation result of each input file yields the
input argument for the desired computation. To put it more
technically: after reading the input file with \texttt{in}, the input
argument is expected to be available in \texttt{ws}.

Besides the input files for the series instances the user must provide
a \textsc{breduce} configuration file \texttt{<name>.breduce}. This configuration file
contains all information on the batch run except for the input
arguments discussed above.

As an example, we apply quantifier elimination to a series of three
instances known as the \emph{rectangle} problem. For this we
\texttt{mkdir} a directory \texttt{rp/}. Inside \texttt{rp/}, we prepare
3 input argument files \texttt{rp1.red}, \dots, \texttt{rp3.red}. Each
of these files contains a formula describing the corresponding
instance of the problem. Recall that the last line of \textsc{reduce} files
must be ``\texttt{end;}''. Finally, we put into \texttt{rp/} our
configuration file \texttt{simple-rp.breduce} reads as follows:
\begin{verbatim}
seriesfilebasename='rp'
seriesinstances='1 2 3'
package='redlog'
initcommands='rlset reals; on rlverbose;'
command='rlqe'
processresult='rlatnum'
\end{verbatim}

\section{Running \textsc{breduce}}
For running \textsc{breduce} within out directory \texttt{rp/}, we use the
command
\begin{verbatim}
breduce simple-rp.breduce
\end{verbatim}
where the extension may be omitted. Now the following happens: There
are three instances computed. Their names \texttt{rp1}, \texttt{rp2},
and \texttt{rp3} are constructed from \texttt{seriesfilebasename} and
\texttt{seriesinstances} given in \texttt{simple-rp.breduce}.

For each instance \texttt{rp<i>}, \textsc{breduce} proceeds as follows: It
silently starts a new non-interactive \textsc{reduce} session, loads the
package \texttt{redlog} specified in \texttt{package} and evaluates
the commands specified in \texttt{initcommands}. Then it reads
\begin{verbatim}
in "rp<i>.red";
\end{verbatim}
and applies quantifier elimination \texttt{rlqe ws} as specified in
\texttt{command}. The result of the quantifier elimination is
processed with \texttt{rlatnum} (count number of atomic formulas) as
specified in \texttt{processresult}.

All information obtained this way is collected in a \LaTeX{} document
\texttt{rp.tex}, the name of which is derived from the name of configuration
file \texttt{rp.breduce}. If there already exists a file
\texttt{rp.tex}, then this is moved to \texttt{rp.tex.old} first. In
addition, there is a directory \texttt{simple-rp/} created, which
contains log files \texttt{rp1.log}, \texttt{rp2.log},
\texttt{rp3.log} of the three \textsc{reduce} sessions.

\begin{figure}
  \begin{center}
    \fbox{\begin{minipage}{0.9\textwidth}
        \noindent sturm using reduce.psl on lennier.local (i386 Darwin)
        \smallskip

        \noindent start of job on  Sun Dec 30 16:41:20 CET 2007
        \noindent\begin{longtable}{lrr}
          \hline
          example & result & time (ms)\\
          \hline
          \endfirsthead
          \hline
          example & result & time (ms)\\
          \hline
          \endhead
          rp1 & 5 & 0\\
          \hline
          rp2 & 50 & 80\\
          \hline
          rp3 & 6909 & 54000\\
          \hline
        \end{longtable}
        \noindent end of job on  Sun Dec 30 16:42:17 CET 2007
        \bigskip\bigskip

        \begin{tt}
          \footnotesize
          working directory was /Users/sturm/rp\\
          configuration file ./simple-rp.breduce:\\
          seriesfilebasename='rp'\\
          seriesinstances='1 2 3'\\
          package='redlog'\\
          initcommands='rlset reals; on rlverbose;'\\
          command='rlqe'\\
          processresult='rlatnum'
        \end{tt}
      \end{minipage}}
    \caption{The formatted \texttt{simple-rp.tex}\label{simple-rp.tex}}
  \end{center}
\end{figure}

Figure \ref{simple-rp.tex} displays the \LaTeX-formatted
\texttt{simple-rp.tex}. Since are using the package
\texttt{longtable}, which allows to typeset tables that split over
several pages, it is sometimes necessary to run \LaTeX{} several times
until the table is properly arranged.

\end{document}
