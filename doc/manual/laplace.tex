
\index{Kazasov, C.}\index{Persons!Kazasov, C.}
\index{Spiridonova, M.}\index{Persons!Spiridonova, M.}
\index{Tomov, V.}\index{Persons!Tomov, V.}

\index{Laplace transform}
\index{Integral transform!Laplace transform}

\hypertarget{operator:LAPLACE}{}
\hypertarget{operator:INVLAP}{}
\ttindextype[LAPLACE]{laplace}{operator}
\ttindextype[LAPLACE]{invlap}{operator}
Reference: \cite{Kazasov:87}.

Some hints on how to use to use this package: \\[\baselineskip]
%
Syntax:
\begin{syntax}
  \f{laplace}(\meta{exp},\meta{var-s},\meta{var-t}) \\[\baselineskip]
%
  \f{invlap}(\meta{exp},\meta{var-s},\meta{var-t})
\end{syntax}
%
where \meta{exp} is the expression to be transformed, \meta{var-s} is the source
variable (in most cases \meta{exp} depends explicitly of this variable) and
\meta{var-t} is the target variable. If \meta{var-t} is omitted, the above operators use
an internal variable \var{lp!\&} or \var{il!\&}, respectively.

The following switches can be used to control the transformations: \\
\ttindexswitch[LAPLACE]{lmon}
\ttindexswitch[LAPLACE]{lhyp}
\ttindexswitch[LAPLACE]{ltrig}
\hypertarget{switch:LMON}{}
\hypertarget{switch:LHYP}{}
\hypertarget{switch:LTRIG}{}
\begin{description}
\item[\sw{lmon}:] If on, sin, cos, sinh and cosh are converted by \f{laplace} into
exponentials,
\item[\sw{lhyp}:] If on, expressions $e^{\tilde\ x}$ are converted by \f{invlap} into 
hyperbolic functions sinh and cosh,
\item[\sw{ltrig}:] If on, expressions $e^{\tilde\ x}$ are converted by \f{invlap} into
trigonometric functions sin and cos.
\end{description}

The system can be extended by adding Laplace transformation rules for
single functions by rules or rule sets.~ In such a rule the source
variable \emph{must} be free, the target variable \emph{must} be \var{il!\&} for \f{laplace} and
\var{lp!\&} for \f{invlap} and the third parameter should be omitted.~ Also rules for
transforming derivatives are entered in such a form.

\pagebreak
\textbf{Examples:}
\begin{verbatim}

    let {laplace(log(~x),x) => -log(gam * il!&)/il!&,

    invlap(log(gam * ~x)/x,x) => -log(lp!&)};


    operator f;

    let{

    laplace(df(f(~x),x),x) => il!&*laplace(f(x),x) - sub(x=0,f(x)),

    laplace(df(f(~x),x,~n),x) => il!&**n*laplace(f(x),x) -

    for i:=n-1 step -1 until 0 sum

    sub(x=0, df(f(x),x,n-1-i)) * il!&**i

    when fixp n,

    laplace(f(~x),x) = f(il!&)

    };

\end{verbatim}


Remarks about some functions: \\[\baselineskip]
The \f{delta} and \f{gamma} functions are known. \\
ONE is the name of the unit step function. \\
INTL is a parametrized integral function
\begin{syntax}
\f{intl}(\meta{expr},\meta{var},0,\meta{obj.var})
\end{syntax}
which means ``Integral of \meta{expr} w.r.t.~\meta{var} taken from 0 to \meta{obj.var}'',
e.g. \texttt{intl($2{*}y^2,y,0,x$)} which is formally a function in $x$.\\
We recommend reading the file \texttt{laplace.tst} for a further introduction.
