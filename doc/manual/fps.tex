\index{Koepf, Wolfram}\index{Persons!Koepf, Wolfram}
\index{Neun, Winfried}\index{Persons!Neun, Winfried}

\subsection{Introduction}
This package can expand functions of certain type into
their corresponding Laurent-Puiseux series as a sum of terms of the form
\[
\sum_{k=0}^{\infty} a_{k} (x-x_{0})^{m k/n + s}
\]
where $m$ is the `symmetry number', $s$ is the `shift number',
$n$ is the `Puiseux number',
and $x_0$ is the `point of development'. The following types are
supported:
\begin{itemize}
\item
textbf{functions of `rational type'}, which are either rational or have a
rational derivative of some order;
\item
\textbf{functions of `hypergeometric type'} where $a(k+m)/a(k)$ is a rational
function for some integer $m$;
\item
\textbf{functions of `explike type'} which satisfy a linear homogeneous
differential equation with constant coefficients.
\end{itemize}

The \package{FPS} package is an implementation of the method
presented in \cite{Koepf:92}. The implementations of this package
for \textsc{Maple} (by D.\ Gruntz) and \textsc{Mathematica} (by W.\ Koepf)
served as guidelines for this one.

Numerous examples can be found in \cite{Koepf:93a,Koepf:93b}, 
most of which are contained in the test file \texttt{fps.tst}. Many 
more examples can be found in the extensive bibliography of Hansen \cite{Hansen:75}.


\subsection{\REDUCE{} operator \texttt{FPS}}

\ttindextype[FPS]{fps}{operator}
\hypertarget{operator:FPS}{}
\texttt{fps(f,x,x0)} tries to find a formal power
series expansion for \texttt{f} with respect to the variable \texttt{x} 
at the point of development \texttt{x0}. 
It also works for formal Laurent (negative exponents) and Puiseux series
(fractional exponents). If the third 
argument is omitted, then \texttt{x0:=0} is assumed.

Examples: \texttt{fps(asin(x)\textasciicircum2,x)} results in
\begin{verbatim}

         2*k  2*k             2  2
        x   *2   *factorial(k) *x
infsum(----------------------------,k,0,infinity)
        factorial(2*k + 1)*(k + 1)
\end{verbatim}
\texttt{fps(sin x,x,pi)} gives
\begin{verbatim}
                   2*k       k
        ( - pi + x)   *( - 1) *( - pi + x)
infsum(------------------------------------,k,0,infinity)
                factorial(2*k + 1)
\end{verbatim}
and \texttt{fps(sqrt(2-x\textasciicircum2),x)} yields
\begin{verbatim}
            2*k
         - x   *sqrt(2)*factorial(2*k)
infsum(--------------------------------,k,0,infinity)
           k             2
          8 *factorial(k) *(2*k - 1)
\end{verbatim}
\ttindextype[FPS]{infsum}{operator}
\hypertarget{operator:INFSUM}{}
Note: The result contains one or more \f{infsum} terms such that it does
not interfere with the {\REDUCE} operator \texttt{sum}. In graphical oriented
REDUCE interfaces this operator results in the usual $\sum$ notation.

If possible, the output is given using factorials. In some cases, the
use of the Pochhammer symbol \texttt{pochhammer(a,k)}$:=a(a+1)\cdots(a+k-1)$
is necessary.

The operator \texttt{fps} uses the operator \texttt{SimpleDE} of the next section.

If an error message of type
\begin{verbatim}
Could not find the limit of:
\end{verbatim}
occurs, you can set the corresponding limit yourself and try a
recalculation. In the computation of \texttt{fps(atan(cot(x)),x,0)},
REDUCE is not able to find the value for the limit 
\texttt{limit(atan(cot(x)),x,0)} since the \texttt{atan} function is multi-valued.
One can choose the branch of \texttt{atan} such that this limit equals
$\pi/2$ so that we may set 
\begin{verbatim}
let limit(atan(cot(~x)),x,0)=>pi/2;
\end{verbatim}
and a recalculation of \texttt{fps(atan(cot(x)),x,0)}
yields the output \texttt{pi - 2*x} which is
the correct local series representation.

\subsection{\REDUCE{} operator \texttt{SimpleDE}}
\ttindextype[FPS]{SimpleDE}{operator}
\hypertarget{operator:SIMPLEDE}{}

\texttt{SimpleDE(f,x)} tries to find a homogeneous linear differential
equation with polynomial coefficients for $f$ with respect to $x$.
Make sure that $y$ is not a used variable.
The setting \texttt{factor df;} is recommended to receive a nicer output form.

Examples: \texttt{SimpleDE(asin(x)\textasciicircum2,x)} then results in
\begin{verbatim}
            2
df(y,x,3)*(x  - 1) + 3*df(y,x,2)*x + df(y,x)
\end{verbatim}
\texttt{SimpleDE(exp(x\textasciicircum (1/3)),x)} gives
\begin{verbatim}
              2
27*df(y,x,3)*x  + 54*df(y,x,2)*x + 6*df(y,x) - y
\end{verbatim}
and \texttt{SimpleDE(sqrt(2-x\textasciicircum2),x)} yields
\begin{verbatim}
          2
df(y,x)*(x  - 2) - x*y
\end{verbatim}
\ttindextype[FPS]{fps\_search\_depth}{shared variable}
\hypertarget{reserved:FPS_SEARCH_DEPTH}{}
The depth for the search of a differential equation for \texttt{f} is
controlled by the variable \texttt{fps\_search\_depth};
A higher value for \texttt{fps\_search\_depth}
will increase the chance to find the solution, but increases the
complexity as well. The default value for \texttt{fps\_search\_depth} 
is $5$. E.~g., for \texttt{fps(sin(x\^{}(1/3)),x)}, or 
\texttt{SimpleDE(sin(x\^{}(1/3)),x)} a setting
\texttt{fps\_search\_depth:=6} is necessary.

\ttindextype{tracefps}{switch}
\hypertarget{switch:TRACEFPS}{}
The output of the \package{FPS} package can be influenced by the
switch \sw{tracefps}. Setting \texttt{on tracefps} causes various
prints of intermediate results.

\subsection{Problems in the current version}
The handling of logarithmic singularities is not yet implemented.

The rational type implementation is not yet complete.

The support of special functions \cite{Koepf:94c}
will be part of the next version.
