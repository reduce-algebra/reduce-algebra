\chapter{Commands for Interactive Use}
\index{Interactive use}
\label{interactive}

{\REDUCE} is designed as an interactive system, but naturally it can
also operate in a batch processing or background mode by taking its
input command by command from the relevant input stream.  There is a
basic difference, however, between interactive and batch use of the
system.  In the former case, whenever the system discovers an
ambiguity at some point in a calculation, such as a forgotten type
assignment for instance, it asks you for the correct interpretation.
In batch operation, it is not practical to terminate the calculation
at such points and require resubmission of the job, so the system
makes the most obvious guess of your intentions and continues the
calculation.

\section{Error Handling:
  \texttt{\small errcont}, \texttt{\small retry}}

\hypertarget{switch:ERRCONT}{}
There is also a difference in the handling of errors.  In the former
case, the computation can continue since you have the opportunity to
correct the mistake.  In batch mode, the error may lead to consequent
erroneous (and possibly time consuming) computations.  So in the
default case, no further evaluation occurs, although the remainder of
the input is checked for syntax errors.  A message \texttt{"Continuing
  with parsing only"} informs you that this is happening.  On the
other hand, the switch \sw{errcont},\ttindexswitch{errcont} if on,
will cause the system to continue evaluating expressions after such
errors occur.

\hypertarget{command:RETRY}{}
When a syntactical error occurs, the place where the system detected
the error is marked with three dollar signs (\texttt{\$\$\$}). In
interactive mode, you can then use \texttt{ed}\ttindextype{ed}{command} to correct
the error, or retype the command.  When a non-syntactical error occurs
in interactive mode, the command being evaluated at the time the last
error occurred is saved, and may later be reevaluated by the command
\texttt{retry}.\ttindextype{retry}{command}

\section{Referencing Previous Results:
  \texttt{\small input}, \texttt{\small ws}, \texttt{\small display}}

\hypertarget{reserved:INPUT}{}
\hypertarget{reserved:WS}{}
It is often useful to be able to reference results of previous
computations during a {\REDUCE} session; see also \ref{Workspace}.
For this purpose, {\REDUCE} maintains a history\index{History} of all
interactive inputs and the results of all interactive computations
during a given session.  These results are referenced by the command
number that {\REDUCE} prints automatically in interactive mode.  To
use a previous input expression in a new computation, write
\texttt{input(}$n$\texttt{)},\ttindex{input} where $n$ is the command
number.  To use a previous output expression, write
\texttt{ws(}$n$\texttt{)}\ttindex{ws} (where WS stands for WorkSpace).
\texttt{ws} used as a variable (rather than a function) references the
previous output expression.  For example:
\begin{verbatim}
1: int(x-1, x);

 x*(x - 2)
-----------
     2
\end{verbatim}
\(\vdots\)
\begin{verbatim}
7: (x^2-1)/(x+1);

x - 1
\end{verbatim}
\(\vdots\)
\begin{verbatim}
15: 2*input(1)-ws(7)^2;

-1

16: 2*ws(1)-ws(7)^2;

-1

17: x := 101;

x := 101

18: ws(7);

100
\end{verbatim}
Inputs 15 and 16 above yield the same result, but input 16 does so
\emph{without} recomputing the integral.  However, an output
expression referenced using \texttt{ws} is re-evaluated in the current
context, as shown by the last two statements above.

Note that input that causes an error, and some commands such as
\texttt{let} statements, file handling and mode changing, do not
produce an output expression, so the output from such input cannot be
accessed.  \texttt{ws} used as a variable returns the last output
expression, which does not necessarily correspond to the last input,
and \texttt{ws} used as a function reports an error if you try to
access non-existent output.  For example:
\begin{verbatim}
1: 6*7;

42

2: 0/0;

***** 0/0 formed

3: ws;

42

4: ws 2;

***** Entry 2 not found

5: let x => 0;

6: ws;

42

7: ws 5;

***** Entry 5 not found
\end{verbatim}

\hypertarget{operator:DISPLAY}{}
The operator \texttt{display}\ttindextype{display}{operator} is available to display previous
inputs.  If its argument is a positive integer, $n$ say, then the
previous $n$ inputs are displayed.  If its argument is \texttt{all} (or in fact
any non-numerical expression), then all previous inputs are displayed.

\section{Interactive Editing:
  \texttt{\small ed}, \texttt{\small editdef}}

\hypertarget{command:ED}{}
It is possible when working interactively to edit any {\REDUCE} input
that comes from your terminal, and also some user-defined procedure
definitions.  At the top level, you can access any previous command
string by the command \texttt{ed(}$n$\texttt{)},\ttindextype{ed}{command}
\index{Command!\texttt{ed}} where $n$
is the desired command number as prompted by the system in interactive
mode.  The command \texttt{ed} (with no argument) accesses the
previous command.

After \texttt{ed} has been called, you can now edit the displayed string using a
string editor with the following commands:

\begin{tabular}{@{\hspace{7mm}}lp{\rboxwidth}}
\texttt{b} & move pointer to beginning \\
\texttt{c}\meta{character} & replace next character by \meta{character} \\
\texttt{d} & delete next character \\
\texttt{e} & end editing and reread text \\
\texttt{f}\meta{character} & move pointer to next
occurrence of \meta{character} \\[1.7pt]
\texttt{i}\meta{string}\meta{escape} &
 insert \meta{string} in front of pointer \\
\texttt{k}\meta{character} & delete all characters
 until \meta{character} \\
\texttt{p} & print string from current pointer \\
\texttt{q} & give up with error exit \\
\texttt{s}\meta{string}\meta{escape} &
 search for first occurrence of \meta{string},
                             positioning pointer just before it \\
\texttt{space} or \texttt{x} & move pointer right
one character.
\end{tabular}

The above table can be displayed online by typing a question mark followed
by a carriage return to the editor. The editor prompts with an angle
bracket. Commands can be combined on a single line, and all command
sequences must be followed by a carriage return to become effective.

Thus, to change the command \texttt{x := a+1;} to \texttt{x := a+2}; and cause
it to be executed, the following edit command sequence could be used:
\begin{verbatim}
        f1c2e<return>
\end{verbatim}
\hypertarget{command:EDITDEF}{}
You can also use the interactive editor to edit a user-defined
procedure that has not been compiled.  To do this, use:
\ttindextype{editdef}{command}
\begin{syntax}
  \texttt{editdef }\meta{id}\texttt{;}
\end{syntax}
where \meta{id} is the name of the procedure.  The procedure definition
will then be displayed in editing mode, and may then be edited and
redefined on exiting from the editor.

Some versions of {\REDUCE} include input editing that uses the
capabilities of modern window systems.  Please consult your system
dependent documentation to see if this is possible.  Such editing
techniques are usually much easier to use then \texttt{ed} or
\texttt{editdef}.

\section{Interactive File Control:
  \texttt{\small int}, \texttt{\small pause}, \texttt{\small cont}}

\hypertarget{switch:INT}{}
If input is coming from an external file, the system treats it as a
batch processed calculation.  If you desire
interactive\index{Interactive use} response in this case, you can
include the command \texttt{on int};\ttindexswitch{int} in the file.
Likewise, you can issue the command \texttt{off int}; in the main
program if you do not desire continual questioning from the system.
Regardless of the setting of the switch \sw{int}, input commands from
a file are not kept in the system, and so cannot be referenced using
\texttt{input} or \texttt{ws}, or edited using \texttt{ed}.  However,
an implementation of {\REDUCE} may provide a link to an external
system editor that can be used for such editing.  The specific
instructions for the particular implementation should be consulted for
information on this.

\hypertarget{CONT-and-PAUSE}{}
Two commands are available in {\REDUCE} for interactive use of files.
\texttt{pause};\ttindextype{pause}{command}\index{Command!pause@\texttt{pause}}
may be inserted at any point in an input file.  When this command is
encountered on input, the system prints the message \texttt{Cont? (Y or N)}
on your terminal and halts.  If you respond \texttt{y} (for
yes), the calculation continues from that point in the file.  If you
respond \texttt{n} (for no), control is returned to the terminal, and
you can input further statements and commands.  Later on you can use
the command
\texttt{cont;}\ttindextype{cont}{command}\index{Command!cont@\texttt{cont}}
to transfer control back to the point in the file following the last
\texttt{pause;} encountered.  A top-level \texttt{pause;} from the
terminal has no effect.
