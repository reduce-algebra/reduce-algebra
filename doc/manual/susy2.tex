\subsection{Introduction}

{ % to limit the scope of the following definitions:

\ifdefined\VerbMath\(
%% For MathJax, must be part of a paragraph to avoid extra vertical space:
\newcommand{\bos}{\mathbf{bos}}
\newcommand{\fer}{\mathbf{fer}}
\newcommand{\fun}{\mathbf{fun}}
\newcommand{\gras}{\mathbf{gras}}
\newcommand{\axp}{\mathbf{axp}}
\newcommand{\der}{\mathbf{der}}
\newcommand{\del}{\mathbf{del}}
\newcommand{\pr}{\mathbf{pr}}
\newcommand{\pg}{\mathbf{pg}}
\newcommand{\d}{\mathbf{d}}
\)%
\else
\newcommand{\bos}{\mathbf{bos}}
\newcommand{\fer}{\mathbf{fer}}
\newcommand{\fun}{\mathbf{fun}}
\newcommand{\gras}{\mathbf{gras}}
\newcommand{\axp}{\mathbf{axp}}
\newcommand{\der}{\mathbf{der}}
\newcommand{\del}{\mathbf{del}}
\newcommand{\pr}{\mathbf{pr}}
\newcommand{\pg}{\mathbf{pg}}
\renewcommand{\d}{\mathbf{d}}
\fi
The main idea of supersymmetry (SuSy) is to treat boson and fermion
operators equally \hyperlink{susy2-bib}{[1,2]}.  This has been realised
by introducing so-called supermultiplets constructed from the boson
and fermion operators and additionally from the Mayorana spinors.
Such supermultiplets possess the proper transformation property under
the transformation of the Lorentz group.  At the moment we have no
experimental confirmation that supersymmetry appears in nature.

The idea of using supersymmetry for the generalization of the soliton
equations~\hyperlink{susy2-bib}{[3--7]} appeared almost in parallel to
the usage of SuSy in the quantum field theory.  The first results,
concerning the construction of classical field theories with fermionic
and bosonic fields depending on time and one space variable can be
found in~\hyperlink{susy2-bib}{[8--12]}.  In many cases, the addition
of fermion fields does not guarantee that the final theory becomes
SuSy invariant and therefore this method was named the fermionic
extension in order to distinguish it from the fully SuSy method.

In order to get a SuSy theory we have to add to a system of $k$
bosonic equations $kN$ fermion and $k(N-1)$ boson fields
$(k=1,2,\ldots,\;N=1,2,\ldots)$ in such a way that the final theory
becomes SuSy invariant.  From the soliton point of view we can
distinguish two important classes of the supersymmetric equations: the
non-extended $(N = 1)$ and extended $(N > 1)$ cases.  Consideration of
the extended case may imply new bosonic equations whose properties
need further investigation.  This may be viewed as a bonus, but this
extended case is no more fundamental than the non-extended one.  The
problem of the supersymmetrization of the nonlinear partial
differential equations has its own history, and at the moment we have
no unique solution~\hyperlink{susy2-bib}{[13--40]}.  We can distinguish
three different methods of supersymmetrization: algebraic, geometric
and direct.

In the first two cases we are looking for the symmetry group of the
given equation and then we replace this group by the corresponding
SuSy group.  As a final product we are able to obtain a SuSy
generalization of the given equation.  The classification into the
algebraic or geometric approach is connected with the kind of symmetry
which appears in the classical case.  For example, if our classical
equation could be described in terms of the geometrical object then
the simple exchange of the classical symmetry group of this object
with its SuSy partner justifies the name geometric.  In the algebraic
case we are looking for the symmetry group of the equation without any
reference to its geometrical origin.  This strategy could be applied
to the so-called hidden symmetry as for example in the case of the
Toda lattice.  These methods each have advantages and disadvantages.
For example, sometimes we obtain the fermionic extensions.  In the
case of the extended supersymmetric Korteweg-de Vries equation we have
three different fully SuSy extensions; however only one of them fits
these two classifications.

In the direct approach we simply replace all objects which appear in
the evolution equation by all possible combinations of the
supermultiplets and their superderivatives so as to conserve the
conformal dimensions.  This is non-unique and yields many different
possibilities.  However, the arbitrariness is reduced if we
additionally investigate super-bi-hamiltonian structure or try to find
its supersymmetric Lax pair.  In many cases this approach is
successful.

The utilization of the above methods can be helped by symbolic
computer algebraic and for this reason we developed the package SuSy2
in the symbolic language REDUCE~\hyperlink{susy2-bib}{[41]}.

We have implemented and ordered the superfunctions in our program,
extensively using the concept of ``noncom operator'' in order to
implement the supersymmetric integro-differential operators.  The
program is meant to perform the symbolic calculations using either
fully supersymmetric supermultiplets or the component version of our
supersymmetry.  We have constructed 25 different commands to allow us
to compute almost all objects encountered in the supersymmetrization
procedure of the soliton equation.

\subsection{Supersymmetry}

The basic object in the supersymmetric analysis is the superfield and
the supersymmetric derivative.  The superfields are the superfermions
or the superbosons~\hyperlink{susy2-bib}{[1]}.  These fields, in the
case of extended $N=2$ supersymmetry, depend, in addition to $x$ and
$t$, upon two anticommuting variables, $\theta_{1}$ and $\theta_{2}$
(such that $\theta_{2}\theta_{1} = -
\theta_{1}\theta_{2},\;\theta_{1}^{2} = \theta_{2}^{2} = 0$).  Their
Taylor expansion with respect to $\theta_{1},\theta_{2}$ is
\begin{equation*}
  b(x,t,\theta_{1},\theta_{2}):=w+\theta_{1}\zeta_{1}+
  \theta_{2}\zeta_{2}+\theta_{2}\theta_{1}u
\end{equation*}
for superbosons and
\begin{equation*}
  f(x,t,\theta_{1},\theta_{2}):=\zeta_{1}+\theta_{1}w+
  \theta_{2}u+\theta_{2}\theta_{1}\zeta_{2}
\end{equation*}
for superfermions, where $w$ and $u$ are classical (commuting)
functions depending on $x$ and $t$, and $\zeta_{1}$ and $\zeta_{2}$
are odd Grassmann-valued functions depending on $x$ and $t$.

On the set of these superfunctions we can define the usual derivative
and the superderivative.  Usually, we encounter two different
realizations of the superderivative: the first we call ``traditional''
and the second ``chiral''.

The traditional realization can be defined by introducing two
superderivatives $D_{1}$ and $D_{2}$:
\begin{align*}
  D_{1} &= \partial_{\theta_{1}}+\theta_{1}\partial,\\
  D_{2} &= \partial_{\theta_{2}}+\theta_{2}\partial,
\end{align*}
with the properties:
\begin{gather*}
  D_{1} D_{1} = D_{2} D_{2} = \partial,\\
  D_{1} D_{2} + D_{2} D_{1} = 0.
\end{gather*}
The chiral realization is defined by
\begin{align*}
  D_{1} &= \partial_{\theta_{1}} - \frac{1}{2}\theta_{2}\partial,\\
  D_{2} &= \partial_{\theta_{2}} - \frac{1}{2}\theta_{1}\partial,
\end{align*}
with the properties:
\begin{gather*}
  D_{1} D_{1} = D_{2} D_{2} = 0,\\
  D_{1} D_{2} + D_{2} D_{1} = -\partial.
\end{gather*}

Below we shall use the names ``traditional'', ``chiral'' or
``chiral1'' algebras to denote the kind of commutation relations
assumed for the superderivatives.  The chiral1 algebras possess,
additionally to the chiral algebra, the commutator of $D_{1}$ and
$D_{2}$ defined by
\begin{equation*}
  D_{3} = D_{1} D_{2} -D_{2} D_{1}.
\end{equation*}
In the \package{SuSy2} package we have implemented the superfunctions
and the algebra of superderivatives.  Moreover, we have defined many
additional procedures which are useful in the supersymmetrization of
the classical nonlinear system of partial differential equation.
Different applications of this package to physical problems can be
found in the papers~\hyperlink{susy2-bib}{[34--38]}.

\subsection{Superfunctions}

In this manual entry, REDUCE procedure (function) and let-rule names
are usually displayed in a bold font, while all other input and output
is usually displayed as normal typeset mathematics.  The value
returned by a procedure (function) is indicated by the notation
\begin{equation*}
  \mathbf{function}(\mathit{arguments}) \Rightarrow
  \mathit{function~value}.
\end{equation*}
However, REDUCE input without corresponding output and REDUCE command
names are usually displayed in a typewriter font.

Superfunctions are represented in this package by
\begin{equation}\label{susy2-eqn1}
  \bos(f,0,0)
\end{equation}
for superbosons and
\begin{equation*}
  \fer(g,0,0)
\end{equation*}
for superfermions.

The first argument denotes the name of the given superobject, the
second denotes the value of the SuSy derivative, and the last denotes
the value of the usual derivative.  The $\bos$ and
$\fer$ objects are declared as \texttt{noncom} operators in
REDUCE\@.  The first argument can take an arbitrary value but with the
following restrictions:
\begin{gather*}
  \bos(0,m,n) = 0, \\
  \fer(0,m,n) = 0,
\end{gather*}
for all values of $m,n$.

The program has the capability to compute the coordinates of arbitrary
SuSy expressions, using expansions in powers of $\theta$.  We have
here four commands:

\begin{enumerate}
\item In order to have the given expression in components use
  \begin{equation*}
    \mathbf{fpart}(\mathit{expression}).
  \end{equation*}
  The output is in the form of a list, in which the first element is
  the zero-order term in $\theta$, the second is the first-order term
  in $\theta_{1}$, the third is the first-order term in $\theta_{2}$
  and the fourth is the term in $\theta_{2}\theta_{1}$.  For example,
  the superfunction \eqref{susy2-eqn1} has the representation
  \begin{equation*}
    \mathbf{fpart}(\bos(f,0,0)) \Rightarrow
    \{ \fun(f_{0},0), \gras(\mathit{ff}_{1},0),
    \gras(\mathit{ff}_{2},0),\fun(f_{1},0) \},
  \end{equation*}
  where $\fun$ denotes the classical function and
  $\gras$ denotes the Grassmann function.  The first argument
  in $\fun$ or $\gras$ denotes the name of the given
  object, while the second denotes the usual derivative.

\item In order to have the bosonic sector only, in which all odd
  Grassmann functions disappear, use
  \begin{equation*}
    \mathbf{bpart}(\mathit{expression}).
  \end{equation*}

  Example:
  \begin{equation*}
    \mathbf{bpart}(\fer(g,0,0)) \Rightarrow
    \{0, \fun(g_{0},0), \fun(g_{1},0), 0\}
  \end{equation*}

\item In order to have the given coordinates use
  \begin{equation*}
    \mathbf{bf\_part}(\mathit{expression},n),
  \end{equation*}
  where $n=0,1,2,3$.

  Example:
  \begin{equation*}
    \mathbf{bf\_part}(\bos(f,0,0),3) \Rightarrow \fun(f_{1},0)
  \end{equation*}

\item In order to have the given coordinates in the bosonic
  sector use
  \begin{equation*}
    \mathbf{b\_part}(\mathit{expression},n),
  \end{equation*}
  where $n=0,1,2,3$.

  Example:
  \begin{equation*}
    \mathbf{b\_part}(\fer(g,0,0),1) \Rightarrow \fun(g_{0},0)
  \end{equation*}
\end{enumerate}

Notice that the program switches on factoring of
$\fer,\bos,\gras,\fun$.  If you remove this factoring then many
commands give wrong results (for example the commands $\mathbf{lyst}$,
$\mathbf{lyst1}$ and $\mathbf{lyst2}$).

\subsection{The Inverse and Exponentials of Superfunctions}

In addition to our definitions of the superfunctions we can also
define the inverse and exponential of the superboson.

The inverse of the given $\bos$ function (not to be confused with the
``inverse function'' encountered in the usual analysis) is defined as
\begin{equation*}
  \bos(f,n,m,-1),
\end{equation*}
for arbitrary $f,n,m$ with the property
$\bos(f,n,m,-1)\,\bos(f,n,m,1)=1$.  The object $\bos(f,n,m,k)$, in
general, denotes the $k$-th power of the $\bos(f,n,m)$ superfunction.
If we use the command \texttt{let inverse} then three-index $\bos$
objects are transformed into four-index $\bos$ objects.

The exponential of the superboson function is
\begin{equation*}
  \axp(\bos(f,0,0)).
\end{equation*}
It is also possible to use $\axp(f)$, but then we should specify what
is $f$.

We have the following representation in components for the inverse and
$\axp$ superfunctions:
\begin{multline*}
  \mathbf{fpart}(\bos(f,0,0,-1)) = \{ \fun(f_{0},0,-1), -\fun(f_{0},0,-1)\,\gras(\mathit{ff}_{1},0), \\
  -\fun(f_{0},0,-1)\,\gras(\mathit{ff}_{2},0), - \fun(f_{0},0,-2)\,\fun(f_{1},0,1) \\
  + 2\,\fun(f_{0},0,-3)\,\gras(\mathit{ff}_{1},0)\,\gras(\mathit{ff}_{2},0) \},
\end{multline*}
\begin{multline*}
  \mathbf{fpart}(\axp(f)) = \{ \mathbf{axx}(\mathbf{bf\_part}(f,0)),
  \mathbf{axx}(\mathbf{bf\_part}(f,0))\,\mathbf{bf\_part}(f,1), \\
  \mathbf{axx}(\mathbf{bf\_part}(f,0))\,\mathbf{bf\_part}(f,2), \mathbf{axx}(\mathbf{bf\_part}(f,0)) \\
  (\mathbf{bf\_part}(f,3)+2\,\mathbf{bf\_part}(f,1)\,\mathbf{bf\_part}(f,2)) \},
\end{multline*}
where $\mathbf{axx}(f)$ denotes the exponentiation of the given
classical function while $\fun(f,m,n)$ denotes the $n$-th power of the
function $\fun(f,m)$.

\subsection{Ordering}

The three different superfunctions $\fer,\bos,\axp$ are ordered among
themselves as
\begin{gather*}
  \fer(f,n,m)\,\bos(h,j,k)\,\axp(g), \\
  \fer(f,n,m)\,\bos(h,j,k,l)\,\axp(g),
\end{gather*}
independently of the arguments.  The superfunctions $\bos$ and $\axp$
commute with themselves, while the superfunctions $\fer$ anticommute
with themselves.  For these superfunctions we introduce the following
ordering.

\begin{itemize}
\item The $\bos$ objects with three and four arguments are ordered as
  follows: the first argument anti-lexicographically, the second and
  third by decreasing order of natural numbers; the last (fourth) is
  not ordered because
  \begin{equation*}
    \bos(f,n,m,k)*\bos(f,n,m,l) \Rightarrow \bos(f,n,m,k+l)
  \end{equation*}

\item The anticommuting $\fer$ objects are ordered as follows: the
  first argument anti-lexicographically, the second and third by
  decreasing order of natural numbers.

  Example:
  \begin{equation*}
    \fer(f,n,m)*\fer(g,k,l) \Rightarrow -\fer(g,k,l)\,\fer(f,n,m)
  \end{equation*}
  for arbitrary $n,m,k,l$.
  \begin{equation*}
    \fer(f,n,m)*\fer(f,n,m) \Rightarrow 0
  \end{equation*}
  for arbitrary $f,n,m$.
  \begin{eqnarray*}
    \bos(f,2,3,7)*\bos(a,0,3)*\bos(f,2,3,-7)   & \Rightarrow & \bos(a,0,3), \\
    \bos(f,2,3,2)*\bos(z,0,3,2)*\bos(f,2,3,-2) & \Rightarrow & \bos(z,0,3,2).
  \end{eqnarray*}

\item For all exponential functions we have
  \begin{equation*}
    \axp(f)*\axp(g) \Rightarrow \axp(f+g).
  \end{equation*}
\end{itemize}

\subsection{(Super)Differential Operators}

We have implemented three different realizations of the supersymmetric
derivatives.  In order to select the traditional realization declare
\texttt{let trad}.  In order to select the chiral or chiral1 algebra
declare \texttt{let chiral} or \texttt{let chiral1}.  By default we
have the traditional algebra.

We have introduced three different types of SuSy operators which act
on the superfunctions and are considered as noncommuting operators in
REDUCE.

For the usual differentiation we introduced two types of operators:
\begin{itemize}
\item right differentations
  \begin{equation*}
    \d(1)*\bos(f,0,0) \Rightarrow  \bos(f,0,1) + \bos(f,0,0)\,\d(1);
  \end{equation*}
\item left differentations
  \begin{equation*}
    \fer(f,0,0)*\d(2) \Rightarrow -\fer(f,0,1) + \d(2)\,\fer(f,0,0).
  \end{equation*}
\end{itemize}
This example illustrates that the third argument in the $\bos$ and
$\fer$ objects can take an arbitrary integer value.

We denote SuSy derivatives as $\der$ and $\del$, which represent the
right and left operations respectively, and are one-argument
operators.  The action of these objects on the superfunctions depends
on the choice of the supersymmetric algebra.

Explicitly, we have for the traditional algebra:
\begin{description}
\item{right SuSy derivative}
  \begin{eqnarray*}
    \der(1)*\bos(f,0,0) &\Rightarrow&  \fer(f,1,0)+\bos(f,0,0)\,\der(1), \\
    \der(2)*\fer(g,0,0) &\Rightarrow&  \bos(g,2,0)-\fer(g,0,0)\,\der(2), \\
    \der(1)*\fer(f,2,0) &\Rightarrow&  \bos(f,3,0)-\fer(f,2,0)\,\der(1), \\
    \der(2)*\bos(f,3,0) &\Rightarrow& -\fer(f,1,1)+\bos(f,3,0)\,\der(2), \\
    \der(1)*\bos(f,0,0,-1) &\Rightarrow& -\fer(f,1,0)\,\bos(f,0,0,-2) + \mbox{}\\
    && \bos(f,0,0,-1)\,\der(1), \\
    \der(2)*\axp(\bos(f,0,0)) &\Rightarrow&
    \fer(f,2,0)\,\axp(\bos(f,0,0)) + \mbox{}\\
    && \axp(\bos(f,0,0))\,\der(2).
  \end{eqnarray*}

\item{left SuSy derivative}
  \begin{eqnarray*}
    \bos(f,0,0)*\mathbf{\del(1)} &\Rightarrow& -\fer(f,1,0)+\del(1)\,\bos(f,0,0), \\
    \fer(g,0,0)*\mathbf{\del(2)} &\Rightarrow&  \bos(g,2,0)-\del(2)\,\fer(g,0,0), \\
    \fer(f,2,0)*\del(2) &\Rightarrow& \bos(f,3,0)-\del(1)\,\fer(f,2,0), \\
    \bos(f,3,0)*\del(2) &\Rightarrow& \fer(f,1,1)+\del(2)\,\bos(f,3,0), \\
    \bos(f,0,0,-1)*\del(1) &\Rightarrow& \fer(f,1,0)\,\bos(f,0,0,-2) + \mbox{}\\
    && \del(1)\,\bos(f,0,0,-1), \\
    \axp(\bos(f,0,0))*\del(2) &\Rightarrow&
    -\fer(f,2,0)\,\axp(\bos(f,0,0)) + \mbox{}\\
    && \del(2)\,\axp(\bos(f,0,0)).
  \end{eqnarray*}
\end{description}

These examples illustrate that the second argument in the $\fer$ and
$\bos$ objects can take values 0, 1, 2, 3 only with the following
meaning: 0 -- no SuSy derivatives, 1 -- first SuSy derivative, 2 --
second SuSy derivative, 3 -- first and second SuSy derivative.

Using the results above we obtain
\begin{multline*}
  \der(1)*\der(2)*\bos(f,0,0) \Rightarrow \\
  \bos(f,3,0) + \bos(f,0,0)\,\der(1)\,\der(2) + \mbox{} \\
  \fer(f,1,0)\,\der(2) - \fer(f,2,0)\,\der(1).
\end{multline*}

For the chiral representation, the meaning of the second argument in
the $\bos$ or $\fer$ object is the same as for the traditional
representation while the actions of SuSy operators on the
superfunctions are different.  For example, we have
\begin{eqnarray*}
  \der(1)*\fer(f,1,0) &\Rightarrow& -\fer(f,1,0)\,\der(1), \\
  \der(1)*\fer(f,2,0) &\Rightarrow& \bos(g,3,0) - \fer(f,2,0)\,\der(1), \\
  \der(2)*\bos(g,3,0) &\Rightarrow& -\fer(g,2,1) + \bos(g,3,0)\,\der(2), \\
  \bos(g,2,0)*\del(2) &\Rightarrow& \del(2)\,\bos(g,2,).
\end{eqnarray*}
For the chiral1 representation we have a different meaning of the
second argument in the $\bos$ and $\fer$ objects: the values $0,1,2$
for this second argument denote the values of the SuSy derivatives
while 3 denotes the value of the commutator.  Explicitly, we have
\begin{eqnarray*}
  \der(3)*\bos(f,0,0) &\Rightarrow& \bos(f,3,0) + 2\,\fer(f,1,0,0)\,\der(2) \\
  && \mbox{} - 2\,\fer(f,2,0)\,\der(1) + \bos(f,0,0)\,\der(3), \\
  \der(1)*\fer(f,2,0) &\Rightarrow& (\bos(f,3,0)-\bos(f,0,1))/2 - \fer(f,2,0)\,\der(1).
\end{eqnarray*}

The supersymmetric operators are always ordered in the case of
traditional algebra as
\begin{eqnarray*}
  \der(2)*\der(1) &\Rightarrow& -\der(1)\,\der(2),\\
  \del(2)*\del(1) &\Rightarrow& -\del(1)\,\del(2), \\
  \der(1)*\del(1) &\Rightarrow& \d(1), \\
  \der(1)*\del(2) &\Rightarrow& -\del(2)\,\der(1);
\end{eqnarray*}
for the chiral algebra we have
\begin{eqnarray*}
  \der(2)*\der(1) &\Rightarrow& -\d(1) - \der(1)\,\der(2),\\
  \del(2)*\del(1) &\Rightarrow& -\d(1) - \del(1)\,\del(2), \\
  \der(1)*\del(1) &\Rightarrow& 0, \\
  \der(1)*\del(2) &\Rightarrow& -\d(1) - \del(2)\,\der(1);
\end{eqnarray*}
while for chiral1 additionally we have
\begin{eqnarray*}
  \der(3)*\der(1) &\Rightarrow& -\der(1)\,\d(1), \\
  \der(1)*\der(3) &\Rightarrow& \der(1)\,\d(1), \\
  \der(3)*\der(2) &\Rightarrow& \der(2)\,\d(1), \\
  \der(2)*\der(3) &\Rightarrow& -\der(2)\,\d(1).
\end{eqnarray*}

Please notice that if we would like to have the components of some
$\bos(f,3,0,-1)$ superfunction in the chiral representation then new
objects appear:
\begin{equation*}
  \mathbf{b\_part}(\bos(f,3,0,-1), 1)  \Rightarrow \fun(f1,0,f0,1,-1),
\end{equation*}
We should consider the five-argument object $\fun$ as
\begin{equation*}
  \fun(f,n,g,m,-k) \Rightarrow (\fun(f,n)-\fun(g,m)/2)^{-k}.
\end{equation*}
\begin{sloppypar}
  Similar interpretation is valid for other commands containing
  objects like $\bos(f,3,n,-k)$.
\end{sloppypar}

\subsection{Action of the Operators}

In order to have the value of the action of the given operator on some
superfunction we introduce two operators $\pr$ and $\pg$.  The
operator
\begin{equation*}
  \pr(n,\mathit{expression})
\end{equation*}
where $n=0,1,2,3$ denotes the value itself of the action of the SuSy
derivatives on the given expression.  For $n=0$ there is no SuSy
derivative, $n=1$ corresponds to $\der(1)$, $n=2$ to $\der(2)$, and
$n=3$ to $\der(1)*\der(2)$.

Example:
\begin{gather*}
  \pr(1,\bos(f,0,0)) \Rightarrow \fer(f,1,0), \\
  \pr(3,\fer(g,0,0)) \Rightarrow \fer(f,3,0).
\end{gather*}

For the usual derivative we reserve the command
\begin{equation*}
  \pg(n,\mathit{expression})
\end{equation*}
where $n=0,1,2,\ldots$ denotes the value of the usual derivative on
the expression.

Example:
\begin{equation*}
  \pg(2,\bos(f,0,0)) \Rightarrow \bos(f,0,2)
\end{equation*}

\subsection{Supersymmetric Integration}

There is one command
$\mathbf{s\_int}(\mathit{number},\mathit{expression},\mathit{list})$
only.  This allows us to compute the supersymmetric integral of
arbitrary polynomial expressions constructed from $\fer$ and $\bos$
objects.  It is valid in the traditional representation of
supersymmetry.  The argument $number$ takes the following values: $0$
corresponds to the usual $x$ integration, $1$ or $2$ to integration
over the first or second supersymmetric argument, while $3$
corresponds to integration over both the first and second arguments.
The argument $list$ is a list of the names of the superfunctions over
which we would like to integrate.  The output of this command is in
the form of the integrated part and non-integrated part.  The
non-integrated part is denoted by $\del(-\mathit{number})$ for
$\mathit{number} = 1,2,3$ and by $\d(-3)$ for $\mathit{number} = 0$.

Example:
\begin{equation*}
  \mathbf{s\_int}(0,2*\bos(f,0,1)*\bos(f,0,1),\{f\}) \Rightarrow \bos(f,0,0)^{2},
\end{equation*}
\begin{equation*}
  \mathbf{s\_int}(1,2*\fer(f,1,0)*\bos(f,0,0),\{f\}) \Rightarrow \bos(f,0,0)^{2},
\end{equation*}
\begin{multline*}
  \mathbf{s\_int}(3,\bos(f,3,0)*\bos(g,0,0)+\bos(f,0,0)*\bos(g,3,0),\{f,g\}) \Rightarrow \\
  \bos(f,0,0)\,\bos(g,0,0) - \mbox{} \\
  \del(-3)\,\big( \fer(f,1,0)\,\fer(g,2,0) - \fer(f,2,0)\,\bos(g,1,0) \big).
\end{multline*}

\subsection{Integration Operators}

We introduced four different types of integration operators: right and
left denoted by $\d(-1)$ and $\d(-2)$ respectively and moreover two
different types of neutral integration operators $\d(-3)$ and
$\d(-4)$.  In the first two cases they act according to the formula
\begin{equation*}
  \d(-1)\,\bos(f,0,0) =
  \sum_{i=1}^{\infty} (-1)^{i}\,\bos(f,0,i-1)\,\d(-1)^{i}
\end{equation*}
for the right integration and
\begin{equation*}
  \bos(f,0,0)\,\d(-2) =
  \sum_{i=1}^{\infty} \d(-2)^{i}\,\bos(f,0,i-1)
\end{equation*}
for the left integration.

Before using these operators the precision of the integration must be
specified by an assignment of the form \texttt{ww := number}, which
sets the actual upper limit to be used on the sums above instead of
infinity.  If required this precision can be changed by reassignment.
Both operators are defined by their action and by the properties
\begin{gather*}
  \d(1)\,\d(-1)=\d(-1)\,\d(1)=\d(2)\,\d(-1)=\d(2)\,\d(-1)=1, \\
  \der(1)\,\d(-1)=\d(-1)\,\der(1), \\
  \d(-1)\,\del(1)=\del(1)\,\d(-1),
\end{gather*}
and analogously for $\d(-2)$ and $\der(2), \del(2)$.

The neutral operator does not show any action on an expression but has
several properties.  More precisely
\begin{gather*}
  \d(1)\,\d(-3)=\d(-3)\,\d(1)=\d(2)\,\d(-3)=\d(-3)\,\d(2)=1, \\
  \der(k)\,\d(-3)=\d(-3)\,\der(k), \\
  \d(-3)\,\del(k)=\del(k)\,\d(-3),
\end{gather*}
while for $\d(-4)$
\begin{gather*}
  \d(1)\,\d(-4)=\d(-4)\,\d(1)=\d(2)\,\d(-4)=\d(-4)\,\d(2)=1, \\
  \der(k)\,\d(-4)=\d(-4)\,\der(k),
\end{gather*}
where $k=1,2$.

From the last two formulas we see that the $\d(-3)$ operator is
transparent under $\del$ operators while the $\d(-4)$ operator stops
the $\del$ action.

Similarly to $\d(-3)$ or $\d(-4)$ it is also possible to use the
neutral differentiation operator denoted by $\d(3)$.  It has the
properties
\begin{gather*}
  \d(3)\,\d(-4)=\d(-4)\,\d(3)=\d(3)\,\d(-3)=\d(-3)\,\d(3)=1, \\
  \der(k)\,\d(3)=\d(3)\,\der(k), \\
  \d(3)\,\del(k)=\del(k)\,\d(3),
\end{gather*}
where $k=1,2$.

We can have also ``accelerated'' integration operators denoted by
$\mathbf{dr}(-n)$ where $n$ is a natural number.  The action of these
operators is exactly the same as $\d(-1)^n$ but instead of using the
integration formulas $n$ times in the case of $\d(-1)^n$,
$\mathbf{dr}(-n)$ uses the following formula only once:
\begin{equation*}
  \mathbf{dr}(-n)\,\bos(f,0,0) = \sum^\mathit{ww}_{s=0}(-1)^{s}
  \begin{pmatrix} n+s-1 \\ n-1 \end{pmatrix}
  \bos(f,0,s)\,\mathbf{dr}(-n-s).
\end{equation*}
Similarly to the $\d(-1)$ case, we have to declare also the
``precision'' of integration if we would like to use the accelerated
integration operators.  The command \texttt{let cutoff} and assignment
of the form \texttt{cut := number} allow us to annihilate the
higher-order terms in the $\mathbf{dr}$ integration procedure.
Moreover, the command \texttt{let drr} automatically changes the usual
integration $\d(-1)$ into accelerated integration $\mathbf{dr}$.  The
command \texttt{let nodrr} changes $\mathbf{dr}$ integration into
$\d(-1)$.

\subsection{Useful Commands}

\subsubsection*{Combinations}

We encounter, in many practical applications, the problem of
constructing different possible combinations of superfunction and
super-pseudo-differential elements with given conformal dimensions.
We provide three different procedures in order to realize this
requirement:
\begin{gather*}
  \mathbf{w\_comb}(\mathit{list},n,m,x), \\
  \mathbf{fcomb}(\mathit{list},n,m,x), \\
  \mathbf{pse\_ele}(n,\mathit{list},m).
\end{gather*}
All these commands are based on the gradations trick, to associate
with superfunctions and superderivatives the scaling parameter
conformal dimension.  We consider here $k/2$ and $k$ ($k$ a positive
integer) gradation only.

The command $\mathbf{w\_comb}$ gives the most general form of
superfunction combinations of given gradation.  It is a four-argument
procedure in which:
\begin{enumerate}
\item the first argument is a list in which each element is a
  three-element list in which the first element is the name of the
  superfunction from which we would like to construct our combinations,
  the second denotes its gradation, and the last can take two values:
  f or b to indicate that the superfunction is respectively
  superfermionic or superbosonic;
\item the second argument is a number, the desired gradation;
\item the third argument is an arbitrary non-numerical value which
  enumerates the free parameters in our combinations;
\item the fourth argument takes one of two values: f or b to indicate
  that whole combinations should be respectively fermionic or bosonic.
\end{enumerate}

Examples:
\begin{eqnarray*}
  \mathbf{w\_comb}(\{\{ f,1,b \},\{g,1,b\}\},2,z,b) &\Rightarrow&
  z1\,\bos(f,3,0) + z2\,\bos(f,0,1) + \mbox{}\\
  && z3\,\bos(f,0,0)^2 \\
  \mathbf{w\_comb}(\{\{ f,1,b\}\},3/2,g,f) &\Rightarrow&
  g1\,\fer(f,1,0) + g2\,\fer(f,2,0)
\end{eqnarray*}

The command $\mathbf{fcomb}$, similarly to $\mathbf{w\_comb}$, gives
the general form of an arbitrary combination of superfunctions modulo
divergence terms.  It is a four-argument command with the same meaning
of arguments as for $\mathbf{w\_comb}$.  This command first calls
$\mathbf{w\_comb}$, then eliminates in the canonical way SuSy
derivatives by integration by parts of $\mathbf{w\_comb}$.  By
canonical we understand that (SuSy) derivatives are removed first from
the superfunction which is first in the list of superfunctions in the
$\mathbf{fcomb}$ command, next from the second, etc.

In order to illustrate the canonical manner of elimination of (SuSy)
derivatives let us consider some expression which is constructed from
$f, g$ and $h$ superfunctions and their (SuSy) derivatives.  This
expression is first split into three sub-expressions called the
\textit{f-expression}, \textit{g-expression} and
\textit{h-expression}.  The \textit{f-expression} contains only
combinations of $f$ with $f$ or $g$ or (and) $h$, while the
\textit{g-expression} contains only combinations of $g$ with $g$ or
$h$ and the \textit{h-expression} contains only combinations of $h$
with $h$.  The command $\mathbf{fcomb}$ removes first (SuSy)
derivatives from $f$ in \textit{f-expression}, then from $g$ in
\textit{g-expression}, and finally from $h$ in \textit{h-expression}.
Consider this example:
\begin{equation*}
  \fer(f,1,0)\,\fer(g,2,0) + \bos(g,0,0)\,\bos(g,3,0).
\end{equation*}
\begin{sloppypar}
  Let us now assume that we have $f,g$ order; then the
  \textit{f-expression} is $\fer(f,1,0)\,\fer(g,2,0)$, while the
  \textit{g-expression} is $\bos(g,0,1)\,\bos(g,3,0)$.  Now canonical
  elimination gives
\end{sloppypar}
\begin{equation*}
  - \bos(f,0,0)\,\bos(g,3,0) + 2\,\bos(g,0,0)\,\bos(g,3,1),
\end{equation*}
while assuming $g,f$ order gives
\begin{equation*}
  - \bos(f,3,0)\,\bos(g,0,0) + 2\,\bos(g,0,0)\,\bos(g,3,1).
\end{equation*}
Example:
\begin{multline*}
  \mathbf{fcomb}(\{\{u,1\}\},4,h) \Rightarrow \\
  h(1)\,\fer(u,2,0)\,\fer(u,1,0)\,\bos(u,0,0) + h(2)\,\bos(u,3,0)\,\bos(u,0,0)^2 + \mbox{}\\
  h(3)\,\bos(u,0,2)\,\bos(u,0,0) + h(4)\,\bos(u,0,0)^4
\end{multline*}

Finally, the command $\mathbf{pse\_ele}$ gives the general form of an
element of the pseudo-SuSy derivative
algebra~\hyperlink{susy2-bib}{[3]}.  Such an element can be written
down symbolically as
\begin{equation*}
  (\bos + \fer\,\der(1)+\fer\,\der(2)+\bos\,\der(1)\,\der(2))\,\d(1)^n
\end{equation*}
for the traditional and chiral representations, or
\begin{equation*}
  (\bos + \fer\,\der(1)+\fer\,\der(2)+\bos\,\der(3))\,\d(1)^n
\end{equation*}
for the chiral1 representation, where $\bos$ and $\fer$ denote
arbitrary superfunctions.  The mentioned command allows us to obtain
such an element of the given gradation which is constructed from some set
of superfunctions of given gradation.  This command takes three
arguments:
\begin{equation*}
  \mathbf{pse\_ele}(\mathit{wx},\mathit{wy},\mathit{wz}).
\end{equation*}
The first argument denotes the gradation of the SuSy-pseudo-element,
and the second denotes the names and gradations of the superfunctions
from which we would like to construct our element.  This second
argument $wy$ is in the form of a list exactly the same as in the
$\mathbf{w\_comb}$ command.  The last argument denotes the names which
enumerate the free parameters in our combination.

\subsubsection*{Parts of the pseudo-SuSy-differential elements}

In order to obtain the components of the (pseudo)-SuSy element we have
three different commands:
\begin{gather*}
  \mathbf{s\_part}(\mathit{expression},m), \\
  \mathbf{d\_part}(\mathit{expression},n), \\
  \mathbf{sd\_part}(\mathit{expression},m,n),
\end{gather*}
where $m,n=0,1,2,3,\ldots$.

The $\mathbf{s\_part}$ command gives the coefficient standing in the
$m$-th SuSy derivative.  However, notice that for $m=3$ we should
consider the coefficients standing in the $\der(1)\,\der(2)$ operator
for the traditional or chiral representations while for the chiral1
representation the terms standing in the $\der(3)$ operator.  The
$\mathbf{d\_part}$ command give the coefficients standing in the same
power of $\d(1)$, while $\mathbf{sd\_part}$ gives the term standing in
the $m$-th SuSy derivative and $n$-th power of the usual derivative.

Example: Given the REDUCE input
\begin{verbatim}
ala := bos(g,0,0) + fer(f,3,0)*der(1) +
  (fer(h,2,0)*der(2) + bos(r,0,0)*der(1)*der(2))*d(1);
\end{verbatim}
we have
\begin{eqnarray*}
  \mathbf{s\_part}(\mathit{ala},3) &\Rightarrow& \fer(f,3,0) \\
  \mathbf{d\_part}(\mathit{ala},1) &\Rightarrow& \fer(h,2,0)\,\der(2) +
  \bos(r,0,0)\,\der(1)\,\der(2) \\
  \mathbf{sd\_part}(\mathit{ala},0,0) &\Rightarrow& \bos(g,0,0)
\end{eqnarray*}

\subsubsection*{Adjoint}

The \emph{adjoint} $\mathit{PP}^*$ of some SuSy operator $\mathit{PP}$
is defined in standard form by
\begin{equation*}
  \langle \alpha, \mathit{PP}\,\beta \rangle =
  \langle \beta, \mathit{PP}^*\,\alpha \rangle
\end{equation*}
where $\alpha$ and $\beta$ are test superboson functions and the
scalar product is defined by
\begin{equation*}
  \langle \alpha, \beta \rangle =
  \int \alpha \beta\,d\theta_{1}\,d\theta_{2},
\end{equation*}
where we use the Berezin integral definition~\hyperlink{susy2-bib}{[1]}
\begin{align*}
  \int \theta_{i}\,d\theta_{j} &= \delta_{i,j}, \\
  \int d\theta_{i} &= 0.
\end{align*}
For this operation we have the command
\begin{equation*}
  \mathbf{cp}(\mathit{expression}).
\end{equation*}
Examples:
\begin{eqnarray*}
  \mathbf{cp}(\der(1)) &\Rightarrow& -\der(1), \\
  \mathbf{cp}(\del(1)*\fer(r,1,0)*\der(1)) &\Rightarrow&
  \fer(r,1,1) + \fer(r,1,0)\,\d(1) - \mbox{}\\
  && \del(1)\,\bos(r,0,1),
\end{eqnarray*}

The last example illustrates that it is possible to define
$\mathbf{cp}(\del(1)\,\fer(r,1,0)\,\der(1))$ in the different but
equivalent manner as $\fer(r,1,0)\,\d(1) - \bos(r,0,1)\,\der(1)$.

From a practical point of view, we do not define conjugation for the
$\d(-1)$ and $\d(-2)$ operators, because then we should define the
precision of the action of the operators $\d(-1)$ and $\d(-2)$, and
even then we would obtain very complicated formulas.  However, if
somebody decides to apply this conjugation to $\d(-1)$ or $\d(-2)$, it
is recommended first to change by hand these operators into $\d(-3)$,
next to compute $\mathbf{cp}$ and change $\d(-3)$ back into $\d(-1)$
or $\d(-2)$ together with the declaration of the precision.

\subsubsection*{Projection}

In many cases, especially in the SuSy approach to soliton theory, we
have to obtain the projection onto the invariant subspace (with
respect to the commutator) of the pseudo-SuSy-differential algebra.
There are three different subspaces~\hyperlink{susy2-bib}{[4]} and
hence we have the two-argument command
\begin{equation*}
  \mathbf{rzut}(\mathit{expression},n)
\end{equation*}
in which $n=0,1,2$.

Example: Given the REDUCE input
\begin{verbatim}
ewa := bos(f,0,0) + bos(f3,0,0)*der(1)*der(2) +
   bos(g,0,0)*d(1) + bos(g3,0,0)*d(1)*der(1)*der(2) +
   fer(f1,1,0)*der(1) + fer(f2,2,0)*der(2) +
   fer(g1,1,0)*d(1)*der(1) + fer(g2,2,0)*d(1)*der(2);
\end{verbatim}
we have
\begin{equation*}
  \mathbf{rzut}(\mathit{ewa},0) \Rightarrow \mathit{ewa},
\end{equation*}
\begin{equation*}
  \mathbf{rzut}(\mathit{ewa},1) \Rightarrow \mathit{ewa}-\bos(f,0,0),
\end{equation*}
\begin{multline*}
  \mathbf{rzut}(\mathit{ewa},2) \Rightarrow
  \bos(f3,0,0)\,\der(1)\,\der(2) + \mbox{} \\
  \big( \fer(g1,1,0)\,\der(1) + \fer(g2,2,0)\,\der(2) +
  \bos(g3,0,0)\,\der(1)\,\der(2) \big)\,\d(1).
\end{multline*}

\subsubsection*{Analogue of \texttt{coeff}}

Motivated by practical applications, we constructed for our
supersymmetric functions three commands, which allow us to obtain a
list of the same combinations of some superfunctions and (SuSy)
derivatives from some given operator-valued expression.  Each command
takes one argument and returns a list.  We use the following REDUCE
input to illustrate each command:
\begin{verbatim}
magda := fer(f,1,0)*fer(f,2,0)*a1 + der(1);
\end{verbatim}

The first command is
\begin{equation*}
  \mathbf{lyst}(\mathit{expression}).
\end{equation*}
For example
\begin{equation*}
  \mathbf{lyst}(\mathit{magda}) \Rightarrow
  \{\fer(f,1,0)\,\fer(f,2,0)\,a1, \der(1)\}.
\end{equation*}

The second command is
\begin{equation*}
  \mathbf{lyst1}(\mathit{expression})
\end{equation*}
with the output in the form of a list in which each element is
constructed from the coefficients and (SuSy) operators of the
corresponding element in the \textbf{lyst} list.  For example
\begin{equation*}
  \mathbf{lyst1}(\mathit{magda}) \Rightarrow \{a1,\der(1)\}.
\end{equation*}

The third command is
\begin{equation*}
  \mathbf{lyst2}(\mathit{expression})
\end{equation*}
with the output in the form of a list in which each element is
constructed from coefficients in the given expression.  For example
\begin{equation*}
  \mathbf{lyst2}(\mathit{magda}) \Rightarrow \{a1,1\}.
\end{equation*}

\subsubsection*{Simplification}

If we encounter during the process of computation an expression such
as
\begin{equation*}
  \fer(f,1,0)\,\d(-3)\,\fer(f,2,0)\,\d(1),
\end{equation*}
it is not reduced further.  To facilitate simplification, we can
replace $\d(1)$ with $\d(2)$, or vice versa.  In order to do this
replacement we have the command
\begin{equation*}
  \mathbf{chan}(\mathit{expression})
\end{equation*}
Example:
\begin{multline*}
  \mathbf{chan}(\fer(f,1,0)*\d(-3)*\fer(f,2,0)*\d(1)) \Rightarrow \\
  -\fer(f,2,0)\,\fer(f,1,0) - \fer(f,1,0)\,\d(-3)\,\fer(f,2,1).
\end{multline*}
Notice that as a result we remove the $\d(1)$ operator.

\subsubsection*{O(2) Invariance}

In many cases in supersymmetric theories we deal with the O(2)
invariance of SuSy indices.  This invariance follows from the physical
assumption of nonprivileging the ``fermionic'' coordinates in the
superspace.  In order to check whether our formula possesses such
invariance we can use
\begin{equation*}
  \mathbf{odwa}(\mathit{expression})
\end{equation*}
This procedure replaces in the given expression $\der(1)$ with
$-\der(2)$ and $\der(2)$ with $\der(1)$.  Next, it changes, in the
same manner, the values of the action of these operators on the
superfunctions.

\subsubsection*{Macierz}

We can define the supercomponent form for the $\mathbf{pse\_ele}$
objects similarly to the representation of the superfunctions.
Usually we can consider such an object as the matrix which acts on the
components of the superfunction.  It is realized in our program using
the command
\begin{equation*}
  \mathbf{macierz}(\mathit{expression},x,y),
\end{equation*}
where $\mathit{expression}$ is the formula under consideration.  The
argument $x$ can take two values, b or f, depending on whether we
would like to consider the bosonic (b) part or fermionic (f) part of
the expression.  The last argument denotes the option in which we act
on the bosonic or fermionic superfunction.  It takes two values: f for
fermionic test superfunction or b for bosonic.  More explicitly, we
obtain
\begin{equation*}
  \mathbf{macierz}(\der(1)*\der(2),b,f) \Rightarrow
  \begin{pmatrix}
    0 & 0 & 0 & 0 \\
    0 & 0 & \d(1) & 0 \\
    0 & -\d(1) & 0 & 0 \\
    -\d(1)^2 & 0 & 0 & 0
  \end{pmatrix},
\end{equation*}
\begin{equation*}
  \mathbf{macierz}(\der(1)*\der(2),f,b) \Rightarrow
  \begin{pmatrix}
    0 & 0 & 0 & 0 \\
    0 & 0 & 0 & \d(1) \\
    -\d(1) & 0 & 0 & 0 \\
    0 & 0 & 0 & 0
  \end{pmatrix}.
\end{equation*}

\subsection{Functional Gradients}

In the SuSy soliton approach we very frequently encounter the problem
of computing the gradient of the given functional.  The usual
definition of the gradient~\hyperlink{susy2-bib}{[2]} is adopted in the
supersymmetry also:
\begin{equation*}
  H'[v] = \langle \operatorname{grad} H, v \rangle =
  \frac{\partial}{\partial\epsilon} H(u+\epsilon v) \mid_{\epsilon=0},
\end{equation*}
where $H$ denotes some functional which depends on $u$, $v$ denotes a
vector along which we compute the gradient, and
$\langle\cdot,\cdot\rangle$ denotes the relevant scalar product.

We implemented all that in our package for the traditional algebra
only.  In order to compute the gradient with respect to some
superfunction use
\begin{equation*}
  \mathbf{gra}(\mathit{expression},f),
\end{equation*}
where $\mathit{expression}$ is the given density of the functional and
$f$ denotes the first argument in the superfunction operator (name of
the superfunction).

Example:
\begin{equation*}
  \mathbf{gra}(\bos(f,3,0)*\fer(f,1,0),f) \Rightarrow -2\,\fer(f,2,1)
\end{equation*}
For practical use we provide two additional commands:
\begin{gather*}
  \mathbf{dyw}(\mathit{expression},f), \\
  \mathbf{war}(\mathit{expression},f).
\end{gather*}
The first computes the variation of $\mathit{expression}$ with respect
to superfunction $f$; the second removes (via integration by parts)
SuSy derivatives from various functions and finally produces a list of
factorized $\fer$ and $\bos$ superfunctions.  When the given
expression is a full (SuSy) derivative, the result of the
$\mathbf{dyw}$ command is 0 and hence this command is very useful in
verifications of (SuSy) divergences of expressions.

When the result of applications of the $\mathbf{dyw}$ command is not
zero then we would like to have the system of equations on the
coefficients standing in the same factorized $\fer$ and $\bos$
superfunction.  We can quickly obtain such a list using the command
$\mathbf{war}(\mathit{expression},f)$ with the same meaning for the
arguments as in the $\mathbf{dyw}$ command.

Examples: Given the REDUCE input
\begin{verbatim}
xxx := fer(f,1,0)*fer(f,2,0) + x*bos(f,3,0)^2;
\end{verbatim}
we obtain
\begin{gather*}
  \mathbf{dyw}(\mathit{xxx},f) \Rightarrow
  \{ -2\,\bos(f,3,0)\,\bos(f,0,0), -2x\,\bos(f,0,2)\,\bos(f,0,0) \}, \\
  \mathbf{war}(\mathit{xxx},f) \Rightarrow \{ -2, -2x \}.
\end{gather*}

\subsection{Conservation Laws}

In many cases we would like to know whether a given expression is a
conservation law for some Hamiltonian equation.  We can quickly check
it using
\begin{equation*}
  \mathbf{dot\_ham}(\mathit{equation},\mathit{expression})
\end{equation*}
where $\mathit{equation}$ is a list of two-element lists in which the
first element denotes the function while the second denotes its flow.
The second argument should be understood as the density of some
conserved current.  For example, for the SuSy version of the Nonlinear
Schr\"odinger Equation~\hyperlink{susy2-bib}{[7]} we could use the
following REDUCE input:
\begin{verbatim}
ew := {{q, -bos(q,0,2) + bos(q,0,0)^3*bos(r,0,0)^2
         2*bos(q,0,0)*pr(3,bos(q,0,0)*bos(r,0,0))},
       {r, bos(r,0,2) - bos(q,0,0)^2*bos(r,0,0)^3 +
         2*bos(r,0,0)*pr(3,bos(q,0,0)*bos(r,0,0))}};
ham := bos(q,0,1)*bos(r,0,0) +
         x*bos(q,0,0)^2*bos(r,0,0)^2;
yyy := dot_ham(ew,ham);
\end{verbatim}
The result of the previous computation is a complicated expression
that is not zero.  We would like to interpret it as a full (SuSy)
divergence and we can quickly verify it by using the command
$\mathbf{war}$.  We can solve the resulting list of equations using
known techniques.  For example, in our previous case we obtain
\begin{gather*}
  \mathbf{war}(yyy,q) \Rightarrow \{-4x,-8x,-4x\}, \\
  \mathbf{war}(yyy,r) \Rightarrow \{4x,8x,4x\},
\end{gather*}
and we conclude that $\mathit{ham}$ is a constant of motion if $x=0$.

It is also possible to apply the command $\mathbf{dot\_ham}$ to the
pseudo-SuSy-differential element.  This is very useful in the SuSy
approach to the Lax operator in which we would like to check the
validity of the formula
\begin{equation*}
  \partial_{t}L := [ L,A ],
\end{equation*}
where $A$ is some (SuSy) operator.

\subsection{Jacobi Identity}

The Jacobi identity for some SuSy Hamiltonian operators is verified
using the relation
\begin{equation*}
  \langle \alpha, P'_{P(\beta)} \gamma \rangle +
  \text{all cyclic permutations of } \alpha,\beta,\gamma = 0,
\end{equation*}
where $P'$ denotes the directional derivative along the $P(\beta)$
vector and $\langle\cdot,\cdot\rangle$ denotes the scalar product.
The directional derivative is defined in the standard manner
as~\hyperlink{susy2-bib}{[44]}:
\begin{equation*}
  F^{'}(u)[v] = \frac{\partial}{\partial\epsilon}
  F(u+\epsilon v)\mid_{\epsilon =0},
\end{equation*}
where $F$ is some functional depending on $u$, and $v$ is a
directional vector.

In this package we have several commands that allow us to verify the
Jacobi identity.  We have the possibility to compute, independently of
verifying the Jacobi identity, the directional derivative for the
given Hamiltonian operator along the given vector using
\begin{equation*}
  \mathbf{n\_gat}(\mathit{pp}, \mathit{wim}),
\end{equation*}
where $\mathit{pp}$ is a scalar or matrix Hamiltonian operator and
$\mathit{wim}$ denotes the components of a vector along which we
compute the derivative.  It has the form of a list in which each
element has the representation
\begin{equation*}
  \bos(f) \Rightarrow \mathit{expression}.
\end{equation*}
The expression $\bos(f)$ above denotes the shift of the $\bos(f,0,0)$
superfunction according to the definition of the directional
derivative.

In order to compute the Jacobi identity we use the command
\begin{equation*}
  \mathbf{fjacob}(\mathit{pp}, \mathit{wim})
\end{equation*}
with the same meaning for $\mathit{pp}$ and $\mathit{wim}$ as in the
$\mathbf{n\_gat}$ command.

Notice that the ordering of the components in the $\mathit{wim}$ list
is important and is connected with the interpretation of the
components of the Hamiltonian operator $\mathit{pp}$ as a set of
Poisson brackets constructed just from elements of the $\mathit{wim}$
list.  For example, in our scheme, the first component of
$\mathit{wim}$ is always connected with the element from which we
create the Poisson bracket and which corresponds to the first element
on the diagonal of $\mathit{pp}$, the second element of $\mathit{wim}$
with the second element on the diagonal of $\mathit{pp}$, etc.

As the result of the application of the $\mathbf{fjacob}$ command to
some Hamiltonian operator we obtain a complicated formula, not
necessarily equal to zero but which should be expressed as a (SuSy)
divergence.  However, we can quickly verify it using the same method
as for the $\mathbf{dot\_ham}$ command, which was described in the
previous subsection.

Usually, after the application of the $\mathbf{fjacob}$ command to
some matrix Hamiltonian operator we obtain a huge expression, which is
too complicated to analyze even when we would like to check its (SuSy)
divergence.  In this case we could extract from the $\mathbf{fjacob}$
expression terms containing given components of vector test functions
fixed by us.  We can use the command
\begin{equation*}
  \mathbf{jacob}(\mathit{pp}, \mathit{wim}, \mathit{mm})
\end{equation*}
where $\mathit{pp}$ and $\mathit{wim}$ have the same meaning as for
the $\mathbf{fjacob}$ command while $\mathit{mm}$ is a three-element
list denoting the components: $\{\alpha,\beta,\gamma\}$.

This command is not prepared to compute in full the Jacobi identity,
which contains the integration operators.  We do not implement here
the symbolic integration of superfunctions in order to simplify the
final results.

\subsection{Objects, Commands and Let Rules}

\subsubsection*{Objects}

\begin{equation*}
  \begin{array}{lllll}
    \bos(f,n,m) & \bos(f,n,m,k) & \fer(f,n,m) & \axp(f) & \fun(f,n) \\
    \fun(f,n,m) & \gras(f,n) & \mathbf{axx}(f) & \d(1) & \d(2) \\
    \d(3) & \d(-1) & \d(-2) & \d(-3) & \d(-4) \\
    \mathbf{dr}(-n) & \der(1) & \der(2) & \del(1) & \del(2)
  \end{array}
\end{equation*}

\subsubsection*{Commands}

\begin{center}
  \begin{tabular}{ll}
    \textbf{fpart}(\textit{expression}) &
    \textbf{bpart}(\textit{expression}) \\
    \textbf{bf\_part}(\textit{expression}, $n$) &
    \textbf{b\_part}(\textit{expression}, $n$) \\
    \textbf{pr}($n$, \textit{expression}) &
    \textbf{pg}($n$, \textit{expression}) \\
    \textbf{w\_comb}($\{\{f,n,x\},\ldots\},m,z,y$) &
    \textbf{fcomb}($\{\{f,n,x\},\ldots\},m,z,y$) \\
    \textbf{pse\_ele}($n,\{\{f,n\},\ldots\},z$) &
    \textbf{s\_part}(\textit{expression}, $n$) \\
    \textbf{d\_part}(\textit{expression}, $n$) &
    \textbf{sd\_}(\textit{expression}, $n,m$) \\
    \textbf{cp}(\textit{expression}) &
    \textbf{rzut}(\textit{expression}, $n$) \\
    \textbf{lyst}(\textit{expression}) &
    \textbf{lyst1}(\textit{expression}) \\
    \textbf{lyst2}(\textit{expression}) &
    \textbf{chan}(\textit{expression}) \\
    \textbf{odwa}(\textit{expression}) &
    \textbf{gra}(\textit{expression}, $f$) \\
    \textbf{dyw}(\textit{expression}, $f$) &
    \textbf{war}(\textit{expression}, $f$) \\
    \textbf{dot\_ham}(\textit{equations}, \textit{expression})&
    \textbf{n\_gat}(\textit{operator}, \textit{list}) \\
    \textbf{fjacob}(\textit{operator}, \textit{list}) &
    \textbf{jacob}(\textit{operator}, \textit{list}, \{$\alpha,\beta,\gamma$\}) \\
    \textbf{macierz}(\textit{expression}, $x,y$) &
    \textbf{s\_int}(\textit{numbers}, \textit{expression}, \textit{list})
  \end{tabular}
\end{center}

\subsubsection*{Let Rules}

\begin{center}\ttfamily
  trad\qquad chiral\qquad chiral1\qquad inverse\qquad drr\qquad nodrr
\end{center}

} % end scope of local definitions

\subsection*{Acknowledgement}

The author would like to thank to Dr~W.~Neun for valuable remarks.

\hypertarget{susy2-bib}{\subsection*{Bibliography}}

Please see the original version of this document, which is available
formatted as
\url{https://reduce-algebra.sourceforge.io/extra-docs/susy2.pdf} and
as the \LaTeX{} source file \texttt{susy2.tex} (in the REDUCE
\texttt{packages/susy2} directory).
