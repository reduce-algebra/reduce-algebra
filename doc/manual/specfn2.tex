
\index{Adamchik, Viktor}\index{People!Adamchik, Viktor}
\index{Neun, Winfried}\index{People!Neun, Winfried}

\hypertarget{SPECFNS2}{}

The package SPECFN2 is designed to provide algebraic and numeric manipulations
for some less commonly used special functions:
\index{Hypergeometric functions}
\begin{itemize}
\item \hyperlink{operator:HYPERGEOMETRIC}{Hypergeometric function};
\item \hyperlink{operator:MEIJERG}{Meijer's G function}.
\end{itemize}

These functions are from the non-core package SPECFN2, which
needs to be loaded before use with the command:
\begin{verbatim}
        load_package specfn2;
\end{verbatim}

More information on the functions provided may be found on the website
\href{https://dlmf.nist.gov/}{DLMF:NIST} although currently not all functions
may conform to these standards.

\index{Pochhammer symbol}
\subsection{Hypergeometric Functions: Introduction}
The (generalised) hypergeometric functions
\[
_pF_q \left( \genfrac{}{}{0pt}{}{a_1, \ldots , a_p}{b_1, \ldots ,b_q} \Bigg\vert z \right)
\]
are defined in textbooks on special functions as
\[
_pF_q \left( \genfrac{}{}{0pt}{}{a_1, \ldots , a_p}{b_1, \ldots ,b_q} \Bigg\vert z \right)
  = \sum_{n=0}^{\infty}\frac{(a_{1})_{n}\dots(a_{p})_{n}}{(b_{1})_{n}\dots(b_{q})_{n}}\frac{z^{n}}{n!}
\]
where $(a)_{n}$ is the Pochhammer symbol
\[
 (a)_{n} = \prod_{k=0}^{n-1} (a+k).
\]

The function 
\[
G_{p q}^{m n} \left( z \  \Bigg\vert \  \genfrac{}{}{0pt}{}{(a_p)}{(b_q)} \right)
\]
has been studied by C.~S.~Meijer beginning in 1936 and has been
called Meijer's G function later on. The complete definition of Meijer's
G function can be found in \cite{Prudnikov:90c}.
Many well-known functions can be written as G functions,
e.g.\ exponentials, logarithms, trigonometric functions, Bessel functions
and hypergeometric functions.

Several hundreds of particular values can be found in \cite{Prudnikov:90c}.

\hypertarget{operator:HYPERGEOMETRIC}{}
\ttindex{hypergeometric}
\subsection{The Hypergeometric Operator}

The operator \f{hypergeometric} expects 3 arguments, namely the 
list of upper parameters (which may be empty), the list of lower
parameters (which may be empty too), and the argument, e.g.\ the input:
\begin{verbatim}
hypergeometric ({},{},z);
\end{verbatim}
yields the output
\begin{verbatim}
 z
e
\end{verbatim}
and the input
\begin{verbatim}
hypergeometric ({1/2,1},{3/2},-x^2);
\end{verbatim}
gives
\begin{verbatim}
 atan(abs(x))
--------------
    abs(x)
\end{verbatim}


Since hundreds of particular cases for the generalised hypergeometric
functions can be found in the literature, one cannot expect that all
cases are known to the \f{hypergeometric} operator.
Nevertheless the set of special cases can be augmented by adding
rules to the \REDUCE{} system, e.g.
\begin{verbatim}
let {hypergeometric({1/2,1/2},{3/2},-(~x)^2) => asinh(x)/x};
\end{verbatim}

\ttindex{MeijerG}
\subsection{Meijer's G Function}
\hypertarget{operator:MEIJERG}{}

The operator \f{MeijerG} expects 3 arguments, namely the 
list of upper parameters (which may be empty), the list of lower
parameters (which may be empty too), and the argument.

The first element of the lists has to be the list of the
first $n$ or $m$ respective parameters, e.g.\ to describe 
\[
G_{1 1}^{1 0} \left( x \  \Bigg\vert \  \genfrac{}{}{0pt}{}{1}{0} \right)
\]
one has to write 
\begin{verbatim}
MeijerG({{},1},{{0}},x); % and the result is:

 sign( - x + 1) + sign(x + 1)
------------------------------
              2
\end{verbatim}
and for
\[
G_{0 2}^{1 0} \left( \frac{x^2}{4} \  \Bigg\vert \ \genfrac{}{}{0pt}{}{}{{1+ \frac{1}{4} },
{1-\frac{1}{4}}} \right)
\]
\begin{verbatim}
MeijerG({{}},{{1+1/4},1-1/4},(x^2)/4) * sqrt pi;


                    2                    2
 sqrt(pi)*sqrt(-----------)*sin(abs(x))*x
                abs(x)*pi
-------------------------------------------
                     4
\end{verbatim}

