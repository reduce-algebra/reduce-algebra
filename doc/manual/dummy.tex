
\index{Dresse, Alain}\index{People!Dresse, Alain}

\subsection{Introduction}
The possibility to handle dummy variables and to manipulate  
dummy summations are important features in many applications. In particular, 
in theoretical physics, the possibility to represent complicated expressions
concisely and to realize simplifications efficiently depend 
on both capabilities.
However, when dummy variables are used, there are many more  ways 
to express a given mathematical objects since 
the names of dummy variables may be chosen  almost arbitrarily.
Therefore, from the point of view of computer algebra
the simplification problem is much more difficult. 
Given a definite ordering, one is, at least, to find a representation which 
is independent of the names chosen for the dummy variables otherwise, 
simplifications are impossible.  
The package does handle any number of dummy variables and summations 
present in expressions which are arbitrary multivariate polynomials  
and which  have operator objects eventually dependent on one (or several) 
dummy variable(s) as some of their indeterminates.
These operators have the same generality as the one existing in {\REDUCE}.  
They can be noncommutative, anticommutative or commutative. They can have 
any kind of symmetry property.
Such polynomials will be called in the following \emph{dummy} polynomials.
Any monomial of this kind will be called  \emph{dummy} monomial.
For any such object, the package allows to find a well defined 
\emph{normal form} in one-to-one correspondance with it.

In section 2, the convention for writing dummy summations is explained 
and the available declarations to introduce or suppress dummy variables
are given.

In section 3, the commands allowing to give 
various algebraic properties to the operators are described.

In section 4, the use of the function \f{canonical} is 
explained and illustrated.

For references, see \cite{BUTLER1985363,ButlerCannon1982a,Butler1982a,Leon1980a,Leon1984a,Leon1991PermutationGA,Linton1991415,McKay1977a,RodinonovElAl1989a,Sims1971a,Sims1971b,Burnel1994a,Caprasse1997a}.

The use of \package{DUMMY} requires that the package \package{ASSIST} 
 version 2.2 be available. It is loaded automatically when \package{DUMMY} is loaded.

 \subsection{Dummy variables and dummy summations}
A dummy variable (let us name it \var{dv}) is an identifier which runs from 
the integer $i_1$ to another integer $i_2$. To the extent that no 
definite space is defined, $i_1$ and $i_2$ are assumed to be some 
integers which are the \emph{same} for all dummy variables.

If \f{f} is any {\REDUCE} operator, then the simplest dummy summation 
associated to \var{dv} is the sum
\[
\sum_{dv=i_1}^{i_2} f(dv)
\]
and is simply written as
\[
f(dv). 
\]
No other rules govern the implicit summations. \var{dv} can appear as many times
we want since the operator \f{f} may depend on an arbitrary number of 
variables. So, the package is potentially applicable to many contexts.
For instance, it is possible to add rules of the kind one encounters in 
tensor calculus.      

Obviously, there are as many ways we want to express the \emph{same} quantity.
If the name of another dummy variable is \var{dum} then 
the previous expression is written as
\[
\sum_{dum=i_1}^{i_2} f(dum)
\]
and the computer algebra system should be able to find that the expression
\[
f(dv)-f(dum);
\]
is equal to $0$.
A very special case which is \emph{allowed} is when \f{f} is the identity 
operator.
So, a generic dummy polynomial will be a sum 
of dummy monomials of the  kind
\[
\prod_i c_i*f_i(dv_1,\ldots ,dv_{k_i},fr_1,\ldots , fr_{l_i})
\]
where $dv_1,\ldots,$ are dummy variables while $fr_1, \ldots, $ 
are ordinary or free variables. 

\ttindextype[DUMMY]{dummy\_base}{declaration}
\ttindextype[DUMMY]{dummy\_name}{declaration}
\hypertarget{command:DUMMY_BASE}{}
\hypertarget{command:DUMMY_NAME}{}
To declare dummy variables, two commands are available:
\begin{itemize}
\item{i.}
\begin{syntax}
     \f{dummy\_base} \meta{idp};
\end{syntax}
where \meta{idp} is the name of any unassigned identifier.
\item{ii.}
\begin{syntax}
     \f{dummy\_names} \meta{d},\meta{dp},\meta{dpp}, \ldots ;
\end{syntax}
\end{itemize}
The first one  declares \var{idp\_1},\ldots,\var{idp\_n} as dummy variables i.e.
all variables of the form \var{idp\_xxx} where $xxx$ is a number
will be dummy variables, such as \var{idp\_1}, \var{idp\_2},\ldots, \var{idp\_23}.
The second one gives special names for dummy variables.
All other identifiers which may appear are assumed to be \emph{free}.
However, there is a restriction: named and base dummy variables 
cannot be declared \emph{simultaneously}. The above declarations are 
mutually \emph{exclusive}. 
Here is an example  showing that:
\begin{verbatim}

     dummy_base dv; ==> dv

               % dummy indices are dv1, dv2, dv3, ...

     dummy_names i,j,k; ==> 
                 
               ***** The created dummy base dv must be cleared
\end{verbatim}

When this is done, an expression like 
\begin{verbatim}
     op(dv1)*sin(dv2)*abs(i)*op(dv2)$
\end{verbatim}
means a sum over $dv_1,dv_2$.
To clear the dummy base, and to create the dummy names $i,j,k$ one is 
to do
\ttindextype[DUMMY]{clear\_dummy\_base}{command}
\hypertarget{command:CLEAR_DUMMY_BASE}{}
\begin{verbatim}

      clear_dummy_base; ==> t

      dummy_names i,j,k; ==> t
                
                 % dummy indices are i,j,k.
\end{verbatim}
When this is done, an expression like
\begin{verbatim}
     op(dv1)*sin(dv2)*abs(x)*op(i)^3*op(dv2)$
\end{verbatim}
means a sum over $i$.
Similarly, the command \f{clear\_dummy\_names}
\ttindextype[DUMMY]{clear\_dummy\_names}{command}
\hypertarget{command:CLEAR_DUMMY_NAMES}{}
clears earlier defined named dummy variables.

One should keep in mind that every application of the above commands 
erases the previous ones. 
It is also possible to display the declared dummy names using 
\f{show\_dummy\_names}:\ttindextype[DUMMY]{show\_dummy\_names}{command}
\hypertarget{command:SHOW_DUMMY_NAMES}{}
\begin{verbatim}
     show_dummy_names(); ==> {i,j,k}
\end{verbatim}
To suppress \emph{all} dummy variables one can enter
\begin{verbatim}
     clear_dummy_names; clear_dummy_base;
\end{verbatim}

\subsection{The Operators and their Properties}
All dummy variables \emph{should appear at first level}
as  arguments of operators. For instance, if $i$ and $j$ are 
dummy variables, the expression 
\begin{verbatim}
     rr:=  op(i,j)-op(j,j)
\end{verbatim}
is allowed but the expression 
\begin{verbatim}
     op(i,op(j)) - op(j,op(j))
\end{verbatim}
is \emph{not} allowed. This is because dummy variables are not detected 
if they appear at a level larger than 1. 
Apart from that there is no restrictions. Operators  may be 
commutative, noncommutative or even anticommutative. Therefore 
they may be elements of an algebra, they may be tensors,
spinors, grassman variables, etc. $\ldots$
By default they are assumed to be \emph{commutative} and without symmetry
properties. The {\REDUCE} command \f{noncom}\ttindex{noncom}{declaration} is taken into account 
and, in addition, the command
\ttindextype[DUMMY]{anticom}{command}
\hypertarget{command:ANTICOM}{}
\begin{verbatim}
     anticom at1, at2;
\end{verbatim}
makes the operators $at_1$ and $at_2$ anticommutative.

One can also give symmetry properties to them. The usual declarations
\f{symmetric} and \f{an\-ti\-sym\-me\-tric} are taken into account. Moreover
and most important they can be endowed with a \emph{partial} symmetry 
through the command \f{symtree}.\ttindextype[DUMMY]{symtree}{command}
\hypertarget{command:SYMTREE}{}
Here are three illustrative examples for the $r$ operator:
\begin{verbatim}
     symtree (r,{!+, 1, 2, 3, 4});
     symtree (r,{!*, 1, {!-, 2, 3, 4}});
     symtree (r, {!+, {!-, 1, 2}, {!-, 3, 4}});
\end{verbatim}
The first one makes the operator (fully) symmetric.
The second one declares it antisymmetric with respect 
to the three last indices. 
The symbols !*,\  !+\  and !-\  at the beginning of each list mean that
the operator has no symmetry, is symmetric or is antisymmetric with respect
to the indices inside the list. Notice that the indices are not denoted
by their names but merely by their natural order of appearance. 1 means the
first written argument of $r$, 2 its second argument etc.
The first command is equivalent to the declaration \texttt{symmetric}
except that the number of indices of $r$ is \emph{restricted} to 
4 i.e. to the number declared in \texttt{symtree}.
In the second example $r$  is stated to have no symmetry with respect 
to the  first index and is declared  to be antisymmetric with respect 
to the three last indices. In the third example, $r$ is made symmetric 
with respect to the interchange  of the pairs of indices 1,2 
and 3,4 respectively and is made antisymmetric separately within
the pairs $(1,2)$ and $(3,4)$. It is the symmetry of the Riemann tensor.
The anticommutation property and the various 
symmetry properties may be suppressed by the commands \texttt{remanticom} and 
\texttt{remsym}. To eliminate partial symmetry properties one can also use 
\texttt{symtree}  itself. For example, assuming that $r$ has the Riemann symmetry,
to eliminate it do\ttindextype[DUMMY]{remanticom}{command}\ttindextype[DUMMY]{remsym}{command}
\hypertarget{command:REMANTICOM}{}
\hypertarget{command:REMSYM}{}
\begin{verbatim}
     symtree (r,{!*, 1, 2, 3, 4});  
\end{verbatim}
However, notice that the number of indices remains fixed and equal to 4
while with \texttt{remsym} it becomes again arbitrary.
\subsection{The \texttt{canonical} Operator}
\ttindextype[DUMMY]{canonical}{operator}
\hypertarget{operator:CANONICAL}{}
\f{canonical} is the most important functionality of the package.
It can be applied on any polynomial whether it is a dummy polynommial 
or not. It returns a normal form uniquely determined 
from the current ordering of the system.  
If the polynomial does not contain any dummy index, it is rewriten
taking into account the various operator properties or symmetries described 
above.
For instance, 
\begin{verbatim}
     symtree (r, {!+, {!-, 1, 2}, {!-, 3, 4}});

     aa:=r(x3,x4,x2,x1)$

     canonical aa; ==>  - r(x1,x2,x3,x4).
\end{verbatim}
If it contains dummy indices, \f{canonical} takes also into account the 
various dummy summations, makes the relevant simplifications, 
eventually rename the dummy indices  and returns the resulting normal form.
Here is a simple example:
\begin{verbatim}
    operator at1,at2;
    anticom at1,at2;

    dummy_names i,j,k; ==> t

    show_dummy_names(); ==> {i,j,k}

    rr:=at1(i)*at2(k) -at2(k)*at1(i)$
                     

    canonical rr; => 2*at1(i)*at2(j)
\end{verbatim}
It is important to notice, in the above example, that in addition to 
the summations over indices $i$ and $k$, and of the anticommutativity 
property of the operators, \f{canonical} has replaced the index $k$
by the index $j$. This substitution is essential to get full simplification.
Several other examples are given in the test file and, there, the output
of \f{canonical} is explained.   

As stated in the previous section, the dependence of operators on dummy 
indices is limited to \emph{first} level. An erroneous result
will be generated if it is not the case as the subsequent example 
illustrates:
\begin{verbatim}
    operator op;

    dummy_names i,j;

    rr:=op(i,op(j))-op(j,op(j))$

    canonical rr; ==> 0
\end{verbatim}
Zero is obtained because, in the second term, \f{canonical} has replaced 
$j$ by $i$ but has left $op(j)$ unchanged because it \emph{does not recognize}
the index $j$ which is inside. This fact has also the consequence that 
it is unable to simplify correctly (or at all) expressions which 
contain some derivatives.
For instance ($i$ and $j$ are dummy indices):
\begin{verbatim}
    aa:=df(op(x,i),x) -df(op(x,j),x)$

    canonical aa; ==> df(op(x,i),x) - df(op(x,j),x)
\end{verbatim} 
instead of zero.
A second limitation is that \f{canonical} does not add anything to the
 problem of simplifications when side relations (like Bianchi identities) 
 are present.
