\chapter{Assigning and Testing Algebraic Properties}

Sometimes algebraic expressions can be further simplified if
there is additional information about the value ranges
of its components. The following section describes 
how to inform {\REDUCE} of such assumptions.


\section{REALVALUED Declaration and Check}
\hypertarget{command:REALVALUED}{}
\hypertarget{command:NOTREALVALUED}{}

The declaration \texttt{realvalued}\ttindextype{realvalued}{declaration} may be used 
to restrict variables to the real numbers. The syntax is:
\begin{verbatim}
        realvalued v1,...vn;
\end{verbatim}
For such variables the operator \texttt{impart}\ttindextype{impart}{operator} gives 
the result zero. Thus, with
\begin{verbatim}
        realvalued x,y;
\end{verbatim}
the expression \texttt{impart(x+sin(y))} is evaluated as zero.
You may also declare an operator as real valued
with the meaning, that this operator maps real arguments always to
real values. Example:
\begin{verbatim}
        operator h; realvalued h,x;
        impart h(x);
   
           0
  
        impart h(w);

           impart(h(w))
\end{verbatim}
Such declarations are not needed for the standard elementary functions.
        
To remove the propery from a variable or an operator use the declaration
\texttt{notrealvalued}\ttindextype{notrealvalued}{declaration} with the syntax:
\begin{verbatim}
        notrealvalued v1,...vn;
\end{verbatim}

\hypertarget{operator:REALVALUEDP}{}
The boolean operator \texttt{realvaluedp}\ttindextype{realvaluedp}{operator}
allows you to check if a variable, an operator, or
an operator expression is known as real valued.
Thus, 
\begin{verbatim}
        realvalued x;
        write if realvaluedp(sin x) then "yes" else "no";
        write if realvaluedp(sin z) then "yes" else "no";
\end{verbatim}
would print first \texttt{yes} and then \texttt{no}. For general
expressions test the impart for checking the value range:
\begin{verbatim}
        realvalued x,y; w:=(x+i*y); w1:=conj w;
        impart(w*w1);

           0

        impart(w*w);

           2*x*y
\end{verbatim}

\section{SELFCONJUGATE Declaration}
\hypertarget{command:SELFCONJUGATE}{}

The declaration \texttt{selfconjugate}\ttindextype{selfconjugate}{declaration} may be used 
to declare an operator to be self-conjuate in the sense that
\texttt{conj(f(z)) = f(conj(z))}. The syntax is:
\begin{verbatim}
        selfconjugate f1,...fn;
\end{verbatim}

Such declarations are not needed for the standard elementary functions nor
for the inverses \texttt{atan, acot, asinh, acsch}. The remaining inverse
functions \texttt{log, asin, acos, atanh, acosh} etc. and
\texttt{sqrt} fail to be self-conjugate on their branch cuts (which are all
subsets of the real axis).

\section{Declaring Complex Conjugates}
\ttindextype{complex\_conjugates}{declaration}
\hypertarget{command:COMPLEX_CONJUGATES}{}
The argument \texttt{u} of a declaration \texttt{complex\_conjugates} should
consist of one or more (comma-separated) lists of two identifiers.
This declaration associates the two identifiers as
mutual complex-conjugates. If the first is an operator, the second is
also declared as an operator, if it is not one already. A fancy print symbol 
is automatically constructed and installed for the second identifier
from that of the first by adding over-lining. For example:
\begin{verbatim} 
      operator f;
      complex_conjugates {f, fbar}, {z, zb};
      conj zb     -> z
      conj(f(z))  -> fbar(zb)  
\end{verbatim}
This will associate \f{f} \& \f{fbar} and \texttt{z} \& \texttt{zb}
as mutual complex conjugates and declare \f{fbar} as an operator.
On graphical interfaces \texttt{zb} and \f{fbar} will be rendered as
$\overline{z}$ and $\overline{f}$ respectively. If the first identifier
already has a \emph{fancy special symbol} defined, this will be over-lined
to produce the fancy print symbol for the second identifier.
Should the user not wish to have a fancy print symbol automatically generated,
they may instead use explicit \f{let} statements as described in the 
subsection on the operator \hyperlink{operator:CONJ}{\texttt{conj}}.

\section{Declaring Expressions Positive or Negative}

Detailed knowlege about the sign of expressions allows {\REDUCE}
to simplify expressions involving exponentials or \texttt{abs}\ttindextype{abs}{operator}.
You can express assumptions about the 
\emph{positivity}\index{positivity} or \emph{negativity}\index{negativity}
of expressions by rules for the operator \texttt{sign}\ttindextype{sign}{operator}.
Examples:
\begin{verbatim}
         abs(a*b*c);
      
            abs(a*b*c);

         let sign(a)=>1,sign(b)=>1; abs(a*b*c);

            abs(c)*a*b

         on precise; sqrt(x^2-2x+1);

            abs(x - 1)

         ws where sign(x-1)=>1;

            x - 1
\end{verbatim}
Here factors with known sign are factored out of an \texttt{abs} expression.
\begin{verbatim}
         on precise; on factor; 

         (q*x-2q)^w;

                      w
           ((x - 2)*q)

         ws where sign(x-2)=>1;

            w        w
           q *(x - 2)

\end{verbatim}
       
In this case the factor $(x-2)^w$ may be extracted from the base
of the exponential because it is known to be positive.

Note that {\REDUCE} knows a lot about sign propagation.
For example, with $x$ and $y$ also $x+y$, $x+y+\pi$ and $(x+e)/y^2$
are known as positive.
Nevertheless, it is often necessary to declare additionally the sign of a 
combined expression. E.g.\ at present a positivity declaration of $x-2$ does not 
automatically lead to sign evaluation for $x-1$ or for $x$.

