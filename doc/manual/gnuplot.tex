\chapter{Graphical Display}

\index{Melenk, Herbert}
\index{People!Melenk, Herbert}

\section{GNUPLOT: Display of Functions and Surfaces}
\label{package:GNUPLOT}
\indexpackage{GNUPLOT}
\index{Graphical display}
\index{Display, graphical}
\newcommand{\Gnuplot}{\texttt{gnuplot}}

Graphical display of functions and data is done via an interface, the
{\REDUCE} \package{GNUPLOT}\footnote{This interface, together with the
code for plotting data, was written by Herbert Melenk.} package, to
the popular
{\Gnuplot}\footnote{\url{https://gnuplot.sourceforge.net/}} graphing
utility.  It allows you to display functions in 2D and surfaces in 3D
on a variety of output devices including X terminals, PC monitors, and
postscript and {\LaTeX} printer files.

The binary distribution of {\REDUCE} for Microsoft Windows includes
{\Gnuplot} version 5.4, which has its own documentation.
\href{https://reduce-algebra.sourceforge.io/web-reduce/about.php?start}{Web
REDUCE} also includes {\Gnuplot} version 5.4.  On other platforms,
{\REDUCE} uses whatever version of {\Gnuplot} is available.

The \package{GNUPLOT} package provides easy to use graphics output for
curves or surfaces which are defined by formulas and/or data sets, and
supports a variety of output devices such as \texttt{VGA screen},
\texttt{postscript}, \texttt{pic}\TeX, \texttt{MS Windows}.  It lets
you generate {\Gnuplot} graphical output directly from inside
{\REDUCE}, either for the interactive display of curves/surfaces or
for the production of pictures on paper.

\subsection{Command \texttt{plot}}
\ttindextype{plot}{command}
\hypertarget{command:PLOT}{}

Under the {\REDUCE} \package{GNUPLOT} package, {\Gnuplot} is used as a
graphical output server, invoked by the command
\texttt{plot(}\ldots{)}.  This command can have a variable number of
parameters:
\begin{itemize}
\item A function to plot, which can be
  \begin{itemize}
    \item an expression with one unknown, e.g.,
      \texttt{u*sin(u)\textasciicircum2};
    \item a list of expressions with one (identical) unknown,
      e.g., \texttt{\{sin(u),\allowbreak cos(u)\}};
    \item an expression with two unknowns, e.g.,
      \texttt{u*sin(u)\textasciicircum2+sqrt(v)};
    \item a list of expressions with two (identical) unknowns,
      e.g., \linebreak
      \texttt{\{x\textasciicircum{2}+y\textasciicircum{2},\allowbreak x\textasciicircum{2}-y\textasciicircum{2}\}};
    \item a parametic expression of the form \texttt{point(<u>,<v>)} (for 2D plots) or
      \texttt{point(<u>,\allowbreak <v>,<w>)} (for 3D plots) where \texttt{<u>,<v>,<w>} are
      expressions which depend of one or two parameters; if there is one
      parameter, the object describes a curve in the plane (only \texttt{<u>} and
      \texttt{<v>}) or in 3D space; if there are two parameters, the object
      describes a surface in 3D. The parameters are treated as independent
      variables.  Example: \texttt{point(sin t,cos t,t/10)};
    \item an equation with a symbol on the left-hand side and an
      expression with one or two unknowns on the right-hand side,
      e.g.,\linebreak[3] \texttt{dome=\allowbreak
        1/(x\textasciicircum2+y\textasciicircum2)};
    \item an equation with an expression on the left-hand side and a
      zero on the right-hand side describing implicitly a
      one-dimensional variety in the plane (implicitly given curve),
      e.g., \texttt{x\textasciicircum3 + x*y\textasciicircum2 - 9x = 0},
      or a two-dimensional surface in three-dimensional Euclidean
      space;
    \item an equation with an expression in two variables on the left-hand side
      and a list of numbers on the right-hand side; the contour lines
      corresponding to the given values are drawn, e.g., \\
      \texttt{x\textasciicircum3 - y\textasciicircum2 + x*y = \{-2,-1,0,1,2\}};
    \item a list of points in 2 or 3 dimensions, e.g.,
      \texttt{\{\{0,0\},\{0,1\},\{1,1\}\}} representing a curve;
    \item a list of lists of points in 2 or 3 dimensions,
      e.g.,\\
      \texttt{\{\{\{0,0\},\{0,1\},\{1,1\}\}, \{\{0,0\},\{0,1\},\{1,1\}\}\}}\\
      representing a family of curves.
  \end{itemize}
  \hypertarget{reserved:intervalop}
\item A range for a variable: this has the form
  \texttt{<variable>=(<lower\_bound> ..\ \allowbreak <upper\_bound>)}
  where \texttt{<lower\_bound>} and \texttt{<upper\_bound>} must be
  expressions which evaluate to numbers. If no range is specified the
  default range for independent variables is \texttt{(-10~..~10)} and
  for the dependent variable it is set to the maximum for the
  {\Gnuplot} executable (using double floats on most IEEE machines).
  However, each of the variables \texttt{plot\_xrange},
  \texttt{plot\_yrange}, \texttt{plot\_zrange} can be assigned an
  interval of the form \texttt{(<lower\_bound> ..\ \allowbreak
    <upper\_bound>)} that sets the default range for the $x$, $y$, $z$
  direction.  The $z$ direction corresponds to the function values or
  surface height and setting a $z$-range flattens the graph at the
  range limits, which is especially useful for functions with
  singularities.

  Additionally, the number of interval subdivisions can be assigned as
  a formal quotient of the form
  \begin{center}
    \texttt{<variable>=(<lower\_bound> ..\ <upper\_bound>)/<it>}
  \end{center}
  where \texttt{<it>} is a positive integer;
  e.g.\ \texttt{(1~..~5)/30} means the interval from 1 to 5 subdivided
  into 30 pieces of equal size.  A subdivision parameter overrides the
  value of the option \texttt{points} for this variable.
\item A plot option, either as a fixed keyword, e.g.,
  \texttt{hidden3d} or as an equation e.g., \texttt{term=pictex}; free
  text such as titles and labels should be enclosed in string quotes.
  Further details are \hyperlink{gnuplot:plot-options}{given below}.
\end{itemize}
Please note that a space has to be inserted between a number and a dot
when specifying an interval, otherwise the \REDUCE{} translator will
be misled.

If a function is given as an equation, the left-hand side is mainly used as a
label for the axis of the dependent variable.

In two dimensions, \texttt{plot} can be called with more than one explicit
function; all curves are drawn in one picture. However, all these must use the
same independent variable name.  One of the functions can be a point set or a
point set list.  Normally all functions and point sets are plotted by lines. A
point set is drawn by points only if functions and the point set are drawn in
one picture.

The same applies to three dimensions with explicit functions. However, an
implicitly given curve must be the sole object for one picture.

In 2D implicit and contour plots, as well as 3D surface plots, the
ordering of the two independent variables is by default the standard
{\REDUCE} ordering, e.g.\ $x$, $y$ rather than $y$, $x$.  However, if
both independent variables are specified explicitly via options of the
form \texttt{<variable> = <range>} then the option ordering is used as
the variable ordering.  In 2D, the first variable is plotted
horizontally and the second vertically, and in 3D the first and second
independent variables and the dependent variable form a right-handed
coordinate system.  (See the examples at the end of this section.)

The functional expressions are evaluated in \texttt{rounded} mode.
This is done automatically, it is not necessary to turn on
\texttt{rounded} mode explicitly.

\textbf{Examples:}
\begin{verbatim}
plot(cos x);
plot(s=sin phi, phi=(-3 .. 3));
plot(sin phi, cos phi, phi=(-3 .. 3));
plot (cos sqrt(x^2 + y^2), x=(-3 .. 3), y=(-3 .. 3),
      hidden3d);
plot {{0,0},{0,1},{1,1},{0,0},{1,0},{0,1},
      {0.5,1.5},{1,1},{1,0}};

% parametric: screw

on rounded;
w := for j := 1:200 collect
                {1/j*sin j, 1/j*cos j, j/200}$
plot w;

% parametric: globe
dd := pi/15$
w := for u := dd step dd until pi-dd collect
    for v := 0 step dd until 2pi collect
      {sin(u)*cos(v), sin(u)*sin(v), cos(u)}$
plot w;

% implicit: superposition of polynomials
plot((x^2+y^2-9)*x*y = 0);
\end{verbatim}


\subsubsection{Piecewise-defined functions}
A composed graph can be defined by a rule-based operator.  In that case each
rule must contain a clause which restricts the rule application to numeric
arguments, e.g.,
\begin{verbatim}
   operator my_step1;
   let {my_step1(~x) => -1 when numberp x and x<-pi/2,
        my_step1(~x) =>  1 when numberp x and x>pi/2,
        my_step1(~x) => sin x
            when numberp x and -pi/2<=x and x<=pi/2};
   plot(my_step1(x));
\end{verbatim}
Of course, such a rule may call a procedure:
\begin{verbatim}
   procedure my_step3(x);
      if x<-1 then -1 else if x>1 then 1 else x;
   operator my_step2;
   let my_step2(~x) => my_step3(x) when numberp x;
   plot(my_step2(x));
\end{verbatim}
The direct use of a procedure with a numeric \texttt{if} clause is impossible.

\hypertarget{gnuplot:plot-options}{\subsubsection{Plot options}}

The following options are specific to the {\REDUCE} \texttt{plot} command:
\begin{itemize}
   \item \texttt{points=<integer>}: the number of unconditionally
     computed data points; for a grid,
     \texttt{points\textasciicircum2} grid points are used.  The
     default value is 20.  The value of \texttt{points} is used only
     for variables for which no individual interval subdivision has
     been included in the range specification.
   \item \texttt{refine=<integer>}: the maximum depth of adaptive
     interval intersections for 2D plots (only).  The default is 8.  A
     value of 0 switches any refinement off.  Note that a high value
     may increase the computing time significantly.
\end{itemize}

\subsubsection{Additional options}

Additional {\Gnuplot} options are supported via the following
syntax in the \texttt{plot} command.

The following options are implemented using the {\Gnuplot}
\texttt{set} (or \texttt{unset} if the option begins with \texttt{no})
command, which offers far more control than is currently exposed via
this interface.  Note that the {\Gnuplot} \texttt{reset} command,
which clears the effect of most \texttt{set} commands (not \texttt{set
  term} or \texttt{set output}) from previous calls of \texttt{plot},
is always issued before any \texttt{set} commands.  See \emph{Commands
/ Set-show} and \emph{Commands / Reset} in the {\Gnuplot}
documentation or the {\Gnuplot} Help for details.
\begin{itemize}
  \item \texttt{title=<string>}: the specified title is put at the top
    of the picture.
  \item \texttt{xlabel=<string>}, \texttt{ylabel=<string>}, and
    \texttt{zlabel=<string>} for surfaces: the specified axis labels
    are displayed.  If omitted the axes are labeled by the independent
    and dependent variable names from the expression. Note that
    \texttt{xlabel}, \texttt{ylabel}, and \texttt{zlabel} here are
    used in the usual sense, $x$ for the horizontal and $y$ for the
    vertical axis in 2D, and $z$ for the perpendicular axis in 3D --
    these names do not refer to the variable names used in the
    expressions.
\begin{verbatim}
plot(1,x,(4*x^2-1)/2,(x*(12*x^2-5))/3, x=(-1 .. 1),
   ylabel="L(x,n)", title="Legendre Polynomials");
\end{verbatim}
  \item \texttt{terminal=name}: prepare output for device type
    \texttt{name}.  Every installation uses a default terminal as
    output device; some installations support additional devices such
    as printers; consult the {\Gnuplot} documentation or the
    {\Gnuplot} Help for details.
  \item \texttt{output="filename"}: redirect the output to a file.
  \item \texttt{size=<string>}: set the size of the plot via the
    {\Gnuplot} command \texttt{set size <string>}, for example\ldots
    \begin{itemize}
    \item \texttt{size="s\_x,s\_y"}: rescale the graph (not the
      window) where $s_x$ and $s_y$ are scaling factors for the $x$-
      and $y$-sizes, e.g.
\begin{verbatim}
plot(1/(x^2+y^2), x=(0.1 .. 5), y=(0.1 .. 5),
   size="0.7,1");
\end{verbatim}
      Defaults are $s_x=1,s_y=1$.  Note that scaling factors greater
      than 1 will often cause the picture to be too big for the
      window.
    \item \texttt{size="ratio -1"}: set the scales so that the unit
      has the same length on both the $x$ and $y$ axes, which is
      essential for geometrical plots to look correct, e.g.
\begin{verbatim}
plot(x^2+y^2-1=0, x=(-2 .. 2), y=(-2 .. 2),
   size="ratio -1");
\end{verbatim}
    \end{itemize}
  \item \texttt{view="r\_x,r\_z"}: set the viewpoint in 3 dimensions by turning
    the object around the $x$ or $z$ axis; the values are degrees (integers).
    Defaults are $r_x=60,r_z=30$.
\begin{verbatim}
plot(1/(x^2+y^2), x=(0.1 .. 5), y=(0.1 .. 5),
   view="30,130");
\end{verbatim}
  \item \texttt{logscale}: set all axes to use logarithmic scales.
  \item \texttt{contour} resp.\ \texttt{nocontour}: in 3 dimensions an
    additional contour map is drawn (default: \texttt{nocontour}).
    Note that \texttt{contour} is an option which is executed by
    {\Gnuplot} by interpolating the precomputed function values.  If
    you want to draw contour lines of a delicate formula, you had
    better use the contour form of the REDUCE \texttt{plot} command as
    described above.
  \item \texttt{surface} resp.\ \texttt{nosurface}: in 3 dimensions the surface
    is drawn, resp.\ suppressed (default: \texttt{surface}).
  \item \texttt{hidden3d}: apply hidden line removal in 3 dimensions.
  \item \texttt{pm3d}: draw a solid coloured (palette-mapped 3d)
    surface in 3 dimensions.
\end{itemize}

The following option is implemented using the {\Gnuplot} command
\texttt{with <style>}, which offers far more control than is currently
exposed via this interface.  See \emph{Plotting styles} in the
{\Gnuplot} documentation or the {\Gnuplot} Help for details.
\begin{itemize}
\item
  \begin{sloppypar}
    \texttt{style=} one of \texttt{lines}, \texttt{points},
    \texttt{linespoints}, \texttt{impulses}, \texttt{dots},
    \texttt{errorbars}, \texttt{boxes}, \texttt{boxerrorbars},
    \texttt{boxxyerrorbars}, \texttt{candlesticks},
    \texttt{financebars}, \texttt{fsteps}, \texttt{histeps},
    \texttt{steps}, \texttt{vector}, \texttt{xerrorbars},
    \texttt{xyerrorbars}, \texttt{yerrorbars}: set the display style.
  \end{sloppypar}
\end{itemize}


\subsection{Paper output}

The following example works for a PostScript printer.  If your printer uses a
different communication, please find the correct setting for the \texttt{terminal}
variable in the {\Gnuplot} documentation.

For a PostScript output, you need to add the options \texttt{terminal=postscript} and \allowbreak
\texttt{output="filename"} to your plot command, e.g.,
\begin{verbatim}
plot(sin x, x=(0 .. 10), terminal=postscript,
     output="sin.ps");
\end{verbatim}


\subsection{Mesh generation for implicit curves}

The basic mesh for finding an implicitly-given curve, the $x,y$ plane is
subdivided into an initial set of triangles.  Those triangles which have an
explicit zero point or which have two points with different signs are refined by
subdivision.  A further refinement is performed for triangles which do not have
exactly two zero neighbours because such places may represent crossings,
bifurcations, turning points or other difficulties.  The initial subdivision and
the refinements are controlled by the option \texttt{points} which is initially
set to 20: the initial grid is refined unconditionally until
approximately \texttt{points * points} equally-distributed points in the $x,y$
plane have been generated.

The final mesh can be visualized in the picture by setting
\begin{verbatim}
    on show_grid;
\end{verbatim}


\subsection{Mesh generation for surfaces}

By default the functions are computed at predefined mesh points: the ranges are
divided by the number associated with the option \texttt{points} in both
directions.

For two dimensions the given mesh is adaptively smoothed when the curves are too
coarse, especially if singularities are present. On the other hand refinement
can be rather time-consuming if used with complicated expressions. You can
control it with the option \texttt{refine}.  At singularities the graph is
interrupted.

In three dimensions no refinement is possible as {\Gnuplot} supports surfaces only
with a fixed regular grid. In the case of a singularity the near neighborhood is
tested; if a point there allows a function evaluation, its clipped value is used
instead, otherwise a zero is inserted.

When plotting surfaces in three dimensions you have the option of hidden line
removal. Because of an error in Gnuplot 3.2 the axes cannot be labeled correctly
when hidden3d is used ; therefore they aren't labelled at all.  Hidden line
removal is not available with point lists.


\subsection{{\Gnuplot} operation}
\ttindextype{plotreset}{command}
\hypertarget{command:PLOTRESET}{}

The command \texttt{plotreset;} deletes the current {\Gnuplot} output
window. The next call to \texttt{plot} will then open a new one.

If {\Gnuplot} is invoked directly by an output pipe (UNIX and Windows), an eventual
error in the {\Gnuplot} data transmission might cause {\Gnuplot} to quit. As {\REDUCE}
is unable to detect the broken pipe, you have to reset the plot system by
calling the command \texttt{plotreset;} explicitly. Afterwards new graphics
output can be produced.

Under Windows 3.1 and Windows NT, {\Gnuplot} has a text and a graph window.  If you
don't want to see the text window, iconify it and activate the
option \texttt{update wgnuplot.ini} from the graph window system menu -- then the
present screen layout (including the graph window size) will be saved and the
text windows will come up iconified in future.  You can also select some more
features there and so tailor the graphic output.  Before you terminate {\REDUCE}
you should terminate the graphic window by calling \texttt{plotreset;}.  If you
terminate {\REDUCE} without deleting the {\Gnuplot} windows, use the command button
from the {\Gnuplot} text window -- it offers an exit function.


\subsection{Saving {\Gnuplot} command sequences}
\hypertarget{switch:TRPLOT}{}
\hypertarget{switch:PLOTKEEP}{}
\ttindexswitch[GNUPLOT]{trplot}\ttindexswitch[GNUPLOT]{plotkeep}
\index{Tracing!GNUPLOT package}
If you want to use the internal {\Gnuplot} command sequence more than once
(e.g., for producing a picture for a publication), you may set
\begin{verbatim}
on trplot, plotkeep;
\end{verbatim}
\texttt{trplot} causes all {\Gnuplot} commands to be written additionally to the
actual {\REDUCE} output.  Normally the data files are erased after calling
{\Gnuplot}, however with \sw{plotkeep} on the files are not erased.


\subsection{Direct Call of {\Gnuplot}}
\ttindextype{gnuplot}{command}
\hypertarget{command:GNUPLOT}{}

{\Gnuplot} has a lot of facilities which are not accessed by the operators and
parameters described above. Therefore genuine {\Gnuplot} commands can be sent by
{\REDUCE}.  Please consult the {\Gnuplot} manual for the available commands and
parameters. The general syntax for a {\Gnuplot} call inside {\REDUCE} is
\begin{verbatim}
    gnuplot(<cmd>,<p_1>,<p_2> ...)
\end{verbatim}
where \texttt{cmd} is a command name and $p_1,p_2, \ldots$ are the parameters,
inside {\REDUCE} separated by commas. The parameters are evaluated by {\REDUCE}
and then transmitted to {\Gnuplot} in {\Gnuplot} syntax. Usually a drawing is built by
a sequence of commands which are buffered by {\REDUCE} or the operating
system. For terminating and activating them use the {\REDUCE}
command \texttt{plotshow}.
\ttindextype{plotshow}{command}
\hypertarget{command:PLOTSHOW}{}
Example:
\begin{verbatim}
     gnuplot(set,polar);
     gnuplot(unset,parametric);
     gnuplot(set,dummy,x);
     gnuplot(plot, x*sin x);
     plotshow;
\end{verbatim}
In this example the function expression is transferred literally to {\Gnuplot},
while {\REDUCE} is responsible for computing the function values
when \texttt{plot} is called.  Note that {\Gnuplot} restrictions with respect to
variable and function names have to be taken into account when using this type
of operation. \textbf{Important}: String quotes are not transferred to the {\Gnuplot}
executable; if the {\Gnuplot} syntax needs string quotes, you must add doubled
stringquotes \emph{inside} the argument string, e.g.,
\begin{verbatim}
     gnuplot(plot, """mydata""", "using 2:1");
\end{verbatim}


\subsection{Examples}

The following are taken from a collection of sample plots
(\texttt{gnuplot.tst}) and a set of tests for plotting special
functions. The pictures are made using the \texttt{qt} {\Gnuplot}
device and using the menu of the graphics window to export to PDF or
PNG.

A simple plot for $\sin(1/x)$:

\newpage

\begin{verbatim}
plot(sin(1/x), x=(-1 .. 1), y=(-3 .. 3));
\end{verbatim}

\unitlength=1cm
\begin{picture}(12,8)(0,0)
%\put(0,0){\Includegraphics[bb=128 93 430 315]{gnuplotex1}}
\put(0,0){\Includegraphics[scale=0.5]{gnuplotex1}}
\end{picture}

Some implicitly-defined curves:
\begin{verbatim}
plot(x^3 + y^3 - 3*x*y = {0,1,2,3},
     y=(-5 .. 5), x=(-2.5 .. 2));
\end{verbatim}

\unitlength=1cm
\begin{picture}(10,8)(0,0)
%\put(-1,-1){\Includegraphics[bb=0 0 360 270]{bild1}}
\put(0,0){\Includegraphics[scale=0.5]{bild1}}
\end{picture}

(Note that the $y$-axis is plotted horizontally since it is specified
first among the options.)

\newpage

A test for hidden surfaces:
\begin{verbatim}
plot(cos sqrt(x^2 + y^2), y=(-3 .. 3), x=(-3 .. 3),
     hidden3d);
\end{verbatim}

\begin{picture}(12,8)(0,0)
%\put(0,0){\Includegraphics[bb=50 0 350 220]{gnuplotex2}}
\put(0,0){\Includegraphics[scale=0.5]{gnuplotex2}}
\end{picture}

(Note the left-handed coordinate system since $y$ is specified before
$x$ among the options; for a right-handed coordinate system specify
$x$ before $y$.)

\newpage

\begin{verbatim}
plot(sinh(x*y)/sinh(2*x*y), y=(-10 .. 10), x=(-10 .. 10),
     hidden3d);
\end{verbatim}

\begin{picture}(12,8)(0,0)
%\put(0,0){\Includegraphics[bb=128 93 430 315]{gnuplotex3}}
\put(0,0){\Includegraphics[scale=0.5]{gnuplotex3}}
\end{picture}

(Note the left-handed coordinate system since $y$ is specified before
$x$ among the options; for a right-handed coordinate system specify
$x$ before $y$.)

\newpage

\begin{verbatim}
on rounded;
w:= {for j:=1 step 0.1 until 20 collect
       {1/j*sin j, 1/j*cos j, j},
     for j:=1 step 0.1 until 20 collect
       {(0.1+1/j)*sin j, (0.1+1/j)*cos j, j} }$
plot w;
\end{verbatim}
\begin{picture}(12,8)(0,0)
%\put(0,0){\Includegraphics[bb=127 93 429 314]{gnuplotex4}}
\put(0,0){\Includegraphics[scale=0.5]{gnuplotex4}}
\end{picture}

\newpage

The following example is taken from: Cox, Little, O'Shea,
\emph{Ideals, Varieties and Algorithms}:
\begin{verbatim}
plot(point(3u+3u*v^2-u^3, 3v+3u^2*v-v^3, 3u^2-3v^2),
     hidden3d, title="Enneper Surface");
\end{verbatim}

\begin{picture}(10,7)(0,1)
%\put(1,1){\Includegraphics[bb=69 69 319 246]{bild2}}
\put(0,0){\Includegraphics[scale=0.5]{bild2}}
\end{picture}

\newpage

The following examples use the \texttt{specfn} package to draw a collection of
Chebyshev T polynomials and Bessel Y functions.
The special function package has to be loaded explicitely
to make the operator ChebyshevT and BesselY available.

\begin{verbatim}
load_package specfn;
plot(chebyshevt(1,x), chebyshevt(2,x), chebyshevt(3,x),
     chebyshevt(4,x), chebyshevt(5,x),
     x=(-1 .. 1), title="Chebyshev t Polynomials");
\end{verbatim}

\begin{picture}(12,7.5)(0,0)
%\put(0,0){\Includegraphics[bb=128 93 430 315]{gnuplotex5}}
\put(0,0){\Includegraphics[scale=0.45]{gnuplotex5}}
\end{picture}
\enlargethispage{1cm}
\begin{verbatim}
plot(bessely(0,x), bessely(1,x), bessely(2,x),
     x=(0.1 .. 10), y=(-1 .. 1),
     title="Bessel functions of 2nd kind");
\end{verbatim}

\begin{picture}(12,7.5)(0,0)
%\put(0,0){\Includegraphics[bb=128 93 430 315]{gnuplotex6}}
\put(0,0){\Includegraphics[scale=0.45]{gnuplotex6}}
\end{picture}
