\chapter{Built-in Prefix Operators}
In the following subsections are descriptions of the most  useful prefix
\index{Prefix}
operators built into {\REDUCE} that are not defined in other sections (such
as substitution operators). Some are fully defined internally as
procedures; others are more nearly abstract operators, with only some of
their properties known to the system.

In many cases, an operator is described by a prototypical header line as
follows. Each formal parameter is given a name and followed by its allowed
type. The names of classes referred to in the definition are printed in
lower case, and parameter names in upper case. If a parameter type is not
commonly used, it may be a specific set enclosed in brackets
\texttt{\{} \ldots{} \texttt{\}}.
Operators that accept formal parameter lists of arbitrary length have the
parameter and type class enclosed in square brackets indicating that zero
or more occurrences of that argument are permitted. Optional parameters
and their type classes are enclosed in angle brackets.

\section{Numerical Operators}\index{Numerical operator}\index{Operator!numerical}
{\REDUCE} includes a number of functions that are analogs of those found
in most numerical systems.  With numerical arguments, such functions
return the expected result.  However, they may also be called with
non-numerical arguments.  In such cases, except where noted, the system
attempts to simplify the expression as far as it can.  In such cases, a
residual expression involving the original operator usually remains.
These operators are as follows:

\subsection{ABS}
\hypertarget{operator:ABS}{}
\texttt{ABS}\ttindex{ABS} returns the absolute value
of its single argument, if that argument has a numerical value.
A non-numerical argument is returned as an absolute value, with an overall
numerical coefficient taken outside the absolute value operator. For example:
\begin{verbatim}
        abs(-3/4)     ->  3/4
        abs(2a)       ->  2*abs(A)
        abs(i)        ->  1
        abs(-x)       ->  abs(X)
\end{verbatim}

\subsection{CEILING}\ttindex{CEILING}
\hypertarget{operator:CEILING}{}
This operator returns the ceiling (i.e., the least integer greater than
the given argument) if its single argument has a numerical value.  A
non-numerical argument is returned as an expression in the original
operator.  For example:

\begin{verbatim}
        ceiling(-5/4) ->  -1
        ceiling(-a)   ->  ceiling( - a)
\end{verbatim}

\subsection{CONJ}\ttindex{CONJ}
\hypertarget{operator:CONJ}{}
This returns the complex conjugate
of an expression, if that argument has a numerical value.  By default the
complex conjugate of a non-numerical argument is returned as an expression
in the operators
\texttt{REPART}\ttindex{REPART} and \texttt{IMPART}\ttindex{IMPART}.
For example:
\begin{verbatim}
        conj(1+i)     -> - i + 1
        conj(a+i*b)   -> - impart(a)*i - impart(b)
                         + repart(a) - repart(b)*i
\end{verbatim}

If rules have been previously defined for the complex conjugate(s)
of one or more non-numerical terms appearing in the argument, these rules are
applied and the expansion in terms of the operators
\texttt{REPART}\ttindex{REPART} and \texttt{IMPART}\ttindex{IMPART} is
suppressed.

For example:
\begin{verbatim}
        realvalued a,b;
        conj(a+i*b)   ->   a - b*i
        let conj z => z!*, conj c => c!*;
        conj(a+b*z*z!*+z*c!*)   ->   a + b*z*z* + c*z*
        conj atan z    ->   atan(z*)
\end{verbatim}
Note that in defining the rule \texttt{conj z => z!*},\ the rule
\texttt{conj z!*  => z} is (in effect) automatically defined and
\emph{should not} be entered by the user.  A more convenient method of
associating two identifiers as mutual complex-conjugates is to use the
\texttt{COMPLEX\_CONJUGATES} declaration as described in the section
\hyperlink{command:COMPLEX_CONJUGATES}{Declaring Complex Conjugates}.

The main use of rules for \texttt{CONJ} is to associate two identifiers
as complex conjugates as in the examples above. In addition rules of the form
\texttt{let conj(z)=>z, conj(w)=>-w} may be used. They imply that
\texttt{z} is real-valued and \texttt{w} is purely imaginary, although
the effect of the first rule can also be obtained by declaring \texttt{z}
to be \texttt{realvalued}.

Rules of the form \texttt{let conj z => <<some-expression>>} may be used,
but are not recommended. More useful results will usually be obtained by
defining the equivalent rule \texttt{let z => conj(<<some-expression>>)}.
Rules of the form \texttt{let conj z => <<some-expression>>} are particularly
problematic if \texttt{<<some-expression>>} involves \texttt{z} itself as
they may be inconsistent, for example \texttt{let conj z => z+1}. Even where
they are consistent, better results may usually achieved by defining alternative rules.
For example, given:
\begin{verbatim}
        realvalued a,b;
        let conj z => 2*a-z, conj w => w-2*b*i;
\end{verbatim}
so that the real part of \texttt{z} is \texttt{a} and the imaginary part
of \texttt{w} is \texttt{b}, more useful results will be obtained by
defining the mathematically equivalent rules:
\begin{verbatim}
        realvalued a,b,x,y;
        let z => a +i*y, w => x + b*i;
\end{verbatim}

Note also that the standard elementary functions and their inverses
(where appropriate) are automatically defined to be
\texttt{SELFCONJUGATE}\ttindextype{SELFCONJUGATE}{declaration} so that
\texttt{conj(f(z))} is simplified to \texttt{f(conj(z))}. User-defined
operators may be declared to be self-conjugate with the declaration
\hyperlink{command:SELFCONJUGATE}{\texttt{SELFCONJUGATE}}.

\subsection{FACTORIAL}\ttindex{FACTORIAL}
\hypertarget{operator:FACTORIAL}{}

If the single argument of \texttt{FACTORIAL} evaluates to a non-negative
integer, its factorial is returned.  Otherwise an expression involving
\texttt{FACTORIAL} is returned. For example:
\begin{verbatim}
        factorial(5)  ->  120
        factorial(a)  ->  factorial(a)
\end{verbatim}

\subsection{FIX}\ttindex{FIX}
\hypertarget{operator:FIX}{}
This operator returns the fixed value (i.e., the integer part of
the given argument) if its single argument has a numerical value.  A
non-numerical argument is returned as an expression in the original
operator.  For example:

\begin{verbatim}
        fix(-5/4)   ->  -1
        fix(a)      ->  fix(a)
\end{verbatim}

\subsection{FLOOR}\ttindex{FLOOR}
\hypertarget{operator:FLOOR}{}
This operator returns the floor (i.e., the greatest integer less than
the given argument) if its single argument has a numerical value.  A
non-numerical argument is returned as an expression in the original
operator.  For example:

\begin{verbatim}
        floor(-5/4)   ->  -2
        floor(a)      ->  floor(a)
\end{verbatim}

\subsection{IMPART}\ttindex{IMPART}
\hypertarget{operator:IMPART}{}
This operator returns the imaginary part of an expression, if that argument
has an numerical value.  A non-numerical argument is returned as an expression
in the operators \texttt{REPART}\ttindex{REPART} and \texttt{IMPART}.  For example:
\begin{verbatim}
        impart(1+i)   -> 1
        impart(sin(3+4*i)) -> cos(3)*sinh(4)
        impart(log(2+i))   -> atan(1/2)
        impart(asin(1+i))  -> acosh(sqrt(5)+2)/2
        impart(a+i*b)      -> impart(a) + repart(b)
\end{verbatim}

For the inverse trigometric and hyperbolic functions with non-numeric arguments the output is usually more compact when the \sw{FACTOR} is on.

\subsection{LEGENDRE\_SYMBOL}
\ttindex{LEGENDRE\_SYMBOL}
\hypertarget{operator:LEGENDRE_SYMBOL}{}
The operator \f{legendre\_symbol}(a,p) denotes the Legendre symbol
\[
  \left(\frac{a}{p}\right) \equiv a^{\frac{p-1}{2}} \pmod{p}
\]
which, by its very definition can only have one of the values $\{-1,0,1\}$.

\subsection{MAX/MIN}
\hypertarget{operator:MAX}{}
\hypertarget{operator:MIN}{}

\texttt{MAX} and \texttt{MIN}\ttindex{MAX}\ttindex{MIN} can take an arbitrary
number of expressions as their arguments.  If all arguments evaluate to
numerical values, the maximum or minimum of the argument list is returned.
If any argument is non-numeric, an appropriately reduced expression is
returned.  For example:
\begin{verbatim}
        max(2,-3,4,5) ->  5
        min(2,-2)     ->  -2.
        max(a,2,3)    ->  max(A,3)
        min(x)        ->  X
\end{verbatim}
\texttt{MAX} or \texttt{MIN} of an empty list returns 0.

\subsection{NEXTPRIME}\ttindex{NEXTPRIME}
\hypertarget{operator:NEXTPRIME}{}

\texttt{NEXTPRIME} returns the next prime greater than its integer argument,
using a probabilistic algorithm.  A type error occurs if the value of the
argument is not an integer.  For example:
\begin{verbatim}
        nextprime(5)      ->  7
        nextprime(-2)     ->  2
        nextprime(-7)     -> -5
        nextprime 1000000 -> 1000003
\end{verbatim}
whereas \texttt{nextprime(a)} gives a type error.

\subsection{RANDOM}\ttindex{RANDOM}
\hypertarget{operator:RANDOM}{}

\texttt{random(}{\em n\/}\texttt{)} returns a random number $r$ in the
range $0\leq r < n$.  A type error occurs if the value of the argument is not a
positive integer in algebraic mode, or positive number in symbolic mode.
For example:
\begin{verbatim}
        random(5)         ->    3
        random(1000)      ->  191
\end{verbatim}
whereas \texttt{random(a)} gives a type error.

\subsection{RANDOM\_NEW\_SEED}\ttindex{RANDOM\_NEW\_SEED}
\hypertarget{operator:RANDOM_NEW_SEED}{}

\texttt{random\_new\_seed(}\emph{n}\texttt{)} reseeds the random number
generator
to a sequence determined by the integer argument $n$.  It can be used to
ensure that a repeatable pseudo-random sequence will be delivered
regardless of any previous use of \texttt{RANDOM}, or can be called early in
a run with an argument derived from something variable (such as the time
of day) to arrange that different runs of a REDUCE program will use
different random sequences.  When a fresh copy of REDUCE is first created
it is as if \texttt{random\_new\_seed(1)} has been obeyed.

A type error occurs if the value of the argument is not a positive integer.

\subsection{REIMPART}\ttindex{REIMPART}
\hypertarget{operator:REIMPART}{}
This returns a two-element list of the real and imaginary parts of an
expression, if that argument has an
numerical value.  A non-numerical argument is returned as an expression in
the operators \texttt{REPART} and \texttt{IMPART}\ttindex{IMPART}.
This is more efficient than calling \texttt{REPART} and \texttt{IMPART}
separately particularly if its argument is complicated.
For example:
\begin{verbatim}
        reimpart(1+i)   -> {1,1}
        reimpart(sin(3+4*i)) -> {cosh(4)*sin(3),cos(3)*sinh(4)}
        reimpart(log(2+i))   -> {log(5)/2,atan(1/2)}
        reimpart(asin(1+i))  -> {acos(sqrt(5)2)/2,acosh(sqrt(5)+2)/2}
        reimpart(a+i*b)      ->
                   { - impart(b) + repart(a),impart(a) + repart(b)}
\end{verbatim}
For the inverse trigometric and hyperbolic functions with non-numeric arguments the output is usually more compact when the \sw{FACTOR} is on.

\subsection{REPART}\ttindex{REPART}
\hypertarget{operator:REPART}{}
This returns the real part of an expression, if that argument has an
numerical value.  A non-numerical argument is returned as an expression in
the operators \texttt{REPART} and \texttt{IMPART}\ttindex{IMPART}.
For example:
\begin{verbatim}
        repart(1+i)   -> 1
        repart(sin(3+4*i)) -> cosh(4)*sin(3)
        repart(log(2+i))   -> log(5)/2
        repart(asin(1+i))  -> acos(sqrt(5)-2)/2
        repart(a+i*b)      ->  - impart(b) + repart(a)
\end{verbatim}

For the inverse trigometric and hyperbolic functions with non-numeric arguments the output is usually more compact when the \sw{FACTOR} is on.

\subsection{ROUND}\ttindex{ROUND}
\hypertarget{operator:ROUND}{}
This operator returns the rounded value (i.e, the nearest integer) of its
single argument if that argument has a numerical value.  A non-numeric
argument is returned as an expression in the original operator.  For
example:
\begin{verbatim}
        round(-5/4)   ->  -1
        round(a)      ->  round(a)
\end{verbatim}

\subsection{SIGN}\ttindex{SIGN}
\hypertarget{operator:SIGN}{}
\texttt{SIGN} tries to evaluate the sign of its argument. If this
is possible \texttt{SIGN} returns one of 1, 0 or -1.  Otherwise, the result
is the original form or a simplified variant. For example:
\begin{verbatim}
        sign(-5)      ->  -1
        sign(-a^2*b)  ->  - sign(b)
\end{verbatim}
Note that even powers of formal expressions are assumed to be
positive only as long as the switch \texttt{COMPLEX} is off.

\section{Mathematical Functions}
\hypertarget{operator:ACOS}{}
\hypertarget{operator:ACOSH}{}
\hypertarget{operator:ACOT}{}
\hypertarget{operator:ACOTH}{}
\hypertarget{operator:ACSC}{}
\hypertarget{operator:ACSCH}{}
\hypertarget{operator:ASEC}{}
\hypertarget{operator:ASECH}{}
\hypertarget{operator:ASIN}{}
\hypertarget{operator:ASINH}{}
\hypertarget{operator:ATAN}{}
\hypertarget{operator:ATANH}{}
\hypertarget{operator:ATAN2}{}
\hypertarget{operator:COS}{}
\hypertarget{operator:COSH}{}
\hypertarget{operator:COT}{}
\hypertarget{operator:COTH}{}
\hypertarget{operator:CSC}{}
\hypertarget{operator:CSCH}{}
\hypertarget{operator:EXP}{}
\hypertarget{operator:HYPOT}{}
\hypertarget{operator:LOG}{}
\hypertarget{operator:LOGB}{}
\hypertarget{operator:LOG10}{}
\hypertarget{operator:SEC}{}
\hypertarget{operator:SECH}{}
\hypertarget{operator:SIN}{}
\hypertarget{operator:SINH}{}
\hypertarget{operator:SQRT}{}
\hypertarget{operator:TAN}{}
\hypertarget{operator:TANH}{}

{\REDUCE} knows that the following represent mathematical functions
\index{Mathematical functions} that can
take arbitrary scalar expressions as their argument(s).
\ttindex{ACOS}\ttindex{ACOSH}\ttindex{ACOT}
\ttindex{ACOTH}\ttindex{ACSC}\ttindex{ACSCH}
\ttindex{ASEC}\ttindex{ASECH}\ttindex{ASIN}
\ttindex{ASINH}\ttindex{ATAN}\ttindex{ATANH}
\ttindex{ATAN2}\ttindex{COS}\ttindex{COSH}
\ttindex{COT}\ttindex{COTH}\ttindex{CSC}
\ttindex{CSCH}\ttindex{EXP}\ttindex{HYPOT}
\ttindex{LOG}\ttindex{LOGB}\ttindex{LOG10}
\ttindex{SEC}\ttindex{SECH}\ttindex{SIN}
\ttindex{SINH}
\ttindex{SQRT}\ttindex{TAN}\ttindex{TANH}

\index{Elementary functions}
\subsection{Elementary Functions}
\textbf{Trigonometric, hyperbolic and exponential functions:}
\begin{verbatim}
    sin cos tan cot csc sec sinh
    cosh tanh coth csch sech exp
\end{verbatim}

\textbf{Their inverse functions:}
\begin{verbatim}
    asin acos atan acot acsc asec asinh acosh
    atanh acoth acsch asech log log10 logb
\end{verbatim}
where \f{log} is the natural logarithm, \f{log10} is the logarithm to
base 10, and \f{logb} has two arguments of which the second is the
logarithmic base.  Note on the CSL GUI and other graphical interfaces
the inverse trig and hyperbolic functions are output as $\arcsin$ etc.

\textbf{Miscellaneous functions:}
\begin{verbatim}
    sqrt  hypot  atan2
\end{verbatim}
The function \texttt{hypot} takes two arguments $x$  and $y$ and
returns the value $\sqrt{x^2+y^2}$ but, when the switch \sw{ROUNDED} is ON,
problems with rounding and possible overflow for large numerical arguments
are reduced.

The function \f{atan2} also takes two arguments $y$ and $x$
respectively and returns a value of $\arctan(y/x)$ in the range
$(-\pi, \pi]$ taking account of the signs of its two arguments and
avoiding an error if $x=0$.

{\REDUCE} knows various elementary identities and properties
of these functions. For example:
\begin{verbatim}
      cos(-x) = cos(x)        sin(-x) = -sin(x)
      cos(n*pi) = (-1)^n      sin(n*pi) = 0
      log(e) = 1              e^(i*pi/2) = i
      log(1) = 0              e^(i*pi) = -1
      log(e^x) = x            e^(3*i*pi/2) = -i
      sin(asin(x) = x         atan(0) = 0
      atan2(0, -1) = pi       atan2(1, 0) = pi/2
\end{verbatim}

The derivatives of all the elementary functions except \f{hypot} are
also known to the system.  Beside these identities, there are a lot of
simplifications for elementary functions defined in {\REDUCE} as
rulelists.  In order to view these, the SHOWRULES operator can be
used, e.g.
\begin{verbatim}
      showrules tan;

{tan(~n*arbint(~i)*pi + ~~x) => tan(x) when fixp(n),

 tan(~x) => trigquot(sin(x),cos(x)) when knowledge_about(sin,x,tan),

      ~x + ~~k*pi              x                 k
 tan(-------------) =>  - cot(--- + i*pi*impart(---))
          ~~d                  d                 d
                                                     k     1
                                    when abs(repart(---))=---,
                                                     d     2

      ~~w + ~~k*pi           w      k                k
 tan(--------------) => tan(--- + (--- - fix(repart(---)))*pi)
          ~~d                d      d                d

                                                         k
  when ((ratnump(rp) and abs(rp)>=1) where rp => repart(---)),
                                                         d

 tan(atan(~x)) => x,

                             2
 df(tan(~x),~x) => 1 + tan(x) }
\end{verbatim}

For further simplification, especially of expressions involving
trigonometric functions, see the TRIGSIMP\indexpackage{TRIGSIMP} package
(chapter~\ref{TRIGSIMP}) documentation.

Functions not listed above may be defined in the special functions
package SPECFN.

The user can add further rules for the reduction of expressions involving
these operators by using the \texttt{LET}\ttindex{LET} command.

\hypertarget{switch:COMBINELOGS}{}
\hypertarget{switch:EXPANDLOGS}{}
In many cases it is desirable to expand product arguments of logarithms,
or collect a sum of logarithms into a single logarithm.  Since these are
inverse operations, it is not possible to provide rules for doing both at
the same time and preserve the {\REDUCE} concept of idempotent evaluation.
As an alternative, REDUCE provides two switches
\sw{EXPANDLOGS}\ttindexswitch{EXPANDLOGS} and
\sw{COMBINELOGS}\ttindexswitch{COMBINELOGS}{} to carry
out these operations.  Both are off by default, and are subject
to the value of the switch \sw{PRECISE}\ttindexswitch{PRECISE}. This switch is on
by default and prevents modifications that may be false in a complex domain.
Thus to expand \texttt{log(3*y)} into a sum of logs, one can say
\begin{verbatim}
        on expandlogs; log(3*y);
\end{verbatim}
whereas to expand \texttt{log(x*y)} into a sum of logs, one needs to say
\begin{verbatim}
        off precise; on expandlogs; log(x*y);
\end{verbatim}
To combine this sum into a single log:
\begin{verbatim}
        off precise; on combinelogs; log(x) + log(y);
\end{verbatim}

These switches affect the logarithmic functions \f{LOG10} (base 10) and \f{LOGB}
(arbitrary base) as well.

At the present time, it is possible to have both switches on at once,
which could lead to infinite recursion.  However, an expression is
switched from one form to the other in this case.  Users should not rely
on this behavior, since it may change in the future.

\hypertarget{switch:COMBINEEXPT}{}
The current version of {\REDUCE} does a poor job of simplifying surds.  In
particular, expressions involving the product of variables raised to
non-integer powers do not usually have their powers combined internally,
even though they are printed as if those powers were combined.  For
example, the expression
\begin{verbatim}
        x^(1/3)*x^(1/6);
\end{verbatim}
will print as
\begin{verbatim}
        sqrt(x)
\end{verbatim}
but will have an internal form containing the two exponentiated terms.
If you now subtract \texttt{sqrt(x)} from this expression, you will
\emph{not} get zero.  Instead, the confusing form
\begin{verbatim}
        sqrt(x) - sqrt(x)
\end{verbatim}
will result.  To combine such exponentiated terms, the switch
\sw{COMBINEEXPT}\ttindexswitch{COMBINEEXPT} should be turned on.

The square root function can be input using the name \texttt{SQRT}, or the
power operation \texttt{\textasciicircum(1/2)}.  On output, unsimplified
square roots
are normally represented by the operator \texttt{SQRT} rather than a
fractional power.  With the default system switch settings, the argument
of a square root is first simplified, and any divisors of the expression
that are perfect squares taken outside the square root argument.  The
remaining expression is left under the square root.
% However, if the switch \texttt{REDUCED}\ttindex{REDUCED} is on,
% multiplicative factors in the argument of the square root are also
% separated, becoming individual square roots. Thus with \texttt{REDUCED} off,
Thus the expression
\begin{verbatim}
        sqrt(-8a^2*b)
\end{verbatim}
becomes
\begin{verbatim}
        2*a*sqrt(-2*b).
\end{verbatim}
% whereas with \texttt{REDUCED} on, it would become
% \begin{verbatim}
%         2*a*i*sqrt(2)*sqrt(b) .
% \end{verbatim}
% The switch \texttt{REDUCED}\ttindex{REDUCED} also applies to other rational
% powers in addition to square roots.

\hypertarget{switch:PRECISE}{}
Note that such simplifications can cause trouble if \var{A} is eventually
given a value that is a negative number.  If it is important that the
positive property of the square root and higher even roots always be
preserved, the switch \sw{PRECISE}\ttindexswitch{PRECISE} should be set on
(the default value).
This causes any non-numerical factors taken out of surds to be represented
by their absolute value form.
With % both \texttt{REDUCED} and
\texttt{PRECISE} on then, the above example would become
\begin{verbatim}
        2*abs(a)*sqrt(-2*b).
\end{verbatim}

\hypertarget{switch:PRECISE_COMPLEX}{}
However, this is incorrect in the complex domain, where
$\sqrt{x^{2}}$ is not identical to $|x|$. To avoid the above
simplification, the switch
\sw{PRECISE\_COMPLEX}\ttindexswitch{PRECISE\_COMPLEX} should be set on
(default is off). For example:
\begin{verbatim}
on precise_complex; sqrt(-8a^2*b);
\end{verbatim}
yields the output
\begin{verbatim}
             2
2*sqrt( - 2*a *b)
\end{verbatim}

If the switch \sw{ROUNDED}\ttindexswitch{ROUNDED} is on, any of the
elementary functions
\begin{verbatim}
    acos acosh acot acoth acsc acsch asec asech
    asin asinh atan atanh atan2 cos cosh cot coth
    csc csch exp hypot log logb log10 sec sech
    sin sinh sqrt tan tanh
\end{verbatim}
when given a numerical argument has its value calculated to the current
degree of floating point precision.  In addition, real (non-integer
valued) powers of numbers will also be evaluated.

If the \sw{COMPLEX}\ttindexswitch{COMPLEX} switch is turned on in addition to \sw{ROUNDED},
these functions will also calculate a real or complex result, again to
the current degree of floating point precision,
if given complex arguments.  For example, with \texttt{on rounded,complex;}
\begin{verbatim}
    2.3^(5.6i)  ->  - 0.0480793490914 - 0.998843519372*i
    cos(2+3i)   ->  - 4.18962569097 - 9.10922789376*i
\end{verbatim}

For \texttt{log} and the inverse trigonometric and hyperbolic functions which are
multi-valued, the principal value is returned. The branch cuts chosen
(except for \texttt{acot}) are now
those recommended by W.~Kahan (\cite{Kahan1987BranchCF})

The exception for \texttt{acot} is necessary as elsewhere in {\REDUCE}
$\mathop{\mathrm{acot}}(-z)$ is taken to be $\pi - \mathop{\mathrm{acot}}(z)$
rather than $-\mathop{\mathrm{acot}}(z)$.
\index{Branch Cuts}
The branch cuts are:\\
\begin{tabular}{ll}
  $\log$, $\mathrm{sqrt}$:     & $ \{r   \mid r \in\mathbb{R} \land r < 0\} $\\
  $\mathop{\mathrm{asin}}$, $\mathop{\mathrm{acos}}$: & $ \{r   \mid r \in\mathbb{R} \land (r > 1 \lor r < -1)\} $\\
  $\mathop{\mathrm{acsc}}$, $\mathop{\mathrm{asec}}$: & $ \{r   \mid r \in\mathbb{R} \land r \neq 0 \land r > -1 \land r < 1\} $\\
  $\mathop{\mathrm{atan}}$, $\mathop{\mathrm{acot}}$: & $ \{r*i \mid r \in\mathbb{R} \land (r > 1 \lor r < -1)\} $\\
  $\mathop{\mathrm{asinh}}$:         & $ \{r*i \mid r \in\mathbb{R} \land (r \geq  1 \lor r \leq  -1)\} $\\
  $\mathop{\mathrm{acsch}}$:         & $ \{r*i \mid r \in\mathbb{R} \land r \neq 0 \land r \geq  -1 \land r \leq  1\} $\\
  $\mathop{\mathrm{acosh}}$:         & $ \{r   \mid r \in\mathbb{R} \land r < 1\} $\\
  $\mathop{\mathrm{asech}}$:         & $ \{r   \mid r \in\mathbb{R} \land (r > 1 \lor r < 0)\} $\\
  $\mathop{\mathrm{atanh}}$:         & $ \{r   \mid r \in\mathbb{R} \land (r > 1 \lor r < -1)\} $\\
  $\mathop{\mathrm{acoth}}$:         & $ \{r   \mid r \in\mathbb{R} \land r > -1 \land r < 1\} $
\end{tabular}

\index{Special functions}
\subsection{Special Functions}
The functions in this section are either built-in or are autoloading
functions from the package \texttt{SPECFN}. On the CSL GUI and other graphical
interfaces many of the functions will be output in standard form; for example
\texttt{BesselJ(nu,x)} will be output as $J_\nu(x)$ and \texttt{Fresnel\_S(u)}
as $\mathrm{S}(u)$. For most of the non-unary special functions in this
section (\f{Lerch\_Phi} is an exception), the last parameter is the `main'
variable and the earlier parameters are the order (or orders)
usually rendered in the literature as subscipts and/or superscripts.

The information provided below is fairly rudimentary; more complete
information may be found in the \hyperlink{SPECFNS}{SPECFN package}.
\hyperlink{SPECFNTAB1}{Quick Reference Tables} are also available.

\index{Integral functions}
\ttindex{Si}\ttindex{Shi}\ttindex{Chi}
\ttindex{Ci}\ttindex{Ei}\ttindex{Li}
\ttindex{Erf} \ttindex{Fresnel\_S}\ttindex{Fresnel\_C}
\textbf{Integral Functions:}
\begin{verbatim}
    Ei Li Si Ci Shi Chi Erf Fresnel_S Fresnel_C
\end{verbatim}
All these functions are unary;
the first six are the exponential, logarithmic, sine and
cosine integrals and their hyperbolic counterparts. \f{Erf},
\f{Fresnel\_S} and \f{Fresnel\_C} are the error function and the Fresnel
sine and cosine integrals respectively.

\index{Polygamma functions}\index{Gamma function}\index{Beta function}
\ttindex{Beta}\ttindex{ibeta}\ttindex{Gamma}
\ttindex{igamma}\ttindex{Polygamma}\ttindex{psi}
\textbf{Beta, Gamma and Related Functions:}
\begin{verbatim}
    Beta ibeta Gamma igamma psi Polygamma
\end{verbatim}
The \f{Gamma} function is unary whilst \f{Beta} is binary.
The binary function \f{igamma} and ternary function \f{ibeta}
are the (normalised) incomplete Gamma and Beta functions respectively.
The unary function \f{psi} is sometimes known as the Digamma function and
the binary function \f{Polygamma} with integer first parameter $n$ is the
$n$th derivative of the function \f{psi}.

\index{Bessel functions}\index{Hankel functions}
\ttindex{BesselI}\ttindex{BesselJ}\ttindex{BesselK}
\ttindex{BesselY}\ttindex{Hankel1}\ttindex{Hankel2}
\textbf{Bessel and Related Functions:}
\begin{verbatim}
    BesselJ BesselY BesselI BesselK Hankel1 Hankel2
\end{verbatim}
All of these functions are binary, their first argument being the
order of the function.

For the special functions below, a second
\hyperlink{SPECFNTAB2}{Quick Reference Table} is available.

\index{Airy functions}
\ttindex{Airy\_Ai}\ttindex{Airy\_Aiprime}
\ttindex{Airy\_Bi}\ttindex{Airy\_Biprime}
\textbf{Airy Functions:}
\begin{verbatim}
    Airy_Ai Airy_Aiprime Airy_Bi Airy_Biprime
\end{verbatim}
These are all unary functions.

\index{Kummer functions}\index{Struve functions}\index{Lommel functions}
\index{Whittaker functions}
\ttindex{KummerM}\ttindex{KummerU}\ttindex{Lommel1}
\ttindex{Lommel2}\ttindex{StruveH}\ttindex{StruveL}
\ttindex{WhittakerM}\ttindex{WhittakerW}
\textbf{Kummer, Lommel, Struve and Whittaker Functions:}
\begin{verbatim}
    KummerM KummerU Lommel1 Lommel2
    StruveH StruveL WhittakerM WhittakerW
\end{verbatim}
The Struve functions are both binary whilst the remaining ones are all ternary.

\index{Riemann Zeta function}\index{Lambert's W function}
\ttindex{zeta} \ttindex{Lambert\_W}
\textbf{Riemann Zeta and Lambert's W Function:}
\begin{verbatim}
    zeta  Lambert_W
\end{verbatim}
These are both unary functions.

\ttindex{dilog}\ttindex{Polylog}\ttindex{Lerch\_Phi}
\index{Polylogarithm and related functions} \index{Dilog function}
\textbf{Polylogarithms and Related Functions}
\nopagebreak
\begin{verbatim}
    dilog Polylog Lerch_Phi
\end{verbatim}
These take one, two and three arguments respectively.

\index{Spherical and Solid Harmonics}
\ttindex{SphericalHarmonicY}\ttindex{SolidHarmonicY}
\textbf{Associated Legendre functions:}
\begin{verbatim}
    SphericalHarmonicY SolidHarmonicY
\end{verbatim}
These functions take four and six arguments respectively.

\subsection{Polynomial Functions}
\index{Polynomial functions}
The polynomial functions below are from the non-core package SPECFN
and for the most part are not autoloading. This package needs to be loaded
before they may be used with the command:
\begin{verbatim}
    load_package specfn;
\end{verbatim}
The names of the {\REDUCE} operators for the polynomial functions below
are mostly built by adding a P to the name of the polynomial,
e.g.\ \f{EulerP} implements the Euler polynomials.

The information in this subsection is fairly rudimentary; more complete
information may be found in the \hyperlink{POLYFN}{SPECFN package}.\\
A \hyperlink{SPECFNTAB3}{Quick Reference Table} is available for all the
polynomial functions below.

\textbf{Orthogonal Polynomials}
\index{Orthogonal polynomials}
\index{Hermite polynomials}\index{Jacobi polynomials}
\index{Legendre polynomials}\index{Laguerre polynomials}
\index{Chebyshev polynomials}\index{Gegenbauer polynomials}
\ttindex{HermiteP}\ttindex{JacobiP}
\ttindex{LegendreP}\ttindex{LaguerreP}
\ttindex{ChebyshevT} \ttindex{ChebyshevU}
\ttindex{GegenbauerP}

Some well-known orthogonal polynomials are available:
\begin{itemize}
\item Hermite polynomials: (\f{HermiteP});
\item Chebyshev polynomials: (\f{ChebyshevT}, \f{ChebyshevU});
\item Legendre polynomials: (\f{LegendreP});
\item Laguerre polynomials: (\f{LaguerreP});
\item Associated Legendre functions: (\f{LegendreP});
\item Generalised Laguerre (or Sonin) polynomials: (\f{LaguerreP});
\item Gegenbauer polynomials: (\f{GegenbauerP});
\item Jacobi polynomials: (\f{JacobiP}).
\end{itemize}
The first three of these functions are binary and the first
argument should be an integer specifying the order of the
required polynomial. The functions \f{LegendreP} and \f{LaguerreP} may be used
either as binary operators or ternary ones and represent the corresponding
`basic' and associated functions respectively.
Finally the Gegenbauer polynomials are ternary whilst the Jacobi polynomials
are quaternary.

Most definitions are equivalent to those in \cite{AbramowitzStegun:72},
except for the ternary associated Legendre functions:
\[P_n^{(m)}(x) = (-1)^m(1-x^2)^{m/2}\frac{\mathrm{d}^m P_n(x)}{\mathrm{d}x^m}\,.\]

% \begin{verbatim}
% P(n,m,x) = (-1)^m *(1-x^2)^(m/2)*df(legendreP(n,x), x, m)
% \end{verbatim}
These are sometimes mistakenly referred to as associated Legendre polynomials,
but they are only polynomial when $m$ is even.

\textbf{Other Polynomial Functions}
\index{Fibonacci polynomials}\index{Other polynomials}
\ttindex{FibonacciP}
\hypertarget{operator:FIBONACCIP}{}

\textbf{Fibonacci Polynomials} are computed by the binary operator
\f{FibonacciP}, where
\texttt{FibonacciP(n,x)} returns the $n$th Fibonacci polynomial
in the variable $x$. If $n$ is an integer, it will be

evaluated using the recursive definition:
\[F_0(x) = 0;\qquad F_1(x) = 1; \qquad F_n(x) = x F_{n-1}(x) + F_{n-2}(x)\,.\]

\index{Euler polynomials}\ttindex{EulerP}
\textbf{Euler Polynomials} are computed by the binary operator
\f{EulerP}, where\linebreak
\texttt{EulerP(n,x)} returns the $n$th Euler polynomial in the variable $x$.

\index{Bernoulli polynomials}\ttindex{BernoulliP}
\textbf{Bernoulli Polynomials} are computed by the binary operator
\f{BernoulliP}, where \texttt{BernoulliP(n,x)} returns the
$n$th Bernoulli polynomial in the variable $x$.

\subsection{Elliptic Functions and Integrals}
All the functions documented in this subsection are autoloading
functions from the package \texttt{ELLIPFN}.  On the CSL GUI and other graphical
interfaces these functions will be output in standard mathematical form;
for example \texttt{jacobisn(x,k)} will be output as $\mathrm{sn}(x,k)$ and
\texttt{weierstrass(x,omega1,omega3)} as $\wp(x,\omega_1,\omega_3)$.

\index{Jacobi Elliptic functions}\index{Elliptic functions}
\ttindex{Jacobiam}
\ttindex{Jacobisn}\ttindex{Jacobicn}
\ttindex{Jacobidn}\ttindex{Jacobins}
\ttindex{Jacobinc}\ttindex{Jacobind}
\ttindex{Jacobisc}\ttindex{Jacobisd}
\ttindex{Jacobicd}\ttindex{Jacobics}
\ttindex{Jacobids}\ttindex{Jacobidc}
\textbf{Jacobi Elliptic Functions:}
\begin{verbatim}
    Jacobisn Jacobicn Jacobidn
\end{verbatim}
and their three reciprocals
\begin{verbatim}
    Jacobins Jacobinc Jacobind
\end{verbatim}
and six quotients
\begin{verbatim}
    Jacobisc Jacobisd Jacobicd
    Jacobics Jacobids Jacobidc
\end{verbatim}
All are binary functions with the second argument being the modulus.
The binary function \f{Jacobiam} is the amplitude.

\index{Elliptic Integrals}\index{Zeta function of Jacobi}
\ttindex{JacobiZeta}\ttindex{JacobiE}
\ttindex{EllipticK}\ttindex{EllipticE}\ttindex{EllipticF}
\textbf{Complete and Incomplete Elliptic Integrals of the First \& Second Kinds:}
\begin{verbatim}
    EllipticK EllipticE EllipticF EllipticE
    JacobiE JacobiZeta
\end{verbatim}
The function \f{EllipticE} may take one or two arguments to
denote the complete and Legendre's form of the incomplete elliptic integrals of
the second kind respectively. The complete integral of the first kind
\f{EllipticK} is unary whilst \f{EllipticF}, \f{JacobiE} and
\f{JacobiZeta} are binary and represent the incomplete integral of
the first kind, Jacobi's form of the incomplete elliptic integral of
the second kind and Jacobi's Zeta function respectively.

\index{Jacobi Theta functions}
\ttindex{EllipticTheta1}\ttindex{EllipticTheta2}
\ttindex{EllipticTheta3}\ttindex{EllipticTheta4}
\textbf{Jacobi's Theta Functions:}
\begin{verbatim}
    EllipticTheta1 EllipticTheta2
    EllipticTheta3 EllipticTheta4
\end{verbatim}
are all binary functions with the second argument being the `parameter' $\tau$,
the nome $q$ being given by $q = \exp(i\pi\tau)$

\index{Weierstrass Elliptic functions}
\index{Sigma functions}
\ttindex{Weierstrass}\ttindex{WeierstrassZeta}
\ttindex{sigma}\ttindex{sigma1}
\ttindex{sigma2}\ttindex{sigma3}
\textbf{Weierstrassian Elliptic Functions:}
\begin{verbatim}
    Weierstrass    WeierstrassZeta
    sigma  sigma1  sigma2   sigma3
    Weierstrass1   WeierstrassZeta1
\end{verbatim}
are all ternary functions with the second and third arguments of the first
six functions being the the lattice period parameters $\omega_1$ and $\omega_3$.
The remaining two functions are alternative versions of the Weierstrass
functions with the second and third arguments being the lattice invariants
$g_2$ and $g_3$.

\index{Inverse Elliptic functions}
\ttindex{arcsn}\ttindex{arccn}\ttindex{arcdn}
\ttindex{arcns}\ttindex{arcnc}\ttindex{arcnd}
\ttindex{arcsc}\ttindex{arccs}\ttindex{arsd}
\ttindex{arcds}\ttindex{arccd}\ttindex{arcdc}
\textbf{Inverse Elliptic Functions:}
\begin{verbatim}
    arcsn  arccn  arcdn   arcns  arcnc  arcnd
    arcsc  arccs  arcsd   arcds  arccd  arcdc
\end{verbatim}
These are all binary functions with the second argument being the modulus $k$.
They are the inverses of the corresponding Jacobi elliptic functions
\texttt{jacobisn}, \texttt{jacobicn} etc.\/(wrt their first argument).

For the elliptic functions above
a \hyperlink{ELLIPFNTAB}{Quick Reference Table} is available.

\section{Combinatorial Numbers}
\index{Combinatorial numbers}
\index{Binomial coefficients}  \ttindex{Binomial}
\textbf{Binomial coefficients} are provided by the binary operator \f{Binomial}.
The value of \texttt{binomial(n, m)}, where $n$ and $m$ are non-negative
integers with $n \geq m$, is the number of ways of choosing $m$ items from a
set of $n$ distinct items as well, of course, as being the coefficient of $x^m$
in the expansion of $(1+x)^n$.

The function call \texttt{binomial(n,m)}, where $n$ and $m$
are non-negative integers, will return the expected integer value
(from Pascal's triangle). For other real numerical values the result will
usually involve the $\Gamma$ function, but if the switch
\sw{ROUNDED} is ON the $\Gamma$ functions will be evaluated numerically.
This will also be the case for complex numerical arguments if the switch
\sw{COMPLEX} is ON.  For non-numeric arguments the result returned will
involve the original oerator \f{binomial}, or its pretty-printed equivalent
on graphical interfaces.

\hypertarget{operator:Stirling1}{}\hypertarget{operator:Stirling2}{}
\index{Stirling numbers}\ttindex{Stirling1}\ttindex{Stirling2}
\textbf{Stirling numbers} of the first and second kind are computed
by the binary operators \f{Stirling1} and \f{Stirling2}
respectively using explicit formulae.
\texttt{stirling1(n, k)} is $(-1)^{n-k}\ \times$
\emph{(the number of permutations of the set} $\{1, 2, \ldots, n\}$
\emph{into exactly $k$ cycles)}.\\
\texttt{stirling2(n, k)} is \emph{the number of partitions of the set}
$\{1, 2, \ldots, n\}$ \emph{into exactly $k$ non-empty subsets}.\\
Here $n$ and $k$ should be non-negative integers with $n \geq k$.

For integer arguments an integer result will be
returned, otherwise a result involving the original operator will be returned.
Note on graphical user interfaces \texttt{Stirling1(n,m)} and
\texttt{Stirling2(n,m)} are rendered as
$\mathrm{s}_n^m$ and $\mathrm{S}_n^m$ respectively.

Stirling numbers are implemented in the non-core package SPECFN and are
not currently autoloading. Before use this package should be loaded with the
command:
\begin{verbatim}
    load_package specfn;
\end{verbatim}
For more information see \hyperlink{STIRL}{here}.

\hypertarget{operator:MOTZKIN}{}
\index{Motzkin numbers}\ttindex{Motzkin}
\textbf{A Motzkin number} $M_{n}$ (named after Theodore Motzkin) is the
number of different ways of drawing non-intersecting chords on a circle between
$n$ points. For a non-negative integer $n$, the operator \texttt{Motzkin(n)}
returns the $n$th Motzkin number, according to the recursion formula
\[
  M_{0} = 1; \qquad M_{1}=1; \qquad
  M_{n+1} = \frac{2n+3}{n+3}M_{n}+ \frac{3n}{n+3}M_{n-1}\,.
\]
The recursion is, of course, optimised as a simple loop to avoid repeated
computation of lower-order numbers.

For the functions in this and the section below
a \hyperlink{SPECFNTAB4}{Quick Reference Table} is available.
It also includes a list of reserved constants known to {\REDUCE}.

\section{Bernoulli, Euler and Fibonacci Numbers}
\ttindex{Bernoulli}\index{Bernoulli numbers}
\hypertarget{operator:BERNOULLI}{}
\textbf{Bernoulli numbers} are provided by the unary operator
\f{Bernoulli}. If $n$ is a non-negative integer, the call \texttt{Bernoulli(n)}
evaluates to the $n$th Bernoulli number; all of the odd Bernoulli numbers,
except \texttt{Bernoulli(1)}, are zero. Otherwise the result involves the
original operator \texttt{bernoulli}; on graphical interfaces this is rendered
as $\mathrm{B}_n$.

\ttindex{Euler}\index{Euler numbers}
\hypertarget{operator:EULER}{}
\textbf{Euler numbers} are computed by the unary operator \f{Euler}.  If $n$
is a non-negative integer, the call \texttt{Euler(n)} returns the $n$th
Euler number; all of the odd Euler numbers are zero.
Otherwise the result returned involves the original operator \texttt{euler};
on graphical interfaces this is rendered
as $\mathrm{E}_n$.

\ttindex{Fibonacci}\index{Fibonacci numbers}
\hypertarget{operator:FIBONACCI}{}
\textbf{Fibonacci numbers} are provided by the unary operator \f{Fibonacci},
where \texttt{Fibonacci(n)} evaluates to the $n$th Fibonacci
number; if $n$ is an integer, this will be evaluated following
the recursive definition:
\[F_0 = 0; \qquad F_1 = 1;\qquad F_n = F_{n-1} + F_{n-2}\,.\]
The recursion is, of course, optimised as a simple loop to avoid repeated
computation of lower-order numbers.
Otherwise the result returned involves the original operator
\texttt{fibonacci}; on graphical interfaces this is rendered
as $\mathrm{F}_n$.


\section{CHANGEVAR operator}
\hypertarget{operator:CHANGEVAR}{}
\ttindextype{CHANGEVAR}{operator}

\noindent
Author: G.~\"{U}\c{c}oluk.

The operator \texttt{CHANGEVAR} does a variable transformation in a set of
differential equations.
Syntax:
\begin{syntaxtable}
  \texttt{changevar(}\meta{depvars},\,\meta{newvars},\,\meta{eqlist},\,\meta{diffeq}\texttt{)}
\end{syntaxtable}
\meta{diffeq} is either a single differential equation or a list of
differential equations, \meta{depvars} are the dependent variables to
be substituted, \meta{newvars} are the new depend variables, and
\meta{eqlist} is a list of equations of the form \meta{depvar}
\texttt{=} \meta{expression} where \meta{expression} is some function
in the new dependent variables.

The three lists \meta{depvars}, \meta{newvars}, and \meta{eqlist} must
be of the same length. If there is only one variable to be
substituted, then it can be given instead of the list. The same applies to the list
of differential equations, i.e., the following two
commands are equivalent
\begin{verbatim}
    changevar(u,y,x=e^y,df(u(x),x) - log(x));

    changevar({u},{y},{x=e^y},{df(u(x),x) - log(x)});
\end{verbatim}
except for one difference: the first command returns the transformed differential
equation, the second one a list with a single element.

\hypertarget{switch:DISPJACOBIAN}{}
The switch \sw{DISPJACOBIAN} governs the display the entries of the inverse Jacobian,
it is \texttt{OFF} per default.\ttindexswitch{DISPJACOBIAN}

The mathematics behind the change of independent variable(s) in differential
equations is quite straightforward. It is basically the application of the
chain rule. If the dependent variable of the differential equation is $F$,
the independent variables are $x_{i}$ and the new independent variables are
$u_{i}$ (where ${\scriptstyle i=1\ldots n}$) then the first derivatives are:
\[
    \frac{\partial F}{\partial x_{i}} = \frac{\partial F}{\partial u_{j}}
                                        \frac{\partial u_{j}}{\partial x_{i}}
\]
We assumed Einstein's summation convention. Here the problem is to
calculate the $\partial u_{j}/\partial x_{i}$ terms if the change of variables
is given by
\[
    x_{i} = f_{i}(u_{1},\ldots,u_{n})
\]
The first thought might be solving the above given equations for $u_{j}$ and
then differentiating them with respect to $x_{i}$, then again making use of the
equations above, substituting new variables for the old ones in the calculated
derivatives. This is not always a  preferable way to proceed. Mainly because
the functions $f_{i}$ may not always be easily invertible. Another approach
that makes use of the Jacobian is better. Consider the above given equations
which relate the old variables to the new ones. Let us differentiate them:
\begin{align*}
  \frac{\partial x_{j}}{\partial x_{i}} & =
        \frac{\partial f_{j}}{\partial x_{i}}   \\
  \delta_{ij} & =
        \frac{\partial f_{j}}{\partial u_{k}}
        \frac{\partial u_{k}}{\partial x_{i}}
\end{align*}
The first derivative is nothing but the $(j,k)$ th entry of the Jacobian matrix.

So if we speak in matrix language
\[ {\bf 1 = J \cdot D} \]
where we defined the Jacobian
\[ {\bf J}_{ij} \stackrel{\triangle}{=}  \frac{\partial f_{i}}{\partial u_{j}} \]
and the matrix of the derivatives we wanted to obtain as
\[ {\bf D}_{ij} \stackrel{\triangle}{=}  \frac{\partial u_{i}}{\partial x_{j}}. \]
If the Jacobian has a non-vanishing determinant then it is invertible and
we are able to write from the matrix equation above:
\[ {\bf  D = J^{-1}} \]
so finally we have what we want
\[
   \frac{\partial u_{i}}{\partial x_{j}} = \left[{\bf J^{-1}}\right]_{ij}
\]
The higher derivatives are obtained by the successive application of the chain
rule and using the definitions of the old variables in terms of the new ones. It
can be easily verified that the only derivatives that are needed to be
calculated are the first order ones which are obtained above.

\subsection{\texttt{CHANGEVAR} example: The 2-dim. Laplace Equation}
The 2-dimensional Laplace equation in cartesian coordinates is:
\[
   \frac{\partial^{2} u}{\partial x^{2}} +
   \frac{\partial^{2} u}{\partial y^{2}} = 0
\]
Now assume we want to obtain the polar coordinate form of Laplace equation.
The change of variables is:
\[
   x = r \cos \theta, \qquad  y = r \sin \theta
\]
The solution using \texttt{CHANGEVAR} is as follows
\begin{verbatim}
  changevar({u},{r,theta},{x=r*cos theta,y=r*sin theta},
            {df(u(x,y),x,2)+df(u(x,y),y,2)} );
\end{verbatim}
Here we could omit the curly braces in the first and last arguments (because
those lists have only one member) and the curly braces in the third argument
(because they are optional), but you cannot leave off the curly braces in the
second argument. So one could equivalently write
\begin{verbatim}
   changevar(u,{r,theta},x=r*cos theta,y=r*sin theta,
             df(u(x,y),x,2)+df(u(x,y),y,2) );
\end{verbatim}
If you have tried out the above example, you will notice that the denominator
contains a $\cos^{2} \theta + \sin^{2} \theta$ which is actually equal to $1$.
This has of course nothing to do with \texttt{CHANGEVAR}.
One has to be overcome these pattern matching problems by the
conventional methods REDUCE provides (a rule, for example,
will fix it).

Secondly you will notice that your \texttt{u(x,y)} operator has changed to
\texttt{u(r,theta)} in the result. Nothing magical  about this. That is just what
we do with pencil and paper. \texttt{u(r,theta)} represents the  the transformed
dependent variable.

\subsection{Another \texttt{CHANGEVAR} example: An Euler Equation}
Consider a differential equation which is of Euler type, for instance:
\[
   x^{3}y''' - 3 x^{2}y'' + 6 x y' - 6 y = 0
\]
where prime denotes differentiation with respect to $x$. As is well known,
Euler type of equations are solved by a change of variable:
\[
   x = e^{u}
\]
So our call to \texttt{CHANGEVAR} reads as follows:
\begin{verbatim}
   changevar(y, u, x=e**u, x**3*df(y(x),x,3)-
             3*x**2*df(y(x),x,2)+6*x*df(y(x),x)-6*y(x));
\end{verbatim}
and returns the result
\begin{verbatim}
   df(y(u),u,3) - 6*df(y(u),u,2) + 11*df(y(u),u) - 6*y(u)
\end{verbatim}


\section{CONTINUED\_FRACTION Operator}
\hypertarget{operator:CONTINUED_FRACTION}{}
\index{approximation}\index{rational number}

The operator \texttt{CONTINUED\_FRACTION}
\ttindextype{CONTINUED\_FRACTION}{operator}
generates the continued fraction expansion of a rational number argument.
For irrational or rounded arguments, it approximates the real number
as a rational number to the current system precision and generates
the continued fraction expansion. Currently the operator \texttt{CF} is a
complete synonym for \texttt{CONTINUED\_FRACTION} although this may change
in future updates of the package \texttt{RATAPRX}.
\ttindextype{CF}{operator}

The operator \texttt{CONTINUED\_FRACTION} accepts one, two or
three arguments: the number to be expanded; an \textbf{optional} maximum size
permitted for the denominator of the convergent
and an \textbf{optional} number of continuents to be generated:
\begin{syntaxtable}
  \texttt{continued\_fraction(}\meta{num}\texttt{)}\\
  \texttt{continued\_fraction(}\meta{num}\texttt{,}\,\meta{size}\texttt{)}\\
  \texttt{continued\_fraction(}\meta{num}\texttt{,}\,%
  \meta{size}\texttt{,}\,\meta{numterms}\texttt{)}
\end{syntaxtable}
The result is the special operator \texttt{contfrac}
\ttindextype{contfrac}{operator}
with three arguments: the original number to be expanded
\meta{num}, secondly the rational number approximation (the final convergent)
and thirdly a list of continuents of the continued fraction (i.e.\ a list of
pairs of partial numerators and denominators)

\hspace{10mm}\texttt{\{t0, \{1, t1\}, \{1, t2\}, .... \}}

which represents the same value according to the definition

\hspace{10mm}\texttt{t0 +1/(t1 + 1/(t2 + ...))}.

Note that, although with the current
algorithm all the partial numerators have the value 1, they are stored
in the list of continuents. This is for compatibility with the output of
other continued fractions functions \texttt{cfrac} and \texttt{cf\_euler}.
This facilitates pretty-printing and the implementation of various equivalence
transformations all of which are documented in the continued fraction
subsection of the \texttt{RATAPRX} manual (Section 16.51).
\ttindextype{cfrac}{operator}
\ttindextype{cf\_euler}{operator}

\textbf{Precision}: the second optional parameter \meta{size} is an upper
bound for the absolute value of the  denominator of the convergent.\\
\textbf{Number of terms}: the third optional parameter \meta{numterms} is
the maximum number of terms (continuents) to be generated.\\
If both optional parameters omitted, the expansion performed is exact for
rational number arguments and for irrational or rounded arguments it is up
to the current system precision.
If both optional parameters are given the expansion is halted when the desired
precision is reached or when the specified maximum number of terms have
been generated whichever is the sooner. If the size parameter is zero,
its value is ignored.  Thus to obtain a continued fraction expansion to,
for example, 10 terms one would specify the \meta{size} parameter to be 0 and
the \meta{numterms} parameter to be 10.

Note that the operator \texttt{contfrac} is not normally seen as the output
is pretty-printed, unless the number of continuents generated is larger
than 12.

\textit{Examples:}
\begin{verbatim}
continued_fraction(6/11);

  6                1
{----,exact,---------------}
  11                 1
             1 + ---------
                       1
                  1 + ---
                       5

continued_fraction(pi,1000);

     355             1
{pi,-----,3 + ----------------}
     113               1
               7 + ----------
                          1
                    15 + ---
                          1

continued_fraction(pi,0,6);

     104348                    1
{pi,--------,3 + ------------------------------}
     33215                       1
                  7 + ------------------------
                                    1
                       15 + -----------------
                                      1
                             1 + -----------
                                         1
                                  292 + ---
                                         1

continued_fraction(pi,1000,3);

     333          1
{pi,-----,3 + ----------}
     106            1
               7 + ----
                    15

continued_fraction(pi,1000,6);

     355             1
{pi,-----,3 + ----------------}
     113               1
               7 + ----------
                          1
                    15 + ---
                          1

continued_fraction e;

{e,

  13580623
 ----------,
  4996032

 {2, {1,1}, {1,2}, {1,1}, {1,1}, {1,4}, {1,1}, {1,1}, {1,6},
  {1,1}, {1,1}, {1,8}, {1,1}, {1,1}, {1,10}, {1,1}, {1,1}, {1,12}}}
\end{verbatim}


\section{DF Operator}
\label{sec:DF-operator}
\hypertarget{operator:DF}{}
The operator \texttt{DF}\ttindextype{DF}{operator} is used to represent partial
differentiation\index{Differentiation} with respect
to one or more variables. It is used with the syntax:
\begin{syntax}
  \texttt{DF(}\meta{EXPRN:algebraic}[,\,\meta{VAR:kernel}<,\,\meta{NUM:integer}>])\,:\,\textit{algebraic}.
\end{syntax}
The first argument is the expression to be differentiated. The remaining
arguments specify the differentiation variables and the number of times
they are applied.

The number \texttt{NUM} may be omitted if it is 1.  For example,
\begin{flushleft}
  \hspace{0.5in}%
  \begin{tabular}{@{}l@{}l}
    \texttt{df(y,x)}            & ~=~$\partial y/\partial x$ \\
    \texttt{df(y,x,2)}          & ~=~$\partial^{2}y/\partial x^{2}$ \\
    \texttt{df(y,x1,2,x2,x3,2)} & ~=~$\partial^{5}y/\partial x_{1}^{2}
    \partial x_{2}\partial x_{3}^{2}$.
  \end{tabular}
\end{flushleft}

The evaluation of \texttt{df(y,x)} proceeds as follows: first, the values of
\var{Y} and \var{X} are found.  Let us assume that \var{X} has no assigned
value, so its value is \var{X}.  Each term or other part of the value of
\var{Y} that contains the variable \var{X} is differentiated by the
standard rules.  If \var{Z} is another variable, not \var{X} itself, then
its derivative with respect to \var{X} is taken to be 0, unless \var{Z}
has previously been declared to \texttt{DEPEND} on \var{X}, in which
case the derivative is reported as the symbol \texttt{df(z,x)}.


\subsection{Switches influencing differentiation}
\hypertarget{switch:COMMUTEDF}{}
\hypertarget{switch:NOCOMMUTEDF}{}
\hypertarget{switch:SIMPNONCOMDF}{}

Consider \texttt{df(u,x,y,z)}, assuming \var{u} depends on each of \texttt{x,y,z}
in some way.  If none of \texttt{x,y,z} is equal to \var{u}
then the order of differentiation is commuted into a canonical form, unless the
switch \sw{NOCOMMUTEDF}\ttindexswitch{NOCOMMUTEDF} is turned on (default is off).
If at least one of \texttt{x,y,z} is equal to \var{u} then the order of
differentiation is \emph{not} fully commuted and the derivative is \emph{not}
simplified
to zero, unless the switch \sw{COMMUTEDF}\ttindexswitch{COMMUTEDF} is turned on.
It is off by default.

If \sw{COMMUTEDF} is off and the switch \sw{SIMPNONCOMDF}\ttindexswitch{SIMPNONCOMDF}
is on then simplify as follows:
\begin{verbatim}
        df(u,x,u)        ->  df(u,x,2) / df(u,x)
        df(u,x,n,u)      ->  df(u,x,n+1) / df(u,x)
\end{verbatim}
\emph{provided} \texttt{u} depends only on the one variable \var{x}.
This simplification removes the non-commutative aspect of the derivative.

\hypertarget{switch:EXPANDDF}{}
If the switch \sw{EXPANDDF}\ttindexswitch{EXPANDDF} is turned on then REDUCE uses
the chain rule to expand symbolic derivatives of indirectly dependent variables
provided the result is unambiguous, i.e. provided there is no direct dependence.
It is off by default.  Thus, for example, given
\begin{verbatim}
        depend f,u,v; depend {u,v},x;
        on expanddf;
        df(f,x)    -> df(f,u)*df(u,x) + df(f,v)*df(v,x)
\end{verbatim}
whereas after
\begin{verbatim}
        depend f,x;
\end{verbatim}
\texttt{df(f,x)} does not expand at all (since the result would be ambiguous
and the algorithm would loop).

\hypertarget{switch:ALLOWDFINT}{}
\hypertarget{switch:DFINT}{}
Turning on the switch \sw{ALLOWDFINT}\ttindexswitch{ALLOWDFINT} allows
``differentiation under the integral sign'', i.e.
\begin{verbatim}
        df(int(y, x), v) -> int(df(y, v), x)
\end{verbatim}
if this results in a simplification.  If the switch \sw{DFINT}\ttindexswitch{DFINT}
is also turned on then this happens regardless of whether the result
simplifies. Both switches are off by default.


\subsection{Adding Differentiation Rules}
\hypertarget{command:LETdf}{}

The \texttt{LET}\ttindex{LET} statement can be used to introduce
rules for differentiation of user-defined operators.  Its general form is
\begin{syntaxtable}
  \texttt{FOR ALL }\meta{var1}\texttt{,}\,\dots\texttt{,}\,\meta{varn} \\
    \qquad \texttt{LET DF(}\meta{operator}\meta{varlist}\texttt{,}\,\meta{vari}\texttt{)
    = }\meta{expression}
\end{syntaxtable}
where
\begin{syntax}
  \meta{varlist}\ \BNFprod\ (\meta{var1},\,\dots,\,\meta{varn}),
\end{syntax}
and
\meta{var1},\,\dots,\,\meta{varn} are the dummy variable arguments of
\meta{operator}.

An analogous form applies to infix operators.

\textit{Examples:}
\begin{verbatim}
        for all x let df(tan x,x) = 1 + tan(x)^2;
\end{verbatim}
(This is how the tan differentiation rule appears in the {\REDUCE}
source.)
\begin{verbatim}
        for all x,y let df(f(x,y),x)=2*f(x,y),
                        df(f(x,y),y)=x*f(x,y);
\end{verbatim}
Notice that all dummy arguments of the relevant operator must be declared
arbitrary by the \texttt{FOR ALL} command, and that rules may be supplied for
operators with any number of arguments.  If no differentiation rule
appears for an argument in an operator, the differentiation routines will
return as result an expression in terms of \texttt{DF}\ttindextype{DF}{operator}.  For
example, if the rule for the differentiation with respect to the second
argument of \texttt{F} is not supplied, the evaluation of \texttt{df(f(x,z),z)}
would leave this expression unchanged. (No \texttt{DEPEND} declaration
is needed here, since \texttt{f(x,z)} obviously ``depends on'' \var{Z}.)

Once such a rule has been defined for a given operator, any future
differentiation\index{Differentiation} rules for that operator must be
defined with the same number of arguments for that operator, otherwise we
get the error message
\begin{verbatim}
   Incompatible DF rule argument length for <operator>
\end{verbatim}


\subsection{Options controlling display of derivatives}
\hypertarget{switch:DFPRINT}{}

If the switch \sw{DFPRINT}\ttindexswitch{DFPRINT} is turned on (it is
off by default) then derivatives are displayed using subscripts, as
illustrated below.  In graphical environments with typeset mathematics
enabled, the (shared) variable
\texttt{FANCY\_PRINT\_DF}\ttindex{FANCY\_PRINT\_DF} can be set to one
of the values \texttt{PARTIAL}\ttindex{PARTIAL},
\texttt{TOTAL}\ttindex{TOTAL} or \texttt{INDEXED}\ttindex{INDEXED} to
control the display of derivatives.  The default value is
\texttt{PARTIAL}.  However, if the switch \texttt{DFPRINT} is on then
\texttt{FANCY\_PRINT\_DF} is ignored.  For example, with the following
settings, derivatives are displayed as follows (assuming
\texttt{DEPEND F,X,Y} and \texttt{OPERATOR G}):

\begin{tabular}{ccc}
  \texttt{Setting} & \texttt{df(f,x,2,y)} & \texttt{df(g(x,y),x,2,y)} \\
  \texttt{fancy\_print\_df:=partial} &
  \(\frac{\partial^3 f}{\partial x^2 \partial y}\) &
  \(\frac{\partial^3 g(x,y)}{\partial x^2 \partial y}\) \\
  \texttt{fancy\_print\_df:=total} & \(\frac{d^3 f}{d x^2 d y}\) &
  \(\frac{d^3 g(x,y)}{d x^2 d y}\) \\
  \texttt{fancy\_print\_df:=indexed} & \(f_{x,x,y}\) & \(g(x,y)_{x,x,y}\) \\
  \texttt{on dfprint} & \(f_{x,x,y}\) & \(g_{x,x,y}\)
\end{tabular}

\section{INT Operator}
\hypertarget{operator:INT}{}
\index{Integration}
\texttt{INT}\ttindex{INT} is an operator in {\REDUCE} for indefinite or definite integration.

\subsection{Indefinite integration}
\index{Indefinite integration}\index{Integration, indefinite}
Indefinite
integration is performed using a
combination of the Risch-Norman algorithm and pattern matching \cite{Norman:77,Harrington:79,Norman:79}.
It is used with the syntax:
\begin{syntax}
  \texttt{INT(}\meta{EXPRN:algebraic}\texttt{,}\,\meta{VAR:kernel}\texttt{)}\,:\,\textit{algebraic}.
\end{syntax}
This will return correctly the indefinite integral for expressions comprising
polynomials, log functions, exponential functions and tan and atan. The
arbitrary constant is not represented. If the integral cannot be done in
closed terms, it returns a formal integral for the answer in one of two ways:
\begin{enumerate}
\item It returns the input, \texttt{INT(\ldots,\ldots)} unchanged.

\item It returns an expression involving \texttt{INT}s of some
      other functions (sometimes more complicated than
      the original one, unfortunately).
\end{enumerate}
Rational functions can be integrated when the denominator is factorizable
by the program. In addition it will attempt to integrate expressions
involving error functions, dilogarithms and other trigonometric
expressions. In these cases it might not always succeed in finding the
solution, even if one exists.

\textit{Examples:}
\begin{verbatim}
        int(log(x),x) ->  X*(LOG(X) - 1),
        int(e^x,x)    ->  E**X.
\end{verbatim}
The program checks that the second argument is a variable and gives an
error if it is not.

\subsection{Definite Integration} \index{Definite integration}
\index{Integration, definite}

If \texttt{INT} is used with the syntax

\begin{verbatim}
   INT(EXPRN:algebraic,VAR:kernel,
        LOWER:algebraic,UPPER:algebraic):algebraic.
\end{verbatim}

The definite integral of \texttt{EXPRN} with respect to \texttt{VAR} is
calculated between the limits \texttt{LOWER} and \texttt{UPPER}.
This is calculated by several methods that are tried one after the other:
pattern matching, by first finding
the indefinite integral and then substituting the limits into this, by contour integration, or
by transforming the integrand into one or two Meijer G-functions.
For details, consult
the documentation on the DEFINT\indexpackage{DEFINT} package.


\subsection{Options}
\hypertarget{switch:TRINT}{}
\hypertarget{switch:TRINTSUBST}{}
\hypertarget{switch:TRDEFINT}{}
\hypertarget{switch:FAILHARD}{}
\hypertarget{switch:NOLNR}{}
\hypertarget{switch:NOINTSUBST}{}

The switch \sw{TRINT}\ttindexswitch{TRINT} when on will trace the operation of the algorithm. It
produces a great deal of output in a somewhat illegible form, and is not
of much interest to the general user. It is normally off.

The switch \sw{TRINTSUBST}\ttindexswitch{TRINTSUBST} when on will trace the heuristic attempts to
solve the integral by substitution. It is normally off.

The switch \sw{TRDEFINT}\ttindexswitch{TRDEFINT} when on will trace the operation of the definite integration algorithm.

If the switch \sw{FAILHARD}\ttindexswitch{FAILHARD} is on the algorithm will terminate with an
error if the integral cannot be done in closed terms, rather than return a
formal integration form. \sw{FAILHARD} is normally off.

The switch \sw{NOLNR}\ttindexswitch{NOLNR} suppresses the use of the linear properties of
integration in cases when the integral cannot be found in closed terms.
It is normally off.

The switch \sw{NOINTSUBST}\ttindexswitch{NOINTSUBST} disables the heuristic attempts to solve
the integral by substitution. It is normally off.

\subsection{Advanced Use}
\hypertarget{operator:ERF}{}

If a function appears in the integrand that is not one of the functions
\texttt{EXP, Erf, TAN, ATAN, LOG, dilog}\ttindex{EXP}\ttindex{Erf}
\ttindex{TAN}\ttindex{ATAN}\ttindex{LOG}\ttindex{dilog}
then the algorithm will make an
attempt to integrate the argument if it can, differentiate it and reach a
known function.  However the answer cannot be guaranteed in this case.  If
a function is known to be algebraically independent of this set it can be
flagged transcendental by
\begin{verbatim}
        flag('(trilog),'transcendental);
\end{verbatim}
in which case this function will be added to the permitted field
descriptors for a genuine decision procedure. If this is done the user is
responsible for the mathematical correctness of his actions.


The standard version does not deal with algebraic extensions. Thus
integration of expressions involving square roots and other like things
can lead to trouble.  The extension package \texttt{ALGINT} 
will analytically integrate a wide range of
expressions involving square roots where the answer exists in that class
of functions. It is an implementation of the work described by J.H.
Davenport \cite{Davenport:81}.

\hypertarget{switch:ALGINT}{}
The extension package is loaded automatically when the switch \sw{ALGINT}
is turned on.
One enters an expression for integration, as with the regular integrator,
for example:
\begin{verbatim}
        int(sqrt(x+sqrt(x**2+1))/x,x);
\end{verbatim}
If one later wishes to integrate expressions without using the facilities of
this package, the switch \sw{ALGINT} \ttindexswitch[ALGINT]{ALGINT} should be turned
off.

\hypertarget{switch:TRA}{}
The switches supported by the standard integrator (e.g., \sw{TRINT})
\ttindexswitch[ALGINT]{TRINT} are also supported by this package.  In addition, the
switch \sw{TRA}, \ttindexswitch[ALGINT]{TRA} if on, will give further tracing
information about the specific functioning of the algebraic integrator.



\section{LENGTH Operator}
\hypertarget{operator:LENGTH}{}
\texttt{LENGTH}\ttindex{LENGTH} is a generic operator for finding the
length of various objects in the system.  The meaning depends on the type
of the object.  In particular, the length of an algebraic expression is
the number of additive top-level terms its expanded representation.

\textit{Examples:}
\begin{verbatim}
        length(a+b)    ->  2
        length(2)      ->  1.
\end{verbatim}
Other objects that support a length operator include arrays, lists and
matrices. The explicit meaning in these cases is included in the description
of these objects.

\section{LIMIT operator}
\index{Computing limits}

LIMITS is a fast limit package for REDUCE for functions which are
continuous except for computable poles and singularities, written by Stanley L.~Kameny, based on some
earlier work by Ian Cohen and John P. Fitch.  The Truncated Power Series
package is used for non-critical points, at which the value of the
function is the constant term in the expansion around that point.
\index{l'H\^opital's rule}
l'H\^opital's rule is used in critical cases, with preprocessing of
$\infty - \infty$ forms and reformatting of product forms in order
to apply l'H\^opital's rule.  A limited amount of bounded arithmetic
is also employed where applicable.

The standard way of calling limit, applying all of the methods, is
\ttindextype{limit}{operator}\hypertarget{operator:LIMIT}{}
\begin{syntax}
  \f{limit(}\meta{exprn:algebraic},\,\meta{var:kernel},\,%
    \meta{limpoint:algebraic})\,:\,\textit{algebraic}
\end{syntax}
The result is the limit of exprn as var approaches limpoint.
To compute the of \(\sin(x)/x\) at the point $0$, enter
\begin{verbatim}
limit(sin(x)/x,x,0);

1
\end{verbatim}
If the limit depends upon the direction of approach to the \texttt{limpoint},                                                                                                        the onesided limit functions \texttt{limit!+} and \texttt{limit!-} may be used:
\ttindextype{limit"!+}{operator}\ttindextype{limit"!-}{operator}
\hypertarget{operator:LIMIT+}{}
\hypertarget{operator:LIMIT-}{}
\begin{syntaxtable}
  \texttt{limit!+(}\meta{exprn:algebraic},\,\meta{var:kernel},\,%
    \meta{limpoint:algebraic}\texttt{)}\,:\,\textit{algebraic} \\
  \texttt{limit!-(}\meta{exprn:algebraic},\,\meta{var:kernel},\,%
    \meta{limpoint:algebraic}\texttt{)}\,:\,\textit{algebraic}
\end{syntaxtable}
they are defined by:
\begin{quote}
\begin{tabular}{l}
 \texttt{limit!+ (limit!-)} (exp,var,limpoint) $\rightarrow$\texttt{limit}(exp*,$\epsilon$,0), \\
  \qquad exp*=sub(var=var+(-)$\epsilon^2$,exp)
\end{tabular}
\end{quote}
for example,
\begin{verbatim}
limit!+(sqrt x/sin x,x,0);

infinity;
\end{verbatim}


\iffalse
\subsection{Diagnostic Functions}

\ttindex{LIMIT0}
\hypertarget{operator:LIMIT0}{}
\vspace{.1in}
\noindent \texttt{LIMIT!-}(EXPRN:{\em algebraic}, VAR:{\em kernel},
LIMPOINT:{\em algebraic}):{\em algebraic}
\noindent \texttt{LIMIT0}(EXPRN:{\em algebraic}, VAR:{\em kernel},
LIMPOINT:{\em algebraic}):{\em algebraic}
\vspace{.1in}

This function will use all parts of the limits package, but it does not
combine log terms before taking limits, so it may fail if there is a sum
of log terms which have a removable singularity in some of the terms.

\ttindex{LIMIT1}
\vspace{.1in}
\noindent \texttt{LIMIT1}(EXPRN:{\em algebraic}, VAR:{\em kernel},
LIMPOINT:{\em algebraic}):{\em algebraic}
\vspace{.1in}

\index{TPS package}
This function uses the TPS branch only, and will fail if the limit point is
singular.

\ttindex{LIMIT2}
\hypertarget{operator:LIMIT2}{}
\begin{quote}
\begin{tabular}{l@{}l}
\texttt{LIMIT2(} & TOP:{\em algebraic}, \\
&BOT:{\em algebraic}, \\
&VAR:{\em kernel}, \\
&LIMPOINT:{\em algebraic}):{\em algebraic}
\end{tabular}
\end{quote}

This function applies l'H\^opital's rule to the quotient (TOP/BOT).

\fi


\section{MAP Operator}\ttindex{MAP}
\hypertarget{operator:MAP}{}

The \texttt{MAP} operator applies a uniform evaluation pattern to all members
of a composite structure: a matrix, a list, or the arguments of an
operator expression.  The evaluation pattern can be a unary procedure, an
operator, or an algebraic expression with one free variable.

It is used with the syntax:
\begin{verbatim}
   MAP(FNC:function,OBJ:object)
\end{verbatim}
Here \texttt{OBJ} is a list, a matrix or an operator expression.
\texttt{FNC} can be one of the following:
\begin{enumerate}
\item the name of an operator with a single argument: the operator
is evaluated once with each element of \texttt{OBJ} as its single argument;
\item an algebraic expression with exactly one free variable, i.e.
a variable preceded by the tilde symbol. The expression
is evaluated for each element of \texttt{OBJ}, with the element
substituted for the free variable;
\item a replacement rule of the form \texttt{var => rep}
where \texttt{var} is a variable (a kernel without a subscript)
and \texttt{rep} is an expression that contains \texttt{var}.
The replacement expression \texttt{rep} is evaluated for each element of
\texttt{OBJ} with
the element substituted for  \texttt{var}. The variable \texttt{var} may be
optionally preceded by a tilde.
\end{enumerate}
The rule form  for \texttt{FNC} is needed when more than
one free variable occurs.

Examples:
\begin{verbatim}
        map(abs,{1,-2,a,-a})  ->  {1,2,ABS(A),ABS(A)}
        map(int(~w,x), mat((x^2,x^5),(x^4,x^5))) ->

                [  3     6 ]
                [ x     x  ]
                [----  ----]
                [ 3     6  ]
                [          ]
                [  5     6 ]
                [ x     x  ]
                [----  ----]
                [ 5     6  ]

        map(~w*6, x^2/3 = y^3/2 -1) -> 2*X^2=3*(Y^3-2)
\end{verbatim}

You can use \texttt{MAP} in nested expressions. However, you cannot
apply \texttt{MAP} to a non-composite object, e.g. an identifier or a number.


\section{MKID Operator}\ttindex{MKID}
\hypertarget{operator:MKID}{}
In many applications, it is useful to create a set of identifiers for
naming objects in a consistent manner. In most cases, it is sufficient to
create such names from two components. The operator \texttt{MKID} is provided
for this purpose. Its syntax is:
\begin{verbatim}
MKID(U:id,V:id|non-negative integer):id
\end{verbatim}
for example
\begin{verbatim}
        mkid(a,3)      -> A3
        mkid(apple,s)  -> APPLES
\end{verbatim}
while \texttt{mkid(a+b,2)} gives an error.

The \texttt{SET}\ttindextype{SET}{statement} statement can be used to give a value to the
identifiers created by \texttt{MKID}, for example
\begin{verbatim}
        set(mkid(a,3),3);
\end{verbatim}
will give \var{A3} the value 2.
Similarly, the \texttt{UNSET}\ttindextype{UNSET}{statement} statement can be used to
remove the value from these identifiers, for example
\begin{verbatim}
        unset(mkid(a,3));
\end{verbatim}

\section{The Pochhammer Notation}
\ttindex{Pochhammer}\index{Pochhammer notation}
\index{Pochhammer symbol}
\hypertarget{POCH}{}
\hypertarget{operator:POCHHAMMER}{}
The Pochhammer notation $(a)_k$ (also called Pochhammer's symbol) is supported
by the binary operator \texttt{Pochhammer(a,k)}.
For a non-negative integer $k$, it is defined as
(\url{http://dlmf.nist.gov/5.2.iii})
\begin{align*}
  (a)_0 &= 1, \\
  (a)_k &= a(a+1)(a+2)\cdots(a+k-1).
\end{align*}
%% Note that MathJax does not support \intertext.
For $a \neq 0, -1, -2, -3, \ldots$, this is equivalent to
\[ (a)_k = \frac{\Gamma(a+k)}{\Gamma(a)}. \]
When $n$ is integral, the defining product is expanded (assuming the switch
\sw{EXP} is ON). With \sw{ROUNDED} off, this expression is evaluated
numerically if $a$ is numerical and $k$ is integral, and otherwise may be
simplified where appropriate.  The simplification rules are based upon
algorithms supplied by Wolfram Koepf \cite{Koepf:92}.

The Pochhammer symbol is used quite extensively in the simplification and
numerical evaluation of \hyperlink{SPECFNS}{special functions}.

\section{PF Operator}\ttindextype{PF}{operator}
\hypertarget{operator:PF}{}

\index{partial fraction}\index{partial fraction decomposition}\index{decomposition!partial fraction}
\texttt{PF(\meta{exp},\meta{var})} transforms the expression \meta{exp} into
a list of partial fractions with respect to the main variable, \meta{var}.
\texttt{PF}
does a complete partial fraction decomposition, and as the algorithms used
are fairly unsophisticated (factorization and the extended Euclidean
algorithm), the code may be unacceptably slow in complicated cases.

\textit{Example:}
Given \texttt{2/((x+1)\textasciicircum2*(x+2))} in the workspace,
\texttt{pf(ws,x);} gives the result
\begin{samepage}
\begin{verbatim}
            2      - 2         2
        {-------,-------,--------------} .
          X + 2   X + 1    2
                          X  + 2*X + 1
\end{verbatim}
\end{samepage}

If you want the denominators in factored form, use \texttt{off exp;}.
Thus, with \texttt{2/((x+1)\textasciicircum2*(x+2))} in the workspace,
the commands \texttt{off exp; pf(ws,x);} give the result
\begin{verbatim}
            2      - 2       2
        {-------,-------,----------} .
          X + 2   X + 1          2
                          (X + 1)
\end{verbatim}

To recombine the terms, \texttt{FOR EACH} \ldots{} \texttt{SUM} can be used.
So with the above list in the workspace,
\texttt{for each j in ws sum j;} returns the result
\begin{verbatim}
             2
     ------------------
                     2
      (X + 2)*(X + 1)
\end{verbatim}

Alternatively, one can use the operations on lists to extract any desired
term.


\section{RESIDUE and POLEORDER operators}
\ttindex{RESIDUE}
\ttindex{POLEORDER}
\hypertarget{operator:RESIDUE}{}
\hypertarget{operator:POLEORDER}{}

The residue
$\mathop{\mathrm{Res}}\limits_{z=a} f(z)$ of a function $f(z)$ at the point $a\in\mathbb{C}$ is defined
as
\[
\mathop{\mathrm{Res}}\limits_{z=a} f(z)=
\frac{1}{2 \pi i}\oint f(z)\,dz
\;,
\]
with integration along a closed curve around $z=a$ with winding number 1.

If $f(z)$ is given by a Laurent series expansion at $z=a$
\[
f(z)=\sum_{k=-\infty}^\infty a_k\,(z-a)^k
\;,
\]
then
\begin{equation}
\mathop{\mathrm{Res}}\limits_{z=a} f(z)=a_{-1}
\;.
\label{eq:Laurent}
\end{equation}
If $a=\infty$, one defines on the other hand
\begin{equation}
\mathop{\mathrm{Res}}\limits_{z=\infty} f(z)=-a_{-1}
\label{eq:Laurent2}
\end{equation}
for given Laurent representation
\[
f(z)=\sum_{k=-\infty}^\infty a_k\,\frac{1}{z^k}
\;.
\]
The operator \f{RESIDUE}\texttt{(f,z,a)} determines the residue of $f$ at the point $z=a$
if $f$ is meromorphic at $z=a$. The calculation of residues at essential
singularities of $f$ is not supported, as are the residues of factorial terms.\footnote{This code was written by Wolfram Koepf.}

\f{POLEORDER}\texttt{(f,z,a)} determines the pole order of $f$ at the point $z=a$
if $f$ is meromorphic at $z=a$.

Note that both functions use the operator \f{taylor} in
connection with representations (\eqnref{eq:Laurent})--(\eqnref{eq:Laurent2}).

Here are some examples:

\begin{verbatim}
2: residue(x/(x^2-2),x,sqrt(2));

 1
---
 2

3: poleorder(x/(x^2-2),x,sqrt(2));

1

4: residue(sin(x)/(x^2-2),x,sqrt(2));

 sqrt(2)*sin(sqrt(2))
----------------------
          4

5: poleorder(sin(x)/(x^2-2),x,sqrt(2));

1

6: residue(1/(x-1)^m/(x-2)^2,x,2);

 - m

7: poleorder(1/(x-1)/(x-2)^2,x,2);

2

8: residue(sin(x)/x^2,x,0);

1

9: poleorder(sin(x)/x^2,x,0);

1

10: residue((1+x^2)/(1-x^2),x,1);

-1

11: poleorder((1+x^2)/(1-x^2),x,1);

1

12: residue((1+x^2)/(1-x^2),x,-1);

1

13: poleorder((1+x^2)/(1-x^2),x,-1);

1

14: residue(tan(x),x,pi/2);

-1

15: poleorder(tan(x),x,pi/2);

1

16: residue((x^n-y^n)/(x-y),x,y);

0

17: poleorder((x^n-y^n)/(x-y),x,y);

0

18: residue((x^n-y^n)/(x-y)^2,x,y);

  n
 y *n
------
  y

19: poleorder((x^n-y^n)/(x-y)^2,x,y);

1

20: residue(tan(x)/sec(x-pi/2)+1/cos(x),x,pi/2);

-2

21: poleorder(tan(x)/sec(x-pi/2)+1/cos(x),x,pi/2);

1

22: for k:=1:2 sum residue((a+b*x+c*x^2)/(d+e*x+f*x^2),x,
    part(part(solve(d+e*x+f*x^2,x),k),2));

 b*f - c*e
-----------
     2
    f

23: residue(x^3/sin(1/x)^2,x,infinity);

  - 1
------
  15

24: residue(x^3*sin(1/x)^2,x,infinity);

-1

25: residue(gamma(x),x,-1);

-1

26: residue(psi(x),x,-1);

-1

27: on fullroots;

28: for k:=1:3 sum
28: residue((a+b*x+c*x^2+d*x^3)/(e+f*x+g*x^2+h*x^3),x,
28: part(part(solve(e+f*x+g*x^2+h*x^3,x),k),2));

0
\end{verbatim}


\section{SELECT Operator}\ttindex{SELECT}
\hypertarget{operator:SELECT}{}
\ttindex{map}\ttindex{list}

The \texttt{SELECT} operator extracts from a list,
or from the arguments of an n--ary operator, elements corresponding
to a boolean predicate. It is used with the syntax:
\begin{syntax}
  \texttt{SELECT(}\meta{FNC:function},\,\meta{LST:list}\texttt{)}
\end{syntax}

\texttt{FNC} can be one of the following forms:
\begin{enumerate}
\item the name of an operator with a single argument: the operator
is evaluated once on each element of \texttt{LST};
\item an algebraic expression with exactly one free variable, i.e.
a variable preceded by the tilde symbol. The expression
is evaluated for each element of \meta{LST}, with the element
substituted for the free variable;
\item a replacement rule of the form \meta{var} \texttt{=>} \meta{rep}
where \meta{var} is a variable (a kernel without subscript)
and \meta{rep} is an expression that contains \meta{var}.
\meta{rep} is evaluated for each element of \texttt{LST} with
the element substituted for  \meta{var}. \meta{var} may be
optionally preceded by a tilde.
\end{enumerate}
The rule form  for \texttt{FNC} is needed when more than
one free variable occurs.

The result of evaluating \texttt{FNC} is
interpreted as a boolean value corresponding to the conventions of
\REDUCE. These values are composed with the leading operator of the
input expression.

\textit{Examples:}\nopagebreak
\begin{verbatim}
    select( ~w>0 , {1,-1,2,-3,3}) -> {1,2,3}
    select(evenp deg(~w,y),part((x+y)^5,0):=list)
           -> {X^5 ,10*X^3*Y^2 ,5*X*Y^4}
    select(evenp deg(~w,x),2x^2+3x^3+4x^4) -> 4X^4 + 2X^2
\end{verbatim}
