
\index{Schwarz, Fritz}\index{People!Schwarz, Fritz}

The package \package{SPDE} provides a set of functions which may be applied
to determine the symmetry group of Lie- or point-symmetries of a
given system of partial differential equations.  Preferably it is
used interactively on a computer terminal. In many cases the
determining system is solved completely automatically. In some
other cases the user has to provide some additional input
information for the solution algorithm to terminate. The package
should only be used in compiled form.

For all theoretical questions, a description of the algorithm and
numerous examples the following articles should be consulted:
\cite{Schwarz:85,Schwarz:88,Schwarz:GMD1987}.
%``Automatically Determining Symmetries of Partial Differential
%Equations'', Computing vol. 34, page 91-106(1985) and vol. 36, page
%279-280(1986), ``Symmetries of Differential Equations: From Sophus
%Lie to Computer Algebra'', SIAM Review, to appear, and Chapter 2
%of the Lecture Notes ``Computer Algebra and Differential Equations
%of Mathematical Physics'', to appear.


\subsection{Description of the System Functions and Variables}

The symmetry analysis of partial differential equations logically
falls into three parts. Accordingly the most important functions
provided by the package are:
\begin{table}[htbp]
\begin{center}
\begin{tabular}{| c | c | }\hline
Function name & Operation \\ \hline \hline
\ttindextype[SPDE]{cresys}{operator}
\f{cresys}(\s{arguments}) & Constructs determining system \\ \hline
\ttindextype[SPDE]{simpsys}{operator}
\f{simpsys}() & Solves determining system \\ \hline
\ttindextype[SPDE]{result}{operator}
\f{result}() & Prints infinitesimal generators \\
&  and commutator table \\ \hline
\end{tabular}\\
\caption{\package{SPDE} Functions}
\end{center}
\end{table}

Some other useful functions for obtaining various kinds of output
are:\pagebreak
\begin{table}[htbp]
\begin{center}
\begin{tabular}{| c | c |} \hline
Function name & Operation \\ \hline \hline
\ttindextype[SPDE]{prsys}{operator}
\f{prsys}() & Prints determining system \\ \hline
\ttindextype[SPDE]{prgen}{operator}
\f{prgen}() & Prints infinitesimal generators \\ \hline
\ttindextype[SPDE]{comm}{operator}
\f{comm}(U,V) & Prints commutator of generators U and V \\ \hline
\end{tabular}\\
\caption{\package{SPDE} Useful Output Functions}\label{spde:useful}
\end{center}
\end{table}

There are several global variables defined by the system which should
not be used for any other purpose than that given in
Table~\ref{spde:intt} and~\ref{spde:op}. The three globals of the type
integer are:
\begin{table}[htbp]
  \hypertarget{reserved:MM}{}
  \hypertarget{reserved:NN}{}
  \hypertarget{reserved:PCLASS}{}
\begin{center}
\begin{tabular}{| c | c |}\hline
Variable name & Meaning \\ \hline \hline
\ttindextype[SPDE]{nn}{reserved variable}
\var{nn} & Number of independent variables \\ \hline
\ttindextype[SPDE]{mm}{reserved variable}
\var{mm} & Number of dependent variables \\ \hline
\ttindextype[SPDE]{pclass}{reserved variable}
\var{pclass}=0, 1 or 2 & Controls amount of output \\ \hline
\end{tabular}\\
\caption{\package{SPDE} Integer valued globals}\label{spde:intt}
\end{center}
\end{table}

In addition there are the following global variables of type
operator:
\begin{table}[htbp]
\begin{center}
\begin{tabular}{| c | c |}\hline
Variable name & Meaning \\ \hline \hline
\ttindextype[SPDE]{x(i)}{operator}
X(I) & Independent variable $x_i$ \\ \hline
\ttindextype[SPDE]{u(alfa)}{operator}
U(ALFA) & Dependent variable $u^{alfa}$ \\ \hline
\ttindextype[SPDE]{U(ALFA,I)}{operator}
U(ALFA,I) & Derivative of $u^{alfa}$ w.r.t. $x_i$ \\ \hline
\ttindextype[SPDE]{deq(i)}{operator}
DEQ(I) & i-th differential equation \\ \hline
\ttindextype[SPDE]{sder(i)}{operator}
SDER(I) & Derivative w.r.t. which DEQ(I) is resolved \\ \hline
\ttindextype[SPDE]{gl(i)}{operator}
GL(I) & i-th equation of determining system \\ \hline
\ttindextype[SPDE]{gen(i)}{operator}
GEN(I) & i-th infinitesimal generator \\ \hline
\ttindextype[SPDE]{xi(i)}{operator}
\ttindextype[SPDE]{ETA(ALFA)}{operator}
\ttindextype[SPDE]{ZETA(ALFA,I)}{operator}
XI(I), ETA(ALFA)  & See definition given in the \\
ZETA(ALFA,I) & references quoted in the introduction. \\ \hline
\ttindextype[SPDE]{c(i)}{operator}
C(I) & i-th function used for substitution \\ \hline
\end{tabular}\\
\caption{\package{SPDE} Operator type global variables}\label{spde:op}
\end{center}
\end{table}

The differential equations of the system at issue have to be assigned
as values to the operator deq i applying the notation which is defined
in Table~\ref{spde:op}. The entries in the third and the last line of
that Table have obvious extensions to higher derivatives.

The derivative w.r.t. which the i-th differential equation deq i is
resolved has to be assigned to sder i. Exception: If there is a single
differential equation and no assignment has been made by the user, the
highest derivative is taken by default.

When the appropriate assignments are made to the variable deq, the
values of \var{nn} and \var{mm} (Table~\ref{spde:useful}) are determined
automatically, i.e. they have not to be assigned by the user.

\ttindextype[SPDE]{cresys}{operator}
The function CRESYS may be called with any number of arguments, i.e.

\begin{verbatim}
  cresys(); or cresys(deq 1,deq 2,... );
\end{verbatim}

 are legal calls. If it is called without any argument, all current
assignments to deq are taken into account. Example: If deq 1, deq 2
and deq 3 have been assigned a differential equation and the symmetry
group of the full system comprising all three equations is desired,
equivalent calls are

\begin{verbatim}
  cresys();   or   cresys(deq 1,deq 2,deq 3);
\end{verbatim}

The first alternative saves some typing. If later in the session the
symmetry group of deq 1 alone has to be determined, the correct call
is

\begin{verbatim}
  cresys deq 1;
\end{verbatim}

\ttindextype[SPDE]{simpsys}{operator}
after the determining system has bee created, \f{simpsys} which has no
arguments may be called for solving it. The amount of intermediate
output produced by \f{simpsys} is controlled by the global variable \var{pclass}
with the default value 0. \ttindextype[SPDE]{pclass}{reserved variable}With \var{pclass} equal to 0, no
intermediate steps are shown. With \var{pclass} equal to 1, all intermediate
steps are displayed so that the solution algorithm may be followed
\index{Tracing!\package{SPDE} package} through in detail. Each time the algorithm
passes through the top of the main solution loop the message

\begin{verbatim}
  Entering main loop
\end{verbatim}

is written. \var{pclass} equal 2 produces a lot of LISP output and is of no
interest for the normal user.

If with \var{pclass}=0 the procedure \f{simpsys} terminates without any
response, the determining system is completely solved.  In some cases
\f{simpsys} does not solve the determining system completely in a single
run. In general this is true if there are only genuine differential
equations left which the algorithm cannot handle at present. If a case
like this occurs, \f{simpsys} returns the remaining equations of the
determining system. To proceed with the solution algorithm,
appropriate assignments have to be transmitted by the user, e.g. the
explicit solution for one of the returned differential equations. Any
new functions which are introduced thereby must be operators of the
form c(k) with the correct dependencies generated by a depend
statement (see section~\ref{sec:dependency}). Its enumeration has to be
chosen in agreement with the current number of functions which have
alreday been introduced.  This value is returned by \f{simpsys} too.

After the determining system has been solved, the procedure \f{result},
which has no arguments, may be called. It displays the infinitesimal
generators and its non-vanishing commutators.


\subsection{How to Use the Package}

In this Section it is explained by way of several examples how the
package \package{SPDE} is used interactively to determine the symmetry group of
partial differential equations. Consider first the diffusion equation
which in the notation given above may be written as

\begin{verbatim}
  deq 1:=u(1,1)+u(1,2,2);
\end{verbatim}

It has been assigned as the value of deq 1 by this statement.  There
is no need to assign a value to sder 1 here because the system
comprises only a single equation.

The determining system is constructed by calling

\begin{verbatim}
  cresys(); or cresys deq 1;
\end{verbatim}

The latter call is compulsory if there are other assignments to the
operator deq i than for i=1.

The error message

\begin{verbatim}
  ***** Differential equations not defined
\end{verbatim}

appears if there are no differential equations assigned to any deq.

If the user wants the determining system displayed for inspection
before starting the solution algorithm he may call

\ttindextype[SPDE]{prsys}{operator}
\begin{verbatim}
  prsys();
\end{verbatim}

and gets the answer

\begin{verbatim}
  gl(1):=2*df(eta(1),u(1),x(2)) - df(xi(2),x(2),2) -
         df(xi(2),x(1))

  gl(2):=df(eta(1),u(1),2) - 2*df(xi(2),u(1),x(2))

  gl(3):=df(eta(1),x(2),2) + df(eta(1),x(1))

  gl(4):=df(xi(2),u(1),2)

  gl(5):=df(xi(2),u(1)) - df(xi(1),u(1),x(2))

  gl(6):=2*df(xi(2),x(2)) - df(xi(1),x(2),2) - df(xi(1),x(1))

  gl(7):=df(xi(1),u(1),2)

  gl(8):=df(xi(1),u(1))

  gl(9):=df(xi(1),x(2))

The remaining dependencies

  xi(2) depends on u(1),x(2),x(1)

  xi(1) depends on u(1),x(2),x(1)

  eta(1) depends on u(1),x(2),x(1)
\end{verbatim}

The last message means that all three functions xi(1), xi(2) and
eta(1) depend on x(1), x(2) and u(1). Without this information the
nine equations gl(1) to gl(9) forming the determining system are
meaningless. Now the solution algorithm may be activated by calling

\ttindextype[SPDE]{simpsys}{operator}
\begin{verbatim}
   simpsys();
\end{verbatim}

\ttindextype[SPDE]{pclass}{reserved variable}
If the print flag \var{pclass} has its default value which is 0 no
intermediate output is produced and the answer is

\begin{verbatim}
  Determining system is not completely solved

  The remaining equations are

  gl(1):=df(c(1),x(2),2) + df(c(1),x(1))

  Number of functions is 16

  The remaining dependencies

  c(1) depends on x(2),x(1)
\end{verbatim}

With \var{pclass} equal to 1 about 6 pages of intermediate output are
obtained. It allows the user to follow through each step of the
solution algorithm.

In this example the algorithm did not solve the determining system
completely as it is shown by the last message. This was to be expected
because the diffusion equation is linear and therefore the symmetry
group contains a generator depending on a function which solves the
original differential equation. In cases like this the user has to
provide some additional information to the system so that the solution
algorithm may continue. In the example under consideration the
appropriate input is

\begin{verbatim}
   df(c(1),x(1)) := - df(c(1),x(2),2);
\end{verbatim}

If now the solution algorithm is activated again by

\begin{verbatim}
  simpsys();
\end{verbatim}

the solution algorithm terminates without any further message, i.e.
there are no equations of the determining system left unsolved. To
obtain the symmetry generators one has to say finally

\begin{verbatim}
  result();
\end{verbatim}

and obtains the answer

\begin{verbatim}
  The differential equation

  deq(1):=u(1,2,2) + u(1,1)


  The symmetry generators are

  gen(1):= dx(1)

  gen(2):= dx(2)

  gen(3):= 2*dx(2)*x(1) + du(1)*u(1)*x(2)

  gen(4):= du(1)*u(1)

  gen(5):= 2*dx(1)*x(1) + dx(2)*x(2)

                       2
  gen(6):= 4*dx(1)*x(1)

         + 4*dx(2)*x(2)*x(1)

                           2
           + du(1)*u(1)*(x(2)  - 2*x(1))

  gen(7):= du(1)*c(1)

  The remaining dependencies

  c(1) depends on x(2),x(1)


  Constraints

  df(c(1),x(1)):= - df(c(1),x(2),2)


  The non-vanishing commutators of the finite subgroup


  comm(1,3):= 2*dx(2)

  comm(1,5):= 2*dx(1)

  comm(1,6):= 8*dx(1)*x(1) + 4*dx(2)*x(2) - 2*du(1)*u(1)

  comm(2,3):= du(1)*u(1)

  comm(2,5):= dx(2)

  comm(2,6):= 4*dx(2)*x(1) + 2*du(1)*u(1)*x(2)

  comm(3,5):=  - (2*dx(2)*x(1) + du(1)*u(1)*x(2))

                          2
  comm(5,6):= 8*dx(1)*x(1)

            + 8*dx(2)*x(2)*x(1)

                                2
            + 2*du(1)*u(1)*(x(2)  - 2*x(1))
\end{verbatim}

The message ``Constraints'' which appears after the symmetry generators
are displayed means that the function c(1) depends on x(1) and x(2)
and satisfies the diffusion equation.

More examples which may used for test runs are given in the final
section.

\index{ansatz of symmetry generator}
If the user wants to test a certain ansatz of a symmetry generator for
given differential equations, the correct proceeding is as follows.
Create the determining system as described above. Make the appropriate
assignments for the generator and call PRSYS() after that.  The
determining system with this ansatz substituted is returned.  Example:
Assume again that the determining system for the diffusion equation
has been created. To check the correctness for example of generator GEN
3 which has been obtained above, the assignments

\begin{verbatim}
  xi(1):=0;  xi(2):=2*x(1);  eta(1):=x(2)*u(1);
\end{verbatim}

have to be made. If now \f{prsys()} is called all gl(k) are zero
proving the correctness of this generator.

Sometimes a user only wants to know some of the functions \f{zeta} for for
various values of its possible arguments and given values of \var{mm} and
\var{nn}. In these cases the user has to assign the desired values of \var{mm} and
\var{nn} and may call the ZETAs after that. Example:

\begin{verbatim}
  mm:=1;  nn:=2;

  factor u(1,2),u(1,1),u(1,1,2),u(1,1,1);

  on list;

  zeta(1,1);

  -u(1,2)*u(1,1)*df(xi(2),u(1))

  -u(1,2)*df(xi(2),x(1))

         2
  -u(1,1) *df(xi(1),u(1))

  +u(1,1)*(df(eta(1),u(1)) -df(xi(1),x(1)))

  +df(eta(1),x(1))


  zeta(1,1,1);

  -2*u(1,1,2)*u(1,1)*df(xi(2),u(1))

  -2*u(1,1,2)*df(xi(2),x(1))

  -u(1,1,1)*u(1,2)*df(xi(2),u(1))

  -3*u(1,1,1)*u(1,1)*df(xi(1),u(1))

  +u(1,1,1)*(df(eta(1),u(1)) -2*df(xi(1),x(1)))

                2
  -u(1,2)*u(1,1) *df(xi(2),u(1),2)

  -2*u(1,2)*u(1,1)*df(xi(2),u(1),x(1))

  -u(1,2)*df(xi(2),x(1),2)

         3
  -u(1,1) *df(xi(1),u(1),2)

         2
  +u(1,1) *(df(eta(1),u(1),2) -2*df(xi(1),u(1),x(1)))

  +u(1,1)*(2*df(eta(1),u(1),x(1)) -df(xi(1),x(1),2))

  +df(eta(1),x(1),2)
\end{verbatim}

If by error no values to \var{mm} or \var{nn} and have been assigned the message

\begin{verbatim}
  ***** Number of variables not defined
\end{verbatim}

is returned. Often the functions zeta are desired for special values
of its arguments eta(alfa) and xi(k). To this end they have to be
assigned first to some other variable. After that they may be
evaluated for the special arguments. In the previous example this may
be achieved by

\begin{verbatim}
  z11:=zeta(1,1)$   z111:=zeta(1,1,1)$
\end{verbatim}

Now assign the following values to xi 1, xi 2 and eta 1:

\begin{verbatim}
  xi 1:=4*x(1)**2; xi 2:=4*x(2)*x(1);

  eta 1:=u(1)*(x(2)**2  - 2*x(1));
\end{verbatim}

They correspond to the generator gen 6 of the diffusion equation which
has been obtained above. Now the desired expressions are obtained by
calling

\begin{verbatim}
  z11;

                               2
 - (4*u(1,2)*x(2) - u(1,1)*x(2)  + 10*u(1,1)*x(1) + 2*u(1))

  z111;

                                   2
 - (8*u(1,1,2)*x(2) - u(1,1,1)*x(2)  + 18*u(1,1,1)*x(1) +
   12*u(1,1))
\end{verbatim}


\subsection{Test File}

This appendix is a test file. The symmetry groups for various
equations or systems of equations are determined. The variable \var{pclass}
has the default value 0 and may be changed by the user before running
it. The output may be compared with the results which are given in the
references.

\begin{verbatim}
%The Burgers equations

deq 1:=u(1,1)+u 1*u(1,2)+u(1,2,2)$

cresys deq 1$ simpsys()$ result()$

%The Kadomtsev-Petviashvili equation

deq 1:=3*u(1,3,3)+u(1,2,2,2,2)+6*u(1,2,2)*u 1

       +6*u(1,2)**2+4*u(1,1,2)$

cresys deq 1$ simpsys()$ result()$

%The modified Kadomtsev-Petviashvili equation

deq 1:=u(1,1,2)-u(1,2,2,2,2)-3*u(1,3,3)

       +6*u(1,2)**2*u(1,2,2)+6*u(1,3)*u(1,2,2)$

cresys deq 1$ simpsys()$ result()$

%The real- and the imaginary part of the nonlinear
%Schroedinger equation

deq 1:= u(1,1)+u(2,2,2)+2*u 1**2*u 2+2*u 2**3$

deq 2:=-u(2,1)+u(1,2,2)+2*u 1*u 2**2+2*u 1**3$

%Because this is not a single equation the two assignments

sder 1:=u(2,2,2)$  sder 2:=u(1,2,2)$

%are necessary.

cresys()$ simpsys()$ result()$

%The symmetries of the system comprising the four equations

deq 1:=u(1,1)+u 1*u(1,2)+u(1,2,2)$

deq 2:=u(2,1)+u(2,2,2)$

deq 3:=u 1*u 2-2*u(2,2)$

deq 4:=4*u(2,1)+u 2*(u 1**2+2*u(1,2))$

sder 1:=u(1,2,2)$ sder 2:=u(2,2,2)$ sder 3:=u(2,2)$
sder 4:=u(2,1)$

%is obtained by calling

cresys()$ simpsys()$

df(c 5,x 1):=-df(c 5,x 2,2)$

df(c 5,x 2,x 1):=-df(c 5,x 2,3)$

simpsys()$  result()$

% The symmetries of the subsystem comprising equation 1
%  and 3 are obtained by

cresys(deq 1,deq 3)$ simpsys()$ result()$

% The result for all possible subsystems is discussed in
% detail in ``Symmetries and Involution Systems: Some
% Experiments in Computer Algebra'', contribution to the
% Proceedings of the Oberwolfach Meeting on Nonlinear
% Evolution Equations, Summer 1986, to appear.
\end{verbatim}
