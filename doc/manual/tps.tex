
\index{Power series} \index{Extendible power series}\index{Power series!extendible}
\index{Barnes, Alan} \index{Padget, Julian}
\index{Persons!Barnes, Alan} \index{Persons!Padget, Julian}
\subsection{Introduction}
\index{Power series!expansions}
This package implements formal Laurent power series expansions in one
variable using the domain mechanism of REDUCE. This means that power
series objects can be added, multiplied, differentiated etc. like other
first class objects in the system. A lazy evaluation scheme is used in
the package and thus terms of the series are not evaluated until they
are required for printing or for use in calculating terms in other
power series. The series are extendible giving the user the impression
that the full infinite series is being manipulated.  The errors that
can sometimes occur using series that are truncated at some fixed depth
(for example when a term in the required series depends on terms of an
intermediate series beyond the truncation depth) are thus avoided.

The package was originally based on an earlier \emph{truncated power series} package
developed by Julian Padget in the 1980's. The name of the original package was
TPS and this was never changed. The alternative (more accurate) name EPS was
perhaps rejected because of possible confusion with the acronym for
\emph{encapsulated PostScript}.

In the first subsection below a brief description of the main operators
available for series expansion are given together with some examples of
their use.

\subsection{Basic Use}
\hypertarget{operator:PS}{}
\ttindextype{ps}{operator}
The most important operator is  \texttt{ps} which is used as follows:
\begin{verbatim}
  ps(EXP:algebraic, VAR:kernel, 
     ABOUT:algebraic):algebraic.
\end{verbatim}

The \texttt{ps} operator returns a Laurent power series object (a tagged domain
element) representing the univariate formal Laurent power series expansion of
\f{EXP} with respect to the dependent variable \f{VAR} about the expansion point
\f{ABOUT}.  \f{EXP} may itself contain power series objects.
If the function has a pole at the expansion point then the correct
Laurent series expansion will be produced.

The algebraic expression \f{ABOUT} should simplify to an expression
which is independent of the dependent variable \f{VAR}, otherwise
an error will result.  If \f{ABOUT} is the identifier \texttt{infinity}
then the power series expansion about $\infty$ is obtained in ascending
powers of \texttt{1/VAR}.

\textbf{Examples}
\begin{verbatim}
   a := ps(sin x, x, 0);
   ps(sin a, x, 0);
   ps(cos x/x^2, x, 0);
   ps(x/(1+x),x,infinity);
\end{verbatim}

\textbf{Operations on Power Series}
\index{Power Series!arithmetic}

As power series objects are domain elements they may be added, subtracted,
multiplied and divided in the normal way. For example if A and B are power
series objects with the \emph{same expansion variable and expansion point}:
\begin{verbatim}
    a+b; a*b;
    1/b; a/b;
\end{verbatim}

will produce power series objects representing the sum, product, reciprocal,
and quotient of the power series objects A and B
respectively.

\index{Power Series!differentiation}
\textbf{Differentiation}

Similarly, if A is a power series object depending on X then the input
{\tt df(a, x);} will produce the power series expansion of the
derivative of A with respect to X.

\index{Power series!integration}
\textbf{Integration}

The power series expansion of an integral may also be obtained (even if
REDUCE cannot evaluate the integral in closed form).  An example of
this is

\begin{verbatim}
    ps(int(exp(exp x),x),x,0);
\end{verbatim}

Note that if the integration variable is the same as the expansion
variable, the integration package is not called. If on the
other hand the two variables are different the integrator is
called to integrate each of the coefficients in the power series
expansion of the integrand.  The constant of integration is zero by
default.

Note that the Laurent series domain is not closed under integration with
respect to the expansion variable; if the term of degree -1 is non-zero a
logarithmic singularity error will occur on integration.

\index{Power series!exponentiation}
\textbf{Exponentiation}
The Laurent series domain is closed under exponentiation by an \emph{integer}
power. Thus, with respect to integer exponentiation, power series are first
class objects and for example the following results in automatic expansion of
the final result:
\begin{verbatim}
  a:= ps(cos x,x,0);
  b:= ps(sin x,x,0);
  a^2+b^(-2);
\end{verbatim}

However, for more general exponents automatic expansion does not occur. For
example given power series \texttt{a} and \texttt{b} defined as above, the
following commands are necessary:
\begin{verbatim}
    ps(a^(1/2),x,0);
    ps(a^pi,x,0);
    ps(a^b,x,0);
\end{verbatim}
Note \emph{any} power of a power series of \emph{order zero} (that is with a
non-zero term of degree zero) can be expanded as a power series (again of
order zero) provided only that the power is non-singular at the expansion point.
As the third example above shows the exponent may itself be a power series.

However in general the Laurent series domain is not closed under exponentiation.
If the result is to be a Laurent series some restrictions 
on the allowed values of the exponent and order of the original series are
necessary. Namely, if the order of the power series is non-zero ($\sigma$ say)
and the exponent is rational with denominator $q$ say, then $\sigma q$ must be
integral.  If the exponent is rational, but $\sigma q$ is not an integer,
a \emph{branch point} error is generated. For other exponents a
\emph{logarithmic singularity} error is usually generated. For example,
\begin{verbatim}
   a := ps(1-cos x,x,0);  % series has order 2
   ps(a^(1/2),x,0);   % series has order 1
   ps(a^(2/3),x,0);   % branch point error
   ps(a^pi,x,0);      % logarithmic singularity error
\end{verbatim}

\textbf{Power series of user defined functions}

New user-defined functions may be expanded provided the user provides
a rule or rule list defining the derivative of the function and optionally
its value at the expansion point. For example
\begin{verbatim}
    operator u;
    let df(u(~x),~x)= exp(e^x);
    let u(0) = e;
    ps(u(sin x),x,0);
\end{verbatim}
Of course the rules defined must be such that the function actually has a
Taylor series expansion about the specified point.

\textbf{Restrictions and Known Bugs}

Currently automatic expansion of quotients with an integer denominator does not
normally occur. One must use:
\begin{verbatim}
    a:=ps(sin x,x,0);
    ps(a/5,x,0);
or
    on rational;    % or on rounded;
    a/5;
\end{verbatim}

Currently the following does not produce a power series object (although the
result is formally valid):
\begin{verbatim}
   a := ps(cos x, x, 0);
   ps(2^a,x,0);
 % instead use:
   ps(2^cos x,x,0);
\end{verbatim}

If A is a power series object and X is a variable
which evaluates to itself then expressions such as \texttt{a*x} or
\texttt{int(a, x);}  do not automtically expand to a single power series object
(although the result returned is formally valid).  Instead expressions such as
 \texttt{ps(a*x,x,0)} and \texttt{ps(int(a,x),x,0} should be used.

Currently the handling of essential sigularities is rather erratic; sometimes
an Essential Singularity or Logarithmic Singularity error message is output,
but often the system fails rather ungracefully.

There is no simple way to write the results of power series
calculation to a file and read them back into REDUCE at a later
stage.

\textbf{Taylor Series Expansion}

The operator \texttt{pstaylor} may be used as follows:
\hypertarget{operator:PSTAYLOR}{}
\ttindextype{pstaylor}{operator}
\begin{verbatim}
  pstaylor(EXP:algebraic, VAR:kernel, 
           ABOUT:algebraic):algebraic.
\end{verbatim}

which uses the classic Taylor series algorithm for expanding \f{EXP} and
returning an extendible Taylor series object.

The \f{pstaylor} operator may be useful in contexts where the  operator \f{ps}
fails to build a suitable recurrence relation automatically and reports too
deep a recursion in \texttt{ps!:unknown!-crule}. A typical example is the
expansion of the $\Gamma$ function about an expansion point which is not a
non-positive integer
\footnote{Actually the TPS code now detects this case and automatically uses
\f{pstaylor} where appropriate.}.

Note, however, that \f{pstaylor} always returns a \emph{Taylor} series whose
order is non-negative.  Attempting to use \f{pstaylor} to expand a function
about a pole will fail with a zero divisor error message.

Also in many cases the use of an automatically generated recurrence relation
built by \f{ps} is more  efficient than using \f{pstaylor}, particularly if a
large number of terms is required; expansion of \f{tan} is a typical example
where the number of terms in the nth derivative grows exponentially.


\subsection{Printing Power Series}

If the command \f{ps} or \f{pstaylor} is terminated by a semi-colon, a power
series object is compiled and then a number of terms of the
power series expansion are evaluated and printed.

\textbf{psexplim Operator}

The expansion is carried out as far as the value specified by an internal
variable (with a default value of 6). This variable can be accessed via the
operator \texttt{psexplim}.
\hypertarget{operator:PSEXPLIM}{}
\ttindextype{psexplim}{operator}
\begin{verbatim}
  psexplim(UPTO:integer):integer.
or
  psexplim():integer
\end{verbatim}
If \texttt{psexplim} is called with an integer value, the internal variable is
updated to the value of \texttt{UPTO} and its previous value is returned.
If \texttt{psexplim} is called with no argument the current value is unaltered
and that value is returned. 

If \texttt{psexplim} is used to increase the expansion limit, sufficient
information is stored in the power series object to enable the additional
terms to be calculated without recalculating the terms already obtained.

If the command is terminated by a dollar symbol, a power series object
is compiled and the first term is calculated, but no output is printed.

\textbf{psprintorder Switch}
\hypertarget{switch:PSPRINTORDER}{}
\ttindexswitch[TPS]{psprintorder}

When the switch \sw{psprintorder} is ON the trailing terms of power series
beyond \texttt{psexplim} are represented in print by a big-O notation,
otherwise, three dots are printed. This switch is ON by default.
However, if expression being expanded is a polynomial in the expansion
variable and all non-zero terms have been output then the big-O or trailing
dots are omitted to indicate that the series is complete.

\subsection{Accessor Functions}
In this section a number of accessor functions which allow the user to extract
information such as the dependent variable, expansion point, a particular term
etc. of a power series object.

\textbf{psdepvar Operator}
\hypertarget{operator:PSDEPVAR}{}
\ttindextype{psdepvar}{operator}
\begin{verbatim}
  psdepvar(TPS:power series object):identifier.
\end{verbatim}
The operator \texttt{psdepvar} returns the expansion variable of the
power series object TPS. TPS should evaluate to a power
series object or an integer, otherwise an error results. If TPS
is an integer, the identifier \texttt{undefined} is returned.

\textbf{psexpansionpt operator}
\hypertarget{operator:PSEXPANSIONPT}{}
\ttindextype{psexpansionpt}{operator}
\begin{verbatim}
  psexpansionpt(TPS:power-series-object):algebraic.
\end{verbatim}
The operator \texttt{psexpansionpt} returns the expansion point of the
power series object TPS. TPS should evaluate to a power
series object or an integer, otherwise an error results. If TPS
is an integer, the identifier \texttt{undefined} is returned. If the
expansion is about infinity, the identifier \texttt{infinity} is
returned.

\textbf{psfunction Operator}
\hypertarget{operator:PSFUNCTION}{}
\ttindextype{psfunction}{operator}
\begin{verbatim}
  psfunction(TPS:power-series-object):algebraic.
\end{verbatim}
The operator \texttt{psfunction} returns the function whose expansion
gave rise to the power series object TPS. TPS should
evaluate to a power series object or an integer, otherwise an error
results.

\textbf{psterm Operator}
\hypertarget{operator:PSTERM}{}
\ttindextype{psterm}{operator}
\begin{verbatim}
  psterm(TPS:power-series-object, 
         NTH:integer):algebraic.
\end{verbatim}
The operator \texttt{psterm} returns the NTH term of the existing
power series object TPS. If NTH does not evaluate to
an integer or TPS to a power series object an error results.  It
should be noted that an integer is treated as a power series.

\textbf{psorder Operator}
\hypertarget{operator:PSORDER}{}
\ttindextype{psorder}{operator}
\begin{verbatim}
  psorder(TPS:power-series-object):integer.
\end{verbatim}
The operator \texttt{psorder} returns the order, that is the degree of
the first non-zero term, of the power series object TPS.
TPS should evaluate to a power series object or an error results. If
TPS is zero, the identifier \texttt{undefined} is returned.

\textbf{pstruncate Operator}
\hypertarget{operator:PSTRUNCATE}{}
\ttindextype{pstruncate}{operator}
\begin{verbatim}
  pstruncate(TPS:power-series-object,
             POWER:integer):algebraic.
\end{verbatim}
This procedure truncates the power series \texttt{TPS} discarding terms
of order higher than \texttt{POWER}. The series is extended automatically
if the value of \texttt{POWER} is greater than the order of last term
calculated to date. For example
\begin{verbatim}
    a := ps(sin x, x, 0);
    pstruncate(a, 11);
\end{verbatim}
will output the eleventh order polynomial resulting in truncating the series
for $sin x$ after the term involving $x^{11}$.

If \texttt{POWER} is less than the order of the series then $0$ is
returned.  If \texttt{POWER} does not simplify to an integer or if
\texttt{TPS} is not a power series object then a Reduce error result.


\subsection{Power Series Reversion}
\hypertarget{operator:PSREVERSE}{}
\ttindextype{psreverse}{operator}
\index{Power series!reversion}
In order to functionally invert a power series the operator \texttt{psreverse}
is used. 
\begin{verbatim}
    psreverse(TPS:power-series-object)
              :power-series-object
\end{verbatim}
Four cases arise:

\begin{enumerate}
\item If the order of the series is 1, then the expansion point of the
inverted series is 0.

\item If the order is 0 \emph{and} if the first order term in TPS
is non-zero, then the expansion point of the inverted series is taken
to be the coefficient of the zeroth order term in TPS.

\item If the order is -1 the expansion point of the inverted series
is the point at infinity.  In all other cases a REDUCE error is
reported because the series cannot be inverted as a power series.
Puiseux \index{Puiseux expansion} expansion would be required to
handle these cases.

\item If the expansion point of TPS is finite it becomes the
zeroth order term in the inverted series. For expansion about 0 or the
point at infinity the order of the inverted series is one.
\end{enumerate}

If TPS is not a power series object after evaluation an error results.

Some examples:
\begin{verbatim}
    ps(sin x,x,0);
    psreverse(ws); % produces series for asin x about x=0.
    ps(exp x,x,0);
    psreverse ws; % produces series for log x about x=1.
    ps(sin(1/x),x,infinity);
    psreverse(ws); % series for 1/asin(x) about x=0.
\end{verbatim}

\subsection{Power Series Composition}
\hypertarget{operator:PSCOMPOSE}{}
\ttindextype{pscompose}{operator}
\index{Power series!composition}
In order to functionally compose two power series the operator
\texttt{pscompose} is used. 
\begin{verbatim}
   pscompose(TPS1:power-series-object,
             TPS2:power-series-object)
             :power-series-object
\end{verbatim}
The power series TPS1 and TPS2 are functionally composed;
that is to say that TPS2 is substituted for the expansion
variable in TPS1 and the result expressed as a power series. The
dependent variable and expansion point of the result coincide with
those of TPS2.  The following conditions apply to power series
composition:

\begin{enumerate}
\item If the expansion point of TPS1 is 0 then the order of the
TPS2 must be at least 1.

\item If the expansion point of TPS1 is finite, it should
coincide with the coefficient of the zeroth order term in TPS2.
The order of TPS2 should also be non-negative in this case.

\item If the expansion point of TPS1 is the point at infinity
then the order of TPS2 must be less than or equal to -1.
\end{enumerate}

If these conditions do not hold the series cannot be composed (with
the current algorithm terms of the inverted series would involve
infinite sums) and a REDUCE error occurs.

Some examples:
\begin{verbatim}
  a:=ps(exp y,y,0);  b:=ps(sin x,x,0);
  pscompose(a,b);
  % Produces the power series expansion of exp(sin x)
  % about x=0.

  a:=ps(exp z,z,1); b:=ps(cos x,x,0);
  pscompose(a,b);
  % Produces the power series expansion of exp(cos x)
  % about x=0.

  a:=ps(cos(1/x),x,infinity);  b:=ps(1/sin x,x,0);
  pscompose(a,b);
  % Produces the power series expansion of cos(sin x)
  % about x=0.
\end{verbatim}

\subsection{pssum Operator}
\hypertarget{operator:PSSUM}{}
\ttindextype{pssum}{operator}
If an expression is known for the nth term of a power series, an extendible
power series object may be constructed by the operator \texttt{pssum}

\begin{verbatim}
   pssum(J:kernel = LOWLIM:integer, 
         COEFF:algebraic, X:kernel, 
         ABOUT:algebraic, POWER:algebraic)
         :power-series-object
\end{verbatim}

The formal power series sum for J from LOWLIM to \texttt{infinity} of
\begin{verbatim}
      COEFF*(X-ABOUT)**POWER
\end{verbatim}
when \texttt{ABOUT} is finite or zero, whereas if ABOUT is \texttt{infinity}
\begin{verbatim}
      COEFF*(1/X)**POWER
\end{verbatim}
is constructed and returned. This enables power series whose general
term is known to be constructed and manipulated using the other
procedures of the power series package.

J and X should be distinct simple kernels. The algebraics
ABOUT,  COEFF and POWER should not depend on the
expansion variable X, similarly the algebraic ABOUT should
not depend on the summation variable J.  The algebraic POWER should be
a strictly increasing integer-valued function of J for J in the range
LOWLIM to \texttt{infinity}.

Some examples:
\begin{verbatim}
   pssum(n=0,1,x,0,n*n);
   % Produces the power series summation for n=0 to
   % infinity of x**(n*n).

   pssum(n=1,n,x,0,n);
   % Produces the power series summation for n=1 to
   % infinity of n*x**n.

   pssum(m=1,(-1)**(m-1)/(2m-1),y,1,2m-1);
   % Produces a power series which is actually the expansion
   % of atan(y-1)  about y=1.

   pssum(j=1,-1/j,x,infinity,j);
   % Produces a power series which is actually the expansion
   % of log(1-1/x) about the point at infinity.

   pssum(n=0,1,x,0,2n**2+3n) + pssum(n=1,1,x,0,2n**2-3n);
   % Produces the power series summation for n=-infinity
   % to +infinity of x**(2n**2+3n).
\end{verbatim}
It should be noted that a formal power series is produced which may not have
a non-zero radius of convergence; the second example above illustrates this.
Nevertheless these formal series may be added, multiplied, differentiated etc.
by the TPS package. Of course, in general the result may also have a zero radius
of convergence.

\subsection{Miscellaneous Operators}

\textbf{pscopy Operator}
\hypertarget{operator:PSCOPY}{}
\ttindextype{pscopy}{operator}
\begin{verbatim}
   pscopy(TPS:power-series-object):power-series-object
\end{verbatim}
This procedure returns a copy of the power series \texttt{TPS}. 
The copy has no shared sub-structures in common with the original
 series.  This enables substitutions to be performed on the series
 without side-effects on previously computed objects. For example:
\begin{verbatim}
    clear a;
    b := ps(sin(a*x)), x, 0);
    b where a => 1;
\end{verbatim}

will result in \texttt{a} being set to 1 in each of the terms of the
power series and the resulting expressions being simplified. Owing to
the way power series objects are implemented using Lisp vectors, this
has the side-effect that the value of \texttt{b} is changed.  This may be
avoided by copying the series with \texttt{ pscopy} before applying the
substitution, thus:
\begin{verbatim}
    b := ps(sin(a*x)), x, 0);
    pscopy b where a => 1;
\end{verbatim}

\textbf{pschangevar Operator}
\hypertarget{operator:PSCHANGEVAR}{}
\ttindextype{pschangevar}{operator}
\begin{verbatim}
    pschangevar(TPS:power-series-object,
                X:kernel):power-series-object
\end{verbatim}
The operator \texttt{pschangevar} changes the dependent variable of the
power series object TPS to the variable X. TPS should evaluate to a power
series object and X to a kernel, otherwise an error results.
Also X should not appear as a parameter in TPS. The power series with the new
dependent variable is returned.

\textbf{psordlim Operator}
\hypertarget{operator:PSORDLIM}{}
\ttindextype{psordlim}{operator}
\begin{verbatim}
   psordlim(UPTO:integer):integer
or
   psordlim():integer
\end{verbatim}
An internal variable is set to the value of \texttt{UPTO} (which should
evaluate to an integer). The value returned is the previous value of
the variable.  The default value is 100.  If \texttt{psordlim} is called
with no argument, the current value is returned.

The significance of this control is that the system attempts to find
the order of the power series required, that is the order is the
degree of the first non-zero term in the power series.  If the order
is greater than the value of this variable an error message is given
and the computation aborts. This prevents infinite loops in certain cases,
for example:
\begin{verbatim}
    a:=ps(1-(cos x)^2,x,0);
    b :=ps((sin x)^2,x,0);
    b-a;
\end{verbatim}
This will also occur in the rather unlikely situation where the expression
being expanded is
\begin{enumerate}
\item identically zero, but is not recognized as such by REDUCE;
\item and its derivatives are not recognized as identically zero by Reduce;
\item  but the values of all derivatives at the expansion point are
simplified to zero by REDUCE.
\end{enumerate}



