
\subsection{Rational Integration}

This package implements the Horowitz/ Rothstein/ Trager algorithms ~\cite{Geddes:92} for the integration of rational functions in \REDUCE. We work within a field $K$ of characteristic $0$ and functions $p,q \in K[x]$. $K$ is normally the field $Q$ of rational numbers, but not always. These procedures return $\int \frac{p}{q} dx.$
 The aim is to be able to integrate any function of the form $p/q$ in $x$, where $p$ and $q$ are polynomials in the field $Q$. The algorithms used avoid algebraic number extensions wherever possible, and in general, express the integral using the minimal algebraic extension field.

\subsubsection{Syntax of \emph{ratint}}
This function has the following syntax:
\begin{center} \textbf{ratint(p,q,var)} \end{center}
where $p$ and $q$ are polynomials in var, so that $p/q$ is a rational function in $var$.
The output of \emph{ratint} is a list of two elements: the first is the polynomial part of the integral, the second is the logarithmic part. The integral is the sum of these parts.
\subsubsection{Examples}
Consider the following examples in \REDUCE (the meaning of the \f{log\_sum} operator will be explained in the next section).
\begin{verbatim}
ratint(1,x^2-2,x);

{0,

                  2    1
 log_sum(beta,beta  - ---,0,log(2*beta*x - 1)*beta)}
                       8

p:=441*x^7+780*x^6-2861*x^5+4085*x^4+7695*x^3+3713*x^2-43253*x
                                                       +24500;

q:=9*x^6+6*x^5-65*x^4+20*x^3+135*x^2-154*x+49;

ratint(p,q,x);


    49   6  226  5   268  4   1332  3   2809  2   752     256
   ---*(x + ---*x  - ---*x  + ----*x  - ----*x  - ---*x + ---)
    2       147      49        49       147       21       9
 {-----------------------------------------------------------      , 0  }
              4    2   3      2          7
             x  - ---*x  - 4*x  + 6*x - ---
                   3                     3



k:=36*x^6+126*x^5+183*x^4+(13807/6)*x^3-407*x^2-(3242/5)*x+(3044/15);
l:=(x^2+(7/6)*x+(1/3))^2*(x-(2/5))^3;

ratint(k,l,x);


   5271    3    39547   2    31018       7142
  ------*(x  + -------*x  - -------*x + -------)
    5           52710        26355       26355
{------------------------------------------------,
       4    11   3    11   2    2        4
      x  + ----*x  - ----*x  - ----*x + ----
            30        25        25       75

  37451            2      91125           2      128000           1
 -------*(log(x - ---) + -------*log(x + ---) - --------*log(x + ---))}
   16              5      37451           3      37451            2



ratint(1,x^2+1,x);

                    2    1
{0,log_sum(beta,beta  + ---,0,log(2*beta*x - 1)*beta)}
                         4
\end{verbatim}

\subsection{The Algorithm}
The following main algorithm is used:

\begin{flushleft}
procedure \textit{ratint}$(p,q,x);$ \\
\% p and q are polynomials in $x$, with coefficients in the \\
\% constant field Q \\
solution\_list $\leftarrow HorowitzReduction(p,q,x)$ \\
$c/d \leftarrow$ part(solution\_list,1) \\
$poly\_part \leftarrow$ part(solution\_list,2) \\
$rat\_part \leftarrow$ part(solution\_list,3) \\
$rat\_part \leftarrow LogarithmicPartIntegral(rat\_part,x)$ \\
return($rat\_part+c/d+poly\_part$) \\
end
\end{flushleft}

The algorithm contains two subroutines, \emph{HorowitzReduction} and \emph{rt}.\ \emph{HorowitzReduction} is an implementation of Horowitz' method to reduce a given rational function into a polynomial part and a logarithmic part. The integration of the polynomial part is a trivial task, and is done by the \emph{int} operator in \REDUCE. The integration of the logarithmic part is done by the routine \emph{rt}, which is an impementation of the Rothstein and Trager method. These two answers are outputed in a list, the complete answer being the sum of these two parts.
\\
These two algorithms are as follows:

\begin{flushleft}
procedure $how(p,q,x)$

for a given rational function $p/q$ in $x$, this algorithm calculates the \\
reduction of $\int(p/q)$ into a polynomial part and logarithmic part.

$poly\_part \leftarrow quo(p,q);$ \hspace{3mm} $p \leftarrow rem(p,q)$;

$d \leftarrow GCD(q,q')$; \hspace{3mm} $b \leftarrow quo(q,d)$; \hspace{3mm}
$m \leftarrow deg(b)$; \\
$n \leftarrow deg(d)$;

$a \leftarrow \sum_{i=1}^{m-1} a_{i}x^{i}$; \hspace{3mm}
   $c \leftarrow \sum_{i=1}^{n-1} c_{i}x^{i}$; \\

$r \leftarrow b*c'-quo(b*d',d)+d*a;$

for $i$ from $0$ to $m+n-1$ do \\
~~~~~~~~~~~~~~~~~\{ \\
~~~~~~~~~~~~~~~~~~~~$eqns(i) \leftarrow coeff(p,i)=coeff(r,i)$; \\
~~~~~~~~~~~~~~~~~\};

$solve(eqns,\{a(0),....,a(m-1),c(0),....,c(n-1)\});$

return($c/d+\int poly\_part + \int a/b$); \\
end;
\end{flushleft}

\begin{flushleft}
procedure RothsteinTrager($a,b,x$)

\% Given a rational function $a/b$ in $x$ with $deg(a)<deg(b)$, \\
with $b$ monic and square free, we calculate $\int(a/b)$ \\
$R(z) \leftarrow residue(a-zb',b)$ \\
$(r_{1}(z)...r_{k}(z)) \leftarrow factors(R(z))$ \\
integral $\leftarrow 0$

for $i$ from $1$ to $k$ do \\
~~~~~~~~~~~~~~~~~~~~\{ \\
~~~~~~~~~~~~~~~~~~~~~$ d \leftarrow degree(r_{i}(z))$ \\
~~~~~~~~~~~~~~~~~~~~~if $d=1$ then \{ \\
~~~~~~~~~~~~~~~~~~~~~~~~~~~~~~~~~~~~~~~~~~~~c $\leftarrow solve(r_{i}(z)=0,z)$ \\
~~~~~~~~~~~~~~~~~~~~~~~~~~~~~~~~~~~~~~~~~~~~v $\leftarrow GCD(a-cb',b)$ \\
~~~~~~~~~~~~~~~~~~~~~~~~~~~~~~~~~~~~~~~~~~~~v $\leftarrow v/lcoeff(v)$ \\

~~~~~~~~~~~~~~~~~~~~~~~~~~~~~~~~~~~~~~~~~~~~$integral \leftarrow integral+c*log(v)$ \\
~~~~~~~~~~~~~~~~~~~~~~~~~~~~~~~~~~~~~~~~~~~~\} \\

~~~~~~~~~~~~~~~~~~~~~else \{ \\
~~~~~~~~~~~~~~~~~~~~~~~~~~~~\%  we need to do a GCD over algebraic number field \\
~~~~~~~~~~~~~~~~~~~~~~~~~~~~v $\leftarrow GCD(a-\alpha*b',b)$ \\
~~~~~~~~~~~~~~~~~~~~~~~~~~~~v $\leftarrow v/lcoff(v) $, \hspace{3 mm}
 where $\alpha=roof\_of(r_{i}(z))$ \\
~~~~~~~~~~~~~~~~~~~~~if $d=2$ then \{ \\
~~~~~~~~~~~~~~~~~~~~~~~~~~~~~~~~~~~~~~~~~~~~\% give answer in terms of radicals  \\
~~~~~~~~~~~~~~~~~~~~~~~~~~~~~~~~~~~~~~~~~~~~c $\leftarrow solve(r_{i}(z)=0,z)$ \\
~~~~~~~~~~~~~~~~~~~~~~~~~~~~~~~~~~~~~~~~~~~~for j from 1 to 2 do \{ \\
~~~~~~~~~~~~~~~~~~~~~~~~~~~~~~~~~~~~~~~~~~~~$v[j] \leftarrow substitute(\alpha=c[j],v)$ \\
~~~~~~~~~~~~~~~~~~~~~~~~~~~~~~~~~~~~~~~~~~~~$integral \leftarrow integral+c[j]*log(v[j])$ \\
~~~~~~~~~~~~~~~~~~~~~~~~~~~~~~~~~~~~~~~~~~~~\} \\
~~~~~~~~~~~~~~~~~~~~~~~~~~~~~~~~~~~~~~~~~~~~else \{ \\
~~~~~~~~~~~~~~~~~~~~~~~~~~~~~~~~~~~~~~~~~~~~\% Need answer in terms of root\_of notation \\
~~~~~~~~~~~~~~~~~~~~~~~~~~~~~~~~~~~~~~~~~~~~for j from 1 to d do \{ \\
~~~~~~~~~~~~~~~~~~~~~~~~~~~~~~~~~~~~~~~~~~~~v[j] $\leftarrow substitute(\alpha=c[j],v)$ \\
~~~~~~~~~~~~~~~~~~~~~~~~~~~~~~~~~~~~~~~~~~~~integral $\leftarrow integral+c[j]*log(v[j])$ \\
~~~~~~~~~~~~~~~~~~~~~~~~~~~~~~~~~~~~~~~~~~~~\% where $c[j]=root\_of(r_{i}(z))$ \} \\
~~~~~~~~~~~~~~~~~~~~~~~~~~~~~~~~~~~~~~~~~~~~\} \\
~~~~~~~~~~~~~~~~~~~~~~~\} \\
~~~~~~~~~~~~~~~~~~~~\} \\
return(integral) \\
end
\end{flushleft}

\subsection{The log\_sum operator}
The algorithms above returns a sum of terms of the form
\[ \sum_{\alpha \mid R(\alpha)=0} \log(S(\alpha,x)), \]

where $R \in K[z]$ is square free, and $S \in K[z,x]$. In the cases where the degree of $R(\alpha)$ is less than two, this is merely a sum of logarithms. For cases where the degree is two or more, I have chosen to adopt this notation as the answer to the original problem of integrating the rational function. For example,
consider the integral
 \[ \int \frac{a}{b}=\int \frac{2x^5-19x^4+60x^3-159+x^2+50x+11}{x^6-13x^5+58x^4-85x^3-66x^2-17x+1}\, dx \]
Calculating the resultant $R(z)=res_x(a-zb',b)$ and factorising gives
  \[ R(z)=-190107645728000(z^3-z^2+z+1)^{2} \]
Making the result monic, we have
\[ R_2(z)=z^3-z^2+z+1 \]
which does not split over the constant field $Q$.
Continuting with the Rothstein Trager algorithm, we now calculate
\[ gcd(a-\alpha\,b',b)=z^2+(2*\alpha-5)*z+\alpha^2, \] where $\alpha$ is a root of $R_2(z)$. \\
Thus we can write
\[ \int \frac{a}{b}= \sum_{\alpha \mid \alpha^3-\alpha^2+\alpha+1=0} \alpha*\log(x^2+2\alpha x-5x+\alpha^2), \]

and this is the answer now returned by \REDUCE, via a function called $log\_sum$. This has the following syntax:
\begin{center}$ log\_sum(\alpha,eqn(\alpha),0,sum\_term,var)$ \end{center}
where $\alpha$ satisfies $eqn=0$, and $sum\_term$ is the term of the summation in the variable $var$. Thus in the above example, we have
\[ \int \frac{a}{b}\,dx= log\_sum(\alpha,\alpha^3-\alpha^2+\alpha+1,0,\alpha*\log(x^2+2\alpha x-5x+\alpha^2),x) \]

Many rational functions that could not be integrated by \REDUCE previously can now be integrated with this package. The above is one example; some more are given on the next page.

\subsubsection{More examples}
\(
\displaystyle \int \frac{1}{x^5+1} \, dx = \frac{1}{5}\log(x + 1)
\)
\[
\mbox{} + 5 log\_sum(\beta,\beta^4+\frac{1}{5}\beta^3+\frac{1}{25}\beta^2+\frac{1}{125}\beta+\frac{1}{625},0,\log(5*\beta+x)*\beta)
\]

which should be read as
\[
\int \frac{1}{x^5+1}\,dx = \frac{1}{5}\log(x+1)+\sum_{\beta \mid\beta^4+\frac{1}{5}\beta^3+\frac{1}{25}\beta^2+\frac{1}{125}\beta+\frac{1}{625}=0}\log(5*\beta+x)\beta
\]

\(
\displaystyle
\int \frac{7x^{13}+10x^8+4x^7-7x^6-4x^3-4x^2+3x+3}{x^{14}-2x^8-2x^7-2x^4-4x^3-x^2+2x+1} \, dx  =
\)
\[
log\_sum(\alpha,\alpha^2  -\alpha -\frac{1}{4},0,\log( - 2\alpha x^2  - 2\alpha x + x^7  + x^2  - 1)*\alpha,x) ,
\]
\[ \int \frac{1}{x^3+x+1} \, dx = log\_sum(\beta,\beta^3-\frac{3}{31}\beta^2-\frac{1}{31},0,\beta \log(-\frac{62}{9}\beta^2+\frac{31}{9} \beta +x+\frac{4}{9})). \]

\subsection{Options}
There are several alternative forms that the answer to the integration problem can take. One output is the $log\_sum$ form shown in the examples above. There is an option with this package to convert this to a ``normal'' sum of logarithms in the case when the degree of $eqn$ in $\alpha$ is two, and $\alpha$ can be expressed in surds. To do this, use the function \emph{convert}, which has the following syntax:
\begin{center}  convert(exp) \end{center}
If exp is free of $log\_sum$ terms, then $\exp$ itself is returned. If $\exp$ contains $log\_sum$ terms, then $\alpha$ is represented as surds, and substituted into the $log\_sum$ expression. For example, using the last example, we have in \REDUCE:
\begin{verbatim}
2: ratint(a,b,x);


  {0,

                    2            1
  log_sum(alpha,alpha  - alpha - ---,0,
                                 4

                         2                7    2
         log( - 2*alpha*x  - 2*alpha*x + x  + x  - 1)*alpha,x)}



3: convert(ws);


 1                           2                7
---*(sqrt(2)*log( - sqrt(2)*x  - sqrt(2)*x + x  - x - 1)
 2

                             2                7
      - sqrt(2)*log(sqrt(2)*x  + sqrt(2)*x + x  - x - 1)

                        2                7
      + log( - sqrt(2)*x  - sqrt(2)*x + x  - x - 1)

                     2                7
      + log(sqrt(2)*x  + sqrt(2)*x + x  - x - 1))
\end{verbatim}

\subsubsection{LogtoAtan function}
The user could then combine these to form a more elegant answer, using the switch combinelogs if one so wished. Another option is to convert complex logarithms to real arctangents ~\cite{Bronstein:97}, which is recommended if definite integration is the goal. This is implemented in \REDUCE via a function \emph{convert\_log}, which has the following syntax:
\begin{center} \textbf{convert\_log(exp)}, \end{center}
where $\exp$ is any $log\_sum$ expression.

The procedure to convert complex logarithms to real arctangents is based on an algorithm by Rioboo. Here is what it does:

Given a field $K$ of characteristic 0 such that $\sqrt{-1} \not\in K$ and
$A, B \in K[x]$ with $B \not = 0$, return a sum $f$ of arctangents of polynomials in $K[x]$ such that
\[  \frac{df}{dx}=\frac{d}{dx} i \log(\frac{A+ i B}{A- i B}) \]

Example:
\[ \int \frac{x^4-3*x^2+6}{x^6-5*x^4+5*x^2+4} \, dx = \sum_{ \alpha \mid 4\alpha+1=0} \alpha \log(x^3+2\alpha x^2-3 x-4 \alpha) \]

Substituting $\alpha=i/2$ and $\alpha=-i/2$ gives the result

\[ \frac{i}{2} \log(\frac{(x^3-3 x)+i (x^2-2)}{(x^3-3 x)-i (x^2-2)}) \]

Applying logtoAtan now with $A=x^3-3 x$, and $B=x^2-2$ we obtain

\[ \int \frac{x^4-3*x^2+6}{x^6-5*x^4+5*x^2+4} \, dx = \arctan(\frac{x^5-3 x^3+x}{2})+\arctan(x^3)+\arctan(x) , \]

and this is the formula which should be used for definite integration.

Another example in \REDUCE is given below:
\begin{verbatim}
1: ratint(1,x^2+1,x);

*** Domain mode rational changed to arnum

                    2    1
{0,log_sum(beta,beta  + ---,0,log(2*beta*x - 1)*beta)}
                         4

13: part(ws,2);

                 2    1
log_sum(beta,beta  + ---,0,log(2*beta*x - 1)*beta)
                      4

14: on combinelogs;

15: convertlog(ws);

 1        - i*x + 1
---*log(------------)*i
 2         i*x + 1


logtoAtan(-x,1,x);

2*atan(x)
\end{verbatim}

\subsection{Hermite's method}
The package also implements Hermite's method to reduce the integral into its polynomial and logarithmic parts, but occasionally, \REDUCE returns the incorrect answer when this algorithm is used. This is due to the REDUCE operator pf, which performs a complete partial fraction expansion when given a rational function as input. Work is presently being done to give the pf operator a facility which tells it that the input is already factored. This would then enable REDUCE to perform a partial fraction decomposition with respect to a square free denominator, which may not necessarily be fully factored over Q.

For a complete explanation of this and the other algorithms used in this package, including the theoretical justification and proofs, please consult ~\cite{Geddes:92}.

\subsection{Tracing the \emph{ratint} program}
The package includes a facility to trace in some detail the inner workings of the \emph{ratint} program. Messages are given at the key stages of the algorithm, together with the results obtained. These messages are displayed when the switch \emph{traceratint} is on, which is done in \REDUCE with the command
\begin{verbatim}
on traceratint;
\end{verbatim}
This switch is off by default. Here is an example of the output obtained with this switch on:

\begin{verbatim}
Loading image file: /silo/tony/red/lisp/psl/solaris/red/reduce.img
REDUCE Development Version, 21-May-97 ...

1: load_package ratint;

2: on traceratint;


3: ratint(1+x,x^2-2*x+1,x);

                                     x + 1
performing Howoritz reduction on --------------
                                   2
                                  x  - 2*x + 1

                   - 2        1
Howoritz gives: {-------,0,-------}
                  x - 1     x - 1

                                 1
computing Rothstein Trager on -------
                               x - 1

integral in Rothstein T is log(x - 1)

   - 2
{-------,log(x - 1)}
  x - 1
\end{verbatim}

\subsection{Bugs, suggestions and comments}
This package was written when the author was working as a placement student at ZIB Berlin.
% All comments should therefore be reported to Winfried Neun, ZIB, Takustrasse 7, D 14195 Berlin Dahlem, Germany \\ (email: neun@zib.de).
