\chapter{Calculations in High Energy Physics}

A set of {\REDUCE} commands is provided for users interested in symbolic
calculations in high energy physics. Several extensions to our basic
syntax are necessary, however, to allow for the different data structures
encountered.

\section{High Energy Physics Operators}
\label{HEPHYS}
\hypertarget{command:INDEX}{}

We begin by introducing three new operators required in these calculations.

\subsection{. (Cons) Operator}\index{Dot product}
Syntax:
\begin{verbatim}
        (exprn1:vector_expression)
                 . (exprn2:vector_expression):algebraic.
\end{verbatim}
The binary \texttt{.} operator, which is normally used to denote the addition
of an element to the front of a list, can also be used in algebraic mode
to denote the scalar product of two Lorentz four-vectors.  For this to
happen, the second argument must be recognizable as a vector expression
\index{High energy vector expression} at the time of
evaluation.  With this meaning, this operator is often referred to as the
\emph{dot} operator.  In the present system, the index handling routines all
assume that Lorentz four-vectors are used, but these routines could be
rewritten to handle other cases.

Components of vectors can be represented by including representations of
unit vectors in the system.  Thus if \texttt{eo} represents the unit vector
\texttt{(1,0,0,0)}, \texttt{(p.eo)} represents the zeroth component of the
four-vector P.  Our metric and notation follows Bjorken and Drell \cite{BjorkenDrell:1965}.
Similarly, an arbitrary component \texttt{p} may be represented by
\texttt{(p.u)}.  If contraction over components of vectors is required, then
the declaration \texttt{index}\ttindextype{index}{declaration} must be used.  Thus
\begin{verbatim}
        index u;
\end{verbatim}
declares \texttt{u} as an index, and the simplification of
\begin{verbatim}
        p.u * q.u
\end{verbatim}
would result in
\begin{verbatim}
        P.Q
\end{verbatim}
The metric tensor $g^{\mu \nu}$ may be represented by \texttt{(u.v)}.  If
contraction over \texttt{u} and \texttt{v} is required, then they should be
declared as indices.

Errors occur if indices are not properly matched in expressions.

\hypertarget{command:REMIND}{}
If a user later wishes to remove the index property from specific vectors,
he can do it with the declaration \texttt{remind}.\ttindextype{remind}{declaration} Thus
\texttt{remind v1,\ldots{},vn;} removes the index flags from the variables 
\var{V1}
through \var{Vn}.  However, these variables remain vectors in the system.

\subsection{G Operator for Gamma Matrices}\index{Dirac $\gamma$ matrix}
\ttindextype{g}{operator}
\hypertarget{operator:G}{}

Syntax:
\begin{verbatim}
        g(id:identifier[,exprn:vector_expression])
                :gamma_matrix_expression.
\end{verbatim}
\texttt{g} is an n-ary operator used to denote a product of $\gamma$ matrices
contracted with Lorentz four-vectors. Gamma matrices are associated with
fermion lines in a Feynman diagram. If more than one such line occurs,
then a different set of $\gamma$ matrices (operating in independent spin
spaces) is required to represent each line. To facilitate this, the first
argument of \texttt{g} is a line identification identifier (not a number)
used to distinguish different lines.

Thus
\begin{verbatim}
        g(l1,p) * g(l2,q)
\end{verbatim}
denotes the product of \texttt{$\gamma$.p} associated with a fermion line
identified as \texttt{l1}, and \texttt{$\gamma$.q} associated with another line
identified as \texttt{l2} and where \texttt{p} and \texttt{q} are Lorentz
four-vectors.  A product of $\gamma$ matrices associated with the same
line may be written in a contracted form.

Thus
\begin{verbatim}
        g(l1,p1,p2,...,p3) = g(l1,p1)*g(l1,p2)*...*g(l1,p3) .
\end{verbatim}
The vector \texttt{a} is reserved in arguments of G to denote the special
$\gamma$ matrix $\gamma^{5}$. Thus
\begin{quote}
\begin{tabular}{lll}
 \texttt{g(l,a)} & $ = \gamma^{5}$ & associated with the line \texttt{l} \\[0.1in]
 \texttt{g(l,p,a)} & $ = \gamma\cdot p \times \gamma^{5}$ & 
           associated with the line \texttt{l}.
\end{tabular}
\end{quote}
$\gamma^{\mu}$ (associated with the line \texttt{l}) may be written as
\texttt{g(l,u)}, with \texttt{u} flagged as an index if contraction over \texttt{u}
is required.

The notation of Bjorken and Drell is assumed in all operations involving
$\gamma$ matrices.

\subsection{EPS Operator}\ttindextype{eps}{operator}
\hypertarget{operator:EPS}{}
Syntax:
\begin{verbatim}
         eps(exprn1:vector_expression,...,exprn4:vector_exp)
            :vector_exp.
\end{verbatim}
The operator \texttt{eps} has four arguments, and is used only to denote the
completely antisymmetric tensor of order 4 and its contraction with Lorentz
four-vectors. Thus
\[ \epsilon_{i j k l} = \left\{ \begin{array}{cl}
                                +1 & \mbox{if $i,j,k,l$ is an even permutation
                                              of 0,1,2,3} \\
                                -1 & \mbox{if $i,j,k,l$ is an odd permutation
                                              of 0,1,2,3} \\
                                0 & \mbox{otherwise}
                              \end{array}
                      \right. \]

A contraction of the form $\epsilon_{i j \mu \nu}p_{\mu}q_{\nu}$ may be
written as \texttt{eps(i,j,p,q)}, with \texttt{i} and \texttt{j} flagged as indices,
and so on.

\section{Vector Variables}
\hypertarget{command:VECTOR}{}

Apart from the line identification identifier in the \texttt{g} operator, all
other arguments of the operators in this section are vectors.  Variables
used as such must be declared so by the type declaration \texttt{vector},
\ttindextype{vector}{declaration} for example:
\begin{verbatim}
        vector  p1,p2;
\end{verbatim}
declares \texttt{p1} and \texttt{p2} to be vectors.  Variables declared as
indices or given a mass\ttindextype{mass}{declaration} are automatically declared
vector by these declarations.

\section{Additional Expression Types}

Two additional expression types are necessary for high energy
calculations, namely

\subsection{Vector Expressions}\index{High energy vector expression}

These follow the normal rules of vector combination. Thus the product of a
scalar or numerical expression and a vector expression is a vector, as are
the sum and difference of vector expressions. If these rules are not
followed, error messages are printed. Furthermore, if the system finds an
undeclared variable where it expects a vector variable, it will ask the
user in interactive mode whether to make that variable a vector or not. In
batch mode, the declaration will be made automatically and the user
informed of this by a message.

\textit{Examples:}

Assuming \texttt{p} and \texttt{q} have been declared vectors, the following are
vector expressions
\begin{verbatim}
        p
        2*q/3
        2*x*y*p - p.q*q/(3*q.q)
\end{verbatim}
whereas \texttt{p*q} and \texttt{p/q} are not.

\subsection{Dirac Expressions}

These denote those expressions which involve $\gamma$ matrices. A $\gamma$
matrix is implicitly a 4 $\times$ 4 matrix, and so the product, sum and
difference of such expressions, or the product of a scalar and Dirac
expression is again a Dirac expression.  There are no Dirac variables in
the system, so whenever a scalar variable appears in a Dirac expression
without an associated $\gamma$ matrix expression, an implicit unit 4
by 4 matrix is assumed.  For example, \texttt{g(l,p) + m} denotes 
\texttt{g(l,p) + m*\meta{unit 4 by 4 matrix}}.  Multiplication of Dirac
expressions, as for matrix expressions, is of course non-commutative.

\section{Trace Calculations}\index{High energy trace}
\hypertarget{command:NOSPUR}{}
\hypertarget{command:SPUR}{}

When a Dirac expression is evaluated, the system computes one quarter of
the trace of each $\gamma$ matrix product in the expansion of the expression.
One quarter of each trace is taken in order to avoid confusion between the
trace of the scalar \texttt{m}, say, and \texttt{m} representing
\texttt{m * \meta{unit 4 by 4 matrix}}.  
Contraction over indices occurring in such expressions is
also performed.  If an unmatched index is found in such an expression, an
error occurs.

The algorithms used for trace calculations are the best available at the
time this system was produced. For example, in addition to the algorithm
developed by Chisholm for contracting indices in products of traces,
{\REDUCE} uses the elegant algorithm of Kahane for contracting indices in
$\gamma$ matrix products.  These algorithms are described in \cite{Chisholm1963}
and \cite{Kahane:1968}.

It is possible to prevent the trace calculation over any line identifier
by the declaration \texttt{nospur}.\ttindextype{nospur}{declaration}  For example,
\begin{verbatim}
        nospur l1,l2;
\end{verbatim}
will mean that no traces are taken of $\gamma$ matrix terms involving the line
numbers \texttt{l1} and \texttt{l2}.  However, in some calculations involving
more than one line, a catastrophic error
\begin{verbatim}
        NOSPUR on more than one line not implemented
\end{verbatim}
can occur (for the reason stated!) If you encounter this error, please let
us know!

A trace of a $\gamma$ matrix expression involving a line identifier which has
been declared \texttt{nospur} may be later taken by making the declaration
\texttt{spur}.\ttindextype{spur}{declaration}

See also the \package{CVIT} package for an alternative
mechanism\extendedmanual{ (section~\ref{sec:CVIT})}.

\section{Mass Declarations}\ttindextype{mass}{declaration}
\hypertarget{command:MASS}{}
\hypertarget{command:MSHELL}{}

It is often necessary to put a particle ``on the mass shell'' in a
calculation.  This can, of course, be accomplished with a \texttt{let}
command such as
\begin{verbatim}
        let p.p= m^2;
\end{verbatim}
but an alternative method is provided by two commands \texttt{mass} and
\texttt{mshell}.\ttindextype{mshell}{command}
\texttt{mass} takes a list of equations of the form:
\begin{syntax}
  \meta{vector variable}\texttt{ = }\meta{scalar variable}
\end{syntax}
for example,
\begin{verbatim}
        mass p1=m, q1=mu;
\end{verbatim}
The only effect of this command is to associate the relevant scalar
variable as a mass with the corresponding vector. If we now say
\begin{syntax}
  \texttt{mshell }\meta{vector variable}\texttt{,}\,\dots\texttt{,}\,\meta{vector variable}\meta{terminator}
\end{syntax}
and a mass has been associated with these arguments, a substitution of the
form
\begin{syntax}
  \meta{vector variable}\,\texttt{.}\,\meta{vector variable}\texttt{ = }%
    \meta{mass}\texttt{\textasciicircum}2
\end{syntax}
is set up. An error results if the variable has no preassigned mass.

\section{Example}

We give here as an example of a simple calculation in high energy physics
the computation of the Compton scattering cross-section as given in
Bjorken and Drell Eqs. (7.72) through (7.74). We wish to compute the trace of
\[
\frac{\alpha^2}{2} \left(\frac{k^\prime}{k}\right)^2
 \left(\frac{\gamma\cdot p_f+m}{2m}\right)\left(\frac{\gamma\cdot e^\prime \gamma\cdot e
 \gamma\cdot k_i}{2k.p_i} + \frac{\gamma\cdot e\gamma\cdot e^\prime
 \gamma\cdot k_f}{2k^\prime\cdot p_i}\right)
\]
\[
 \left(\frac{\gamma\cdot p_i+m}{2m}\right)
 \left(\frac{\gamma\cdot k_i\gamma\cdot e\gamma\cdot e^\prime}{2k.p_i} +
 \frac{\gamma\cdot k_f\gamma\cdot e^\prime\gamma\cdot e}{2k^\prime\cdot p_i}
 \right)
\]
where $k_i$ and $k_f$ are the four-momenta of incoming and outgoing photons
(with polarization vectors $e$ and $e^\prime$ and laboratory energies 
$k$ and $k^\prime$
respectively) and $p_i$, $p_f$ are incident and final electron four-momenta.

Omitting therefore an overall factor
$\displaystyle\frac{\alpha^2}{2m^2}\left(\frac{k^\prime}{k}\right)^2$ we need to find one quarter of the trace of
\[
 \left( \gamma\cdot p_f + m\right)
 \left(\frac{\gamma\cdot e^\prime \gamma\cdot e\gamma\cdot k_i}{2k.p_i} +
  \frac{\gamma\cdot e\gamma\cdot e^\prime \gamma\cdot k_f}{2k^\prime.p_i}\right)\times
\]
\[
 \qquad\left(
  \gamma\cdot p_i + m\right)
 \left(\frac{\gamma\cdot k_i\gamma\cdot e\gamma\cdot e^\prime}{2k.p_i} +
  \frac{\gamma\cdot k_f\gamma\cdot e^\prime \gamma\cdot e}{2k^\prime.p_i}\right) 
\]
A straightforward \REDUCE{} program for this, with appropriate substitutions
(using \texttt{p1} for $p_i$, \texttt{pf} for $p_f$, \texttt{ki}
for $k_i$ and \texttt{kf} for $k_f$) is
\begin{verbatim}
 on div; % this gives output in same form as Bjorken and Drell.
 mass ki= 0, kf= 0, p1= m, pf= m; vector e,ep;
 % if e is used as a vector, it loses its scalar identity
 %      as the base of natural logarithms.
 mshell ki,kf,p1,pf;
 let p1.e= 0, p1.ep= 0, p1.pf= m^2+ki.kf, p1.ki= m*k,p1.kf=
     m*kp, pf.e= -kf.e, pf.ep= ki.ep, pf.ki= m*kp, pf.kf=
     m*k, ki.e= 0, ki.kf= m*(k-kp), kf.ep= 0, e.e= -1,
     ep.ep=-1;
 operator gp;
 for all p let gp(p)= g(l,p)+m;
 comment this is just to save us a lot of writing;
 gp(pf)*(g(l,ep,e,ki)/(2*ki.p1) + g(l,e,ep,kf)/(2*kf.p1))
   * gp(p1)*(g(l,ki,e,ep)/(2*ki.p1) + g(l,kf,ep,e)/
     (2*kf.p1))$
 write "The Compton cxn is ",ws;
\end{verbatim}

(We use \texttt{p1} instead of \texttt{pi} in the above to avoid confusion with
the reserved variable \texttt{pi}).

This program will print the following result
\begin{verbatim}
                         2    1      -1    1   -1
The Compton cxn is 2*e.ep  + ---*k*kp   + ---*k  *kp - 1
                              2            2
\end{verbatim}

\section{Extensions to More Than Four Dimensions}
\hypertarget{command:VECDIM}{}

In our discussion so far, we have assumed that we are working in the
normal four dimensions of QED calculations. However, in most cases, the
programs will also work in an arbitrary number of dimensions. The command
\ttindextype{vecdim}{command}
\begin{syntax}
  \texttt{vecdim }\meta{expression}\meta{terminator}
\end{syntax}
sets the appropriate dimension. The dimension can be symbolic as well as
numerical. Users should note however, that the \texttt{eps} operator and the
$\gamma_{5}$ symbol (\texttt{a}) are not properly defined in other than four
dimensions and will lead to an error if used.

\section{The CVIT algorithm}
\hypertarget{switch:CVIT}{}

An alternative algorithm for computing traces of products of gamma matrices is available,
based on treating of gamma-matrices as 3-j symbols (details may be found in
\cite{Ilyin:89,Kennedy:1982}).

This alternative algorithm is used when the switch \sw{cvit} is set to on. With \sw{cvit} off,
calculations of Diracs matrices traces are performed using standard \REDUCE{} facilities.

For more information see section~\ref{sec:CVIT}.
\endinput

\newpage

\section{The CVIT package}
\indexpackage{CVIT}
\label{sec:CVIT}
This package provides an alternative method for computing traces of Dirac
gamma matrices, based on an algorithm by Cvitanovich that treats gamma
matrices as 3-j symbols.

Authors: V.Ilyin, A.Kryukov, A.Rodionov, A.Taranov.
\index{Ilyin, V.}\index{People!Ilyin, V.}
\index{Kryukov, A.}\index{People!Kryukov, A.}
\index{Rodionov, A.}\index{People!Rodionov, A.}
\index{Taranov, A.}\index{People!Taranov, A.}

In modern high energy physics the calculation of Feynman diagrams are
still very important. One of the difficulties of these calculations
are trace calculations. So the calculation of traces of Dirac's
$\gamma$-matrices were one of first task of computer algebra systems.
All available algorithms are based on the fact that gamma-matrices
constitute a basis of a Clifford algebra:
\[
  \left\{G_{m},G_{n}\right\} = 2g_{mn}.
\]

We present the implementation of an alternative algorithm based on
treating of gamma-matrices as 3-j symbols (details may be found in
\cite{Ilyin:89,Kennedy:1982}).

The program consists of 5 modules described below.

\newpage
\begin{verbatim}

                    MODULES CROSS REFERENCES
   +--------+
   | REDUCE |
   |________|                  |ISIMP1
    ISIMP2|         +-----------------------+
          +--->-----| RED_TO_CVIT_INTERFACE |
                    |_______________________|
                 CALC_SPUR|          |REPLACE_BY_VECTOR
                          |          |REPLACE_BY_VECTORP
                          |          |GAMMA5P
                          ^          V
                         +--------------+
                         | CVITMAPPING  |
                         |______________|
                                ^
                                |PRE-CALC-MAP
                                |CALC_MAP_TAR
                                |CALC_DENTAR
                                |
                         +-------------+
                         | INTERFIERZ  |
                         |_____________|
                            |         |MK-NUMR
                            |         |STRAND-ALG-TOP
                            |         ^
               MAP-TO-STRAND|    +------------+
                   INCIDENT1|    | EVAL-MAPS  |
                            |    |____________|
                            ^               |DELETEZ1
                            |               |CONTRACT-STRAND
                  +----------------+        |COLOR-STRAND
                  | MAP-TO-STRAND  |---->---+
                  |________________|


\end{verbatim}


\subsection*{Module RED\_TO\_CVIT\_INTERFACE}

\begin{center}
Author: A.P.Kryukov \\
Purpose:interface REDUCE and CVIT package
\end{center}

RED\_TO\_CVIT\_INTERFACE module is intended for connection of REDUCE
with main module of CVIT package. The main idea is to preserve
standard REDUCE syntax for high energy calculations.  For realization
of this we redefine {\ SYMBOLIC PROCEDURE ISIMP1} from HEPhys module of
REDUCE system.

After loading CVIT package user may use switch CVIT which is {\tt ON} by
default.  If switch CVIT is {\tt OFF} then calculations of Diracs matrices
traces are performed using standard REDUCE facilities. If CVIT switch
is {\tt ON} then CVIT package will be active.

{\tt RED\_TO\_CVIT\_INTERFACE} module performs some primitive simplification
and control input data independently.  For example it remove $G_mG_m$,
check parity of the number of Dirac matrices in each trace \emph{etc}.
There is one principal restriction concerning G5-matrix. There are no
closed form for trace in non-integer dimension case when trace include
G5-matrix.  The next restriction is that if the space-time dimension
is integer then it must be even (2,4,6,...).  If these and other
restrictions are violated then the user get corresponding error
message. List of messages is included.

\begin{center}
\begin{verbatim}
                  LIST OF IMPORTED FUNCTIONS
-------------------------------------------------
 Function              From module
-------------------------------------------------
 ISIMP2                HEPhys
 CALC_SPUR             CVITMAPPING
-------------------------------------------------
\end{verbatim}

\begin{verbatim}
                  LIST OF EXPORTED FUNCTION
-------------------------------------------------
 Function              To module
-------------------------------------------------
 ISIMP1                HEPhys (redefine)
 REPLACE_BY_VECTOR     EVAL_MAP
 REPLACE_BY_VECTORP    EVAL__MAP
 GAMMA5P               CVITMAPPING, EVAL_MAP
-------------------------------------------------
\end{verbatim}
\end{center}



\subsection*{Module CVITMAPPING}

\begin{center}
Author: A.Ya.Rodionov \\
Purpose: graphs reduction
\end{center}

CVITMAPPING module is intended for diagrams calculation according to
Cvitanovic - Kennedy algorithm. The top function of this module
CALC\_SPUR is called from RED\_TO\_CVIT\_INTERFACE interface module.
The main idea of the algorithm consists in diagram simplification
according to rules (1.9') and (1.14) from [1].  The input data - trace
of Diracs gamma matrices (G-matrices) has a form of a list of
identifiers lists with cyclic order. Some of identifiers may be
identical.  In this case we assume summation over dummy indices. So
trace Sp(GbGr).Sp(GwGbGcGwGcGr) is represented as list ((b r) (w b c w
c r)).

The first step is to transform the input data to ``map'' structure and
then to reduce the map to a ``simple'' one. This transformation is made
by function TRANSFORM\_MAP\_ (top function). Transformation is made in
three steps. At the first step the input data are transformed to the
internal form - a map (by function PREPARE\_MAP\_). At the second step
a map is subjected to Fierz transformations (1.14) (function
MK\_SIMPLE\_MAP\_). At this step of optimization can be maid (if
switch CVITOP is on) by function MK\_FIRZ\_OP.  In this case Fierzing
starts with linked vertices with minimal distance (number of vertices)
between them.  After Fierz transformations map is further reduced by
vertex simplification routine MK\_SIMPLE\_VERTEX using (1.9').
Vertices reduced to primitive ones, that is to vertices with three or
less edges.  This is the last (third) step in transformation from
input to internal data.

The next step is optional.  If switch CVITBTR is on factorisation of
bubble (function FIND\_BUBBLES1) and triangle (function
FIND\_TRIANGLES1) submaps is made.  This factorisation is very
efficient for ``wheel'' diagrams and unnecessary for ``lattice'' diagrams.
Factorisation is made recursively by substituting composed edges for
bubbles and composed vertices for triangles.  So check (function
SORT\_ATLAS) must be done to test possibility of future marking
procedure.  If the check fails then a new attempt to reorganize atlas
(so we call complicated structure witch consists of MAP, COEFFicient
and DENOMinator) is made. This cause backtracking (but very seldom).
Backtracking can be traced by turning on switch CVITRACE. FIND\_BUBLTR
is the top function of this program's branch.

Then atlases must be prepared (top function WORLD\_FROM\_ATLAS) for
final algebraic calculations.  The resulted object called ``world''
consists of edges names list (EDGELIST), their marking variants
(VARIANTS) and WORLD1 structure. WORLD1 structure differs from WORLD
structure in one point.  It contains MAP2 structure instead of MAP
structure. MAP2 is very complicated structure and consist of VARIANTS,
marking plan and GSTRAND.  (GSTRAND constructed by PRE!-CALC!-MAP\_
from INTERFIERZ module.)  By marking we understand marking of edges
with numbers according to Cvitanovic - Kennedy algorithm.

The last step is performed by function CALC\_WORLD. At this step
algebraic calculations are done.  Two functions CALC\_MAP\_TAR and
CALC\_DENTAR from INTERFIERZ module make algebraic expressions in the
prefix form. This expressions are further simplified by function
{\tt REVAL}.  This is the REDUCE system general function for algebraic
expressions simplification. {\tt REVAL} and {\tt SIMP!*} are the only REDUCE
functions used in this module.

There are also some functions for printing several internal
structures: PRINT\_ATLAS, PRINT\_VERTEX, PRINT\_EDGE, PRINT\_COEFF,
PRINT\_DENOM.  This functions can be used for debugging.

If an error occur in module CVITMAPPING the error message ``ERROR IN
MAP CREATING ROUTINES'' is displayed.  Error has number 55.  The switch
CVITERROR allows to give full information about error: name of
function where error occurs and names and values of function's
arguments. If CVITERROR switch is on and backtracking fails message
about error in SORT\_ATLAS function is printed.  The result of
computation however will be correct because in this case factorized
structure is not used. This happens extremely seldom.


\begin{verbatim}
                  List of imported function
-------------------------------------------------
 function              from module
-------------------------------------------------
 REVAL                 REDUCE
 SIMP!*                REDUCE
 CALC_MAP_TAR          INTERFIERZ
 CALC_DENTAR           INTERFIERZ
 PRE!-CALC!-MAP_       INTERFIERZ
 GAMMA5P               RED_TO_CVIT_INTERFACE
-------------------------------------------------
\end{verbatim}

\begin{verbatim}
                  List of exported function
-------------------------------------------------
 function              to module
-------------------------------------------------
 CALC_SPUR             REDUCE - CVIT interface
-------------------------------------------------
\end{verbatim}

\begin{verbatim}
                        Data structure
 WORLD     ::=  (EDGELIST,VARIANTS,WORLD1)
 WORLD1    ::=  (MAP2,COEFF,DENOM)
 MAP2      ::=  (MAPS,VARIANTS,PLAN)
 MAPS      ::=  (EDGEPAIR . GSTRAND)
 MAP1      ::=  (EDGEPAIR . MAP)
 MAP       ::=  list of VERTICES (unordered)
 EDGEPAIR  ::=  (OLDEDGELIST . NEWEDGELIST)
 COEFF     ::=  list of WORLDS (unordered)
 ATLAS     ::=  (MAP,COEFF,DENOM)
 GSTRAND   ::=  (STRAND*,MAP,TADPOLES,DELTAS)
 VERTEX    ::=  list of EDGEs (with cyclic order)
 EDGE      ::=  (NAME,PROPERTY,TYPE)
 NAME      ::=  ATOM
 PROPERTY  ::=  (FIRSTPAIR . SECONDPAIR)
 TYPE      ::=  T or NIL
 ------------------------------------------------
 *Define in module MAP!-TO!-STRAND.

\end{verbatim}

\subsection*{Modules INTERFIERZ, EVAL\_MAPS, AND MAP-TO-STRAND}

\begin{center}
Author: A.Taranov \\
Purpose: evaluate single Map
\end{center}

Module INTERFIERZ exports to module CVITMAPPING three functions:
PRE-CALC-MAP\_, CALC-MAP\_TAR, CALC-DENTAR.

Function PRE-CALC-MAP\_ is used for preliminary processing of a map. It
returns a list of the form (STRAND NEWMAP TADEPOLES DELTAS) where
STRAND is strand structure described in MAP-TO-STRAND module.  NEWMAP
is a map structure without ``tadepoles'' and ``deltas''.  ``Tadepole'' is a
loop connected with map with only one line (edge). ``Delta'' is a single
line disconnected from a map.  TADEPOLES is a list of ``tadepole''
submaps.  DELTAS is a list (CONS E1 E2) where E1 and E2 are

Function CALC\_MAP\_TAR takes a list of the same form as returned by
PRE-CALC-MAP\_, a-list, of the form (...  edge . weight ...  )  and
returns a prefix form of algebraic expression corresponding to the map
numerator.

Function CALC-DENTAR returns a prefix form of algebraic expression
corresponding to the map denominator.

Module EVAL-MAP exports to module INTERFIERZ functions MK-NUMR and
STRAND-ALG-TOP.

Function MK-NUMR returns a prefix form for some combinatorial
coefficient (Pohgammer symbol).

Function STRAND-ALG-TOP performs an actual computation of a prefix
form of algebraic expression corresponding to the map numerator. This
computation is based on a ``strand'' structure constructed from the
``map'' structure.

Module MAP-TO-STRAND exports functions MAP-TO-STRAND, INCIDENT1 to
module INTERFIERZ and functions DELETEZ1, CONTRACT-STRAND,
COLOR-STRAND to module EVAL-MAPS.

Function INCIDENT1 is a selector in ``strand'' structure.  DELETEZ1
performs auxiliary optimization of ``strand''.  MAP-TO-STRAND transforms
``map'' to ``strand'' structure.  The latter is describe in program
module.

CONTRACT-STRAND do strand vertex simplifications of ``strand'' and
COLOR-STRAND finishes strand generation.

\begin{verbatim}

            Description of STRAND  data structure.
 STRAND ::=<LIST OF VERTEX>
 VERTEX ::=<NAME> . (<LIST OF ROAD> <LIST OF ROAD>)
 ROAD   ::=<ID> . NUMBER
 NAME   ::=NUMBER
\end{verbatim}


\subsubsection*{LIST OF MESSAGES}

\begin{itemize}

\item{CALC\_SPUR: $<$vecdim$>$ IS NOT EVEN SPACE-TIME DIMENSION}
 The dimension of space-time $<$vecdim$ $is integer but not
even. Only even numeric dimensions are allowed.

\item{NOSPUR NOT YET IMPLEMENTED}
 Attempt to calculate trace when NOSPUR switch is on.  This facility
is not implemented now.

\item{G5 INVALID FOR VECDIM NEQ 4}
 Attempt to calculate trace with gamma5-matrix for space-time
dimension not equal to 4.

\item{CALC\_SPUR: $<$expr$>$ HAS NON-UNIT DENOMINATOR}
The <expr> has non-unit denominator.

\item{THREE INDICES HAVE NAME $<$name$>$}
 There are three indices with equal names in evaluated expression.

\end{itemize}

\begin{verbatim}
                  List of switches
------------------------------------------------------------
 switch           default    comment
------------------------------------------------------------
 CVIT             ON         If it is on then use Kennedy-
                             Cvitanovic algorithm else use
                             standard facilities.
 CVITOP           OFF        Fierz optimization switch
 CVITBTR          ON         Bubbles and triangles
                             factorisation switch
 CVITRACE         OFF        Backtracking tracing switch
------------------------------------------------------------
\end{verbatim}


\begin{verbatim}
                 Functions cross references*.

CALC_SPUR
 |
 +-->SIMP!* (REDUCE)
     |
     +-->CALC_SPUR0
         |
         |--->TRANSFORM_MAP_
         |    |
         |    |--->MK_SIMPLE_VERTEX
         |    +--->MK_SIMPLE_MAP_
         |         |
         |         +--->MK_SIMPLE_MAP_1
         |              |
         |              +--->MK_FIERS_OP
         |
         |--->WORLD_FROM_ATLAS
         |    |
         |    +--->CONSTR_WORLDS
         |         |
         |         +---->MK_WORLD1
         |               |
         |               +--->MAP_2_FROM_MAP_1
         |                    |
         |                    |--->MARK_EDGES
         |                    +--->MAP_1_TO_STRAND
         |                         |
         |                         +-->PRE!-CALC!-MAP_
         |                             (INTERFIRZ)
         |
         |--->CALC_WORLD
         |    |
         |    |--->CALC!-MAP_TAR (INTERFIRZ)
         |    |--->CALC!-DENTAR (INTERFIRZ)
         |    +--->REVAL (REDUCE)
         |
         +--->FIND_BUBLTR
              |
              +--->FIND_BUBLTR0
                   |
                   |--->SORT_ATLAS
                   +--->FIND_BUBLTR1
                        |
                        |--->FIND_BUBLES1
                        +--->FIND_TRIANGLES1
*Unmarked functions are from CVITMPPING module.

\end{verbatim}

