\chapter{Changes since Version 3.8}

\paragraph*{New packages}

assert
bibasis
breduce
cde
cdiff
clprl
gcref
guardian
lalr
lessons
libreduce
listvecops
logoturtle
lpdo
redfront
reduce4
sstools
utf8

\paragraph*{Core package rlisp}

Support for namespaces (::)

Default value in switch statement

Support for utf8 characters

in\_tex command

\paragraph*{Core package poly}

Improvements for differentiation: new switches \sw{expanddf}, \sw {allowdfint} etc
(from odesolve)

New operator \f{reimpart}

\paragraph*{Core package alg}

New switch \sw{precise\_complex}

Improvements for switch \sw{combineexpt} (exptchk.red)

New command \f{unset}

New operators \f{continued\_fraction}, \f{totaldeg}

Improvements to the \f{conj} operator, added \f{selfconjugate} declaration.

Added \f{complex\_conjugates} declaration to associate pairs of identifiers
as mutual complex-conjugates.

\paragraph*{Core Package mathpr}

New switch \sw{unicode\_in\_off\_nat} to have unicode characters
displayed as such when nat is off.

\paragraph*{Core Package solve}

New boolean operator \f{polyp}(p,var), to determine whether p is a pure polynomial
in var, ie. the coefficients of p do not contain var.

\paragraph*{Core Package matrix}

New keyword \k{matrixproc} for declaration of matrix-valued procedures, and listproc for declaration of list-valued procedures.

\paragraph*{Operators now defined in the \REDUCE core}

From changevar: \f{changevar},

From polyrat: \f{divide}, \f{pseudo\_divide}, \f{pseudo\_div}, \f{pseudo\_quotient}, \f{pseudo\_remainder},

From specfn:
\f{Li}, \f{Si}, \f{Ci}, \f{Shi}, \f{Chi},
\f{Fresnel\_S}, \f{Fresnel\_C},
\f{gamma}, \f{igamma}, \f{psi}, \f{polygamma}, \f{beta}, \f{ibeta},
\f{euler},
\f{bernoulli}, \f{pochhammer}, \f{lerch\_phi}, \f{polylog}, \f{zeta},
\f{besselj},
\f{bessely},
\f{besseli},
\f{besselk},
\f{hankel1},
\f{hankel2},
\f{kummerM},
\f{kummerU},
\f{struveh},
\f{struvel},
\f{lommel1},
\f{lommel2},
\f{whittakerm},
\f{whittakerw},
\f{Airy\_Ai},
\f{Airy\_Bi},
\f{Airy\_AiPrime},
\f{Airy\_biprime},
\f{binomial},
\f{solidharmonic},
\f{sphericalharmonic},
\f{fibonacci},\f{fibonaccip},
\f{motzkin},
\f{hypergeometric}, \f{MeijerG}

From limit:
\f{limits}, \f{limit!+}, \f{limit!-},

From taylor and tps
\f{taylor}, \f{implicit\_taylor}, \f{inverse\_taylor},
\f{taylororiginal}, \f{taylortemplate}, \f{taylorcoefflist}, \f{taylortostandard}, \f{taylorcombine}, \f{taylorseriesp},
\f{taylorrevert},
\f{ps},
\f{psexplim}, \f{psordlim}, \f{psterm}, \f{psorder}, \f{pssetorder}, \f{psdepvar}, \f{psexpansionpt},
\f{psfunction}, \f{pschangevar}, \f{psreverse}, \f{pscompose}, \f{pssum}, \f{pstaylor}, \f{pscopy}, \f{pstruncate}

From fps: \f{fps}, \f{simplede}, \f{infsum}

From compact: \f{compact}

From residue: \f{residue}, \f{poleorder}

From ineq and rsolve: \f{ineq\_solve}, \f{r\_solve}, \f{i\_solve},

From roots:
\f{realroots}, \f{isolater}, \f{rlrootno},
\f{roots}, \f{roots\_at\_prec}, \f{root\_val}, \f{nearestroot}, \f{firstroot},
\f{getroot}, \f{mkpoly}, \f{gfnewt}, \f{gfroot},

From laplace: \f{laplace}, \f{invlap},

From defint:
\f{laplace\_transform}, \f{hankel\_transform}, \f{y\_transform}, \f{k\_transform}, \f{struveh\_transform},
\f{fourier\_sin}, \f{fourier\_cos},

From arnum: \f{defpoly}, \f{split\_field}

From zeilberg:
\f{extended\_gosper}, \f{extended\_sumrecursion},
\f{gosper}, \f{hyperrecursion}, \f{hypersum}, \f{hyperterm},
\f{sumtohyper}, \f{sumrecursion},
\f{simplify\_gamma}, \f{simplify\_gamma2}, \f{simplify\_gamman}, \f{simplify\_combinatorial},

From trigsimp: \f{trigsimp}, \f{triggcd}, \f{trigfactorize}

From ratint: \f{ratint}, \f{log\_sum}

From odesolve:
\f{odesolve}, \f{dsolve}, \f{root\_of\_unity}, \f{plus\_or\_minus},

From gnuplot:
New operator \f{plot}, new commands \f{gnuplot}, \f{plotshow} and \f{plotreset}.

New switches \sw{tracefps}, \sw{zb\_factor}, \sw{zb\_proof}, \sw{zb\_trace}

Constants now part of the core:\\
\var{catalan}, \var{euler\_gamma}, \var{golden\_ratio},
\var{khinchin}.

Variables as part of the core: \f{gosper\_representation},
\f{zb\_direction}, \f{zb\_order}, \f{zeilberger\_representation},


Consistent branch cuts for complex numerical functions.



\paragraph{Package specfn}

psi (digamma) function can now be calculated numerically for complex arguments.

\paragraph{Package specfn}

New functions: \f{theta1d}, \f{theta2d}, \f{theta3d} and \f{theta4d}
--- numerical evaluation derivatives of theta functions.

\paragraph{Package specfn}

\f{Weierstrass}, \f{WeierstrassZeta}, \f{sigma}, \f{sigma1}, \f{sigma2},
\f{sigma3} and \f{sigma4}  --- rules and numerical code added.

\paragraph{Package defint}

Added tracing output printing of which is controlled by the switch \sw{trdefint}.

\paragraph{TeXmacs interface}

Print prompt numbers by setting the switch \sw{promptnumbers} to on by default.

\paragraph{Package excalc}

New command \f{killing\_vector}.

\paragraph{Package specfn}

\f{arcsn}, \f{arccn}, \f{arcdn}, \f{arcns}, \f{arcnc}, \f{arcnd},
\f{arcsc}, \f{arcsd}, \f{arccs}, \f{arccd}, \f{arcds}, \f{arcdc}
--- rules and numerical code added for inverse Jacobian elliptic funtions,
real arguments and values only.

\paragraph{Package ellipfn}

New package comprising modules: \f{ellipfn}, \f{efjacobi}, \f{efellint},
\f{efjacinv}, \f{eftheta} and \f{efweier}.
These are essentially copies of the modules \f{sfellip}, \f{sfellipi},
\f{sfellipinv}, \f{sftheta} and \f{sfweier} which have been removed from
package \f{specfn}.
Documentation moved into a new user contributed package in \texttt{ellipfn.tex}.
