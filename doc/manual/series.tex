\chapter{Series expansion}

Expanding an algebraic expression into a series can be done by standard \REDUCE operators, namely \f{df}, \f{sub}, and possibly \f{limit}.
Nevertheless, there are many cases where this straightforward method fails.
\REDUCE offers two different operators for this purpose:
\begin{description}
\item[\f{taylor}] computes a truncated power series.
\item[\f{ps}] computes extendible power series.
  \item[\f{fps}] computes formal power series.
\end{description}

\section{Taylor expansion}
This package carries out the Taylor expansion of an expression in one
or more variables and efficient manipulation of the resulting Taylor
series. Capabilities include basic operations (addition, subtraction,
multiplication and division) and also application of certain algebraic
and transcendental functions.\footnote{This code was written by Rainer Schöpf.}


\subsection{Basic Use}

The most important operator is `\verb+TAYLOR+'. \ttindextype{TAYLOR}{operator}
It is used as follows:
\hypertarget{operator:TAYLOR}{}
\begin{verbatim}
  TAYLOR(EXP:algebraic,
         VAR:kernel,VAR0:algebraic,ORDER:integer[,...])
         :algebraic.
\end{verbatim}
where \f{EXP} is the expression to be expanded. It can be any \REDUCE{}
object, even an expression containing other Taylor kernels. \f{VAR} is
the kernel with respect to which \f{EXP} is to be expanded. \f{VAR0}
denotes the point about which and \f{ORDER} the order up to which
expansion is to take place. If more than one \f{(VAR, VAR0, ORDER)} triple
is specified \f{TAYLOR} will expand its first argument independently
with respect to each variable in turn. For example,
\begin{verbatim}
  taylor(e^(x^2+y^2),x,0,2,y,0,2);
\end{verbatim}
will calculate the Taylor expansion up to order $X^{2}*Y^{2}$:
\begin{verbatim}
       2    2    2  2      3  3
  1 + y  + x  + y *x  + O(x ,y )
\end{verbatim}
Note that once the expansion has been done it is not possible to
calculate higher orders.
Instead of a kernel, \f{VAR} may also
be a list of kernels. In this case expansion will take place in a way
so that the \emph{sum} of the degrees of the kernels does not exceed
\f{ORDER}.
If \f{VAR0} evaluates to the special identifier \f{INFINITY}, expansion is
done in a series in 1/VAR instead of \f{VAR}.

The expansion is performed variable per variable, i.e.\ in the example
above by first expanding $\exp(x^{2}+y^{2})$ with respect to $x$ and
then expanding every coefficient with respect to $y$.

\ttindextype{IMPLICIT\_TAYLOR}{operator}
\hypertarget{operator:IMPLICIT_TAYLOR}{}
There are two
extra operators to compute the Taylor expansions of implicit and
inverse functions:
\begin{verbatim}
  IMPLICIT_TAYLOR(F:algebraic,
                  VAR:kernel,DEPVAR:kernel,
                  VAR0:algebraic,DEPVAR0:algebraic,
                  ORDER:integer)
           :algebraic
\end{verbatim}
takes a function F depending on two variables VAR and DEPVAR and
computes the Taylor series of the implicit function DEPVAR(VAR)
given by the equation F(VAR,DEPVAR) = 0, around the point VAR0.  
(Violation of the necessary condition F(VAR0,DEPVAR0)=0 causes an error.)
For example,
\begin{verbatim}
  implicit_taylor(x^2 + y^2 - 1,x,y,0,1,5);
\end{verbatim}
gives the output
\begin{verbatim}
       1   2    1   4      6
  1 - ---*x  - ---*x  + O(x )
       2        8
\end{verbatim}

\hypertarget{operator:INVERSE_TAYLOR}{}
The operator
\begin{verbatim}
  INVERSE_TAYLOR(F:algebraic,VAR:kernel,DEPVAR:kernel,
                 VAR0:algebraic,ORDER:integer)
         : algebraic
\end{verbatim}
takes a function F depending on VAR1 and computes the Taylor series of
the inverse of F with respect to VAR2. For example,
\begin{verbatim}
  inverse_taylor(exp(x)-1,x,y,0,8);
\end{verbatim}
yields
\begin{verbatim}
       1   2    1   3    1   4    1   5                  9
  y - ---*y  + ---*y  - ---*y  + ---*y  + (3 terms) + O(y )
       2        3        4        5
\end{verbatim}


\ttindextype{TAYLORPRINTTERMS}{variable}\hypertarget{reserved:TAYLORPRINTTERMS}{}
When a Taylor kernel is printed, only a certain number of (non-zero)
coefficients are shown. If there are more, an expression of the form
\f{($n$ terms)} is printed to indicate how many non-zero
terms have been suppressed. The number of terms printed is given by
the value of the shared algebraic variable \f{TAYLORPRINTTERMS}.
Allowed values are integers and the special identifier \f{ALL}. The
latter setting specifies that all terms are to be printed. The default
setting is $5$.

\ttindextype{PART}{operator}\ttindex{PART}
The \f{PART} operator can be used to extract subexpressions of a
Taylor expansion in the usual way. All terms can be accessed,
irregardless of the value of the variable \f{TAYLORPRINTTERMS}.


\ttindexswitch{TAYLORKEEPORIGINAL}{TAYLOR}
If the switch \hyperlink{switch:TAYLORKEEPORIGINAL}{\f{TAYLORKEEPORIGINAL}}
is set to \f{ON} the
original expression EXP is kept for later reference.
It can be recovered by means of the operator

\hypertarget{operator:TAYLORORIGINAL}{}
\hspace*{2em} \texttt{TAYLORORIGINAL}(EXP:{\em exprn}):{\em exprn}

An error is signalled if EXP is not a Taylor kernel or if the original
expression was not kept, i.e.\ if \f{TAYLORKEEPORIGINAL} was
\f{OFF} during expansion.  The template of a Taylor kernel, i.e.\
the list of all variables with respect to which expansion took place
together with expansion point and order can be extracted using
\ttindex{TAYLORTEMPLATE}.

\hypertarget{operator:TAYLORTEMPLATE}{}
\hspace*{2em} \texttt{TAYLORTEMPLATE}(EXP:{\em exprn}):{\em list}

This returns a list of lists with the three elements (VAR,VAR0,ORDER).
As with \f{TAYLORORIGINAL},
an error is signalled if EXP is not a Taylor kernel.

The operator
\hypertarget{operator:TAYLORTOSTANDARD}{}\\
\hspace*{2em} \texttt{TAYLORTOSTANDARD}(EXP:{\em exprn}):{\em exprn}

converts all Taylor kernels in EXP into standard form and
\ttindex{TAYLORTOSTANDARD} resimplifies the result.

The boolean operator
\hypertarget{operator:TAYLORSERIESP}{}\\
\hspace*{2em} \texttt{TAYLORSERIESP}(EXP:{\em exprn}):{\em boolean}

may be used to determine if EXP is a Taylor kernel.
\ttindex{TAYLORSERIESP} (Note that this operator is subject to the same
restrictions as, e.g., \f{ORDP} or \f{NUMBERP}, i.e.\ it may only be used in
boolean expressions in \f{IF} or \f{LET} statements. 

Finally there is

\hypertarget{operator:TAYLORCOMBINE}{}
\hspace*{2em} \texttt{TAYLORCOMBINE}(EXP:{\em exprn}):{\em exprn}

which tries to combine all Taylor kernels found in EXP into one.
\ttindex{TAYLORCOMBINE}
Operations currently possible are:
\index{Taylor series!arithmetic}
\begin{itemize}
  \item Addition, subtraction, multiplication, and division.
  \item Roots, exponentials, and logarithms.
  \item Trigonometric and hyperbolic functions and their inverses.
\end{itemize}
Application of unary operators like \f{LOG} and \f{ATAN} will
nearly always succeed. For binary operations their arguments have to be
Taylor kernels with the same template. This means that the expansion
variable and the expansion point must match. Expansion order is not so
important, different order usually means that one of them is truncated
before doing the operation.

\ttindex{TAYLORKEEPORIGINAL} \ttindex{TAYLORCOMBINE}
If \hyperlink{switch:TAYLORKEEPORIGINAL}{\f{TAYLORKEEPORIGINAL}} is set to \f{ON} and if all Taylor
kernels in \f{exp} have their original expressions kept
\hyperlink{operator:TAYLORCOMBINE}{\f{TAYLORCOMBINE}} will also combine these and store the result
as the original expression of the resulting Taylor kernel.
\ttindexswitch{TAYLORAUTOEXPAND}{TAYLOR}
There is also the switch \hyperlink{switch:TAYLORAUTOEXPAND}{\f{TAYLORAUTOEXPAND}} (see below).

There are a few restrictions to avoid mathematically undefined
expressions: it is not possible to take the logarithm of a Taylor
kernel which has no terms (i.e. is zero), or to divide by such a
beast.  There are some provisions made to detect singularities during
expansion: poles that arise because the denominator has zeros at the
expansion point are detected and properly treated, i.e.\ the Taylor
kernel will start with a negative power.  (This is accomplished by
expanding numerator and denominator separately and combining the
results.)  Essential singularities of the known functions (see above)
are handled correctly.

\index{Taylor series!differentiation}
Differentiation of a Taylor expression is possible.  If you
differentiate with respect to one of the Taylor variables the order
will decrease by one.

\index{Taylor series!substitution}
Substitution is a bit restricted: Taylor variables can only be replaced
by other kernels.  There is one exception to this rule: you can always
substitute a Taylor variable by an expression that evaluates to a
constant.  Note that \REDUCE{} will not always be able to determine
that an expression is constant.

\index{Taylor series!integration}
Only simple taylor kernels can be integrated. More complicated
expressions that contain Taylor kernels as parts of themselves are
automatically converted into a standard representation by means of the
\hyperlink{operator:TAYLORTOSTANDARD}{\f{TAYLORTOSTANDARD}} operator. 
In this case a suitable warning is printed.

\index{Taylor series!reversion} It is possible to revert a Taylor
series of a function $f$, i.e., to compute the first terms of the
expansion of the inverse of $f$ from the expansion of $f$. This is
done by the operator

\hypertarget{operator:TAYLORREVERT}{}
\hspace*{2em} \texttt{TAYLORREVERT}(EXP:{\em exprn},OLDVAR:{\em kernel},
                                 NEWVAR:{\em kernel}):{\em exprn}

EXP must evaluate to a Taylor kernel with OLDVAR being one of its
expansion variables. Example:
\begin{verbatim}
  taylor (u - u**2, u, 0, 5)$
  taylorrevert (ws, u, x);
\end{verbatim}
gives
\begin{verbatim}
       2      3      4       5      6
  x + x  + 2*x  + 5*x  + 14*x  + O(x )
\end{verbatim}

This package introduces a number of new switches:
\begin{description}

\ttindexswitch{TAYLORAUTOCOMBINE}{TAYLOR}
\item[\sw{TAYLORAUTOCOMBINE}] \hypertarget{switch:TAYLORAUTOCOMBINE}{}causes
    Taylor expressions to be automatically combined during the simplification
    process.  This is equivalent to applying \f{TAYLORCOMBINE} to
    every expression that contains Taylor kernels.
    Default is \f{ON}.

\ttindexswitch{TAYLORAUTOEXPAND}{TAYLOR}
\item[\sw{TAYLORAUTOEXPAND}] \hypertarget{switch:TAYLORAUTOEXPAND}{} makes Taylor expressions ``contagious''
    in the sense that \f{TAYLORCOMBINE} tries to Taylor expand
    all non-Taylor subexpressions and to combine the result with the
    rest. Default is \f{OFF}.

\ttindexswitch{TAYLORKEEPORIGINAL}{TAYLOR}\hypertarget{switch:TAYLORKEEPORIGINAL}{}
\item[\sw{TAYLORKEEPORIGINAL}] forces the
    package to keep the original expression, i.e.\ the expression
    that was Taylor expanded.  All operations performed on the
    Taylor kernels are also applied to this expression  which can
    be recovered using the operator \f{TAYLORORIGINAL}.
    Default is \f{OFF}.

\ttindexswitch{TAYLORPRINTORDER}{TAYLOR}\hypertarget{switch:TAYLORPRINTORDER}{}
\item[\sw{TAYLORPRINTORDER}] causes the
    remainder to be printed in big-$O$ notation.  Otherwise, three
    dots are printed. Default is \f{ON}.

\ttindexswitch{VERBOSELOAD}{TAYLOR}
\item[\sw{VERBOSELOAD}] will cause
    \REDUCE{} to print some information when the Taylor package is
    loaded.  This switch is already present in \textsf{PSL} systems.
    Default is \f{OFF}.

\end{description}
\index{Defaults! TAYLOR package}

\subsection{Caveats}
\index{Caveats!TAYLOR package}

\f{TAYLOR} should always detect non-analytical expressions in
its first argument. As an example, consider the function $xy/(x+y)$
that is not analytical in the neighborhood of $(x,y) = (0,0)$: Trying
to calculate
\begin{verbatim}
  taylor(x*y/(x+y),x,0,2,y,0,2);
\end{verbatim}
causes an error
\begin{verbatim}
***** Not a unit in argument to QUOTTAYLOR
\end{verbatim}
Note that it is not generally possible to apply the standard \REDUCE{}
operators to a Taylor kernel. For example, \f{PART}, \f{COEFF},
or \f{COEFFN} cannot be used. Instead, the expression at hand has
to be converted to standard form first using the \f{TAYLORTOSTANDARD}
operator.

\subsection{Warning messages}
\index{Warnings!TAYLOR package}
\begin{description}

\item[\msg{*** Cannot expand further... truncation done}]\mbox{}\\
    You will get this warning if you try to expand a Taylor kernel to
    a higher order.

\item[\msg{*** Converting Taylor kernels to standard representation}]\mbox{}\\
    This warning appears if you try to integrate an expression
    containing Taylor kernels.

\end{description}

\subsection{Error messages}
\index{Errors!TAYLOR package}
\begin{description}

\item[\msg{***** Branch point detected in ...}]\mbox{}\\
    This occurs if you take a rational power of a Taylor kernel
    and raising the lowest order term of the kernel to this
    power yields a non analytical term (i.e.\ a fractional power).

\item[\msg{***** Cannot replace part ... in Taylor kernel}]\mbox{}\\
\ttindextype{PART}{Operator}%
    The \f{PART} operator can only be used to either replace the
    template of a Taylor kernel (part 2) or the original expression
    that is kept for reference (part 3).    

\item[\msg{***** Computation loops (recursive definition?): ...}]\mbox{}\\
    Most probably the expression to be expanded contains an operator
    whose derivative involves the operator itself.

\item[\msg{***** Error during expansion (possible singularity)}]\mbox{}\\
    The expression you are trying to expand caused an error.
    As far as I know this can only happen if it contains a function
    with a pole or an essential singularity at the expansion point.
    (But one can never be sure.)

\item[\msg{***** Essential singularity in ...}]\mbox{}\\
    An essential singularity was detected while applying a
    special function to a Taylor kernel.

\item[\msg{***** Expansion point lies on branch cut in ...}]\mbox{}\\
    The only functions with branch cuts this package knows of are
    (natural) logarithm, inverse circular and hyperbolic tangent and
    cotangent.  The branch cut of the logarithm is assumed to lie on
    the negative real axis.  Those of the arc tangent and arc
    cotangent functions are chosen to be compatible with this: both
    have essential singularities at the points $\pm i$.  The branch
    cut of arc tangent is the straight line along the imaginary axis
    connecting $+1$ to $-1$ going through $\infty$ whereas that of arc
    cotangent goes through the origin.  Consequently, the branch cut
    of the inverse hyperbolic tangent resp.\ cotangent lies on the
    real axis and goes from $-1$ to $+1$, that of the latter across
    $0$, the other across $\infty$.

    The error message can currently only appear when you try to
    calculate the inverse tangent or cotangent of a Taylor
    kernel that starts with a negative degree.
    The case of a logarithm of a Taylor kernel whose constant term
    is a negative real number is not caught since it is
    difficult to detect this in general.

\item[\msg{***** Input expression non-zero at given point}]\mbox{}\\
    Violation of the necessary condition F(VAR0,DEPVAR0)=0 for the arguments of
    \f{IMPLICIT\_TAYLOR}.

\item[\msg{***** Invalid substitution in Taylor kernel: ...}]\mbox{}\\
    You tried to substitute a variable that is already present in the
    Taylor kernel or on which one of the Taylor variables depend.

\item[\msg{***** Not a unit in ...}]\mbox{}\\
    This will happen if you try to divide by or take the logarithm of
    a Taylor series whose constant term vanishes.

\item[\msg{***** Not implemented yet (...)}]\mbox{}\\
    Sorry, but I haven't had the time to implement this feature.
    Tell me if you really need it, maybe I have already an improved
    version of the package.

\item[\msg{***** Reversion of Taylor series not possible: ...}]\mbox{}\\
\ttindex{TAYLORREVERT}
    You tried to call the \f{TAYLORREVERT} operator with
    inappropriate arguments. The second half of this error message
    tells you why this operation is not possible.

\item[\msg{***** Taylor kernel doesn't have an original part}]\mbox{}\\
\ttindex{TAYLORORIGINAL} \ttindex{TAYLORKEEPORIGINAL}
    The Taylor kernel upon which you try to use \f{TAYLORORIGINAL}
    was created with the switch \f{TAYLORKEEPORIGINAL}
    set to \f{OFF}
    and does therefore not keep the original expression.

\item[\msg{***** Wrong number of arguments to TAYLOR}]\mbox{}\\
    You try to use the operator \f{TAYLOR} with a wrong number of
    arguments.

\item[\msg{***** Zero divisor in TAYLOREXPAND}]\mbox{}\\
    A zero divisor was found while an expression was being expanded.
    This should not normally occur.

\item[\msg{***** Zero divisor in Taylor substitution}]\mbox{}\\
    That's exactly what the message says.  As an example consider the
    case of a Taylor kernel containing the term \f{1/x} and you try
    to substitute \f{x} by \f{0}.

\item[\msg{***** ... invalid as kernel}]\mbox{}\\
    You tried to expand with respect to an expression that is not a
    kernel.

\item[\msg{***** ... invalid as order of Taylor expansion}]\mbox{}\\
    The order parameter you gave to \f{TAYLOR} is not an integer.

\item[\msg{***** ... invalid as Taylor kernel}]\mbox{}\\
\ttindex{TAYLORORIGINAL} \ttindex{TAYLORTEMPLATE}
    You tried to apply \f{TAYLORORIGINAL} or \f{TAYLORTEMPLATE}
    to an expression that is not a Taylor kernel.

\item[\msg{***** ... invalid as Taylor Template element}]\mbox{}\\
    You tried to substitute the \f{TAYLORTEMPLATE} part of a Taylor
    kernel with a list a incorrect form. For the correct form see the
    description of the \f{TAYLORTEMPLATE} operator.

\item[\msg{***** ... invalid as Taylor variable}]\mbox{}\\
    You tried to substitute a Taylor variable by an expression that is
    not a kernel.

\item[\msg{***** ... invalid as value of TaylorPrintTerms}]\mbox{}\\
\ttindex{TAYLORPRINTTERMS}
    You have assigned an invalid value to \hyperlink{reserved:TAYLORPRINTTERMS}{\f{TAYLORPRINTTERMS}}.
    Allowed values are: an integer or the special identifier
    \f{ALL}.

\item[\msg{TAYLOR PACKAGE (...): this can't happen ...}]\mbox{}\\
    This message shows that an internal inconsistency was detected.
    This is not your fault, at least as long as you did not try to
    work with the internal data structures of \REDUCE. Send input
    and output to me, together with the version information that is
    printed out.

\end{description}

\subsection{Comparison to other packages}

At the moment there is only one \REDUCE{} package that I know of:
the truncated power series package by Alan Barnes and Julian Padget.
In my opinion there are two major differences:
\begin{itemize}
  \item The interface. They use the domain mechanism for their power
        series, I decided to invent a special kind of kernel. Both
        approaches have advantages and disadvantages: with domain
        modes, it is easier
        to do certain things automatically, e.g., conversions.
  \item The concept of a truncated series. Their idea is to remember
        the original expression and to compute more coefficients when
        more of them are needed. My approach is to truncate at a
        certain order and forget how the unexpanded expression
        looked like.  I think that their method is more widely
        usable, whereas mine is more efficient when you know in
        advance exactly how many terms you need.
\end{itemize}



\section{TPS: extendible power series}
\documentstyle[fullpage]{article}
\begin{document}

\title{TRUNCATED POWER SERIES}
 
\author{Alan Barnes \\
Dept. of Computer Science and Applied Mathematics \\
Aston University, Aston Triangle, \\
Birmingham B4 7ET \\
GREAT BRITAIN \\
e-mail: barnesa@kirk.aston.ac.uk \\
 \\
Julian Padget \\
School of Mathematics, University of Bath \\
Claverton Down, Bath, BA2 7AY \\
GREAT BRITAIN \\
e-mail: jap@maths.bath.ac.uk}

\maketitle

\section*{INTRODUCTION}

This package implements formal Laurent series expansions in one
variable using the domain mechanism of REDUCE. This means that power
series objects can be added, multiplied, differentiated etc. like other
first class objects in the system. A lazy evaluation scheme is used in
the package and thus terms of the series are not evaluated until they
are required for printing or for use in calculating terms in other
power series. The series are extendible giving the user the impression
that the full infinite series is being manipulated.  The errors that
can sometimes occur using series that are truncated at some fixed depth
(for example when a term in the required series depends on terms of an
intermediate series beyond the truncation depth) are thus avoided.

Below we give a brief description of the operators available in the
power series package together with some examples of their use.

\subsection*{PS OPERATOR}

Syntax:
\begin{verbatim}
 PS(EXPRN:algebraic,DEPVAR:kernel,ABOUT:algebraic):ps object.
\end{verbatim}
The {\tt PS} operator returns a  power series object (a tagged domain element)
representing the univariate formal power series expansion of {\tt EXPRN} with
respect to the dependent variable {\tt DEPVAR} about the expansion point
{\tt ABOUT}.  {\tt EXPRN} may itself contain power series objects.
 
The algebraic expression {\tt ABOUT} should simplify to an expression
which is independent of the dependent variable {\tt DEPVAR}, otherwise
an error will result.  If {\tt ABOUT} is the identifier {\tt INFINITY}
then the power series expansion about {\tt DEPVAR} = $\infty$ is
obtained in ascending powers of {\tt 1/DEPVAR}.
 
If the command is terminated by a semi-colon, a power series object
representing {\tt EXPRN} is compiled and then a number of terms of the
power series expansion are evaluated and printed.  The expansion is
carried out as far as the value specified by {\tt PSEXPLIM}.  If,
subsequently, the value of {\tt PSEXPLIM} is increased, sufficient
information is stored in the power series object to enable the
additional terms to be calculated without recalculating the terms
already obtained.
 
If the command is terminated by a dollar symbol, a power series object
is compiled, but at most one term is calculated at this stage.
 
If the function has a pole at the expansion point then the correct
Laurent series expansion will be produced.
 
\noindent The following examples are valid uses of {\tt PS}:
\begin{verbatim}
    psexplim 6;
    ps(log x,x,1);
    ps(e**(sin x),x,0);
    ps(x/(1+x),x,infinity);
    ps(sin x/(1-cos x),x,0);
\end{verbatim}

New user-defined functions may be expanded provided the user provides
LET rules giving
\begin{enumerate}
\item the value of the function at the expansion point
\item a differentiation rule for the new function.
\end{enumerate}

\noindent For example
\begin{verbatim}
    OPERATOR SECH;
    FORALL X LET DF(SECH X,X)= - SECH X * TANH X;
    LET SECH 0 = 1;
    PS(SECH(X**2),X,0);
\end{verbatim}
 
The power series expansion of an integral may also be obtained (even if
REDUCE cannot evaluate the integral in closed form).  An example of
this is
\begin{verbatim}
    PS(INT(e**X/X,X),X,1);
\end{verbatim}
 
Note that if the integration variable is the same as the expansion
variable then REDUCE's integration package is not called; if on the
other hand the two variables are different then the integrator is
called to integrate each of the coefficients in the power series
expansion of the integrand.  The constant of integration is zero by
default.  If another value is desired, then the shared variable {\tt
PSINTCONST} should be set to required value.

For example in algebraic mode
\begin{verbatim}
        PSINTCONST:=A**2;
\end{verbatim}
would set the value of this constant to be (the value of) {\tt A**2}.
The setting of this constant has no effect on the value returned by
the REDUCE integrator. If the expansion and integration variables are
the same and {\tt PSINTCONST} depends on this variable an error
results.
 
 
\subsection*{PSEXPLIM OPERATOR}

Syntax:
\begin{verbatim}
 PSEXPLIM(UPTO:integer):integer,
  or PSEXPLIM ():integer.
\end{verbatim} 
Calling this operator sets an internal variable of the
TPS package to the value of {\tt
UPTO} (which should evaluate to an integer).  The value returned is
the previous value of this variable.  The default value is six.
 
If {\tt PSEXPLIM} is called with no argument, the current value for
the expansion limit is returned.
 

\subsection*{PSORDLIM OPERATOR}

Syntax:
\begin{verbatim}
 PSORDLIM(UPTO:integer):integer,
  or PSORDLIM ():integer.
\end{verbatim}
An internal variable is set to the value of {\tt UPTO} (which should
evaluate to an integer). The value returned is the previous value of
the variable.  The default value is 15.

If {\tt PSORDLIM} is called with no argument, the current value is
returned.

The significance of this control is that the system attempts to find
the order of the power series required, that is the order is the
degree of the first non-zero term in the power series.  If the order
is greater than the value of this variable an error message is given
and the computation aborts. This prevents infinite loops in examples
such as
\begin{verbatim}
        PS(1 - (sin x)**2 - (cos x)**2,x,0);
\end{verbatim}
where the expression being expanded is identically zero, but is not
recognized as such by REDUCE.


\subsection*{PSTERM OPERATOR}

Syntax:
\begin{verbatim}
 PSTERM(TPS:power series object,NTH: integer):algebraic
\end{verbatim}
The operator {\tt PSTERM} returns the {\tt NTH} term of the existing
power series object {\tt TPS}. If {\tt NTH} does not evaluate to
an integer or {\tt TPS} to a power series object an error results.  It
should be noted that an integer is treated as a power series.


\subsection*{PSORDER OPERATOR}

Syntax:
\begin{verbatim}
 PSORDER(TPS:power series object):integer
\end{verbatim}
The operator {\tt PSORDER} returns the order, that is the degree of
the first non-zero term, of the power series object {\tt TPS}. {\tt
TPS} should evaluate to a power series object or an error results. If
{\tt TPS} is zero, the identifier {\tt UNDEFINED} is returned.

\subsection*{PSSETORDER OPERATOR}

Syntax:
\begin{verbatim}
 PSSETORDER(TPS:power series object,ORD:integer):integer
\end{verbatim}
The operator {\tt PSSETORDER} sets the order of the power series {\tt
TPS} to the value {\tt ORD}, which should evaluate to an integer. If
{\tt TPS} does not evaluate to a power series object, then an error
occurs. The value returned by this operator is the previous order of
{\tt TPS}, or 0 if the order of {\tt TPS} was undefined.  This
operator is useful for setting the order of the power series of a
function defined by a differential equation in cases where the power
series package is inadequate to determine the order automatically.


\subsection*{PSDEPVAR OPERATOR}

Syntax:
\begin{verbatim}
 PSDEPVAR(TPS:power series object):identifier
\end{verbatim}
The operator {\tt PSDEPVAR} returns the expansion variable of the
power series object {\tt TPS}. {\tt TPS} should evaluate to a power
series object or an integer, otherwise an error results. If {\tt TPS}
is an integer, the identifier {\tt UNDEFINED} is returned.

\subsection*{PSEXPANSIONPT OPERATOR}

Syntax:
\begin{verbatim}
 PSEXPANSIONPT(TPS:power series object):algebraic
\end{verbatim}
The operator {\tt PSEXPANSIONPT} returns the expansion point of the
power series object {\tt TPS}. {\tt TPS} should evaluate to a power
series object or an integer, otherwise an error results. If {\tt TPS}
is integer, the identifier {\tt UNDEFINED} is returned. If the
expansion is about infinity, the identifier {\tt INFINITY} is
returned.

\subsection*{PSFUNCTION OPERATOR}

Syntax:
\begin{verbatim}
 PSFUNCTION(TPS:power series object):algebraic
\end{verbatim}
The operator {\tt PSFUNCTION} returns the function whose expansion
gave rise to the power series object {\tt TPS}. {\tt TPS} should
evaluate to a power series object or an integer, otherwise an error
results.

\subsection*{PSCHANGEVAR OPERATOR}

Syntax:
\begin{verbatim}
 PSCHANGEVAR(TPS:power series object,X:kernel):power series object
\end{verbatim}
The operator {\tt PSCHANGEVAR} changes the dependent variable of the
power series object {\tt TPS} to the variable {\tt X}. {\tt TPS}
should evaluate to a power series object and {\tt X} to a kernel,
otherwise an error results.  Also {\tt X} should not appear as a
parameter in {\tt TPS}. The power series with the new dependent
variable is returned.

\subsection*{PSREVERSE OPERATOR}

Syntax:
\begin{verbatim}
 PSREVERSE(TPS:power series object):power series
\end{verbatim}
Power series reversion.  The power series {\tt TPS} is functionally
inverted.  Four cases arise:
\begin{enumerate}
\item if the order of the series is 1, then the expansion point of the
inverted series is 0. 

\item if the order is 0 {\em and} if the first order term in {\tt TPS}
is non-zero, then the expansion point of the inverted series is taken
to be the coefficient of the zeroth order term in {\tt TPS}.

\item if the order is -1 the expansion point of the inverted series
is the point at infinity.  In all other cases a REDUCE error is
reported because the series cannot be inverted as a power series. Puiseux
expansion would be required to handle these cases.

\item If the expansion point of {\tt TPS} is finite it becomes the
zeroth order term in the inverted series. For expansion about 0 or the
point at infinity the order of the inverted series is one.
\end{enumerate}
If {\tt TPS} is not a power series object after evaluation an error results.

\noindent Here are some examples:
\begin{verbatim}
        PS(sin x,x,0);
        PSREVERSE(ws); % produces the series for asin x about x=0.
        PS(exp x,x,0);
        PSREVERSE ws; % produces the series for log x about x=1.
        PS(sin(1/x),x,infinity);
        PSREVERSE(ws); % produces the series for 1/asin(x) about x=0.
\end{verbatim}

\subsection*{PSCOMPOSE OPERATOR}

Syntax:
\begin{verbatim}
 PSCOMPOSE(TPS1:power series,TPS2:power series):power series
\end{verbatim}
Power Series Composition.

The power series {\tt TPS1} and {\tt TPS2} are functionally composed.
That is to say that {\tt TPS2} is substituted for the expansion
variable in {\tt TPS1} and the result expressed as a power series. The
dependent variable and expansion point of the result coincide with
those of {\tt TPS2}.  The following conditions apply to power series
composition:

\begin{enumerate}
\item If the expansion point of {\tt TPS1} is 0 then the order of the
{\tt TPS2} must be at least 1.

\item If the expansion point of {\tt TPS1} is finite, it should
coincide with the coefficient of the zeroth order term in {\tt TPS2}.
The order of {\tt TPS2} should also be non-negative in this case.

\item If the expansion point of {\tt TPS1} is the point at infinity
then the order of {\tt TPS2} must be less than or equal to -1.

\end{enumerate}

If these conditions do not hold the series cannot be composed (with
the current algorithm terms of the inverted series would involve
infinite sums) and a REDUCE error occurs.

\noindent Examples of power series composition include the following.
\begin{verbatim}
       A:=PS(exp y,y,0);  B:=PS(sin x,x,0); 
       PSCOMPOSE(A,B);
    % produces the power series expansion of exp(sin x) about x=0.

       A:=PS(exp z,z,1); B:=PS(cos x,x,0);
       PSCOMPOSE(A,B);
    % produces the power series expansion of exp(cos x) about x=0.

      A:=PS(cos(1/x),x,infinity);  B:=ps(1/sin x,x,0);
      PSCOMPOSE(A,B);
    % produces the power series expansion of cos(sin x) about x=0.
\end{verbatim}

\subsection*{PSSUM OPERATOR}

Syntax:
\begin{verbatim}
 PSSUM(J:kernel = LOWLIM:integer, COEFF: algebraic, X: kernel,
       ABOUT: algebraic; POWER: algebraic):power series
\end{verbatim}
The formal power series sum for {\tt J} from {\tt LOWLIM} to {\tt INFINITY} of 
\begin{verbatim}
      COEFF*(X-ABOUT)**POWER
\end{verbatim}
or if {\tt ABOUT} is given as {\tt INFINITY}
\begin{verbatim}
      COEFF*(1/X)**POWER
\end{verbatim}
is constructed and returned. This enables power series whose general
term is known to be constructed and manipulated using the other
procedures of the power series package. 

{\tt J} and {\tt X} should be distinct simple kernels. The algebraics
{\tt ABOUT},  {\tt COEFF} and {\tt POWER} should not depend on the
expansion variable {\tt X}, similarly the algebraic {\tt ABOUT} should
not depend on the summation variable {\tt J}.  The algebraic {\tt
POWER} should be a strictly increasing integer valued function of {\tt
J} for {\tt J} in the range {\tt LOWLIM} to {\tt INFINITY}.

\begin{verbatim}
   PSSUM(N=0,1,x,0,N*N);
% produces the power series summation for n=0 to infinity of x**(n*n)

   PSSUM(m=1,(-1)**(m-1)/(2m-1),y,1,2m-1);
   % produces the power series expansion of atan(y-1) about y=1

   PSSUM(j=1,-1/j,x,infinity,j);
% produces the power series expansion of log(1-1/x) about the point at
% infinity

   PSSUM(n=0,1,x,0,2n**2+3n) + PSSUM(n=1,1,x,0,2n**2-3n);
% produces the power series summation for n=-infinity to +infinity of
%    x**(2n**2+3n)
\end{verbatim}

\subsection*{ARITHMETIC OPERATIONS}
 
As power series objects are domain elements they may be combined
together in algebraic expressions in algebraic mode of REDUCE in the
normal way.
 
For example if {\tt A} and {\tt B} are power
series objects then the commands such as:
\begin{verbatim}
    A*B;
    A**2+B**2;
\end{verbatim}
will produce power series objects representing the product and the sum
of the squares of the power series objects {\tt A} and {\tt B}
respectively.
 
\subsection*{DIFFERENTIATION}
 
If {\tt A} is a power series object depending on {\tt X} then the input
{\tt DF(A,X)}; will produce the power series expansion of the derivative
of {\tt A} with respect to {\tt X}.


\section*{Restrictions and Known Bugs}

If {\tt A} and {\tt B} are power series objects and {\tt X} is a variable
which evaluates to itself then currently expressions such as {\tt A/B} and
{\tt A*X} do not evaluate to a single power series object (although the
results are in each case formally valid).  Instead use {\tt PS(A/B,X,0)}
and {\tt PS(A*X,X,0)} {\em etc.}.  The failure of the system to simplify
quotients to a single power series is due to an infelicity in the REDUCE
simplifier which will be corrected in future releases of REDUCE.

Similarly expressions such as {\tt sin(A)} where {\tt A} is a PS object
currently will not be expanded.
 
{\em e.g.}
\begin{verbatim}
    A:=PS(1/(1+X),X,0);
    B:=sin (A); 
\end{verbatim}
will not expand {\tt sin(1/(1+x))} as a power series . In fact
\begin{verbatim}
          sin(1 - x + x**2 - x**3 + .....)
\end{verbatim}
will be returned. However,
\begin{verbatim} 
    B:=PS(SIN(A),X,0);
\end{verbatim}
or
\begin{verbatim} 
    B:=PS(SIN(1/(1+X)),X,0);
\end{verbatim}
should work as intended.

The handling of functions with essential singularities is currently
erratic: usually an error message Essential Singularity or Logarithmic
Singularity occurs but occasionally a division by zero error or some
drastic error like (for PSL) binding stack overflow may occur.
 
Mixed mode arithmetic of power series objects with other domain
elements is quite restricted: only integers and floats can currently
be converted to power series objects.
 
The printing of power series currently leaves something to be
desired: often line-breaks appear in the middle of terms.

There is no simple way to write the results of power series
calculation to a file and read them back into REDUCE at a later
stage.

\end{document}



\newpage

\section{FPS: Automatic calculation of formal power series}
\indexpackage{FPS}

This package can expand a specific class of functions into their
corresponding Laurent-Puiseux series.\footnote{This package was written by Wolfram Koepf and Winfried Neun.}

\index{Koepf, Wolfram}\index{Persons!Koepf, Wolfram}
\index{Neun, Winfried}\index{Persons!Neun, Winfried}

\subsection{Introduction}
This package can expand functions of certain type into
their corresponding Laurent-Puiseux series as a sum of terms of the form
\[
\sum_{k=0}^{\infty} a_{k} (x-x_{0})^{m k/n + s}
\]
where $m$ is the `symmetry number', $s$ is the `shift number',
$n$ is the `Puiseux number',
and $x_0$ is the `point of development'. The following types are
supported:
\begin{itemize}
\item
textbf{functions of `rational type'}, which are either rational or have a
rational derivative of some order;
\item
\textbf{functions of `hypergeometric type'} where $a(k+m)/a(k)$ is a rational
function for some integer $m$;
\item
\textbf{functions of `explike type'} which satisfy a linear homogeneous
differential equation with constant coefficients.
\end{itemize}

The \package{FPS} package is an implementation of the method
presented in \cite{Koepf:92}. The implementations of this package
for \textsc{Maple} (by D.\ Gruntz) and \textsc{Mathematica} (by W.\ Koepf)
served as guidelines for this one.

Numerous examples can be found in \cite{Koepf:93a,Koepf:93b}, 
most of which are contained in the test file \texttt{fps.tst}. Many 
more examples can be found in the extensive bibliography of Hansen \cite{Hansen:75}.


\subsection{\REDUCE{} operator \texttt{FPS}}

\ttindextype[FPS]{fps}{operator}
\hypertarget{operator:FPS}{}
\texttt{fps(f,x,x0)} tries to find a formal power
series expansion for \texttt{f} with respect to the variable \texttt{x} 
at the point of development \texttt{x0}. 
It also works for formal Laurent (negative exponents) and Puiseux series
(fractional exponents). If the third 
argument is omitted, then \texttt{x0:=0} is assumed.

Examples: \texttt{fps(asin(x)\textasciicircum2,x)} results in
\begin{verbatim}

         2*k  2*k             2  2
        x   *2   *factorial(k) *x
infsum(----------------------------,k,0,infinity)
        factorial(2*k + 1)*(k + 1)
\end{verbatim}
\texttt{fps(sin x,x,pi)} gives
\begin{verbatim}
                   2*k       k
        ( - pi + x)   *( - 1) *( - pi + x)
infsum(------------------------------------,k,0,infinity)
                factorial(2*k + 1)
\end{verbatim}
and \texttt{fps(sqrt(2-x\textasciicircum2),x)} yields
\begin{verbatim}
            2*k
         - x   *sqrt(2)*factorial(2*k)
infsum(--------------------------------,k,0,infinity)
           k             2
          8 *factorial(k) *(2*k - 1)
\end{verbatim}
\ttindextype[FPS]{infsum}{operator}
\hypertarget{operator:INFSUM}{}
Note: The result contains one or more \f{infsum} terms such that it does
not interfere with the {\REDUCE} operator \texttt{sum}. In graphical oriented
REDUCE interfaces this operator results in the usual $\sum$ notation.

If possible, the output is given using factorials. In some cases, the
use of the Pochhammer symbol \texttt{pochhammer(a,k)}$:=a(a+1)\cdots(a+k-1)$
is necessary.

The operator \texttt{fps} uses the operator \texttt{SimpleDE} of the next section.

If an error message of type
\begin{verbatim}
Could not find the limit of:
\end{verbatim}
occurs, you can set the corresponding limit yourself and try a
recalculation. In the computation of \texttt{fps(atan(cot(x)),x,0)},
REDUCE is not able to find the value for the limit 
\texttt{limit(atan(cot(x)),x,0)} since the \texttt{atan} function is multi-valued.
One can choose the branch of \texttt{atan} such that this limit equals
$\pi/2$ so that we may set 
\begin{verbatim}
let limit(atan(cot(~x)),x,0)=>pi/2;
\end{verbatim}
and a recalculation of \texttt{fps(atan(cot(x)),x,0)}
yields the output \texttt{pi - 2*x} which is
the correct local series representation.

\subsection{\REDUCE{} operator \texttt{SimpleDE}}
\ttindextype[FPS]{SimpleDE}{operator}
\hypertarget{operator:SIMPLEDE}{}

\texttt{SimpleDE(f,x)} tries to find a homogeneous linear differential
equation with polynomial coefficients for $f$ with respect to $x$.
Make sure that $y$ is not a used variable.
The setting \texttt{factor df;} is recommended to receive a nicer output form.

Examples: \texttt{SimpleDE(asin(x)\textasciicircum2,x)} then results in
\begin{verbatim}
            2
df(y,x,3)*(x  - 1) + 3*df(y,x,2)*x + df(y,x)
\end{verbatim}
\texttt{SimpleDE(exp(x\textasciicircum (1/3)),x)} gives
\begin{verbatim}
              2
27*df(y,x,3)*x  + 54*df(y,x,2)*x + 6*df(y,x) - y
\end{verbatim}
and \texttt{SimpleDE(sqrt(2-x\textasciicircum2),x)} yields
\begin{verbatim}
          2
df(y,x)*(x  - 2) - x*y
\end{verbatim}
\ttindextype[FPS]{fps\_search\_depth}{shared variable}
\hypertarget{reserved:FPS_SEARCH_DEPTH}{}
The depth for the search of a differential equation for \texttt{f} is
controlled by the variable \texttt{fps\_search\_depth};
A higher value for \texttt{fps\_search\_depth}
will increase the chance to find the solution, but increases the
complexity as well. The default value for \texttt{fps\_search\_depth} 
is $5$. E.~g., for \texttt{fps(sin(x\^{}(1/3)),x)}, or 
\texttt{SimpleDE(sin(x\^{}(1/3)),x)} a setting
\texttt{fps\_search\_depth:=6} is necessary.

\ttindextype{tracefps}{switch}
\hypertarget{switch:TRACEFPS}{}
The output of the \package{FPS} package can be influenced by the
switch \sw{tracefps}. Setting \texttt{on tracefps} causes various
prints of intermediate results.

\subsection{Problems in the current version}
The handling of logarithmic singularities is not yet implemented.

The rational type implementation is not yet complete.

The support of special functions \cite{Koepf:94c}
will be part of the next version.


