
\index{Schruefer, Eberhard}\index{People!Schruefer, Eberhard}

This package implements vector operations on lists\index{List!vector operations}.
Addition, multiplication, division, and exponentiation work elementwise.
For example, after
\ttindextype{+ (plus)}{operator!lists}
\ttindextype{- (minus)}{operator!lists}
\ttindextype{* (times)}{operator!lists}
\ttindextype{/ (quotient)}{operator!lists}
\ttindextype{** (expt)}{operator!lists}
\ttindextype{"\textasciicircum{} (expt)}{operator!lists}
\begin{verbatim}
  A := {a1,a2,a3,a4};
  B := {b1,b2,b3,b4};
\end{verbatim}
\texttt{c*A} will simplify to \texttt{\{c*a1,..,c*a4\}},
\texttt{A + B} to \texttt{\{a1+b1,...,a4+b4\}}, and
\texttt{A*B} to \texttt{\{a1*b1,...,a4*b4\}}.
Linear operations work as expected:
\begin{verbatim}
c1*A + c2*B;

{a1*c1 + b1*c2,

  a2*c1 + b2*c2,

  a3*c1 + b3*c2,

  a4*c1 + b4*c2}
\end{verbatim}
A division and an exponentation example:
\begin{verbatim}
{a,b,c}/{3,g,5};

  a   b   c
{---,---,---}
  3   g   5

ws^3;

   3    3    3
  a    b    c
{----,----,-----}
  27    3   125
       g
\end{verbatim}
The new operator \texttt{*.} (\texttt{ldot})\ttindextype{*.}{ (ldot) operator}
\ttindextype{ldot}{operator} implements
the dot product:
\begin{verbatim}
{a,b,c,d} *. {5,7,9,11/d};

5*a + 7*b + 9*c + 11
\end{verbatim}
For accessing list elements, the new operator \texttt{\_} (\texttt{lnth})
\ttindextype{\_}{(lnth) operator for lists}\ttindextype{LNTH}{operator}
can be used instead of the \f{PART} operator.
Note that the infix operator \texttt{\_} must be separated from the
name of the list variable to which it is applied, otherwise REDUCE
will treat the list name and operator as parts of a single identifier.
\begin{verbatim}
l := {1,{2,3},4}$

lnth(l,3);

4

l _2*3;

{6,9}

l _2 _2;

3
\end{verbatim}
It can also be used to modify a list (unlike \f{PART}, which returns a modified list):
\begin{verbatim}
part(l,2,2):=three;

{1,{2,three},4}

l;

{1,{2,3},4}

l _ 2 _2 :=three;

three

l;

{1,{2,three},4}
\end{verbatim}
Operators are distributed over lists:
\begin{verbatim}
a *. log b;

log(b1)*a1 + log(b2)*a2 + log(b3)*a3 + log(b4)*a4

df({sin x*y,x^3*cos y},x,2,y);

{ - sin(x), - 6*sin(y)*x}

int({sin x,cos x},x);

{ - cos(x),sin(x)}
\end{verbatim}

Finally, here are two examples of using vector operations on lists
within procedures that return lists:
\begin{verbatim}
listproc normalize v;
   v / sqrt(v *. v);
\end{verbatim}
\begin{verbatim}
listproc spat3(u,v,w);
   begin scalar x,y;
     x := u *. w;
     y := u *. v;
     return v*x - w*y
   end;
\end{verbatim}
