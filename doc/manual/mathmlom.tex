To use the OpenMath/MathML Interface, the package must first be loaded
explicitly by executing the command
\begin{verbatim}
     load_package mathmlom;
\end{verbatim}

%% FJW: The switches described in "mathmlom_user.pdf" don't appear to
%% work!

The following commands translate subsequent input from one standard to
the other:
\begin{description}
  \item[\texttt{om2mml();}] translates OpenMath into MathML;
  \item[\texttt{mml2om();}] translates MathML into OpenMath.
\end{description}
Execute one of the above commands, then input one complete expression
using the appropriate standard.  REDUCE will outputs its intermediate
Lisp representation followed by the expression translated to the other
standard, and then revert to normal REDUCE input syntax.

Here is a simple example of translating OpenMath into MathML taken
from the end of the file \texttt{mathmlom.rlg}.  The following input
\begin{verbatim}
     om2mml();
     <OMOBJ>
       <OMA>
         <OMS name="rational" cd="nums1"/>
         <OMI>4</OMI>
         <OMI>2</OMI>
       </OMA>
     </OMOBJ>
\end{verbatim}
produces the following output:
\begin{verbatim}
     Intermediate representation:
     (rational nil 4 2)

     <math>
        <cn type="rational">4<sep/>2</cn>
     </math>
\end{verbatim}
