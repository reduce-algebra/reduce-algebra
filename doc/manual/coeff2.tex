%
% Redistribution and use in source and binary forms, with or without
% modification, are permitted provided that the following conditions are met:
%
%    * Redistributions of source code must retain the relevant copyright
%      notice, this list of conditions and the following disclaimer.
%    * Redistributions in binary form must reproduce the above copyright
%      notice, this list of conditions and the following disclaimer in the
%      documentation and/or other materials provided with the distribution.
%
% THIS SOFTWARE IS PROVIDED BY THE COPYRIGHT HOLDERS AND CONTRIBUTORS "AS IS"
% AND ANY EXPRESS OR IMPLIED WARRANTIES, INCLUDING, BUT NOT LIMITED TO,
% THE IMPLIED WARRANTIES OF MERCHANTABILITY AND FITNESS FOR A PARTICULAR
% PURPOSE ARE DISCLAIMED. IN NO EVENT SHALL THE COPYRIGHT OWNERS OR
% CONTRIBUTORS
% BE LIABLE FOR ANY DIRECT, INDIRECT, INCIDENTAL, SPECIAL, EXEMPLARY, OR
% CONSEQUENTIAL DAMAGES (INCLUDING, BUT NOT LIMITED TO, PROCUREMENT OF
% SUBSTITUTE GOODS OR SERVICES; LOSS OF USE, DATA, OR PROFITS; OR BUSINESS
% INTERRUPTION) HOWEVER CAUSED AND ON ANY THEORY OF LIABILITY, WHETHER IN
% CONTRACT, STRICT LIABILITY, OR TORT (INCLUDING NEGLIGENCE OR OTHERWISE)
% ARISING IN ANY WAY OUT OF THE USE OF THIS SOFTWARE, EVEN IF ADVISED OF THE
% POSSIBILITY OF SUCH DAMAGE.
%
% 2020.02.27     M. Ito

\index{Kako, Fujiko}\index{People!Kako, Fujiko}
\index{Ito, Masaaki}\index{People!Ito, Masaaki}

\ttindextype[COEFF2]{coeff2}{operator}
\ttindextype[COEFF2]{nm}{operator}
\ttindextype[COEFF2]{eval2}{operator}
In \REDUCE,  we can use the \f{coeff} operator which returns a list of coefficients of a polynomial
with respect to  specified variables. 
On the other hand, the \hypertarget{operator:COEFF2}{}\f{coeff2} operator gives a polynomial in which each coefficient is replaced by
special variables \verb|#1,#2|,$\cdots$. It is used with the same syntax as the \f{coeff} operator:
\begin{syntax}
  \f{coeff2(}\meta{exprn:polynomial},\,\meta{var:kernel}\texttt{)}\,:\,\textit{algebraic}
\end{syntax}
\textit{Example:}
\begin{verbatim}
off allfac;
f := (a+b)^2*x^2*y+(c+d)^2*x*y;
f2 := coeff2(f,x,y);
g := (2*c+d)*x^2+(3+a)*x*y^3;
g2 := coeff2(g,x,y);
\end{verbatim}
would result in the output
\begin{verbatim}
      2  2            2      2  2      2                    2
f := a *x *y + 2*a*b*x *y + b *x *y + c *x*y + 2*c*d*x*y + d *x*y

          2
f2 := #1*x *y + #2*x*y

          3        2      2        3
g := a*x*y  + 2*c*x  + d*x  + 3*x*y

          2         3
g2 := #3*x  + #4*x*y
\end{verbatim}
If you want to retrieve the values of  special variables \verb|#1,#2|,$\cdots$,
we can use the operator \hypertarget{operator:NM}{}\f{nm}. The syntax for this is:
\begin{syntax}
  \f{nm(}\meta{n:integer}\texttt{)}\,:\,\textit{algebraic}
\end{syntax}
It returns the value of the variable \verb|#n|. For example, to get the value of 
\verb|#1| in the above, one could say:
\begin{verbatim}
nm(1);
\end{verbatim}
yields the result
\begin{verbatim}
 2            2
a  + 2*a*b + b
\end{verbatim}
It is also possible to evaluate an expression including  special variables \verb|#1,#2,|$\cdots$
by using the \hypertarget{operator:EVAL2}{}\f{eval2} operator.
The syntax for this is:
\begin{syntax}
  \f{eval2(}\meta{exprn:rational}\texttt{)}\,:\,\textit{algebraic}
\end{syntax}
\textit{Example:}
\begin{verbatim}
coeff2(f2*g2,x,y);

    4         3  4       3         2  4
#5*x *y + #6*x *y  + #7*x *y + #8*x *y

nm(8);

#2*#4

eval2(ws);

   2                2      2              2
a*c  + 2*a*c*d + a*d  + 3*c  + 6*c*d + 3*d

\end{verbatim}
\ttindextype[COEFF2]{reset}{operator}
The user may remove all values of special variables \verb|#1,#2|,$\cdots$
by the operator \hypertarget{operator:RESET}{}\f{reset}, in the form
\begin{verbatim}
       reset( );
\end{verbatim}

