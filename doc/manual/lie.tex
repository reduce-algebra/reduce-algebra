
\index{Sch\"obel, Carsten}\index{People!Sch\"obel, Carsten}
\index{Sch\"obel, Franziska}\index{People!Sch\"obel, Franziska}

\subsection{liendmc1}

With the help of the functions in this module real $n$-dimensional Lie algebras
$L$ with a derived algebra $L^{(1)}$ of dimension $1$ can be classified. $L$ has
to be defined by its structure constants $c_{ij}^k$ in the basis
$\{X_1,\ldots,X_n\}$ with $[X_i,X_j]=c_{ij}^k X_k$. The user must define an
\f{array lienstrucin(n,n,n)} with n being the dimension of the Lie algebra $L$.
The structure constants lienstrucin($i,j,k$):=$c_{ij}^k$ for $i<j$ should be
given. Then the procedure \f{liendimcom1} can be called. Its syntax is:
\ttindextype[LIE]{liendimcom1}{procedure}
\hypertarget{procedure:LIENDIMCOM1}{}
\begin{syntax}
   \f{liendimcom1(}\meta{number}\f{)} .
\end{syntax}
\meta{number} corresponds to the dimension $n$. The procedure simplifies
the structure of $L$ performing real linear transformations. The returned
value is a list of the form
\begin{enumerate}[(i)]
  \item \{LIE\_ALGEBRA(2),COMMUTATIVE(n-2)\} or
  \item \{HEISENBERG(k),COMMUTATIVE(n-k)\}
\end{enumerate}
with $3\leq k\leq n$, $k$ odd.

The concepts correspond to the following theorem (\texttt{LIE\_ALGEBRA(2)}
$\rightarrow L_2$, \texttt{HEISENBERG(k)} $\rightarrow H_k$ and
\texttt{COMMUTATIVE(n-k)} $\rightarrow C_{n-k}$):\\[0.2cm]
\textbf{Theorem.} Every real $n$-dimensional Lie algebra $L$ with a 1-dimensional
derived algebra can be decomposed into one of the following forms:
\begin{enumerate}[(1)]
\item $C(L)\cap L^{(1)}=\{0\}\, :\; L_2\oplus C_{n-2}$
or
\item $C(L)\cap L^{(1)}=L^{(1)}\, :\; H_k\oplus C_{n-k}\quad
  (k=2r-1,\, r\geq 2)$,
\end{enumerate}
with
\pagebreak[2]
\begin{enumerate}[1.]
\item $C(L)=C_j\oplus (L^{(1)}\cap C(L))$
and dim$\,C_j=j$ ,
\item $L_2$ is generated by
$Y_1,Y_2$ with $[Y_1,Y_2]=Y_1$ ,
\item $H_k$ is generated by $\{Y_1,\ldots,Y_k\}$ with\\
 $[Y_2,Y_3]=\cdots =[Y_{k-1},Y_k]=Y_1$.
\end{enumerate}
(cf. \cite{Schoebel:93})

\ttindextype[LIE]{lie\_list}{variable}
\ttindextype[LIE]{lientrans}{matrix}
\hypertarget{reserved:LIE_LIST}{}
\hypertarget{reserved:LIENTRANS}{}
The returned list is also stored as \var{lie\_list}. The matrix \var{lientrans} gives the
transformation from the given basis $\{X_1,\ldots ,X_n\}$ into the standard
basis $\{Y_1,\ldots ,Y_n\}$: $Y_j=($LIENTRANS$)_j^k X_k$.

A more detailed output can be obtained by turning on the switch \sw{tr\_lie}:
\ttindexswitch[LIE]{tr\_lie}
\hypertarget{switch:TR_LIE}{}
before the procedure \f{liendimcom1} is called.

The returned list could be an input for a data bank in which mathematical
relevant properties of the obtained Lie algebras are stored.

\subsection{lie1234}

This part of the package classifies real low-dimensional Lie algebras $L$
of the dimension
$n:=$dim$\,L=1,2,3,4$. $L$ is also given by its structure constants $c_{ij}^k$
in the basis $\{X_1,\ldots,X_n\}$ with $[X_i,X_j]=c_{ij}^k X_k$. An ARRAY
LIESTRIN($n,n,n$) has to be defined and LIESTRIN($i,j,k$):=$c_{ij}^k$ for
$i<j$ should be given. Then the procedure \f{lieclass} can be called
whose syntax is:
\ttindextype[LIE]{lieclass}{procedure}
\hypertarget{procedure:LIECLASS}{}
\begin{syntax}
   \f{lieclass(}\meta{number}\f{)} .
\end{syntax}
\meta{number} should be the dimension of the Lie algebra $L$. The procedure
stepwise simplifies the commutator relations of $L$ using properties of
invariance like the dimension of the centre, of the derived algebra,
unimodularity etc.  The returned value has the form:
\begin{verbatim}
   {LIEALG(n),COMTAB(m)},
\end{verbatim}
where $m$ corresponds to the number of the standard form (basis:
$\{Y_1,\ldots,Y_n\}$) in an enumeration scheme. The corresponding enumeration
schemes are listed below (cf. \cite{Schoebel:92},\cite{MacCallum:99}).
In case that the standard form in the enumeration scheme depends on one (or two)
parameter(s) $p_1$ (and $p_2$) the list is expanded to:
\begin{verbatim}
   {LIEALG(n),COMTAB(m),p1,p2}.
\end{verbatim}
\ttindextype[LIE]{lie\_class}{variable}
\ttindextype[LIE]{liemat}{matrix}
\hypertarget{reserved:LIE_CLASS}{}
\hypertarget{reserved:LIEMAT}{}
This returned value is also stored as \var{lie\_class}. The linear transformation from
the basis $\{X_1,\ldots,X_n\}$ into the basis of the standard form
$\{Y_1,\ldots,Y_n\}$ is given by the matrix \var{liemat}:
$Y_j=($LIEMAT$)_j^k X_k$.

By turning on the switch \sw{tr\_lie}
before the procedure \f{lieclass} is called the output contains not only the
list \var{lie\_class} but also the non-vanishing commutator relations in the
standard form.

By the value $m$ and the parameters further examinations of the Lie algebra
are possible, especially if in a data bank mathematical relevant properties
of the enumerated standard forms are stored.

\subsection{Enumeration schemes for lie1234}

\begin{setlength}{\extrarowheight}{1.5mm}
\begin{longtable}{l|l}
  returned list \var{lie\_class} & the corresponding commutator relations \\
  \hline
  \endhead
{LIEALG(1),COMTAB(0)}&commutative case\\\hline
{LIEALG(2),COMTAB(0)}&commutative case\\
{LIEALG(2),COMTAB(1)}&$[Y_1,Y_2]=Y_2$\\\hline
{LIEALG(3),COMTAB(0)}&commutative case\\
{LIEALG(3),COMTAB(1)}&$[Y_1,Y_2]=Y_3$\\
{LIEALG(3),COMTAB(2)}&$[Y_1,Y_3]=Y_3$\\
{LIEALG(3),COMTAB(3)}&$[Y_1,Y_3]=Y_1,[Y_2,Y_3]=Y_2$\\
{LIEALG(3),COMTAB(4)}&$[Y_1,Y_3]=Y_2,[Y_2,Y_3]=Y_1$\\
{LIEALG(3),COMTAB(5)}&$[Y_1,Y_3]=-Y_2,[Y_2,Y_3]=Y_1$\\
{LIEALG(3),COMTAB(6)}&$[Y_1,Y_3]=-Y_1+p_1 Y_2,[Y_2,Y_3]=Y_1,p_1\neq 0$\\
{LIEALG(3),COMTAB(7)}&$[Y_1,Y_2]=Y_3,[Y_1,Y_3]=-Y_2,[Y_2,Y_3]=Y_1$\\
{LIEALG(3),COMTAB(8)}&$[Y_1,Y_2]=Y_3,[Y_1,Y_3]=Y_2,[Y_2,Y_3]=Y_1$\\\hline
{LIEALG(4),COMTAB(0)}&commutative case\\
{LIEALG(4),COMTAB(1)}&$[Y_1,Y_4]=Y_1$\\
{LIEALG(4),COMTAB(2)}&$[Y_2,Y_4]=Y_1$\\[0,1cm]
{LIEALG(4),COMTAB(3)}&$[Y_1,Y_3]=Y_1,[Y_2,Y_4]=Y_2$\\
{LIEALG(4),COMTAB(4)}&$[Y_1,Y_3]=-Y_2,[Y_2,Y_4]=Y_2,$\\
                     &$[Y_1,Y_4]=[Y_2,Y_3]=Y_1$\\
{LIEALG(4),COMTAB(5)}&$[Y_2,Y_4]=Y_2,[Y_1,Y_4]=[Y_2,Y_3]=Y_1$\\
{LIEALG(4),COMTAB(6)}&$[Y_2,Y_4]=Y_1,[Y_3,Y_4]=Y_2$\\
{LIEALG(4),COMTAB(7)}&$[Y_2,Y_4]=Y_2,[Y_3,Y_4]=Y_1$\\
{LIEALG(4),COMTAB(8)}&$[Y_1,Y_4]=-Y_2,[Y_2,Y_4]=Y_1$\\
{LIEALG(4),COMTAB(9)}&$[Y_1,Y_4]=-Y_1+p_1 Y_2,[Y_2,Y_4]=Y_1,p_1\neq 0$\\
{LIEALG(4),COMTAB(10)}&$[Y_1,Y_4]=Y_1,[Y_2,Y_4]=Y_2$\\
{LIEALG(4),COMTAB(11)}&$[Y_1,Y_4]=Y_2,[Y_2,Y_4]=Y_1$ \\
{LIEALG(4),COMTAB(12)}&$[Y_1,Y_4]=Y_1+Y_2,[Y_2,Y_4]=Y_2+Y_3,$\\
                      &$[Y_3,Y_4]=Y_3$\\
{LIEALG(4),COMTAB(13)}&$[Y_1,Y_4]=Y_1,[Y_2,Y_4]=p_1 Y_2,[Y_3,Y_4]=p_2 Y_3,$\\
                      &$p_1,p_2\neq 0$\\
{LIEALG(4),COMTAB(14)}&$[Y_1,Y_4]=p_1 Y_1+Y_2,[Y_2,Y_4]=-Y_1+p_1 Y_2,$\\
                      &$[Y_3,Y_4]=p_2 Y_3,p_2\neq 0$\\
{LIEALG(4),COMTAB(15)}&$[Y_1,Y_4]=p_1 Y_1+Y_2,[Y_2,Y_4]=p_1 Y_2,$\\
                      &$[Y_3,Y_4]=Y_3,p_1\neq 0$\\
{LIEALG(4),COMTAB(16)}&$[Y_1,Y_4]=2 Y_1,[Y_2,Y_3]=Y_1,$\\
                      &$[Y_2,Y_4]=(1+p_1) Y_2,[Y_3,Y_4]=(1-p_1) Y_3,$\\
                      &$p_1\geq 0$\\
{LIEALG(4),COMTAB(17)}&$[Y_1,Y_4]=2 Y_1,[Y_2,Y_3]=Y_1,$\\
                      &$[Y_2,Y_4]=Y_2-p_1 Y_3,[Y_3,Y_4]=p_1 Y_2+Y_3,$\\
                      &$p_1\neq 0$\\
{LIEALG(4),COMTAB(18)}&$[Y_1,Y_4]=2 Y_1,[Y_2,Y_3]=Y_1,$\\
                      &$[Y_2,Y_4]=Y_2+Y_3,[Y_3,Y_4]=Y_3$\\
{LIEALG(4),COMTAB(19)}&$[Y_2,Y_3]=Y_1,[Y_2,Y_4]=Y_3,[Y_3,Y_4]=Y_2$\\
{LIEALG(4),COMTAB(20)}&$[Y_2,Y_3]=Y_1,[Y_2,Y_4]=-Y_3,[Y_3,Y_4]=Y_2$\\
{LIEALG(4),COMTAB(21)}&$[Y_1,Y_2]=Y_3,[Y_1,Y_3]=-Y_2,[Y_2,Y_3]=Y_1$\\
{LIEALG(4),COMTAB(22)}&$[Y_1,Y_2]=Y_3,[Y_1,Y_3]=Y_2,[Y_2,Y_3]=Y_1$
\end{longtable}
\end{setlength}