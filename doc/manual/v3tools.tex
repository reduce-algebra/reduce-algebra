\subsection{Purpose}

Procedures in this package perform computationally expensive and
non-trivial computations with polynomials of scalar vector products of
3-component vectors.  The individual procedures can be grouped into
the following categories:
\begin{itemize}
\item initialization of the package,
\item generation of scalar vector expressions according to a multiple
      weighting scheme where each vector has multiple weights,
\item conversions of scalar vector expressions between different
      representations,
\item the computation of Poisson brackets between vector expressions,
\item a computation determining whether a given scalar vector expression
      is functionally dependent on a list of other scalar vector expressions.
\end{itemize}
A possible application of these procedures is the classification of
integrable `vectorial' Hamiltonians, the study whether obtained
expressions are functionally independent of known expressions and the
compactification of results for publication.

These routines were designed and implemented to support the
classification of integrable Hamiltonians.  Examples are chosen from
this application.  More details are given in \cite{Sokolov_2006}.

\subsection{Notation}

The chosen conventions aim to display large scalar vector expressions
as compactly as possible.  All vectors have 3 components and are
represented by a single letter, like $A$ or $B$.  In this manual we
use capital letters in mathematical formulas and lower case letters
for REDUCE input and output.  REDUCE is not case sensitive but output
(normally) displays lower case letters.  Vector components have an
extra digit 1, 2 or 3, like $A2, B3$.  Products of vectors are
represented by a single identifier where the following convention is
used: $\mathit{ABCD}\ldots$ stands for $(A\ ,\ B \times (C \times (D
\times\ldots)))$ where $(\ ,\ )$ denotes the symmetric scalar product
and $\times$ the skew-symmetric vector product.  For example, we have
$\mathit{AB} = (A,B)$, which is the normal scalar product $A \cdot B$,
and $\mathit{ABC}=(A,B\times C)$, which is the normal scalar triple
product $A \cdot B \times C$.  In the following we distinguish between
three forms of scalar vector expressions:
\begin{itemize}
\item \textbf{the component form} involving only the components
  $A1,A2,A3,B1,\ldots$,
\item \textbf{the standard vector form} involving only scalar products
  $\mathit{AB}\,(=(A,B))$ and triple products $\mathit{ABC}\,(=(A,B
  \times C))$,
\item \textbf{the extended vector form} involving any product
  $\mathit{ABCD}\ldots$.
\end{itemize}

Because of the commutativity of the scalar product $(A,B)=(B,A)$ there
seem to be two identifiers $\mathit{AB}$ and $\mathit{BA}$ suited to
represent the same product.  Similarly for triple products we have the
relation
$\mathit{ABC}=\mathit{BCA}=\mathit{CAB}=-\mathit{ACB}=-\mathit{CBA}=-\mathit{BAC}$.
In order to have an unambiguous notation such that vanishing rational
expressions simplify to zero we adopt the convention that for scalar
products and triple products only identifiers are used in which
letters are sorted alphabetically.  For example, the program uses
$\mathit{AB},\mathit{AV},\mathit{ABV},\mathit{BUV}$ but not
$\mathit{BA},\mathit{VA},\mathit{BVA},\mathit{VBU}$.  In order to
simplify manual input one can assign expressions using identifiers,
like $\mathit{BA},\mathit{BUA}$, and afterwards convert them into
proper notation using the procedure \texttt{e2s} (see below or the
beginning of the file \texttt{packages/crack/v3tools.tst}).

In order to use non-vectorial parameters, like $\mathit{ALPHA}$ or
$\mathit{BETA2}$ every non-vectorial parameter is preceded by
\texttt{!\&}, e.g.\ one would input \texttt{!\&alpha} or
\texttt{!\&beta2}.

\subsection{Initialization}
Scalar products and scalar triple products of 3-component vectors
satisfy identities, for example, the four vectors $A,B,U,V$ satisfy
\[ 0 = AB\ BUV - ABU\ BV + ABV\ BU - AUV\ BB. \]
Therefore, scalar vector expressions, i.e.\ polynomials of scalar
vector products, usually do not have a unique form.  They can be
re-written using these identities, for example, in order to minimize
the number of terms in the polynomial or to bring the expression into
a canonical form.  Sometimes computations have to be done modulo these
identities, more precisely, modulo a Gr\"{o}bner basis of these
identities.  The computation of the identities and their Gr\"{o}bner
basis has to be done only once with the procedure \texttt{vinit} which
initializes the global variables \texttt{v\_}, \texttt{gbase\_},
\texttt{heads\_}.  The procedure \texttt{vinit} takes as argument a
list of all vectors that occur in the computation, for example,
\texttt{vinit(\{u,v,w,z\})}.  In the default case of vectors
\texttt{a}, \texttt{b}, \texttt{u}, \texttt{v} the global variables
\texttt{v\_}, \texttt{gbase\_}, \texttt{heads\_} are already assigned
and \texttt{vinit} does not have to be called initially.  But if some
of the vectors satisfy a special relation, like $\mathit{AB}=0$, then
this relation should be assigned (like \texttt{ab:=0;}) and
\texttt{vinit} should be run afterwards.  Here is an overview of the
three global variables:
\begin{description}
\item[\texttt{v\_}] : a list of vectors involved in all the
  computations.  The global variable \texttt{v\_} is needed in the
  procedure \texttt{poisson\_v}.  The default value is
  \texttt{v\_=\{a,b,u,v\}}.
\item[\texttt{gbase\_}] : a list of polynomials which are a
  Gr\"{o}bner basis for all identities between scalar products
  $\mathit{AB}$ and triple products $\mathit{ABC}$ of any vectors in
  the list \texttt{v\_}.
\item[\texttt{heads\_}] : a list of leading terms of the polynomials
  in \texttt{gbase\_}.
\end{description}

\subsection{Main Routines}

Which routines work only for homogeneous vectorial expressions and
which for any vectorial expressions?
\begin{description}
\item[\texttt{vinit}] : initializes all three global variables
  \texttt{vl\_}, \texttt{gbase\_}, \texttt{heads\_} for a given list
  of vectors.  This procedure is not necessary if the list of vectors
  is $A,B,U,V$.  The procedure is necessary if some of the vectors
  satisfy extra conditions, like $\mathit{AB}=0$.  Example:

  \verb|vinit({a,b,c,d});|

\item[\texttt{e2s}] : converts a vectorial polynomial from extended
  vector form into standard vector form by replacing products
  $\mathit{ABCD}\ldots$ through scalar and triple products.  Example:

  \verb|e2s(buuav);| \quad $\longrightarrow$ \quad \verb|abv*uu - auv*bu|

  It is also useful to convert from standard vector form into standard
  vector form if one wants to sort factors in scalar and triple
  products lexicographically.  This is useful for working with this
  package on expressions that have been generated outside this package
  and that do not obey the lexicographical ordering of factors.
  Example:

  \verb|e2s(avb*uu + uva*bu);| \quad $\longrightarrow$ \quad
  \verb|- abv*uu + auv*bu|

\item[\texttt{v2c}] : converts an expression from any vector form
  (extended or standard) into component form. Example:

  \verb|v2c(abu);| \quad $\longrightarrow$%
\begin{verbatim}
a1*b2*u3 - a1*b3*u2 - a2*b1*u3 + a2*b3*u1 +
a3*b1*u2 - a3*b2*u1
\end{verbatim}
\item[\texttt{c2sl}] : converts an expression from component form into
  standard vector form.  The resulting vector expression is returned
  partitioned as a list of sublists.  Each sublist contains
  expressions with the same multiplicity of each vector (the same
  homogeneity).  The first expression of each list is the vector
  equivalent of the corresponding input terms in component form.  All
  other expressions in each sublist are identically vanishing vector
  expressions, i.e.\ vector identities with the same multiplicity as
  the first expression in each sublist.  That means that the second,
  third, \ldots\ expression in each sublist multiplied with a constant
  could be added to the first expression of each sublist in order to
  change its form but not its value.  Example:
\begin{verbatim}
c2sl(a1*b2*u3 - a1*b3*u2 - a2*b1*u3 +
a2*b3*u1 + a3*b1*u2 - a3*b2*u1 +
a1*b1*b2*u3*v1 - a1*b1*b2*u1*v3 + a1*b1*b3*u1*v2 -
a1*b1*b3*u2*v1 - a1*b2**2*u2*v3 + a1*b2**2*u3*v2 -
a1*b3**2*u2*v3 + a1*b3**2*u3*v2 + a2*b1**2*u1*v3 -
a2*b1**2*u3*v1 + a2*b1*b2*u2*v3 - a2*b1*b2*u3*v2 +
a2*b2*b3*u1*v2 - a2*b2*b3*u2*v1 + a2*b3**2*u1*v3 -
a2*b3**2*u3*v1 - a3*b1**2*u1*v2 + a3*b1**2*u2*v1 +
a3*b1*b3*u2*v3 - a3*b1*b3*u3*v2 - a3*b2**2*u1*v2 +
a3*b2**2*u2*v1 - a3*b2*b3*u1*v3 + a3*b2*b3*u3*v1);
\end{verbatim}
$\longrightarrow$%
\begin{verbatim}
{{abu},{abu*bv - abv*bu,
        ab*buv - abu*bv + abv*bu - auv*bb}}
\end{verbatim}
The input expression is equal to the sum of the first expressions of
  all sublists, i.e.\ equal to $\mathit{ABU} + \mathit{ABU}
  \mathit{BV} - \mathit{ABV} \mathit{BU}$.  The second expression of
  the second sublist is an identity, i.e.\ $0=\mathit{AB} \mathit{BUV}
  - \mathit{ABU} \mathit{BV} + \mathit{ABV} \mathit{BU} - \mathit{AUV}
  \mathit{BB}$, where each vector occurs in each term equally often as
  in each term of the first expression in the second sublist.

\item[\texttt{c2s}] : is equivalent to \texttt{c2sl} with the
  difference that all expressions of all sublists are added up with
  all identity expressions obtaining coefficients \verb|!&c1|,
  \verb|!&c2|,\ldots.  Example:

  \texttt{c2s( \textit{same input as for c2sl above} );}
   \quad $\longrightarrow$%
\begin{verbatim}
abu + abu*bv - abv*bu +
      !&c1*(ab*buv - abu*bv + abv*bu - auv*bb)
\end{verbatim}

\item[\texttt{s2s}] : generates and uses vector identities to reduce
  the length of an expression in standard vector form.  Example:

  \verb|s2s( - abu*vv + abv*uv + av*buv);| \quad $\longrightarrow$ \quad
  \verb|auv*bv|

\item[\texttt{s2e}] : like \texttt{s2s} but also uses identities
  involving products $\mathit{ABCD}\ldots$ and thus transforms
  standard vector form into extended vector form.  Example:

  \verb|s2e(abv*uu - auv*bu);| \quad $\longrightarrow$ \quad \verb|buuav|

  In its current form it only uses products that involve all the
  vectors involved in each term of the standard vector form.  In the
  above example it would not try to use products with 4 factors but
  only with 5 factors, like \texttt{buuav}.

\item[\texttt{genpro}] : generation of all scalar and triple products
  for a given vector list.  Example:

  \verb|genpro({a,b,u,v});| \quad $\longrightarrow$ \\
  \verb|{aa,ab,au,av,bb,bu,bv,uu,uv,vv,abu,abv,auv,buv}|

\item[\texttt{genpro\_wg}] : similar to \texttt{genpro} with the
  difference that each vector is accompanied by a list of weights and
  the resulting scalar and triple products also come with a list of
  weights.  Example:

\begin{verbatim}
genpro_wg({{a,1,0,0},{b,0,1,0},{u,0,0,1},
           {v,1,0,1}});
\end{verbatim}%
$\longrightarrow$%
\begin{verbatim}
{{aa,2,0,0},{ab,1,1,0},{au,1,0,1},{av,2,0,1},
 {bb,0,2,0},{bu,0,1,1},{bv,1,1,1},{uu,0,0,2},
 {uv,1,0,2},{vv,2,0,2},{abu,1,1,1},{abv,2,1,1},
 {auv,2,0,2},{buv,1,1,2}}
\end{verbatim}

\item[poisson\_c] : computes the poisson bracket of two arbitrary
  expressions.  This procedure needs as input the Poisson structure
  matrix which we call \texttt{struc\_cons} below.  This is encoded as
  a list of lists, specifying all non-vanishing Poisson brackets
  between any two dynamical variables, here
  \texttt{u1,u2,u3,v1,v2,v3}.  For example, the first sublist
  \texttt{\{u1,u2,u3\}} of \texttt{struc\_cons} below, encodes the
  Poisson bracket relation $\{U1,U2\}=-\{U2,U1\}=U3$.  Poisson
  brackets associated to other Lie-algebras are appended to the file
  \texttt{packages/crack/v3tools.red}.  The structure matrix below
  encodes the Lie-Poisson bracket $e(3)$ for \texttt{!\&kap=0},
  $so(4)$ for \texttt{!\&kap=1} and $so(3,1)$ for \texttt{!\&kap=-1}.
  \texttt{poisson\_c} needs \emph{all} of its arguments in component
  form.  Example:
\begin{verbatim}
ham:=v2c(ab*uu-2*au*bu+buv)$
fi :=v2c(bu*(2*auv-!&kap*uu+vv))$
!&kap:=-a1**2-a2**2-a3**2$
struc_cons:={{u1,u2, u3}, {u2,u3, u1}, {u3,u1, u2},
             {u1,v2, v3}, {u2,v3, v1}, {u3,v1, v2},
             {u1,v3,-v2}, {u2,v1,-v3}, {u3,v2,-v1},
             {v1,v2, !&kap*u3}, {v2,v3, !&kap*u1},
             {v3,v1, !&kap*u2}}$
\end{verbatim}
\verb|poisson_c(ham,fi,struc_cons);| \quad $\longrightarrow$ \quad 0

The example shows that \texttt{fi} is a first integral of \texttt{ham}
under the assumption \texttt{!\&kap = - aa}.  Note that
\texttt{!\&kap} could not have been expressed in terms of components
of vectors before assigning \texttt{fi} as then the argument to
\texttt{v2c} would not be a purely vectorial expression.

\item[poisson\_v] : computes the Poisson bracket for two vector
  expressions.  It uses the global variable \texttt{gbase\_}.  When
  \texttt{poisson\_v} is called the first time it computes the Poisson
  bracket between any scalar and triple product once using the
  procedure \texttt{poisson\_c} and stores these elementary Poisson
  brackets under the global operator \texttt{poi\_}.  Therefore, the
  third argument to \texttt{poisson\_v} (the Poisson structure matrix)
  has to be given in component form.  Example: Compare the following
  with the example for the call of \texttt{poisson\_c}.
\begin{verbatim}
ham:=ab*uu-2*au*bu+buv$
fi :=bu*(2*auv-!&kap*uu+vv)$
!&kap:=-a1**2-a2**2-a3**2$
struc_cons:={{u1,u2, u3}, {u2,u3, u1}, {u3,u1, u2},
             {u1,v2, v3}, {u2,v3, v1}, {u3,v1, v2},
             {u1,v3,-v2}, {u2,v1,-v3}, {u3,v2,-v1},
             {v1,v2, !&kap*u3}, {v2,v3, !&kap*u1},
             {v3,v1, !&kap*u2}}$
!&kap:=-aa$
\end{verbatim}
\verb|poisson_v(ham,fi,struc_cons);| \quad $\longrightarrow$ \quad 0

  When calling \texttt{poisson\_v} a second time the computation
  proceeds much faster as the Poisson brackets between scalar and
  triple products had been stored with the operator \texttt{poi\_}.
  Note that at first \texttt{!\&kap} is expressed in component form to
  have \texttt{struc\_cons} in component form but afterwards
  \texttt{!\&kap} is expressed in vector form to have \texttt{ham,fi}
  in vector form at the time of calling \texttt{poisson\_v}.

\item[gfi] : generates a vector expression with unknown coefficients
  according to a specific list of weights $\{w_1,w_2,\ldots\}$.  The
  number of weights is arbitrary but fixed.  For example, the vector
  $V$ could be given weights $\{w_1,w_2,w_3\}=\{1,0,1\}$ which will be
  input as \texttt{\{v,1,0,1\}} in the second argument to
  \texttt{gfi}.  In the following example vectors $A,B,U,V$ are each
  given 3 weights $\{w_1,w_2,w_3\}$ and the most general polynomial is
  generated where the sums of all $\{w_1,w_2,w_3\}$ values of all
  vectors in each term are equal to $\{2,1,3\}$.  Example:
\begin{verbatim}
gfi({2,1,3},
    {{a,1,0,0},{b,0,1,0},{u,0,0,1},{v,1,0,1}},
    {aa, ab, bb, bu, uv, -aa*uu + vv,
     ab*uu - 2*au*bu + buv},
    heads_);
\end{verbatim}
$\longrightarrow$
\begin{verbatim}
{&r1*abu*uv + &r2*abv*uu + &r3*bv*uv +
 &r4*auv*bu + &r5*bu*vv +
       2
 &r6*au *bu + &r7*ab*au*uu,
 &r7,&r6,&r5,&r4,&r3,&r2,&r1}
\end{verbatim}
  The meaning of the parameters:
  \begin{enumerate}
  \item The first parameter of \texttt{gfi} is a list of the total
    weights of each term in the generated polynomial.
  \item The second parameter is the list of vectors to be used for
    generating the expression, each followed by its list of weight
    values, like \texttt{\{a,1,0,0\}}.
  \item The third parameter is a list of homogeneous vector
    expressions.  The polynomial returned by \texttt{gfi} must not
    include any expression that is functionally dependent on the
    expressions in this list.  In other words, from the most general
    polynomial with proper homogeneity that is generated at first
    within \texttt{gfi} terms are dropped so that it is not possible
    to choose in the remaining polynomial the undetermined
    coefficients \texttt{!\&r1}, \texttt{!\&r2}, \ldots\ such that an
    expression results which is functionally dependent only on the
    expressions of the third parameter list.  In this example we want
    to formulate an ansatz for a first integral that is functionally
    independent of the expressions \texttt{aa}, \texttt{ab},
    \texttt{bb} (because $A,B$ are assumed to be constant vectors and
    $U,V$ to be dynamical vectors), \texttt{bu} (a known first
    integral), \texttt{uv}, \texttt{-aa*uu + vv} (two Casimirs,
    i.e.\ first integrals) and \texttt{ab*uu - 2*au*bu + buv} which is
    the Hamiltonian.  The third parameter prevents the generation of
    the term \texttt{bu*aa*uu} which also has weights $\{2,1,3\}$ (as
    \texttt{b} has weights $\{0,1,0\}$, \texttt{u} has weights
    $\{0,0,1\}$ and \texttt{a} has weights $\{1,0,0\}$).  The term
    \texttt{bu*aa*uu} is dropped because the resulting polynomial
    would otherwise contain the functionally dependent expression
    \texttt{bu*(-aa*uu+vv)}.

  \item The fourth parameter is the list of leading terms of the
    Gr\"{o}bner basis of identities between the vectors.  The
    resulting polynomial contains only terms which are not a multiple
    of any one of the terms in this list.  By excluding terms which
    are not multiples of leading terms of identities in the
    Gr\"{o}bner basis one defines a canonical form which vanishes only
    if all terms vanish.
  \end{enumerate}

  The result of \texttt{gfi} is a list.  Its first element is the most
  general polynomial that has the required properties.  The list of
  undetermined coefficients \texttt{!\&r}\ldots\ is appended.

\item[\texttt{fnc\_dep}] : investigates whether a given vectorial
  expression (the first parameter) is functionally dependent on the
  elements of a list (the second parameter).  The third parameter is
  the list of occurring vectors with their weights.  In the following
  example it is to be checked whether the first integral
  \texttt{finew} is new or merely the known one \texttt{fiold} in
  disguised form.  Example:
{\small\begin{verbatim}
ham:=ab*uu - 1/2*au*bu + buv$
fiold:=bu**2*((bu*aa-ab*au)**2*uu +
              2*(bu*aa-ab*au)*(bu*auv-buv*au) -
              vv*abu**2 - (bu*av-bv*au)**2 -
              uu*vv*ab**2+aa*uu*vv*bb)$
finew:=( - 16*aa**2*bu**4*uu + 96*aa*ab**2*bb*uu**3 -
96*aa*ab*au*bb*bu*uu**2 + 32*aa*ab*au*bu**3*uu +
32*aa*abu*bu**3*uv - 32*aa*abv*bu**3*uu +
24*aa*au**2*bb*bu **2*uu - 16*aa*bu**4*vv +
56*ab**4*uu**3 - 84*ab**3*au*bu*uu**2 +
168*ab**2*abu*bv*uu**2 - 168*ab**2*abv*bu*uu**2 +
26*ab**2*au**2*bu**2*uu + 360*ab**2*auv*bb*uu**2 +
168*ab**2*bb*uu**2*vv - 184*ab**2*bb*uu*uv**2 -
168*ab**2*bu**2*uu*vv + 336*ab**2*bu*bv*uu*uv -
168*ab**2*bv**2*uu**2 + 264*ab*abu*au*bb*uu*uv -
32*ab* abu*au*bu**2*uv - 168*ab*abu*au*bu*bv*uu +
192*ab*abu*av*bb*uu**2 - 456*ab*abv*au*bb*uu**2 +
200*ab*abv*au*bu**2*uu - 7*ab*au**3*bu**3 -
180*ab*au*bb*bu*uu*vv + 44*ab*au*bb*bu*uv**2 +
192*ab*au*bb*bv*uu*uv + 116*ab*au*bu**3*vv -
168*ab*au* bu**2*bv*uv + 84*ab*au*bu*bv**2*uu +
192*ab*av*bb*bu*uu*uv - 192*ab*av*bb*bv*uu **2 -
264*abu*au**2*bb*bv*uu + 42*abu*au**2*bu**2*bv -
96*abu*au*av*bb*bu*uu + 96*abu*bb*bu*uv*vv -
136*abu*bb*bv*uu*vv + 24*abu*bb*bv*uv**2 -
56*abu*bu**2*bv* vv + 112*abu*bu*bv**2*uv -
56*abu*bv**3*uu + 360*abv*au**2*bb*bu*uu -
42*abv*au **2*bu**3 + 40*abv*bb*bu*uu*vv -
24*abv*bb*bu*uv**2 + 56*abv*bu**3*vv -
112*abv* bu**2*bv*uv + 56*abv*bu*bv**2*uu -
264*au**2*auv*bb**2*uu + 42*au**2*auv*bb*bu** 2 +
96*au**2*bb**2*uu*vv - 40*au**2*bb*bu**2*vv -
96*au**2*bb*bv**2*uu + 16*au** 2*bu**2*bv**2 -
192*au*av*bb**2*uu*uv + 192*au*av*bb*bu*bv*uu -
32*au*av*bu**3* bv - 136*auv*bb**2*uu*vv +
24*auv*bb**2*uv**2 + 40*auv*bb*bu**2*vv +
112*auv*bb* bu*bv*uv - 56*auv*bb*bv**2*uu +
96*av**2*bb**2*uu**2 - 96*av**2*bb*bu**2*uu +
16*av**2*bu**4 - 96*bb**2*uu*vv**2 +
96*bb**2*uv**2*vv + 96*bb*bu**2*vv**2 -
192*bb*bu*bv*uv*vv + 96*bb*bv**2*uu*vv)/8$

if fnc_dep(finew,{aa,ab,bb,bu,uv,uu*aa-vv,ham,fiold},
           {{A,1,0,0},{B,0,1,0},{U,0,0,1},{V,1,0,1}})
then write "This is a known first integral."
else write "This is a new first integral!"$
\end{verbatim}}
$\longrightarrow$
{\small\begin{verbatim}
The expression in question is functionally dependent on
the list of expressions {p_1,p_2,...} in the following way:
              2                3                 2
10*p_2*p_3*p_5 *p_7 + 7*p_2*p_7  + 12*p_3*p_6*p_7  - 2*p_8

This is a known first integral.
\end{verbatim}}
\end{description}

\subsection{Complete list of global variables}

\begin{center}
\begin{tabular}{l|l|l}
  Type      & Name & Purpose \\ \hline
  algebraic & \texttt{v\_}, \texttt{gbase\_},
                   & needed for \texttt{poisson\_v()}, \texttt{gfi()}, \\
            & ~~\texttt{heads\_}
                   & ~~\texttt{fnc\_dep()} \\
            & \texttt{!\&c1}, \texttt{!\&c2}, \ldots
                   & constant coefficients introduced by \texttt{c2s()} \\ \hline
  symbolic  &  \texttt{fino\_}, \texttt{wgths\_}
                   & used internally in \texttt{gen()} \\
            &  \texttt{!\&r1}, \texttt{!\&r2}, \ldots
                   & constant coefficients introduced by \texttt{gfi()} \\
            &  \texttt{tr\_vec}
                   & global tracing flag \\ \hline
  operator  &  \texttt{poi\_}
                   & stores Poisson brackets between \\
           &       & ~~scalar and triple products
\end{tabular}
\end{center}

\subsection{Requirements}

\begin{verbatim}
     load_package v3tools;
\end{verbatim}
