\chapter{Polynomials and Rationals}
\label{sec:core-polyrat}

Many operations in computer algebra are concerned with polynomials
\index{Polynomial} and rational functions\index{Rational function}.  In
this section, we review some of the switches and operators available for
this purpose.  These are in addition to those that work on general
expressions (such as \texttt{df} and \texttt{int}) described elsewhere.  In the
case of operators, the arguments are first simplified before the
operations are applied.  In addition, they operate only on arguments of
prescribed types, and produce a type mismatch error if given arguments
which cannot be interpreted in the required mode with the current switch
settings.  For example, if an argument is required to be a kernel and
\texttt{a/2} is used (with no other rules for \texttt{a}), an error
\begin{verbatim}
        a/2 invalid as kernel
\end{verbatim}
will result.

With the exception of those that select various parts of a polynomial or
rational function, these operations have potentially significant effects on
the space and time associated with a given calculation. The user should
therefore experiment with their use in a given calculation in order to
determine the optimum set for a given problem.

One such operation provided by the system is an operator \texttt{length}
\ttindextype{length}{operator} which returns the number of top level terms in the
numerator of its argument.  For example,
\begin{verbatim}
        length ((a+b+c)^3/(c+d));
\end{verbatim}
has the value 10.  To get the number of terms in the denominator, one
would first select the denominator by the operator \texttt{den}\ttindex{den}
and then call \texttt{length}, as in
\begin{verbatim}
        length den ((a+b+c)^3/(c+d));
\end{verbatim}
Other operations currently supported, the relevant switches and operators,
and the required argument and value modes of the latter, follow.

\section{Controlling the Expansion of Expressions}
\hypertarget{switch:EXP}{}
\ttindexswitch{exp}

The switch \sw{exp}\ttindexswitch{exp} controls the expansion of expressions.  If
it is off, no expansion of powers or products of expressions occurs.
Users should note however that in this case results come out in a normal
but not necessarily canonical form.  This means that zero expressions
simplify to zero, but that two equivalent expressions need not necessarily
simplify to the same form.

\textit{Example:} With \sw{exp} on, the two expressions
\begin{verbatim}
        (a+b)*(a+2*b)
\end{verbatim}
and
\begin{verbatim}
        a^2+3*a*b+2*b^2
\end{verbatim}
will both simplify to the latter form.  With \sw{exp}
off, they would remain unchanged, unless the complete factoring 
(\sw{allfac}) option were in force. \sw{exp} is normally on.

Several operators that expect a polynomial as an argument behave
differently when \sw{exp} is off, since there is often only one term at
the top level.  For example, with \sw{exp} off
\begin{verbatim}
        length((a+b+c)^3/(c+d));
\end{verbatim}
returns the value 1.

\section{Factorization of Polynomials}\index{Factorization}
\hypertarget{switch:FACTOR}{}
\hypertarget{switch:IFACTOR}{}

{\REDUCE} is capable of factorizing univariate and multivariate polynomials
that have integer coefficients, finding all factors that also have integer
coefficients. The package for doing this was written by Dr. Arthur C.
Norman and Ms. P. Mary Ann Moore at The University of Cambridge. It is
described in \cite{NormanMoore:1981A}.

The easiest way to use this facility is to turn on the switch
\sw{factor},\ttindexswitch{factor} which causes all expressions to be output in
a factored form.  For example, with \sw{factor} on, the expression
\texttt{a\textasciicircum2-b\textasciicircum2} is returned as 
\texttt{(a+b)*(a-b)}.

\hypertarget{operator:FACTORIZE}{}
It is also possible to factorize a given expression explicitly.  The
operator \texttt{factorize}\ttindex{factorize} that invokes this facility is
used with the syntax
\begin{verbatim}
     factorize(exprn:polynomial[,intexp:prime integer]):list,
\end{verbatim}
the optional argument of which will be described later. Thus to find and
display all factors of the cyclotomic polynomial $x^{105}-1$, one could
write:
\begin{verbatim}
        factorize(x^105-1);
\end{verbatim}
The result is a list of factor,exponent pairs.
In the above example, there is no overall numerical factor in the result,
so the results will consist only of polynomials in x.  The number of such
polynomials can be found by using the operator \texttt{length}.\ttindextype{length}{operator}
If there is a numerical factor, as in factorizing $12x^{2}-12$,
the first member of the result will be a list with two elements: the numerical factor and its multiplicity (1).
It will however not be factored further.  Prime factors of such numbers
can be found, using a probabilistic algorithm, by turning on the switch
\sw{ifactor}.\ttindexswitch{ifactor}  For example,
\begin{verbatim}
        on ifactor; factorize(12x^2-12);
\end{verbatim}
would result in the output
\begin{verbatim}
        {{2,2},{3,1},{x + 1,1},{x - 1,1}}.
\end{verbatim}

If the first argument of \texttt{factorize} is an integer, it will be
decomposed into its prime components, whether or not \sw{ifactor} is on.

Note that the \sw{ifactor} switch only affects the result of \texttt{factorize}.
It has no effect if the \sw{factor}\ttindexswitch{factor} switch is also on.

The order in which the factors occur in the result (with the exception of
a possible overall numerical coefficient which comes first) can be system
dependent and should not be relied on. Similarly it should be noted that
any pair of individual factors can be negated without altering their
product, and that {\REDUCE} may sometimes do that.

The factorizer works by first reducing multivariate problems to univariate
ones and then solving the univariate ones modulo small primes. It normally
selects both evaluation points and primes using a random number generator
that should lead to different detailed behavior each time any particular
problem is tackled. If, for some reason, it is known that a certain
(probably univariate) factorization can be performed effectively with a
known prime, \texttt{p} say, this value of \texttt{p} can be handed to
\texttt{factorize}\ttindex{factorize} as a second
argument. An error will occur if a non-prime is provided to \texttt{factorize} in
this manner. It is also an error to specify a prime that divides the
discriminant of the polynomial being factored, but users should note that
this condition is not checked by the program, so this capability should be
used with care.

\ttindexswitch{modular}
Factorization can be performed over a number of polynomial coefficient
domains in addition to integers. The particular description of the relevant
domain should be consulted to see if factorization is supported. For
example, the following statements will factorize $x^{4}+1$ modulo 7:
\begin{verbatim}
        setmod 7;
        on modular;
        factorize(x^4+1);
\end{verbatim}
\hypertarget{switch:TRFAC}{}
\hypertarget{switch:OVERVIEW}{}
The factorization module is provided with a trace facility that may be useful
as a way of monitoring progress on large problems, and of satisfying
curiosity about the internal workings of the package. The most simple use
of this is enabled by issuing the {\REDUCE} command\ttindexswitch{trfac}
\texttt{on trfac;} .
Following this, all calls to the factorizer will generate informative
messages reporting on such things as the reduction of multivariate to
univariate cases, the choice of a prime and the reconstruction of full
factors from their images.  Further levels of detail in the trace are
intended mainly for system tuners and for the investigation of suspected
bugs.  For example, \sw{trallfac} gives tracing information at all levels
of detail.
%The switch that can be set by \texttt{on timings;} makes it
%possible for one who is familiar with the algorithms used to determine
%what part of the factorization code is consuming the most resources.
\texttt{on overview;}\ttindexswitch{overview} reduces the amount of detail presented in other forms of
trace.  Other forms of trace output are enabled by directives of the form
\begin{verbatim}
        symbolic set!-trace!-factor(<number>,<filename>);
\end{verbatim}
where useful numbers are 1, 2, 3 and 100, 101, ... .  This facility is
intended to make it possible to discover in fairly great detail what just
some small part of the code has been doing --- the numbers refer mainly to
depths of recursion when the factorizer calls itself, and to the split
between its work forming and factorizing images and reconstructing full
factors from these.  If \texttt{nil} is used in place of a filename the trace
output requested is directed to the standard output stream.  After use of
this trace facility the generated trace files should be closed by calling
\begin{verbatim}
        symbolic close!-trace!-files();
\end{verbatim}
\textit{NOTE:} Using the factorizer with \sw{mcd}\ttindexswitch{mcd} off will
result in an error.

\section{Cancellation of Common Factors}
\hypertarget{switch:GCD}{}
Facilities are available in {\REDUCE} for cancelling common factors in the
numerators and denominators of expressions, at the option of the user. The
system will perform this greatest common divisor computation if the switch
\sw{gcd}\ttindexswitch{gcd} is on. (\sw{gcd} is normally off.)

A check is automatically made, however, for common variable and numerical
products in the numerators and denominators of expressions, and the
appropriate cancellations made.

When \sw{gcd} is on, and \sw{exp} is off, a check is made for square
free factors in an expression.  This includes separating out and
independently checking the content of a given polynomial where
appropriate. (For an explanation of these terms, see \cite{Hearn:79}.)

\textit{Example:} With \sw{exp}\ttindex{exp} off and \sw{gcd}\ttindex{gcd}
on,
the polynomial \texttt{a*c+a*d+b*c+b*d} would be returned as \texttt{(a+b)*(c+d)}.

\hypertarget{switch:EZGCD}{}
Under normal circumstances, GCDs are computed using an algorithm described
in the above paper. It is also possible in {\REDUCE} to compute GCDs using
an alternative algorithm, called the EZGCD Algorithm, which uses modular
arithmetic.  The switch \sw{ezgcd}\ttindexswitch{ezgcd}, if on in addition to
\sw{gcd}, makes this happen.

In non-trivial cases, the EZGCD algorithm is almost always better
than the basic algorithm, often by orders of magnitude.  We therefore
\emph{strongly} advise users to use the \sw{ezgcd} switch where they have the
resources available for supporting the package.

For a description of the EZGCD algorithm, \cite{Moses1973TheEG}.

\textit{NOTE:}
This package shares code with the factorizer, so a certain amount of trace
information can be produced using the factorizer trace switches.

\hypertarget{switch:HEUGCD}{}
An implementation of the heuristic GCD algorithm, first introduced by B.W.
Char, K.O.  Geddes and G.H.  Gonnet, as described in \cite{Davenport:85}, is also available on an
experimental basis.  To use this algorithm, the switch \sw{heugcd}\ttindexswitch{heugcd}
should be on in addition to \sw{gcd}.  Note that
if both \sw{ezgcd} and \sw{heugcd} are on, the former takes precedence.


\subsection{Determining the GCD of Two Polynomials}
\hypertarget{operator:GCD}{}
This operator, used with the syntax
\begin{verbatim}
        gcd(exprn1:polynomial,exprn2:polynomial):polynomial,
\end{verbatim}
returns the greatest common divisor of the two polynomials \texttt{exprn1} and
\texttt{exprn2}.

\textit{Examples:}
\begin{verbatim}
        gcd(x^2+2*x+1,x^2+3*x+2) ->  x+1
        gcd(2*x^2-2*y^2,4*x+4*y) ->  2*x+2*y
        gcd(x^2+y^2,x-y)         ->  1.
\end{verbatim}

\section{Working with Least Common Multiples}
\hypertarget{switch:LCM}{}

Greatest common divisor calculations can often become expensive if
extensive work with large rational expressions is required. However, in
many cases, the only significant cancellations arise from the fact that
there are often common factors in the various denominators which are
combined when two rationals are added. Since these denominators tend to be
smaller and more regular in structure than the numerators, considerable
savings in both time and space can occur if a full GCD check is made when
the denominators are combined and only a partial check when numerators are
constructed. In other words, the true least common multiple of the
denominators is computed at each step. The switch \sw{lcm}\ttindexswitch{lcm}
is available for this purpose, and is normally on.

\hypertarget{operator:LCM}{}
In addition, the operator \texttt{lcm},\ttindextype{lcm}{operator} used with the syntax
\begin{verbatim}
        lcm(exprn1:polynomial,exprn2:polynomial):polynomial,
\end{verbatim}
returns the least common multiple of the two polynomials \texttt{exprn1} and
\texttt{exprn2}.

\textit{Examples:}
\begin{verbatim}
        lcm(x^2+2*x+1,x^2+3*x+2) ->  x**3 + 4*x**2 + 5*x + 2
        lcm(2*x^2-2*y^2,4*x+4*y) ->  4*(x**2 - y**2)
\end{verbatim}

\section{Controlling Use of Common Denominators}
\hypertarget{switch:MCD}{}

When two rational functions are added, {\REDUCE} normally produces an
expression over a common denominator. However, if the user does not want
denominators combined, he or she can turn off the switch \sw{mcd}
\ttindexswitch{mcd} which controls this process.  The latter switch is
particularly useful if no greatest common divisor calculations are
desired, or excessive differentiation of rational functions is required.

\textit{CAUTION:}  With \sw{mcd} off, results are not guaranteed to come out in
either normal or canonical form.  In other words, an expression equivalent
to zero may in fact not be simplified to zero.  This option is therefore
most useful for avoiding expression swell during intermediate parts of a
calculation.

\sw{mcd} is normally on.

\section{Euclidean Division}
\ttindextype{divide}{operator}
\ttindextype{poly\_quotient}{operator}
\ttindextype{remainder}{operator}
\ttindextype{mod}{operator}
\hypertarget{operator:DIVIDE}{}
\hypertarget{operator:POLY_QUOTIENT}{}
\hypertarget{operator:REMAINDER}{}
\hypertarget{operator:MOD}{}

The operators \texttt{divide}, \texttt{poly\_quotient} and
\texttt{mod} / \texttt{remainder} implement Euclidean division of
polynomials (over the current number domain).
The \texttt{remainder} operator is used with the syntax
\begin{verbatim}
     remainder(exprn1:polynomial,exprn2:polynomial):
          polynomial.
\end{verbatim}
It returns the remainder when \texttt{exprn1} is divided by \texttt{exprn2}.  This
is the true remainder based on the internal ordering of the variables, and
not the pseudo-remainder. 

\textit{Examples:}
\begin{verbatim}
     remainder((x + y)*(x + 2*y), x + 3*y)  ->  2*y^2
     remainder(2*x + y, 2)                  ->  y
\end{verbatim}

\textit{CAUTION:} In the default case, remainders are calculated over the
integers.  If you need the remainder with respect to another domain, it
must be declared explicitly.

\textit{Example:}
\begin{verbatim}
     remainder(x^2 - 2, x + sqrt(2));  ->  x^2 - 2
     load_package arnum;
     defpoly sqrt2^2 - 2;
     remainder(x^2 - 2, x + sqrt2);    ->  0
\end{verbatim}
(Note the use of \texttt{sqrt2} in place of \texttt{sqrt(2)} in the
second call of \texttt{remainder}.)

The infix operator \texttt{mod} is an alias for \texttt{remainder}
when at least one operand is explicitly polynomial, e.g.
\begin{verbatim}
     (x^2 + y^2) mod (x - y);

        2
     2*y
\end{verbatim}
However, when both operands are integers, \texttt{mod} implements the
integer modulus operation, regardless of the current number domain,
e.g.
\begin{verbatim}
     7 mod 4  ->  3
\end{verbatim}

The Euclidean division operator \texttt{divide} is used with the syntax
\begin{verbatim}
     divide(exprn1:polynomial,exprn2:polynomial):
          list(polynomial,polynomial)
\end{verbatim}
and returns both the quotient and the remainder together as the first
and second elements of a list, e.g.
\begin{verbatim}
     divide(x^2 + y^2, x - y);

               2
     {x + y,2*y }
\end{verbatim}
It can also be used as an infix operator:
\begin{verbatim}
     (x^2 + y^2) divide (x - y);

               2
     {x + y,2*y }
\end{verbatim}
The infix operator \texttt{poly\_quotient} returns only the quotient,
i.e.\ the first element of the list returned by \texttt{divide}.

All Euclidean division operators (when used in prefix form) accept an
optional third argument, which specifies the main variable to be used
during the division.  The default is the leading kernel in the current
global ordering.  Specifying the main variable does not change the
ordering of any other variables involved, nor does it change the
global environment.  For example
\begin{verbatim}
     divide(x^2 + y^2, x - y, y);

                    2
     { - (x + y),2*x }
\end{verbatim}

Specifying $x$ as main variable gives the same behaviour as the
default shown earlier, i.e.
\begin{verbatim}
     divide(x^2 + y^2, x - y, x);

               2
     {x + y,2*y }
\end{verbatim}

All Euclidean division operators accept a (possibly nested) list as
first argument/operand and map over that list, e.g.
\begin{verbatim}
     {x, x + 1, x^2 - 1} mod x - 1;

     {1,2,0}
\end{verbatim}

\section{Polynomial Pseudo-Division}
\index{Polynomial Pseudo-Division}
\index{Pseudo-Division}

The polynomial division discussed above is normally most useful for a
univariate polynomial over a field, otherwise the division is likely
to fail giving trivially a zero quotient and a remainder equal to the
dividend.  (A ring of univariate polynomials is a Euclidean domain
only if the coefficient ring is a field.)  For example, over the
integers:
\begin{verbatim}
     divide(x^2 + y^2, 2(x - y));

         2    2
     {0,x  + y }
\end{verbatim}

\hypertarget{operator:PSEUDO_DIVIDE}{}
\hypertarget{operator:PSEUDO_QUOTIENT}{}
\hypertarget{operator:PSEUDO_REMAINDER}{}
\ttindextype{PSEUDO\_DIVIDE}{operator}
\ttindextype{PSEUDO\_QUOTIENT}{operator}
\ttindextype{PSEUDO\_REMAINDER}{operator}

The division of a polynomial $u(x)$ of degree $m$ by a polynomial
$v(x)$ of degree $n \le m$ can be performed over any commutative ring
with identity (such as the integers, or any polynomial ring) if the
polynomial $u(x)$ is first multiplied by $\mathrm{lc}(v,x)^{m-n+1}$
(where lc denotes the leading coefficient).  This is called
\emph{pseudo-division}.  The polynomial pseudo-division operators
\texttt{pseudo\_divide}, \texttt{pseudo\_quotient} and
\texttt{pseudo\_remainder} are implemented as prefix operators (only).
When multivariate polynomials are pseudo-divided it is important which
variable is taken as the main variable, because the leading
coefficient of the divisor is computed with respect to this variable.
Therefore, if this is allowed to default and there is any ambiguity,
i.e.\ the polynomials are multivariate or contain more than one
kernel, the pseudo-division operators output a warning message to
indicate which kernel has been selected as the main variable -- it is
the first kernel found in the internal forms of the dividend and
divisor.  (As usual, the warning can be turned off by setting the
switch \sw{msg} to off.)  For example
\begin{verbatim}
     pseudo_divide(x^2 + y^2, x - y);

     *** Main division variable selected is x 

               2
     {x + y,2*y }

     pseudo_divide(x^2 + y^2, x - y, x);

               2
     {x + y,2*y }

     pseudo_divide(x^2 + y^2, x - y, y);

                    2
     { - (x + y),2*x }
\end{verbatim}

If the leading coefficient of the divisor is a unit (invertible
element) of the coefficient ring then division and pseudo-division
should be identical, otherwise they are not, e.g.
\begin{verbatim}
     divide(x^2 + y^2, 2(x - y));

         2    2
     {0,x  + y }

     pseudo_divide(x^2 + y^2, 2(x - y));

     *** Main division variable selected is x

                   2
     {2*(x + y),8*y }
\end{verbatim}

The pseudo-division gives essentially the same result as would
division over the field of fractions of the coefficient ring (apart
from the overall factors [contents] of the quotient and remainder),
e.g.
\begin{verbatim}
     on rational;

     divide(x^2 + y^2, 2(x - y));

       1             2
     {---*(x + y),2*y }
       2

     pseudo_divide(x^2 + y^2, 2(x - y));

     *** Main division variable selected is x

                   2
     {2*(x + y),8*y }
\end{verbatim}

Polynomial division and pseudo-division can only be applied to what
REDUCE regards as polynomials, i.e.\ rational expressions with
denominator 1, e.g.
\begin{verbatim}
     off rational;

     pseudo_divide((x^2 + y^2)/2, x - y);

             2    2
            x  + y
     ***** --------- invalid as polynomial
               2
\end{verbatim}
All pseudo-division operators accept a (possibly nested) list as first
argument/operand and map over that list.

Pseudo-division is implemented using an algorithm (\cite{Knuth:TAoCP2}, Algorithm R, page 407) that does
not perform any actual division at all (which proves that it applies
over a ring).  It is more efficient than the naive algorithm, and it
also has the advantage that it works over coefficient domains in which
REDUCE may not be able to perform in practice divisions that are
possible mathematically.  An example of this is coefficient domains
involving algebraic numbers, such as the integers extended by
$\sqrt{2}$, as illustrated in the file \texttt{polydiv.tst}.

The implementation attempts to be reasonably efficient, except that it
always computes the quotient internally even when only the remainder
is required (as does the standard remainder operator).

\section{RESULTANT Operator}\ttindextype{resultant}{operator}
\hypertarget{operator:RESULTANT}{}
\hypertarget{switch:BEZOUT}{}

This is used with the syntax
\begin{verbatim}
     resultant(exprn1:polynomial,exprn2:polynomial,var:kernel):
        polynomial.
\end{verbatim}
It computes the resultant of the two given polynomials with respect to the
given variable, the coefficients of the polynomials can be taken from any
domain. The result can be identified as the determinant of a
Sylvester matrix, but can often also be thought of informally as the
result obtained when the given variable is eliminated between the two input
polynomials. If the two input polynomials have a non-trivial GCD their
resultant vanishes.

The switch \sw{bezout}\ttindexswitch{bezout} controls the computation of the
resultants. It is off by default. In this case a subresultant algorithm
is used. If the switch Bezout is turned on, the resultant is computed via
the Bezout Matrix. However, in the latter case, only polynomial coefficients
are permitted.

%\begin{samepage}
The sign conventions used by the resultant function follow those in \cite{Loos:1982}.
namely, with \texttt{a} and \texttt{b} not dependent on \texttt{x}:

\begin{verbatim}
                               deg(p)*deg(q)
   resultant(p(x),q(x),x)= (-1)             *resultant(q,p,x)

                            deg(p)
   resultant(a,p(x),x)   = a

   resultant(a,b,x)      = 1
\end{verbatim}
%\end{samepage}

\textit{Examples:}

\begin{samepage}
\begin{verbatim}
                                     2
   resultant(x/r*u+y,u*y,u)   ->  - y
\end{verbatim}
\end{samepage}

\textit{calculation in an algebraic extension:}

\begin{samepage}
\begin{verbatim}
   load arnum;
   defpoly sqrt2**2 - 2;

   resultant(x + sqrt2,sqrt2 * x +1,x) -> -1
\end{verbatim}
\end{samepage}

\textit{or in a modular domain:}

\begin{samepage}
\begin{verbatim}
   setmod 17;
   on modular;

   resultant(2x+1,3x+4,x)    -> 5
\end{verbatim}
\end{samepage}
\section{DECOMPOSE Operator}\ttindextype{decompose}{operator}
\hypertarget{operator:DECOMPOSE}{}

The \texttt{decompose} operator takes a multivariate polynomial as argument,
and returns an expression and a list of equations from which the
original polynomial can be found by composition.  Its syntax is:
\begin{verbatim}
     decompose(exprn:polynomial):list.
\end{verbatim}
For example:
\begin{verbatim}
     decompose(x^8-88*x^7+2924*x^6-43912*x^5+263431*x^4-
                    218900*x^3+65690*x^2-7700*x+234)
                   2                  2            2
              -> {u  + 35*u + 234, u=v  + 10*v, v=x  - 22*x}
                                     2
     decompose(u^2+v^2+2u*v+1)  -> {w  + 1, w=u + v}
\end{verbatim}
Users should note however that, unlike factorization, this decomposition
is not unique.

\section{INTERPOL operator}\ttindextype{interpol}{operator}
\hypertarget{operator:INTERPOL}{}

Syntax:
\begin{syntax}
  \texttt{interpol(}\meta{values}\texttt{,}\,\meta{variable}\texttt{,}\,\meta{points}\texttt{);}
\end{syntax}

where \meta{values} and \meta{points} are lists of equal length and
\texttt{<variable>} is an algebraic expression (preferably a kernel).

\texttt{interpol} generates an interpolation polynomial \emph{f} in the given
variable of degree \texttt{length}(\meta{values})-1.  The unique polynomial 
\emph{f}
is defined by the property that for corresponding elements \emph{v} of
\meta{values} and \emph{p} of \meta{points} the relation $f(p)=v$ holds.

The Aitken-Neville interpolation algorithm is used which guarantees a
stable result even with rounded numbers and an ill-conditioned problem.

\section{Obtaining Parts of Polynomials and Rationals}

These operators select various parts of a polynomial or rational function
structure. Except for the cost of rearrangement of the structure, these
operations take very little time to perform.

For those operators in this section that take a kernel \texttt{var} as their
second argument, an error results if the first expression is not a
polynomial in \texttt{var}, although the coefficients themselves can be
rational as long as they do not depend on \texttt{var}.  However, if the
switch \texttt{ratarg}\ttindexswitch{ratarg} is on, denominators are not checked
for dependence on \texttt{var}, and are taken to be part of the coefficients.

\subsection{DEG Operator}\ttindextype{DEG}{operator}
\hypertarget{operator:DEG}{}

This operator is used with the syntax
\begin{verbatim}
        deg(exprn:polynomial,var:kernel):integer.
\end{verbatim}
It returns the leading degree\index{Degree} of the polynomial \texttt{exprn}
in the variable \texttt{var}.  If \texttt{var} does not occur as a variable in
\texttt{exprn}, 0 is returned.

\textit{Examples:}
\begin{verbatim}
        deg((a+b)*(c+2*d)^2,a) ->  1
        deg((a+b)*(c+2*d)^2,d) ->  2
        deg((a+b)*(c+2*d)^2,e) ->  0.
\end{verbatim}
Note also that if \sw{ratarg} is on,
\begin{verbatim}
        deg((a+b)^3/a,a)       ->  3
\end{verbatim}
since in this case, the denominator \texttt{a} is considered part of the
coefficients of the numerator in \texttt{a}.  With \sw{ratarg} off, however,
an error would result in this case.

\subsection{DEN Operator}\ttindextype{den}{operator}
\hypertarget{operator:DEN}{}

This is used with the syntax:
\begin{verbatim}
        den(exprn:rational):polynomial.
\end{verbatim}
It returns the denominator of the rational expression \texttt{exprn}.  If
\texttt{exprn} is a polynomial, 1 is returned.

\textit{Examples:}
\begin{verbatim}
        den(x/y^2)   ->  Y**2
        den(100/6)   ->  3
                [since 100/6 is first simplified to 50/3]
        den(a/4+b/6) ->  12
        den(a+b)     ->  1
\end{verbatim}

\subsection{LCOF Operator}\ttindextype{lcof}{operator}
\hypertarget{operator:LCOF}{}

\texttt{lcof} is used with the syntax
\begin{verbatim}
        lcof(exprn:polynomial,var:kernel):polynomial.
\end{verbatim}
It returns the leading coefficient\index{Leading coefficient} of the
polynomial \texttt{exprn} in the variable \texttt{var}.  If \texttt{var} does not
occur as a variable in \texttt{exprn}, \texttt{exprn} is returned.
%\extendedmanual{\newpage}
\textit{Examples:}
\begin{verbatim}
        lcof((a+b)*(c+2*d)^2,a) ->  c**2+4*c*d+4*d**2
        lcof((a+b)*(c+2*d)^2,d) ->  4*(a+b)
        lcof((a+b)*(c+2*d),e)   ->  a*c+2*a*d+b*c+2*b*d
\end{verbatim}

\subsection{LPOWER Operator}\ttindextype{lpower}{operator}
\hypertarget{operator:LPOWER}{}

%\begin{samepage}
Syntax:
\begin{verbatim}
        lpower(exprn:polynomial,var:kernel):polynomial.
\end{verbatim}
\f{lpower} returns the leading power of \texttt{exprn} with respect to \texttt{var}.
If \texttt{exprn} does not depend on \texttt{var}, 1 is returned.
%\end{samepage}

\textit{Examples:}
\begin{verbatim}
        lpower((a+b)*(c+2*d)^2,a) ->  a
        lpower((a+b)*(c+2*d)^2,d) ->  d**2
        lpower((a+b)*(c+2*d),e)   ->  1
\end{verbatim}

\subsection{LTERM Operator}\ttindextype{lterm}{operator}
\hypertarget{operator:LTERM}{}

\begin{samepage}
Syntax:
\begin{verbatim}
        lterm(exprn:polynomial,var:kernel):polynomial.
\end{verbatim}
\f{lterm} returns the leading term of \texttt{exprn} with respect to \texttt{var}.
If \texttt{exprn} does not depend on \texttt{var}, \texttt{exprn} is returned.
\end{samepage}

\textit{Examples:}
\begin{verbatim}
        lterm((a+b)*(c+2*d)^2,a) ->  a*(c**2+4*c*d+4*d**2)
        lterm((a+b)*(c+2*d)^2,d) ->  4*d**2*(a+b)
        lterm((a+b)*(c+2*d),e)   ->  a*c+2*a*d+b*c+2*b*d
\end{verbatim}

{\COMPATNOTE} In some earlier versions of \REDUCE, \texttt{lterm} returned
\texttt{0} if the \texttt{exprn} did not depend on \texttt{var}.  In the present
version, \texttt{exprn} is always equal to 
\texttt{lterm(exprn,var) + reduct(exprn,var)}.

\subsection{MAINVAR Operator}\ttindextype{mainvar}{operator}
\hypertarget{operator:MAINVAR}{}

Syntax:
\begin{verbatim}
        mainvar(exprn:polynomial):expression.
\end{verbatim}
Returns the main variable (based on the internal polynomial representation)
of \texttt{exprn}. If \texttt{exprn} is a domain element, 0 is returned.

\textit{Examples:}

Assuming \texttt{a} has higher kernel order than \texttt{b}, \texttt{c}, or \texttt{d}:
\begin{verbatim}
        mainvar((a+b)*(c+2*d)^2) ->  a
        mainvar(2)               ->  0
\end{verbatim}

\subsection{NUM Operator}\ttindextype{num}{operator}
\hypertarget{operator:NUM}{}

Syntax:
\begin{verbatim}
        num(exprn:rational):polynomial.
\end{verbatim}
Returns the numerator of the rational expression \texttt{exprn}.  If \texttt{exprn}
is a polynomial, that polynomial is returned.

\textit{Examples:}
\begin{verbatim}
        num(x/y^2)  ->  x
        num(100/6)   ->  50
        num(a/4+b/6) ->  3*a+2*b
        num(a+b)     ->  a+b
\end{verbatim}

\subsection{REDUCT Operator}\ttindextype{reduct}{operator}
\hypertarget{operator:REDUCT}{}

Syntax:
\begin{verbatim}
        reduct(exprn:polynomial,var:kernel):polynomial.
\end{verbatim}
Returns the reductum of \texttt{exprn} with respect to \texttt{var} (i.e., the
part of \texttt{exprn} left after the leading term is removed).  If \texttt{exprn}
does not depend on the variable \texttt{var}, 0 is returned.

\textit{Examples:}
\begin{verbatim}
     reduct((a+b)*(c+2*d),a) ->  b*(c + 2*d)
     reduct((a+b)*(c+2*d),d) ->  c*(a + b)
     reduct((a+b)*(c+2*d),e) ->  0
\end{verbatim}

{\COMPATNOTE} In some earlier versions of \REDUCE, \texttt{reduct} returned
\texttt{exprn} if it did not depend on \texttt{var}.  In the present version,
\texttt{exprn} is always equal to \texttt{lterm(exprn,var) + reduct(exprn,var)}.

\subsection{TOTALDEG Operator}\ttindextype{totaldeg}{operator}
\hypertarget{operator:TOTALDEG}{}

Syntax:
\begin{verbatim}
     totaldeg(a*x^2+b*x+c, x)  => 2
     totaldeg(a*x^2+b*x+c, {a,b,c})  => 1
     totaldeg(a*x^2+b*x+c, {x, a})  => 3
     totaldeg(a*x^2+b*x+c, {x,b})  => 2
     totaldeg(a*x^2+b*x+c, {p,q,r})  => 0
\end{verbatim}
\texttt{totaldeg(u, kernlist)} finds the total degree of the polynomial \texttt{u} in
the variables in \texttt{kernlist}. If \texttt{kernlist} is not a list it is treated
as a simple single variable.
The denominator of \texttt{u} is ignored, and "degree" here does not pay attention
to fractional powers. Mentions of a kernel within the argument to any
operator or function (eg sin, cos, log, sqrt) are ignored. Really \texttt{u} is
expected to be just a polynomial.

\section{Polynomial Coefficient Arithmetic}\index{Coefficient}
{\REDUCE} allows for a variety of numerical domains for the numerical
coefficients of polynomials used in calculations.  The default mode is
integer arithmetic, although the possibility of using real coefficients
\index{Real coefficient} has been discussed elsewhere.  Rational
coefficients have also been available by using integer coefficients in
both the numerator and denominator of an expression, using the 
\texttt{on div}\ttindexswitch{div} option to print the coefficients as rationals.
However, {\REDUCE} includes several other coefficient options in its basic
version which we shall describe in this section.  All such coefficient
modes are supported in a table-driven manner so that it is
straightforward to extend the range of possibilities.  A description of
how to do this is given in \cite{Bradford:86}.

\subsection{Rational Coefficients in Polynomials}\index{Coefficient}
\hypertarget{switch:RATIONAL}{}
\index{Rational coefficient}
Instead of treating rational numbers as the numerator and denominator of a
rational expression, it is also possible to use them as polynomial
coefficients directly. This is accomplished by turning on the switch
\sw{rational}.\ttindexswitch{rational}

\textit{Example:} With \sw{rational} off, the input expression \texttt{a/2}
would be converted into a rational expression, whose numerator was \texttt{a}
and denominator 2.  With \sw{rational} on, the same input would become a
rational expression with numerator \texttt{1/2*a} and denominator \texttt{1}.
Thus the latter can be used in operations that require polynomial input
whereas the former could not.

\subsection{Real Coefficients in Polynomials}\index{Coefficient}
\index{Real coefficient}
\hypertarget{switch:ROUNDED}{}
\hypertarget{switch:ROUNDBF}{}
\hypertarget{operator:PRECISION}{}
The switch \sw{rounded}\ttindexswitch{rounded} permits the use of arbitrary
sized real coefficients in polynomial expressions.  The actual precision
of these coefficients can be set by the operator \texttt{precision}.
\ttindextype{precision}{operator} For example, \texttt{precision 50;} sets the precision to
fifty decimal digits.  The default precision is system dependent and can
be found by \texttt{precision 0;}.  In this mode, denominators are
automatically made monic, and an appropriate adjustment is made to the
numerator.

\textit{Example:} With \sw{ROUNDED} on, the input expression \texttt{a/2} would
be converted into a rational expression whose numerator is \texttt{0.5*a} and
denominator \texttt{1}.

Internally, {\REDUCE} uses floating point numbers up to the precision
supported by the underlying machine hardware, and so-called \emph{bigfloats} 
for higher precision or whenever necessary to represent numbers
whose value cannot be represented in floating point.  The internal
precision is two decimal digits greater than the external precision to
guard against roundoff inaccuracies.  Bigfloats represent the fraction and
exponent parts of a floating-point number by means of (arbitrary
precision) integers, which is a more precise representation in many cases
than the machine floating point arithmetic, but not as efficient.  If a
case arises where use of the machine arithmetic leads to problems, a user
can force {\REDUCE} to use the bigfloat representation at all precisions by
turning on the switch \sw{roundbf}.\ttindexswitch{roundbf}  In rare cases,
this switch is turned on by the system, and the user informed by the
message
\begin{verbatim}
        ROUNDBF turned on to increase accuracy
\end{verbatim}

\hypertarget{command:PRINT_PRECISION}{}
Rounded numbers are normally printed to the specified precision.  However,
if the user wishes to print such numbers with less precision, the printing
precision can be set by the command \texttt{print\_precision}.
\ttindextype{print\_precision}{command} For example, \texttt{print\_precision 5;} will
cause such numbers to be printed with five digits maximum.

\hypertarget{switch:NOCONVERT}{}
Under normal circumstances when \sw{rounded} is on, {\REDUCE} converts the
number 1.0 to the integer 1.  If this is not desired, the switch
\sw{noconvert}\ttindexswitch{noconvert} can be turned on.

\hypertarget{switch:BFSPACE}{}
Numbers that are stored internally as bigfloats are normally printed with
a space between every five digits to improve readability.  If this
feature is not required, it can be suppressed by turning off the switch
\sw{bfspace}.\ttindexswitch{bfspace}

Further information on the bigfloat arithmetic may be found in T. Sasaki,
``Manual for Arbitrary Precision Real Arithmetic System in {\REDUCE}'',
Department of Computer Science, University of Utah, Technical Note No.
TR-8 (1979).

\hypertarget{switch:ADJPREC}{}
When a real number is input, it is normally truncated to the precision in
effect at the time the number is read.  If it is desired to keep the full
precision of all numbers input, the switch \sw{adjprec}\ttindexswitch{adjprec}
(for \emph{adjust precision}) can be turned on.  While on, \sw{adjprec}
will automatically increase the precision, when necessary, to match that
of any integer or real input, and a message printed to inform the user of
the precision increase.

\hypertarget{switch:ROUNDALL}{}
When \sw{rounded} is on, rational numbers are normally converted to
rounded representation.  However, if a user wishes to keep such numbers in
a rational form until used in an operation that returns a real number,
the switch \sw{roundall}\ttindexswitch{roundall} can be turned off.  This
switch is normally on.

Results from rounded calculations are returned in rounded form with two
exceptions: if the result is recognized as \texttt{0} or \texttt{1} to the
current precision, the integer result is returned.

\subsection{Modular Number Coefficients in Polynomials}\index{Coefficient}
\hypertarget{switch:MODULAR}{}
\hypertarget{command:SETMOD}{}
\index{Modular coefficient}
{\REDUCE} includes facilities for manipulating polynomials whose
coefficients are computed modulo a given base.  To use this option, two
commands must be used; \texttt{setmod }\meta{integer}\ttindex{setmod}, to set
the prime modulus, and \texttt{on modular}\ttindexswitch{modular} to cause the
actual modular calculations to occur.
For example, with \texttt{setmod 3;} and \texttt{on modular;}, the polynomial
\texttt{(a+2*b)\textasciicircum3} would become 
\texttt{a\textasciicircum3+2*b\textasciicircum3}.

The argument of \texttt{setmod} is evaluated algebraically, except that
non-modular (integer) arithmetic is used.  Thus the sequence
\begin{verbatim}
        setmod 3; on modular; setmod 7;
\end{verbatim}
will correctly set the modulus to 7.

\hypertarget{switch:BALANCED_MOD}{}
Modular numbers are by default represented by integers in the interval
[0,p-1] where p is the current modulus.  Sometimes it is more convenient
to use an equivalent symmetric representation in the interval
[-p/2+1,p/2], or more precisely
[-floor((p-1)/2), ceiling((p-1)/2)],
especially if the modular numbers map objects that include
negative quantities.  The switch \sw{balanced\_mod}\ttindexswitch{balanced\_mod}
allows you to select the symmetric representation for output.

Users should note that the modular calculations are on the polynomial
coefficients only.  It is not currently possible to reduce the exponents
since no check for a prime modulus is made (which would allow
$x^{p-1}$ to be reduced to 1 mod p).  Note also that any division by a
number not co-prime with the modulus will result in the error ``Invalid
modular division''.

\subsection{Complex Number Coefficients in Polynomials}\index{Coefficient}
\index{Complex coefficient}
\hypertarget{switch:COMPLEX}
Although {\REDUCE} routinely treats the square of the variable \emph{i} as
equivalent to $-1$, this is not sufficient to reduce expressions involving
\emph{i} to lowest terms, or to factor such expressions over the complex
numbers.  For example, in the default case,
\begin{verbatim}
        factorize(a^2+1);
\end{verbatim}
gives the result
\begin{verbatim}
        {{a**2+1,1}}
\end{verbatim}
and
\begin{verbatim}
        (a^2+b^2)/(a+i*b)
\end{verbatim}
is not reduced further.  However, if the switch
\sw{complex}\ttindexswitch{complex} is turned on, full complex arithmetic is then
carried out.  In other words, the above factorization will give the result
\begin{verbatim}
        {{a + i,1},{a - i,1}}
\end{verbatim}
and the quotient will be reduced to \texttt{a-i*b}.

The switch \sw{complex} may be combined with \sw{rounded} to give complex
real numbers; the appropriate arithmetic is performed in this case.
Similarly, combining \sw{complex} with \sw{rational} performs polynomial
arithmetic with complex rational numbers.

\hypertarget{switch:RATIONALIZE}{}
Complex conjugation is used to remove complex numbers from denominators of
expressions.  To do this if \sw{complex} is off, you must turn the switch
\sw{rationalize}\ttindexswitch{rationalize} on.

\iffalse
\section{ROOT\_VAL Operator}
\hypertarget{operator:ROOT_VAL}{}

The \texttt{root\_val}\ttindex{root\_val}  operator takes a single univariate polynomial as
argument, and returns a list of root values at system precision (or
greater if required to separate roots).  It is used with the syntax
\begin{verbatim}
        root_val(exprn:univariate polynomial):list.
\end{verbatim}
For example, the sequence
\begin{verbatim}
        on rounded; root_val(x^3-x-1);
\end{verbatim}
gives the result
\begin{verbatim}
        {0.562279512062*i - 0.662358978622, - 0.562279512062*i

          - 0.662358978622,1.32471795724}
\end{verbatim}
\fi

\section{Finding Roots}
\index{root finding}
\hypertarget{package:ROOTS}{}

The simplest way to find roots of a univariate polynomial with real
or complex coefficients is to call \f{solve} with the switch
\sw{rounded} set to on. For example, the evaluation of
\begin{verbatim}
        on rounded,complex;
        solve(x**3+x+5,x);
\end{verbatim}
yields the result
\begin{verbatim}
{x=0.757990113846 + 1.65034755069*i,x=0.757990113846 - 1.65034755069*i,x

 = - 1.51598022769}
\end{verbatim}

In the following, the independent use of the roots finder is
described. It can be used to find some or all of the roots of
univariate polynomials with real or complex coefficients, to the
accuracy specified by the user.\footnote{This code was written by Stanley L.~Kameny.}

\subsection{Root Finding Strategies}

For all polynomials handled by the root finding package, strategies of
factoring are employed where possible to reduce the amount of required
work.  These include square-free factoring and separation of complex
polynomials into a product of a polynomial with real coefficients and one
with complex coefficients.  Whenever these succeed, the resulting smaller
polynomials are solved separately, except that the root accuracy takes
into account the possibility of close roots on different branches.  One
other strategy used where applicable is the powergcd method of reducing
the powers of the initial polynomial by a common factor, and deriving the
roots in two stages, as roots of the reduced power polynomial.  Again
here, the possibility of close roots on different branches is taken into
account.

\subsection{Top Level Functions}

The top level functions can be called either as symbolic operators from
algebraic mode, or they can be called directly from symbolic mode with
symbolic mode arguments.  Outputs are expressed in forms that print out
correctly in algebraic mode.


\subsubsection{Functions that refer to real roots only}

Three top level functions refer only to real roots.  Each of these
functions can receive 1, 2 or 3 arguments.

The first argument is the polynomial p, that can be complex and can
have multiple or zero roots.  If arg2 and arg3 are not present, all real
roots are found.  If the additional arguments are present, they restrict
the region of consideration.
\ttindextype{negative}{reserved variable}\ttindextype{positive}{reserved variable}\ttindex{exclude}\ttindex{infinity}
\begin{itemize}
\item If arguments are (p,arg2) then
Arg2 must be \var{positive} or \var{negative}.  If arg2=\var{negative} then only
negative roots of p are included; if arg2=\var{positive} then only positive
roots of p are included. Zero roots are excluded.

\item If arguments are (p,arg2,arg3) then
\ttindex{exclude} \ttindex{positive} \ttindex{negative} \ttindex{infinity}
Arg2 and Arg3 must be r (a real number) or  \f{exclude} r,  or a member of
the list \var{positive}, \var{negative}, \var{infinity}, \var{-infinity}.  \f{exclude} r causes the
value r to be excluded from the region.  The order of the sequence
arg2, arg3 is unimportant.  Assuming that arg2 $\leq$ arg3 when both are
numeric, then

\begin{tabular}{l c l}
\{\var{-infinity},\var{infinity}\} & is equivalent to & \{\} represents all roots; \\
\{arg2,\var{negative}\} & represents & $-\infty < r < arg2$; \\
\{arg2,\var{positive}\} & represents & $arg2 < r < \infty$;
\end{tabular}

In each of the following, replacing an {\em arg} with \f{exclude} {\em arg}
converts the corresponding inclusive $\leq$ to the exclusive $<$

\begin{tabular}{l c l}
\{arg2,\var{-infinity}\} & represents & $-\infty < r \leq arg2$; \\
\{arg2,\var{infinity}\} & represents & $arg2 \leq r < \infty$; \\
\{arg2,arg3\} & represents & $arg2 \leq r \leq arg3$;
\end{tabular}

\item If zero is in the interval the zero root is included.
\end{itemize}

\begin{description}

\item[\f{realroots}]
  \ttindextype{realroots}{operator} \index{Sturm Sequences}
  \hypertarget{operator:REALROOTS}{}
This function finds the real roots of the polynomial p,
using the \f{REALROOT} package to isolate real roots by the method of Sturm
sequences, then polishing the root to the desired accuracy.  Precision
of computation is guaranteed to be sufficient to separate all real roots
in the specified region.  (cf. \sw{multiroot} for treatment of multiple
roots.)

\item[\f{isolater}]
  \ttindextype{isolater}{operator}\hypertarget{operator:ISOLATER}{}
This function produces a list of rational intervals, each
containing a single real root of the polynomial p, within the specified
region, but does not find the roots.

\item[\f{rlrootno}]
  \ttindextype{rlrootno}{operator}\hypertarget{operator:RLROOTNO}{}
This function computes the number of real roots of p in
the specified region, but does not find the roots.
\end{description}

\subsubsection{Functions that return both real and complex roots}

\begin{description}
\item[\f{roots} p;]
\ttindextype{roots}{operator}
\hypertarget{operator:ROOTS}{}
\hypertarget{reserved:ROOTSREAL}{}
\hypertarget{reserved:ROOTSCOMPLEX}{}
This is the main top level function of the roots package.
It will find all roots, real and complex, of the polynomial p to an
accuracy that is sufficient to separate them and which is a minimum of 6
decimal places.  The value returned by \f{roots} is a
list of equations for all roots.  In addition, \f{roots} stores separate lists
of real roots and complex roots in the global variables \var{rootsreal} and
\var{rootscomplex}. \ttindextype{rootsreal}{global variable} \ttindextype{rootscomplex}{global variable}
\par
The order of root discovery by \f{roots} is highly variable from system to
system, depending upon very subtle arithmetic differences during the
computation.  In order to make it easier to compare results obtained on
different computers, the output of \f{roots} is sorted into a standard order:
a root with smaller real part precedes a root with larger real part; roots
with identical real parts are sorted so that larger imaginary part
precedes smaller imaginary part. (This is done so that for complex pairs,
the positive imaginary part is seen first.)

However, when a polynomial has been factored (by square-free factoring or
by separation into real and complex factors) then the root sorting is
applied to each factor separately.  This makes the final resulting order
less obvious.  However it is consistent from system to system.

\item[\f{roots\_at\_prec} p;]
\ttindextype{roots\_at\_prec}{operator}\hypertarget{operator:ROOTS_AT_PREC}{}
Same as \f{roots} except that roots values are
returned to a minimum of the number of decimal places equal to the current
system precision.

\item[\f{root\_val} p;]
  \ttindextype{root\_val}{operator}\hypertarget{operator:ROOT_VAL}{}
Same as \f{roots\_at\_prec}, except that instead of
returning a list of equations for the roots, a list of the root value is
returned.  This is the function that SOLVE calls.

\item[\f{nearestroot}(p,s);]
  \ttindextype{nearestroot}{operator}\hypertarget{operator:NEARESTROOT}{}
This top level function uses an iterative method
to find the root to which the method converges given the initial starting
origin s, which can be complex.  If there are several roots in the
vicinity of s and s is not significantly closer to one root than it is to
all others, the convergence could arrive at a root that is not truly the
nearest root.  This function should therefore be used only when the user
is certain that there is only one root in the immediate vicinity of the
starting point s.

\item[\f{firstroot} p;]
  \ttindextype{firstroot}{operator}\hypertarget{operator:FIRSTROOT}{}
\f{roots} is called, but only the first root determined by
\f{roots} is computed.  Note that this is not in general the first root that
would be listed in \f{roots} output, since the \f{roots} outputs are sorted into
a canonical order.  Also, in some difficult root finding cases, the first
root computed might be incorrect.
\end{description}


\subsubsection{Other top level functions}

\begin{description}
\item[\f{getroot}(n,rr);]
  \ttindextype{getroot}{operator} \ttindex{roots} \ttindex{realroots} \ttindex{nearestroots}
\hypertarget{operator:GETROOT}{}
If rr has the form of the output of ROOTS, REALROOTS,
or NEARESTROOTS; \f{GETROOT} returns the rational, real, or complex value of
the root equation.  An error occurs if $n<1$ or $n>$ the number of roots in
rr.

\item[\f{mkpoly} rr;]
  \ttindextype{mkpoly}{operator}\hypertarget{operator:MKPOLY}{}
This operator can be used to reconstruct a polynomial
whose root equation list is rr and whose denominator is 1.  Thus one can
verify that if $rr := roots~p$, and $rr1 := roots~mkpoly~rr$, then
$rr1 = rr$. (This will be true if \sw{multiroot} and \sw{ratroot} are ON,
and \sw{rounded} is off.)
However, $mkpoly~rr - num~p = 0$ will be true if and only if all roots of p
have been computed exactly.
\end{description}

\subsubsection{Functions available for diagnostic or instructional use only}
\hypertarget{operator:GFNEWT}{}
\hypertarget{operator:GFROOT}{}
\begin{description}
\ttindextype{gfnewt}{operator}
\item[\f{gfnewt}(p,r,cpx);] This function will do a single pass through the
function \f{gfnewton} for polynomial p and root r.  If cpx=T, then any
complex part of the root will be kept, no matter how small.

\ttindextype{gfroot}{operator}
\item[\f{gfroot}(p,r,cpx);] This function will do a single pass through the
function \f{GFROOTFIND} for polynomial p and root r.  If cpx=T, then any
complex part of the root will be kept, no matter how small.
\end{description}

\subsection{Switches Used in Input}

The input of polynomials in algebraic mode is sensitive to the switches
\sw{complex}, \sw{rounded}, and \sw{adjprec}.  The correct choice of
input method is important since incorrect choices will result in
undesirable truncation or rounding of the input coefficients.

Truncation or rounding may occur if \sw{rounded} is on and
one of the following is true:

\begin{enumerate}
\item a coefficient is entered in floating point form or rational form.
\item \sw{complex} is on and a coefficient is imaginary or complex.
\end{enumerate}

Therefore, to avoid undesirable truncation or rounding, then:

\begin{enumerate}
\item \sw{rounded} should be off and input should be
in integer or rational form; or
\item \sw{rounded} can be on if it is acceptable to truncate or round
input to the current value of system precision; or both \sw{rounded} and
\sw{adjprec} can be on, in which case system precision will be adjusted
to accommodate the largest coefficient which is input; or
\item if the
input contains complex coefficients with very different magnitude for the
real and imaginary parts, then all three switches \sw{rounded}, \sw{adjprec} and
\sw{complex} must be on.
\end{enumerate}

\begin{description}
\item[integer and complex modes] (\k{off} \sw{rounded}) any real
polynomial can be input using integer coefficients of any size; integer or
rational coefficients can be used to input any real or complex polynomial,
independent of the setting of the switch \sw{complex}.  These are the most
versatile input modes, since any real or complex polynomial can be input
exactly.

\item[modes rounded and complex-rounded] (on \sw{rounded}) polynomials can be
input using
integer coefficients of any size.  Floating point coefficients will be
truncated or rounded, to a size dependent upon the system.  If complex
is on, real coefficients can be input to any precision using integer
form, but coefficients of imaginary parts of complex coefficients will
be rounded or truncated.
\end{description}

\subsection{Internal and Output Use of Switches}

The {\REDUCE} arithmetic mode switches \sw{rounded} and \sw{complex}
control the behavior of the root finding package.
These switches are returned in the same state in which they were set
initially, (barring catastrophic error).

\begin{description}
\ttindexswitch{complex}
\item[\sw{complex}] The root finding package controls the switch \sw{complex}
internally, turning the switch on if it is processing a complex
polynomial.
For a polynomial with real coefficients, the
\ttindex{nearestroot}
starting point argument for \f{nearestroot} can be given in algebraic mode in
complex form as rl + im * I and will be handled correctly, independent of
the setting of the switch \sw{complex}. Complex roots will be computed
and printed correctly regardless of the setting of the switch \sw{complex}.
However, if \sw{complex} is off, the imaginary part will print
out ahead of the real part, while the reverse order will be obtained if
\sw{complex} is on.

\ttindexswitch{rounded}
\item[\sw{rounded}] The
root finding package performs computations using the arithmetic mode that
is required at the time, which may be integer, Gaussian integer, rounded,
or complex rounded.  The switch \sw{bftag} is used internally to govern
the mode of computation and precision is adjusted whenever necessary.  The
initial position of switches \sw{rounded} and \sw{complex} are ignored.
At output, these switches will emerge in their initial positions.
\end{description}

\subsection{Root Package Switches}

%%Note: switches \sw{automode}, \sw{isoroot} and \sw{accroot}, present in
%%earlier versions, have been eliminated.

\begin{description}
\item[\sw{ratroot}]
  \hypertarget{switch:RATROOT}{}
\ttindexswitch[ROOTS]{ratroot}
(Default off) If \sw{RATROOT} is on all root equations are
output in rational form.  Assuming that the mode is \sw{complex} (i.e.
\sw{rounded} is off,) the root equations are
guaranteed to be able to be input into \REDUCE{} without truncation or
rounding errors. (Cf. the function \f{mkpoly} described above.)

\item[\sw{multiroot}]
  \hypertarget{switch:MULTIROOT}{}
\ttindexswitch[ROOTS]{multiroot}
(Default on) Whenever the polynomial has complex
coefficients or has real coefficients and has multiple roots, as
\ttindex{sqfrf} determined by the Sturm function, the function \f{sqfrf}
is called automatically to factor the polynomial into square-free factors.
If \sw{multiroot} is on, the multiplicity of the roots will be indicated
in the output of \f{roots} or \f{realroots} by printing the root output
repeatedly, according to its multiplicity.  If \sw{multiroot} is off,
each root will be printed once, and all roots should be normally be
distinct. (Two identical roots should not appear.  If the initial
precision of the computation or the accuracy of the output was
insufficient to separate two closely-spaced roots, the program attempts to
increase accuracy and/or precision if it detects equal roots.  If,
however, the initial accuracy specified was too low, and it was not
possible to separate the roots, the program will abort.)

\index{Tracing!ROOTS package}
\item[\sw{trroot}]
  \hypertarget{switch:TRROOT}{}
\ttindexswitch[ROOTS]{trroot}
(Default off) If switch \sw{trroot} is on, trace messages
are printed out during the course of root determination, to show the
progress of solution.

\item[\sw{rootmsg}]
  \hypertarget{switch:ROOTMSG}{}
\ttindexswitch[ROOTS]{rootmsg}
(Default off) If switch
\sw{rootmsg} is on in addition to switch \sw{trroot}, additional
messages are printed out to aid in following the progress of Laguerre and
Newton complex iteration.  These messages are intended for debugging use
primarily.


\end{description}


\subsection{Operational Parameters and Parameter Setting.}

\begin{description}
\item[\var{ROOTACC!\#}]
\hypertarget{reserved:ROOTACC}{}
\hypertarget{operator:ROOTACC}{}
\ttindextype{rootacc"!\#}{global variable (ROOTS package)}
\ttindextype{rootacc}{operator (ROOTS package)}
  (Default 6) This parameter can be set using the function
\f{rootacc} n; which causes \var{rootacc!\#} to be set to max(n,6).
(If roots are closely spaced, a higher number of
significant places is computed where needed.)

\item[system precision]
  \index{system precision}
The roots package, during its operation, will
change the value of system precision but will restore the original value
of system precision at termination except that the value of system
precision is increased if necessary to allow the full roots output to be
printed.

\item[\f{PRECISION} n;]
  \ttindextype{precision}{operator!in ROOTS package@in \textsc{Roots} package}
If the user sets system precision, using the command
\texttt{precision n;} then the effect is to increase the system precision to n, and
to have the same effect on \f{roots} as \f{rootacc} n; ie. roots will now be
printed with minimum accuracy n.  The original conditions can then be
restored by using the command \texttt{PRECISION RESET;} or \texttt{PRECISION NIL;}.

\item[\f{ROOTPREC} n;]
  \hypertarget{operator:ROOTPREC}{}
\ttindextype{rootprec}{operator (ROOTS package)}
The roots package normally sets the computation mode and
precision automatically.  However, if \f{rootprec} n; is
called and $n$ is greater than the initial system precision then all root
computation will be done initially using a minimum system precision n.
Automatic operation can be restored by input of \f{rootprec} 0;.
\end{description}


\subsection{Avoiding truncation of polynomials on input}

The roots package will not internally truncate polynomials.  However, it
is possible that a polynomial can be truncated by input reading functions
of the embedding lisp system, particularly when input is given in floating
point (rounded) format.

To avoid any difficulties, input can be done in integer or Gaussian
integer format, or mixed, with integers or rationals used to represent
quantities of high precision. There are many examples of this in the
test package.  It is usually best to let the roots package
determine the precision needed to compute roots.

The number of digits that can be safely represented in floating point in
the lisp system are contained in the global variable \var{!!nfpd}.
Similarly, the maximum number of significant figures in floating point
output are contained in the global variable \var{!!flim}.  The roots
package computes these values, which are needed to control the logic of
the program. \ttindextype{"!"!flim}{global variable} \ttindextype{"!"!nfpd}{global variable}

The values of intermediate root iterations (that are printed when \sw{TRROOT}
is on) are given in bigfloat format even when the actual values
are computed in floating point.  This avoids intrusive rounding of root
printout.

