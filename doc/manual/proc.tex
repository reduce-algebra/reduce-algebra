\chapter{Procedures}\ttindextype{procedure}{command}
\hypertarget{command:PROCEDURE}{}

It is often useful to name a statement for repeated use in calculations
with varying parameters, or to define a complete evaluation procedure for
an operator. {\REDUCE} offers a procedural declaration for this purpose. Its
general syntax is:
\begin{syntax}
  [\meta{procedural type}]\texttt{ procedure }\meta{name}[\meta{varlist}]\texttt{;}
    \meta{statement}\texttt{;}
\end{syntax}
where
\begin{syntax}
  \meta{varlist}\ \BNFprod\ \texttt{(}\meta{variable}\texttt{,}\,\dots\texttt{,}\,\meta{variable}\texttt{)}
\end{syntax}
This will be explained more fully in the following sections.

In the algebraic mode of {\REDUCE} the \meta{procedural type} can be
omitted, since the default is \texttt{algebraic}.  Procedures of type 
\texttt{integer} or \texttt{real} may also be used.  In the former case, the system
checks that the value of the procedure is an integer.  At present, such
checking is not done for a real procedure, although this will change in
the future when a more complete type checking mechanism is installed.
Users should therefore only use these types when appropriate.  An empty
variable list may also be omitted.

All user-defined procedures are automatically declared to be operators.

In order to allow users relatively easy access to the whole {\REDUCE} source
program, system procedures are not protected against user redefinition. If
a procedure is redefined, a message
\begin{verbatim}
        *** <procedure name> redefined
\end{verbatim}
is printed. If this occurs, and the user is not redefining his own
procedure, he is well advised to rename it, and possibly start over
(because he has {\em already\/} redefined some internal procedure whose correct
functioning may be required for his job!)

All required procedures should be defined at the top level, since they
have global scope throughout a program. In particular, an attempt to
define a procedure within a procedure will cause an error to occur.

\section{Procedure Heading}\index{Procedure!heading}

Each procedure has a heading consisting of the word \texttt{procedure}
(optionally preceded by the word \texttt{algebraic}), followed by the name of
the procedure to be defined, and followed by its formal parameters -- the
symbols that will be used in the body of the definition to illustrate
what is to be done.  There are three cases:
\begin{enumerate}
\item No parameters. Simply follow the procedure name with a terminator
(semicolon or dollar sign).
\begin{verbatim}
        procedure abc;
\end{verbatim}

When such a procedure is used in an expression or command, \texttt{abc()}, with
empty parentheses, must be written.

\item One parameter.  Enclose it in parentheses \emph{or} just leave at
least one space, then follow with a terminator.
\begin{verbatim}
        procedure abc(x);
\end{verbatim}
or
\begin{verbatim}
        procedure abc x;
\end{verbatim}

\item More than one parameter. Enclose them in parentheses, separated by
commas, then follow with a terminator.
\begin{verbatim}
        procedure abc(x,y,z);
\end{verbatim}
\end{enumerate}
Referring to the last example, if later in some expression being evaluated
the symbols \texttt{abc(u,p*q,123)} appear, the operations of the procedure
body will be carried out as if \texttt{x} had the same value as \texttt{u} does,
\texttt{y} the same value as \texttt{p*q} does, and \texttt{z} the value 123.  The
values of \texttt{x}, \texttt{y}, \texttt{z}, after the procedure body operations
are completed are unchanged.  So, normally, are the values of \texttt{u},
\texttt{p}, \texttt{q}, and (of course) 123. (This is technically referred to as
call by value.)\index{Call by value}

The reader will have noted the word \emph{normally} a few lines earlier. The
call by value protections can be bypassed if necessary, as described
elsewhere.

\section{Procedure Body}\index{Procedure!body}

Following the delimiter that ends the procedure heading must be a 
\emph{single} statement defining the action to be performed or the value to be
delivered.  A terminator must follow the statement.  If it is a semicolon,
the name of the procedure just defined is printed.  It is not printed if a
dollar sign is used.

If the result wanted is given by a formula of some kind, the body is just
that formula, using the variables in the procedure heading.

\textit{Simple Example:}

If \texttt{f(x)} is to mean \texttt{(x+5)*(x+6)/(x+7)}, the entire procedure
definition could read
\begin{verbatim}
        procedure f x; (x+5)*(x+6)/(x+7);
\end{verbatim}
Then \texttt{f(10)} would evaluate to 240/17, \texttt{f(a-6)} to
\texttt{a*(a-1)/(a+1)}, and so on.

\textit{More Complicated Example:}

Suppose we need a function \texttt{p(n,x)} that, for any positive integer
\texttt{n}, is the Legendre polynomial\index{Legendre polynomials} of order
$n$. We can define this operator using the
textbook formula defining these functions:
\[
p_n(x) = \left.\frac{1}{n!}\  
\frac{{d}^n}{dy^n}\ \frac{1}{{(y^2 - 2xy + 1)}^{\frac{1}{2}}}\right\vert_{y=0}
\]
Put into words, the Legendre polynomial $p_n(x)$ is the result of
substituting $y=0$ in the $n^{th}$ partial derivative with respect to $y$
of a certain fraction involving $x$ and $y$, then dividing that by $n!$.

This verbal formula can easily be written in {\REDUCE}:
\begin{verbatim}
        procedure p(n,x);
           sub(y=0,df(1/(y^2-2*x*y+1)^(1/2),y,n))
               /(for i:=1:n product i);
\end{verbatim}
Having input this definition, the expression evaluation
\begin{verbatim}
        2p(2,w);
\end{verbatim}
would result in the output
\begin{verbatim}
           2
        3*w  - 1 .
\end{verbatim}
If the desired process is best described as a series of steps, then a group
or compound statement can be used.
%\extendedmanual{\newpage}

\textit{Example:}

The above Legendre polynomial example can be rewritten as a series of steps
instead of a single formula as follows:
\begin{verbatim}
        procedure p(n,x);
          begin scalar seed,deriv,top,fact;
               seed:=1/(y^2 - 2*x*y +1)^(1/2);
               deriv:=df(seed,y,n);
               top:=sub(y=0,deriv);
               fact:=for i:=1:n product i;
               return top/fact
          end;
\end{verbatim}
Procedures may also be defined recursively.  In other words, the procedure
body\index{Procedure!body} can include references to the procedure name
itself, or to other procedures that themselves reference the given
procedure.  As an example, we can define the Legendre polynomial through
its standard recurrence relation:
\begin{verbatim}
        procedure p(n,x);
           if n<0 then rederr "Invalid argument to P(N,X)"
            else if n=0 then 1
            else if n=1 then x
            else ((2*n-1)*x*p(n-1,x)-(n-1)*p(n-2,x))/n;
\end{verbatim}

\hypertarget{operator:REDERR}{}
The operator \texttt{rederr}\ttindextype{rederr}{operator} in the above example provides
for a simple error exit from an algebraic procedure (and also a block).
It can take a string as argument.

It should be noted however that all the above definitions of \texttt{p(n,x)} are
quite inefficient if extensive use is to be made of such polynomials, since
each call effectively recomputes all lower order polynomials. It would be
better to store these expressions in an array, and then use say the
recurrence relation to compute only those polynomials that have not already
been derived. We leave it as an exercise for the reader to write such a
definition.

\section{Matrix- and list-valued Procedures}
\ttindextype{matrixproc}{keyword}
\hypertarget{reserved:MATRIXPROC}{}

Normally, procedures can only return scalar values. In order for a procedure to
return a matrix, it has to be declared of 
type \k{matrixproc}:\index{Procedure!matrix-valued}
\begin{verbatim}
        matrixproc SkewSym1 (w);
           mat((0,-w(3,1),w(2,1)),
               (w(3,1),0,-w(1,1)),
               (-w(2,1), w(1,1), 0));
\end{verbatim}
Following this declaration, the call to \texttt{SkewSym1} can be used as a matrix, e.g.
\begin{verbatim}
        X := SkewSym1(mat((qx),(qy),(qz)));


             [  0     - qz   qy  ]
             [                   ]
        x := [ qz      0     - qx]
             [                   ]
             [ - qy   qx      0  ]

        X * mat((rx),(ry),(rz));


        [ qy*rz - qz*ry  ]
        [                ]
        [ - qx*rz + qz*rx]
        [                ]
        [ qx*ry - qy*rx  ]
\end{verbatim}
\hypertarget{reserved:LISTPROC}{}
Similarly, by using the keyword \k{listproc}\ttindextype{listproc}{keyword}, an algebraic
procedure can be declared to return a list:\index{Procedure!list-valued}
\begin{verbatim}
listproc spat3(u,v,w);
   begin scalar x,y;
     x := u *. w;
     y := u *. v;
     return v*x - w*y
   end;
\end{verbatim}


\section{Using LET Inside Procedures}

By using \texttt{let}\ttindex{let} instead of an assignment in the procedure
body\index{Procedure!using \texttt{let} inside body} it is possible to bypass the call-by-value
\index{Call by value} protection.  If \texttt{X} is a formal parameter or local
variable of the procedure (i.e. is in the heading or in a local
declaration), and \texttt{let} is used instead of \texttt{:=} to make an
assignment to \texttt{x}, e.g.

\begin{verbatim}
        let x = 123;
\end{verbatim}
then it is the variable that is the value of \texttt{x} that is changed.
This effect also occurs with local variables defined in a block.  If the
value of \texttt{x} is not a variable, but a more general expression, then it
is that expression that is used on the left-hand side of the \texttt{let}
statement.  For example, if \texttt{x} had the value \texttt{p*q}, it is as if
\texttt{let p*q = 123} had been executed.

\section{LET Rules as Procedures}

The \texttt{let}\ttindex{let} statement offers an alternative syntax and
semantics for procedure definition.

In place of
\begin{verbatim}
        procedure abc(x,y,z); <procedure body>;
\end{verbatim}
one can write
\begin{verbatim}
        for all x,y,z let abc(x,y,z) = <procedure body>;
\end{verbatim}
There are several differences to note.

If the procedure body contains an assignment to one of the formal
parameters, e.g.
\begin{verbatim}
        x := 123;
\end{verbatim}
in the \texttt{procedure} case it is a variable holding a copy of the first
actual argument that is changed.  The actual argument is not changed.

In the \texttt{let} case, the actual argument is changed.  Thus, if \texttt{abc}
is defined using \texttt{let}, and \texttt{abc(u,v,w)} is evaluated, the value
of \texttt{u} changes to 123.  That is, the \texttt{let} form of definition
allows the user to bypass the protections that are enforced by the call
by value conventions of standard \texttt{procedure} definitions.

\textit{Example:}  We take our earlier \texttt{factorial}\ttindex{factorial}
procedure and write it as a \texttt{let} statement.
\begin{verbatim}
        for all n let factorial n =
                    begin scalar m,s;
                    m:=1; s:=n;
                l1: if s=0 then return m;
                    m:=m*s;
                    s:=s-1;
                    go to l1
                end;
\end{verbatim}
The reader will notice that we introduced a new local variable, \texttt{s},
and set it equal to \texttt{n}.  The original form of the procedure contained
the statement \texttt{n:=n-1;}.  If the user asked for the value of
\texttt{factorial(5)} then \texttt{n} would correspond to, not just have the value
of, 5, and {\REDUCE} would object to trying to execute the statement
5 := $5-1$.

If \texttt{pqr} is a procedure with no parameters,
\begin{verbatim}
        procedure pqr;
           <procedure body>;
\end{verbatim}
it can be written as a \texttt{let} statement quite simply:
\begin{verbatim}
        let pqr = <procedure body>;
\end{verbatim}
To call \emph{procedure} \texttt{pqr}, if defined in the latter form, the empty
parentheses would not be used: use \texttt{pqr} not \texttt{pqr()} where a call
on the procedure is needed.

The two notations for a procedure with no arguments can be combined. \texttt{pqr}
can be defined in the standard \texttt{procedure} form. Then a \texttt{let}
statement
\begin{verbatim}
        let pqr = pqr();
\end{verbatim}
would allow a user to use \texttt{pqr} instead of \texttt{pqr()} in calling the
procedure.

A feature available with \texttt{let}-defined procedures and not with procedures
defined in the standard way is the possibility of defining partial
functions.\index{Function!partial}\index{Partial function}
\begin{verbatim}
    for all x such that numberp x let uvw(x)=<procedure body>;
\end{verbatim}
Now \texttt{uvw} of an integer would be calculated as prescribed by the procedure
body, while \texttt{uvw} of a general argument, such as \texttt{z} or \texttt{p+q}
(assuming these evaluate to themselves) would simply stay \texttt{uvw(z)}
or \texttt{uvw(p+q)} as the case may be.


\section{REMEMBER Statement}\ttindextype{remember}{statement}
\hypertarget{command:REMEMBER}{}

Setting the remember option for an algebraic procedure by
\begin{syntax}
     \texttt{remember (}\meta{procname:procedure}\texttt{);}
\end{syntax}
saves all intermediate results of such procedure evaluations, including
recursive calls.  Subsequent calls to the procedure can then be determined
from the saved results, and thus the number of evaluations (or the
complexity) can be reduced.  This mode of evalation costs extra memory, of
course.  In addition, the procedure must be free of side--effects.

The following examples show the effect of the remember statement
on two well--known examples.

\begin{samepage}
\begin{verbatim}
procedure H(n);      % Hofstadter's function
 if numberp n then
 << cnn := cnn +1;   % counts the calls
 if n < 3 then 1 else H(n-H(n-1))+H(n-H(n-2))>>;

remember h;

<< cnn := 0; H(100); cnn>>;

100

% H has been called 100 times only.

procedure A(m,n);    % Ackermann function

 if m=0 then n+1 else
  if n=0 then A(m-1,1) else
  A(m-1,A(m,n-1));

remember a;

A(3,3);

\end{verbatim}
\end{samepage}
