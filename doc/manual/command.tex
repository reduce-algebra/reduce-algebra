\chapter{Commands and Declarations}

A command\index{Command} is an order to the system to do something.  Some
commands cause visible results (such as calling for input or output);
others, usually called declarations\index{Declaration}, set options,
define properties of variables, or define procedures.  Commands are
formally defined as a statement followed by a terminator
\begin{syntaxtable}
  \meta{command} &\BNFprod& \meta{statement}\meta{terminator} \\
  \meta{terminator} &\BNFprod& \texttt{;}|\texttt{\$}
\end{syntaxtable}
Some {\REDUCE} commands and declarations are described in the following
sub-sections.

\hypertarget{reserved:ARRAY}{\section{Array Declarations}}

\index{Declaration!\texttt{array}}
Array\ttindextype{array}{declaration} declarations in {\REDUCE} are similar to FORTRAN
dimension statements.  For example:
\begin{verbatim}
        array a(10),b(2,3,4);
\end{verbatim}
Array indices each range from 0 to the value declared. An element of an
array is referred to in standard FORTRAN notation, e.g. \texttt{a(2)}.

We can also use an expression for defining an array bound, provided the
value of the expression is a positive integer. For example, if \var{x} has the
value 10 and \var{y} the value 7 then
\texttt{array c(5*x+y)} is the same as \texttt{array c(57)}.

If an array is referenced by an index outside its range, an error occurs.
If the array is to be one-dimensional, and the bound a number or a variable
(not a more general expression) the parentheses may be omitted:
\begin{verbatim}
        array a 10, c 57;
\end{verbatim}
The operator \texttt{length}\ttindex{length}\index{array length} applied to an array name
returns a list of its dimensions.

All array elements are initialized to 0 at declaration time. In other words,
an array element has an \emph{instant evaluation}\index{Instant evaluation}
property and cannot stand for itself.  If this is required, then an
operator should be used instead.

Array declarations can appear anywhere in a program. Once a symbol is
declared to name an array, it can not also be used as a variable, or to
name an operator or a procedure. It can however be re-declared to be an
array, and its size may be changed at that time. An array name can also
continue to be used as a parameter in a procedure, or a local variable in
a compound statement, although this use is not recommended, since it can
lead to user confusion over the type of the variable.

Arrays once declared are global in scope, and so can then be referenced
anywhere in the program. In other words, unlike arrays in most other
languages, a declaration within a block (or a procedure) does not limit
the scope of the array to that block, nor does the array go away on
exiting the block (use \texttt{clear} instead for this purpose).

\section{Mode Handling Declarations}\index{Mode}
\hypertarget{command:ON}{}
\hypertarget{command:OFF}{}
\hypertarget{switch:TIME}{}
\hypertarget{switch:DEMO}{}

\index{Mode handling declarations}\index{Declaration!mode handling}
\index{Declaration!\texttt{on}}\index{Declaration!\texttt{off}}
The \texttt{on}\ttindextype{on}{declaration} and \texttt{off}\ttindextype{off}{declaration} declarations are
available to the user for controlling various system options.  Each option
is represented by a \emph{switch}\index{Switch} name. \texttt{on} and
\texttt{off} take a list of switch names as argument and turn them on and off
respectively, e.g.,
\begin{verbatim}
       on time;
\end{verbatim}
causes the system to print a message after each command giving the elapsed
CPU time since the last command, or since \sw{time}\ttindexswitch{time} was
last turned off, or the session began.  Another useful switch with
interactive use is \sw{demo},\ttindexswitch{demo} which causes the system to
pause after each command in a file (with the exception of comments)
until a \key{Return} is typed on the terminal.  This
enables a user to set up a demonstration file and step through it command
by command.

As with most declarations, arguments to \texttt{on} and \texttt{off} may be
strung together separated by commas.  For example,
\begin{verbatim}
        off time,demo;
\end{verbatim}
will turn off both the time messages and the demonstration switch.

We note here that while most \texttt{on} and \texttt{off} commands are obeyed
almost instantaneously, some trigger slower actions such as
reading in necessary modules from secondary storage.

A diagnostic message is printed if \texttt{on}\ttindex{on} or \texttt{off}
\ttindex{off} are used with a switch that is not known to the system.  For
example, if you misspell \sw{demo} and type
\begin{verbatim}
     on demq;
\end{verbatim}
you will get the message\index{Switch}
\begin{verbatim}
        ***** demq not defined as switch.
\end{verbatim}

\section{END}
\hypertarget{reserved:END}{}

The identifier \texttt{end}\ttindex{end} has two separate uses.

1) Its use in a \texttt{begin} \ldots{} \texttt{end} bracket has been
discussed in connection with compound statements.

2) Files to be read using \texttt{in} should end with an extra \texttt{end};
command.  The reason for this is explained in the section on the \texttt{in}
command.  This use of \texttt{end} does not allow an immediately
preceding \texttt{end} (such as the \texttt{end} of a procedure definition), so
we advise using \texttt{;end;} there.

%3) A command \texttt{END}; entered at the top level transfers control to the
%Lisp system\index{Lisp} which is the host of the {\REDUCE} system.  All
%files opened by \texttt{IN} or \texttt{OUT} statements are closed in the
%process.  \texttt{END;} does not stop {\REDUCE}.  Those familiar with Lisp can
%experiment with typing identifiers and ({\tt <function name> <argument
%list>}) lists to see the value returned by Lisp. (No terminators, other
%than the RETURN key, should be used.) The data structures created during
%the {\REDUCE} run are accessible.

%You remain in this Lisp mode until you explicitly re-enter {\REDUCE} by
%saying \texttt{(BEGIN)} at the Lisp top level.  In most systems, a Lisp error
%also returns you to {\REDUCE} (exceptions are noted in the operating
%instructions for your particular {\REDUCE} implementation).  In either
%case, you will return to {\REDUCE} in the same mode, algebraic or
%symbolic, that you were in before the \texttt{END};.  If you are in
%Lisp mode\index{Lisp mode} by mistake -- which is usually the case,
%the result of typing more \texttt{END}s\ttindex{END} than \texttt{BEGIN}s --
%type \texttt{(BEGIN)} in parentheses and hit the RETURN key.

\hypertarget{command:QUIT}{}
\hypertarget{command:BYE}{\section{BYE Command}}\ttindextype{bye}{command}

\index{Command!bye@\texttt{bye}}
The command \texttt{bye}; (or alternatively \texttt{quit};)\ttindex{quit}
stops the execution
of {\REDUCE}, closes all open output files, and returns you to the calling
program (usually the operating system).  Your {\REDUCE} session is
normally destroyed.

\section{SHOWTIME Command}\ttindextype{SHOWTIME}{command}
\hypertarget{command:SHOWTIME}{}

\index{Command!showtime@\texttt{showtime}}
\texttt{showtime}; prints the elapsed time since the last call of this
command or, on its first call, since the current {\REDUCE} session began.
The time is normally given in milliseconds and gives the time as measured
by a system clock.  The operations covered by this measure are system
dependent.

\section{DEFINE Command}
\hypertarget{command:DEFINE}{}

\ttindextype{define}{command}\index{Command!define@\texttt{define}}
The command \texttt{define} allows a user to supply a new name for
any identifier or replace it by any well-formed expression.  Its argument
is a list of expressions of the form
\begin{syntaxtable}
  \meta{identifier} &\texttt{=}& \meta{number}|\meta{identifier}|\meta{operator}|\\
                    &          & \meta{reserved word}|\meta{expression}
\end{syntaxtable}

\textit{Example:}
\begin{verbatim}
        define be==,x=y+z;
\end{verbatim}
means that \texttt{be} will be interpreted as an equal sign, and \texttt{x}
as the expression \texttt{y+z} from then on.  This renaming is done at parse
time, and therefore takes precedence over any other replacement declared
for the same identifier.  It stays in effect until the end of the
{\REDUCE} run.

The identifiers \texttt{algebraic} and \texttt{symbolic} have properties which
prevent \texttt{define} from being used on them.  To define
\texttt{alg} to be a synonym for \texttt{algebraic}, use the more complicated
construction
\begin{verbatim}
        put('alg,'newnam,'algebraic);
\end{verbatim}

\section{RESETREDUCE Command}
\hypertarget{command:RESETREDUCE}{}
\ttindextype{resetreduce}{command}\index{Command!resetreduce@\texttt{resetreduce}}

The command \texttt{resetreduce} works through the
history of previous commands, and clears any values which have been
assigned, plus any rules, arrays and the like.  It also sets the various
switches to their initial values.  It is not complete, but does work for
most things that cause a gradual loss of space.
