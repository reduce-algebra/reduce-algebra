
\hypertarget{sec:appendixd}{\chapter{Description of Algorithms}}
\label{sec:algorithms}

\section{Definite Integration}
\label{sec:definiteIntegration}
%% For MathJax, must be part of a paragraph to avoid extra space:
\ifdefined\VerbMath{\makehashletter
\(
\newcommand{\MeijerGparams}[2]{\genfrac{}{}{0pt}{}{#1}{#2}}
\)%
}\else
\newcommand{\MeijerGparams}[2]{\genfrac{}{}{0pt}{}{#1}{#2}}
\fi
This section  describes part of \REDUCE's definite
integration package that is able to calculate the definite integrals of
many functions, including several special functions.  There are other
parts of this package, such as Stan Kameny's code for contour integration,
that are not included here.  The integration process described here is not
the more normal approach of initially calculating the indefinite integral,
but is instead the rather unusual idea of representing each function as a
Meijer G-function (a formal definition of the Meijer G-function can be
found in \cite {Prudnikov:90c}), and then calculating the integral by using
the following Meijer G integration formula.
\index{Meijer's G function!use for definite integration}

\begin{equation}
\int_{0}^{\infty} x^{\alpha-1} G^{s t}_{u v}
\left( \sigma x \  \Bigg\vert \  \MeijerGparams{( c_u)}{(d_v)} \right)
G^{m n}_{p q} \left( \omega x^{l/k} \  \Bigg\vert \ \MeijerGparams{(a_p)}{(b_q)}
\right) dx = k G^{i j}_{k l} \left( \xi \ \Bigg\vert \
\MeijerGparams{(g_k)}{(h_l)} \right) \label{defint:eq1}
\end{equation}

The resulting Meijer G-function is then retransformed, either directly
or via a hypergeometric function simplification, to give
the answer. A more detailed account of this theory can be found in
\cite {Adamchik90}.

\subsection{Integration between zero and infinity}

As an example, if one wishes to calculate the following integral

\[
\int_{0}^{\infty} x^{-1} e^{-x} \sin(x) \, dx
\]

then initially the  correct Meijer G-functions are found, via a
pattern matching
process, and are substituted into eq.~\eqnref{defint:eq1} to give

\[
\sqrt{\pi} \int_{0}^{\infty} x^{-1} G^{1 0}_{0 1} \left(x
\ \Bigg\vert \
\MeijerGparams{.}{0}\right) G^{1 0}_{0 2}\left(\frac{x^{2}}{4}
\ \Bigg\vert \ \MeijerGparams{. \; . }{\frac{1}{2} \; 0} \right) dx
\]

The cases for validity of the integral are then checked. If these
are found to be satisfactory then the formula is calculated and we
obtain the following Meijer G-function

\[
G^{1 2}_{2 2} \left(1 \ \Bigg\vert \ \MeijerGparams{\frac{1}{2} \; 1}{\frac{1}{2} \; 0} \right)
\]

This is reduced to the following hypergeometric function

\[
_2F_1 (\frac{1}{2},1;\frac{3}{2};-1)
\]

which is then calculated to give the correct answer of

\[
\frac{\pi}{4}
\]

The above formula (\eqnref{defint:eq1}) is also true for the
integration of a single Meijer G-function by replacing the second
Meijer G-function with a trivial Meijer G-function.

A list of numerous particular Meijer G-functions is available in
\cite {Prudnikov:90c}.

\subsection{Integration over other ranges}

Although the description so far has been limited to the computation of
definite integrals between 0 and infinity, it can also be extended to
calculate integrals between 0 and some specific upper bound, and
by further extension, integrals between any two bounds.  One approach is
to use the Heaviside function, i.e.

\[
\int_{0}^{\infty} x^{2} e^{-x} H(1-x)\,dx = \int_{0}^{1} x^{2} e^{-x}dx
\]

Another approach, again not involving the normal indefinite integration
process, again uses Meijer G-functions, this time by means of the
following formula

\begin{equation}
\int_{0}^{y} x^{\alpha-1} G^{m n}_{p q}
\left( \sigma x \  \Bigg\vert \ \MeijerGparams{(a_u)}{(b_v)} \right) dx=%
y^{\alpha}\,G^{m \; n+1}_{p+1 \; q+1} \left( \sigma y \  \Bigg\vert \
\MeijerGparams{( a_1..a_n,1-\alpha,a_{n+1}..a_p)}{(b_1..b_m,-\alpha,b_{m+1}..b_q)} \right) \label{defint:eq2}
\end{equation}

For a more detailed look at the theory behind this see
\cite{Adamchik90}.

For example, if one wishes to calculate the following integral

\[
\int_{0}^{y} \sin(2 \sqrt{x}) \, dx
\]

then initially the correct Meijer G-function is found, by a pattern
matching process, and is substituted
into eq.~\eqnref{defint:eq2} to give

\[
\int_{0}^{y} G^{1 0}_{0 2}\left(x
\ \Bigg\vert \ \MeijerGparams{. \; .}{\frac{1}{2} \; 0} \right) dx
\]

which then in turn gives

\[
y \; G^{1 1}_{1 3}\left(y \ \Bigg\vert \ \MeijerGparams{0}{\frac{1}{2} -\!1 \; 0} \right) dx
\]

and returns the result

\[
\frac{\sqrt{\pi} \, J_{3/2}(2 \, \sqrt{\,y}) \, y}{y^{1/4}}
\]

\subsubsection{Examples}

\[
\int_{0}^{\infty} e^{-x} dx
\]


\begin{verbatim}
 int(e^(-x),x,0,infinity);

 1
\end{verbatim}

\[
\int_{0}^{\infty} x \sin(1/x) \, dx
\]

\begin{verbatim}
 int(x*sin(1/x),x,0,infinity);

           1
int(x*sin(---),x,0,infinity)
           x
\end{verbatim}

\[
\int_{0}^{\infty} x^2 \cos(x) \, e^{-2x} dx
\]

\begin{verbatim}
 int(x^2*cos(x)*e^(-2*x),x,0,infinity);

   4
 -----
  125
\end{verbatim}

\[
\int_{0}^{\infty} x e^{-1/2x} H(1-x) \,dx = \int_{0}^{1} x e^{-1/2x} dx
\]

\begin{verbatim}
 int(x*e^(-1/2x)*Heaviside(1-x),x,0,infinity);

  2*(2*sqrt(e) - 3)
 -------------------
       sqrt(e)
\end{verbatim}

\[
\int_{0}^{1} x \log(1+x) \,dx
\]

\begin{verbatim}
 int(x*log(1+x),x,0,1);

  1
 ---
  4
\end{verbatim}

\[
\int_{0}^{y} \cos(2x) \,dx
\]

\begin{verbatim}
 int(cos(2x),x,y,2y);

 sin(4*y) - sin(2*y)
---------------------
          2
\end{verbatim}
\subsection{Integral Transforms}

A useful application of the definite integration package is in the
calculation of various integral transforms. The transforms
available are as follows:

\begin{itemize}
\item Laplace transform
\item Hankel transform
\item Y-transform
\item K-transform
\item StruveH transform
\item Fourier sine transform
\item Fourier cosine transform
\end{itemize}

\subsubsection{Laplace transform}
\index{Laplace transform}
\index{Integral transform!Laplace transform}
\hypertarget{operator:LAPLACE_TRANSFORM}{}
\ttindextype{laplace\_transform}{operator}

The Laplace transform

\[
f(s) = \mathcal{L}\{\mathrm{F}(t)\} = \int_{0}^{\infty} e^{-st}F(t)\,dt
\]

can be calculated by using the \f{laplace\_transform} operator.

This requires as parameters
\begin{itemize}
\item the function to be integrated
\item the integration variable.
\end{itemize}

For example

\[
\mathcal{L}\{e^{-at}\}
\]

is entered as

\begin{verbatim}
        laplace_transform(e^(-a*x),x);
\end{verbatim}

and returns the result

\[
\frac{1}{s+a}
\]

\subsubsection{Hankel transform}
\index{Hankel transform}
\index{Integral transform!Hankel transform}
\hypertarget{operator:HANKEL_TRANSFORM}{}
\ttindextype{hankel\_transform}{operator}

The Hankel transform
\[
f(\omega) = \int_{0}^{\infty} F(t) \,J_{\nu}(2\sqrt{\omega t}) \,dt
\]

can be calculated by using the \f{hankel\_transform} operator, e.g.,

\begin{verbatim}
        hankel_transform(f(x),x);
\end{verbatim}

This is used in the same way as the \f{laplace\_transform} operator.

\subsubsection{Y-transform}
\index{Y-transform}
\index{Integral transform!Y-transform}
\hypertarget{operator:Y_TRANSFORM}{}
\ttindextype{Y\_transform}{operator}

The Y-transform

\[
f(\omega) = \int_{0}^{\infty} F(t) \,Y_{\nu}(2\sqrt{\omega t}) \,dt
\]

can be calculated by using the \f{Y\_transform} operator, e.g.,

\begin{verbatim}
        Y_transform(f(x),x);
\end{verbatim}

This is used in the same way as the \f{laplace\_transform} operator.

\subsubsection{K-transform}
\index{K-transform}
\index{Integral transform!K-transform}
\hypertarget{operator:K_TRANSFORM}{}
\ttindextype{K\_transform}{operator}

The K-transform

\[
f(\omega) = \int_{0}^{\infty} F(t) \,K_{\nu}(2\sqrt{\omega t}) \,dt
\]

can be calculated by using the \f{K\_transform} operator, e.g.,

\begin{verbatim}
        K_transform(f(x),x);
\end{verbatim}

This is used in the same way as the \f{laplace\_transform} operator.

\subsubsection{StruveH transform}
\index{StruveH transform}
\index{Integral transform!StruveH transform}
\hypertarget{operator:STRUVEH_TRANSFORM}{}
\ttindextype{struveh\_transform}{operator}

The StruveH transform

\[
f(\omega) = \int_{0}^{\infty} F(t) \,StruveH(\nu,2\sqrt{\omega t}) \,dt
\]

can be calculated by using the \f{struveh\_transform} operator, e.g.,

\begin{verbatim}
        struveh_transform(f(x),x);
\end{verbatim}

This is used in the same way as the \f{laplace\_transform} operator.

\subsubsection{Fourier sine transform}
\index{Fourier sine transform}
\index{Integral transform!Fourier sine transform}
\hypertarget{operator:FOURIER_SIN}{}
\ttindextype{fourier\_sin}{operator}

The Fourier sine transform

\[
f(s) = \int_{0}^{\infty} F(t) \sin(st) \,dt
\]

can be calculated by using the \f{fourier\_sin} operator, e.g.,
\begin{verbatim}
        fourier_sin(f(x),x);
\end{verbatim}

This is used in the same way as the \f{laplace\_transform} operator.

\subsubsection{Fourier cosine transform}
\index{Fourier cosine transform}
\index{Integral transform!Fourier cosine transform}
\hypertarget{operator:FOURIER_COS}{}
\ttindextype{fourier\_cos}{operator}

The Fourier cosine transform

\[
f(s) = \int_{0}^{\infty} F(t) \cos(st) \,dt
\]

can be calculated by using the \f{fourier\_cos} operator, e.g.,
\begin{verbatim}
        fourier_cos(f(x),x);
\end{verbatim}

This is used in the same way as the \f{laplace\_transform} operator.

\subsubsection{The print\_conditions function}
\hypertarget{operator:PRINT_CONDITIONS}{}
\ttindextype{PRINT\_CONDITIONS}{operator}

The required conditions for the validity of the transform integrals
can be viewed using the following command:

\begin{verbatim}
        print_conditions().
\end{verbatim}

For example after calculating the following laplace transform

\begin{verbatim}
        laplace_transform(x^k,x);
\end{verbatim}

using the \f{print\_conditions} operator would produce

\begin{verbatim}
repart(sum(ai) - sum(bj)) + 1/2 (q + 1 - p)>(q - p) repart(s)

 and ( - min(repart(bj))<repart(s))<1 - max(repart(ai))

 or mod(arg(eta))=pi*delta

 or ( - min(repart(bj))<repart(s))<1 - max(repart(ai))

 or mod(arg(eta))<pi*delta
\end{verbatim}

where
\[
\begin{array}{l}
delta = s+t-\frac{u-v}{2}\\
eta = 1-\alpha(v-u)-\mu-\rho\\
\mu = \sum_{j=1}^{q} b_{j} - \sum_{i=1}^{p} a_{i} + \frac{p-q}{2} + 1\\
\rho = \sum_{j=1}^{v} d{j} - \sum_{i=1}^{u} c_{i} + \frac{u-v}{2} + 1\\
s,t,u,v,p,q,\alpha \; \text{as in eq.~\ref{defint:eq1}}
\end{array}
\]

\subsection{Extending the Tables}

The relevant Meijer G representation for any function is found by a
pattern-matching process which is carried out on a list of Meijer
G-function definitions. This list, although extensive, can never hope
to be complete and therefore the user may wish to add more definitions.
Definitions can be added by adding the following lines:

\begin{verbatim}
  defint_choose(f(~x),~var => f1(n,x);

  symbolic putv(mellin!-transforms!*,n,'
                (() (m n p q) (ai) (bj) (C) (var)));

\end{verbatim} 
where $f(x)$ is the new function, $i = 1,\ldots, p$, $j=1,\ldots,q$, $C$ a constant,
%where i = 1..p, j=1..q, C = a constant,
var = variable, n = an indexing number, where all expression must be in internal prefix form.

For example when considering $\cos(x)$ we have
the \textit{Meijer G representation --}
\[
\cos x = \sqrt{\pi} \,G^{1 0}_{0 2}\left(\frac{x^{2}}{4} \ \Bigg\vert
\ \MeijerGparams{ . \; . }{ 0 \; \frac{1}{2}} \right) dx
\]
i.e. $m=1$, $n=0$, $p=0$, $q=2$, $a_{i}=\{\}$, $b_{j} = \{0,1/2)$, $C=\sqrt{\pi}$.
The corresponding \textit{internal definite integration package representation} ist
\begin{verbatim}
  defint_choose(cos(~x),~var)   => f1(3,x);
\end{verbatim}
where 3 is the indexing number corresponding to the 3
in the following formula
\begin{verbatim}
  symbolic putv(mellin!-transforms!*,3,'
                (() (1 0 0 2) () (nil (quotient 1 2))
                (sqrt pi) (quotient (expt x 2) 4)));
\end{verbatim} 

or the more interesting example of the Bessel function $J_{n}(x)$:

\textit{Meijer G representation --}
\[
  J_{n}(x)
G^{1 0}_{0 2} \left(\frac{x^{2}}{4} \ \Bigg\vert
\ \MeijerGparams{. \; .}{\frac{n}{2} \; {\frac{-n}{2}}} \right) dx
\]

\textit{Internal definite integration package representation --}

\begin{verbatim}
  defint_choose(besselj(~n,~x),~var) => f1(50,x,n);

  symbolic putv(mellin!-transforms!*,50,'
                ((n) (1 0 0 2) () ((quotient n 2)
                                   (minus quotient n 2)) 1
                                   (quotient (expt x 2) 4)));
\end{verbatim} 



\subsection{Acknowledgements}
\index{Adamchik, Victor}\index{People!Adamchik, Victor}
Kerry Gaskell would like to thank Victor Adamchik whose implementation of the
definite integration package for {\REDUCE} is vital to this
interface.


\newpage

\section{The CVIT package}
\indexpackage{CVIT}
\label{sec:CVIT}
This package provides an alternative method for computing traces of Dirac
gamma matrices, based on an algorithm by Cvitanovich that treats gamma
matrices as 3-j symbols.

Authors: V.Ilyin, A.Kryukov, A.Rodionov, A.Taranov.
\index{Ilyin, V.}\index{People!Ilyin, V.}
\index{Kryukov, A.}\index{People!Kryukov, A.}
\index{Rodionov, A.}\index{People!Rodionov, A.}
\index{Taranov, A.}\index{People!Taranov, A.}

In modern high energy physics the calculation of Feynman diagrams are
still very important. One of the difficulties of these calculations
are trace calculations. So the calculation of traces of Dirac's
$\gamma$-matrices were one of first task of computer algebra systems.
All available algorithms are based on the fact that gamma-matrices
constitute a basis of a Clifford algebra:
\[
  \left\{G_{m},G_{n}\right\} = 2g_{mn}.
\]

We present the implementation of an alternative algorithm based on
treating of gamma-matrices as 3-j symbols (details may be found in
\cite{Ilyin:89,Kennedy:1982}).

The program consists of 5 modules described below.

\newpage
\begin{verbatim}

                    MODULES CROSS REFERENCES
   +--------+
   | REDUCE |
   |________|                  |ISIMP1
    ISIMP2|         +-----------------------+
          +--->-----| RED_TO_CVIT_INTERFACE |
                    |_______________________|
                 CALC_SPUR|          |REPLACE_BY_VECTOR
                          |          |REPLACE_BY_VECTORP
                          |          |GAMMA5P
                          ^          V
                         +--------------+
                         | CVITMAPPING  |
                         |______________|
                                ^
                                |PRE-CALC-MAP
                                |CALC_MAP_TAR
                                |CALC_DENTAR
                                |
                         +-------------+
                         | INTERFIERZ  |
                         |_____________|
                            |         |MK-NUMR
                            |         |STRAND-ALG-TOP
                            |         ^
               MAP-TO-STRAND|    +------------+
                   INCIDENT1|    | EVAL-MAPS  |
                            |    |____________|
                            ^               |DELETEZ1
                            |               |CONTRACT-STRAND
                  +----------------+        |COLOR-STRAND
                  | MAP-TO-STRAND  |---->---+
                  |________________|


\end{verbatim}


\subsection*{Module RED\_TO\_CVIT\_INTERFACE}

\begin{center}
Author: A.P.Kryukov \\
Purpose:interface REDUCE and CVIT package
\end{center}

RED\_TO\_CVIT\_INTERFACE module is intended for connection of REDUCE
with main module of CVIT package. The main idea is to preserve
standard REDUCE syntax for high energy calculations.  For realization
of this we redefine {\ SYMBOLIC PROCEDURE ISIMP1} from HEPhys module of
REDUCE system.

After loading CVIT package user may use switch CVIT which is {\tt ON} by
default.  If switch CVIT is {\tt OFF} then calculations of Diracs matrices
traces are performed using standard REDUCE facilities. If CVIT switch
is {\tt ON} then CVIT package will be active.

{\tt RED\_TO\_CVIT\_INTERFACE} module performs some primitive simplification
and control input data independently.  For example it remove $G_mG_m$,
check parity of the number of Dirac matrices in each trace \emph{etc}.
There is one principal restriction concerning G5-matrix. There are no
closed form for trace in non-integer dimension case when trace include
G5-matrix.  The next restriction is that if the space-time dimension
is integer then it must be even (2,4,6,...).  If these and other
restrictions are violated then the user get corresponding error
message. List of messages is included.

\begin{center}
\begin{verbatim}
                  LIST OF IMPORTED FUNCTIONS
-------------------------------------------------
 Function              From module
-------------------------------------------------
 ISIMP2                HEPhys
 CALC_SPUR             CVITMAPPING
-------------------------------------------------
\end{verbatim}

\begin{verbatim}
                  LIST OF EXPORTED FUNCTION
-------------------------------------------------
 Function              To module
-------------------------------------------------
 ISIMP1                HEPhys (redefine)
 REPLACE_BY_VECTOR     EVAL_MAP
 REPLACE_BY_VECTORP    EVAL__MAP
 GAMMA5P               CVITMAPPING, EVAL_MAP
-------------------------------------------------
\end{verbatim}
\end{center}



\subsection*{Module CVITMAPPING}

\begin{center}
Author: A.Ya.Rodionov \\
Purpose: graphs reduction
\end{center}

CVITMAPPING module is intended for diagrams calculation according to
Cvitanovic - Kennedy algorithm. The top function of this module
CALC\_SPUR is called from RED\_TO\_CVIT\_INTERFACE interface module.
The main idea of the algorithm consists in diagram simplification
according to rules (1.9') and (1.14) from [1].  The input data - trace
of Diracs gamma matrices (G-matrices) has a form of a list of
identifiers lists with cyclic order. Some of identifiers may be
identical.  In this case we assume summation over dummy indices. So
trace Sp(GbGr).Sp(GwGbGcGwGcGr) is represented as list ((b r) (w b c w
c r)).

The first step is to transform the input data to ``map'' structure and
then to reduce the map to a ``simple'' one. This transformation is made
by function TRANSFORM\_MAP\_ (top function). Transformation is made in
three steps. At the first step the input data are transformed to the
internal form - a map (by function PREPARE\_MAP\_). At the second step
a map is subjected to Fierz transformations (1.14) (function
MK\_SIMPLE\_MAP\_). At this step of optimization can be maid (if
switch CVITOP is on) by function MK\_FIRZ\_OP.  In this case Fierzing
starts with linked vertices with minimal distance (number of vertices)
between them.  After Fierz transformations map is further reduced by
vertex simplification routine MK\_SIMPLE\_VERTEX using (1.9').
Vertices reduced to primitive ones, that is to vertices with three or
less edges.  This is the last (third) step in transformation from
input to internal data.

The next step is optional.  If switch CVITBTR is on factorisation of
bubble (function FIND\_BUBBLES1) and triangle (function
FIND\_TRIANGLES1) submaps is made.  This factorisation is very
efficient for ``wheel'' diagrams and unnecessary for ``lattice'' diagrams.
Factorisation is made recursively by substituting composed edges for
bubbles and composed vertices for triangles.  So check (function
SORT\_ATLAS) must be done to test possibility of future marking
procedure.  If the check fails then a new attempt to reorganize atlas
(so we call complicated structure witch consists of MAP, COEFFicient
and DENOMinator) is made. This cause backtracking (but very seldom).
Backtracking can be traced by turning on switch CVITRACE. FIND\_BUBLTR
is the top function of this program's branch.

Then atlases must be prepared (top function WORLD\_FROM\_ATLAS) for
final algebraic calculations.  The resulted object called ``world''
consists of edges names list (EDGELIST), their marking variants
(VARIANTS) and WORLD1 structure. WORLD1 structure differs from WORLD
structure in one point.  It contains MAP2 structure instead of MAP
structure. MAP2 is very complicated structure and consist of VARIANTS,
marking plan and GSTRAND.  (GSTRAND constructed by PRE!-CALC!-MAP\_
from INTERFIERZ module.)  By marking we understand marking of edges
with numbers according to Cvitanovic - Kennedy algorithm.

The last step is performed by function CALC\_WORLD. At this step
algebraic calculations are done.  Two functions CALC\_MAP\_TAR and
CALC\_DENTAR from INTERFIERZ module make algebraic expressions in the
prefix form. This expressions are further simplified by function
{\tt REVAL}.  This is the REDUCE system general function for algebraic
expressions simplification. {\tt REVAL} and {\tt SIMP!*} are the only REDUCE
functions used in this module.

There are also some functions for printing several internal
structures: PRINT\_ATLAS, PRINT\_VERTEX, PRINT\_EDGE, PRINT\_COEFF,
PRINT\_DENOM.  This functions can be used for debugging.

If an error occur in module CVITMAPPING the error message ``ERROR IN
MAP CREATING ROUTINES'' is displayed.  Error has number 55.  The switch
CVITERROR allows to give full information about error: name of
function where error occurs and names and values of function's
arguments. If CVITERROR switch is on and backtracking fails message
about error in SORT\_ATLAS function is printed.  The result of
computation however will be correct because in this case factorized
structure is not used. This happens extremely seldom.


\begin{verbatim}
                  List of imported function
-------------------------------------------------
 function              from module
-------------------------------------------------
 REVAL                 REDUCE
 SIMP!*                REDUCE
 CALC_MAP_TAR          INTERFIERZ
 CALC_DENTAR           INTERFIERZ
 PRE!-CALC!-MAP_       INTERFIERZ
 GAMMA5P               RED_TO_CVIT_INTERFACE
-------------------------------------------------
\end{verbatim}

\begin{verbatim}
                  List of exported function
-------------------------------------------------
 function              to module
-------------------------------------------------
 CALC_SPUR             REDUCE - CVIT interface
-------------------------------------------------
\end{verbatim}

\begin{verbatim}
                        Data structure
 WORLD     ::=  (EDGELIST,VARIANTS,WORLD1)
 WORLD1    ::=  (MAP2,COEFF,DENOM)
 MAP2      ::=  (MAPS,VARIANTS,PLAN)
 MAPS      ::=  (EDGEPAIR . GSTRAND)
 MAP1      ::=  (EDGEPAIR . MAP)
 MAP       ::=  list of VERTICES (unordered)
 EDGEPAIR  ::=  (OLDEDGELIST . NEWEDGELIST)
 COEFF     ::=  list of WORLDS (unordered)
 ATLAS     ::=  (MAP,COEFF,DENOM)
 GSTRAND   ::=  (STRAND*,MAP,TADPOLES,DELTAS)
 VERTEX    ::=  list of EDGEs (with cyclic order)
 EDGE      ::=  (NAME,PROPERTY,TYPE)
 NAME      ::=  ATOM
 PROPERTY  ::=  (FIRSTPAIR . SECONDPAIR)
 TYPE      ::=  T or NIL
 ------------------------------------------------
 *Define in module MAP!-TO!-STRAND.

\end{verbatim}

\subsection*{Modules INTERFIERZ, EVAL\_MAPS, AND MAP-TO-STRAND}

\begin{center}
Author: A.Taranov \\
Purpose: evaluate single Map
\end{center}

Module INTERFIERZ exports to module CVITMAPPING three functions:
PRE-CALC-MAP\_, CALC-MAP\_TAR, CALC-DENTAR.

Function PRE-CALC-MAP\_ is used for preliminary processing of a map. It
returns a list of the form (STRAND NEWMAP TADEPOLES DELTAS) where
STRAND is strand structure described in MAP-TO-STRAND module.  NEWMAP
is a map structure without ``tadepoles'' and ``deltas''.  ``Tadepole'' is a
loop connected with map with only one line (edge). ``Delta'' is a single
line disconnected from a map.  TADEPOLES is a list of ``tadepole''
submaps.  DELTAS is a list (CONS E1 E2) where E1 and E2 are

Function CALC\_MAP\_TAR takes a list of the same form as returned by
PRE-CALC-MAP\_, a-list, of the form (...  edge . weight ...  )  and
returns a prefix form of algebraic expression corresponding to the map
numerator.

Function CALC-DENTAR returns a prefix form of algebraic expression
corresponding to the map denominator.

Module EVAL-MAP exports to module INTERFIERZ functions MK-NUMR and
STRAND-ALG-TOP.

Function MK-NUMR returns a prefix form for some combinatorial
coefficient (Pohgammer symbol).

Function STRAND-ALG-TOP performs an actual computation of a prefix
form of algebraic expression corresponding to the map numerator. This
computation is based on a ``strand'' structure constructed from the
``map'' structure.

Module MAP-TO-STRAND exports functions MAP-TO-STRAND, INCIDENT1 to
module INTERFIERZ and functions DELETEZ1, CONTRACT-STRAND,
COLOR-STRAND to module EVAL-MAPS.

Function INCIDENT1 is a selector in ``strand'' structure.  DELETEZ1
performs auxiliary optimization of ``strand''.  MAP-TO-STRAND transforms
``map'' to ``strand'' structure.  The latter is describe in program
module.

CONTRACT-STRAND do strand vertex simplifications of ``strand'' and
COLOR-STRAND finishes strand generation.

\begin{verbatim}

            Description of STRAND  data structure.
 STRAND ::=<LIST OF VERTEX>
 VERTEX ::=<NAME> . (<LIST OF ROAD> <LIST OF ROAD>)
 ROAD   ::=<ID> . NUMBER
 NAME   ::=NUMBER
\end{verbatim}


\subsubsection*{LIST OF MESSAGES}

\begin{itemize}

\item{CALC\_SPUR: $<$vecdim$>$ IS NOT EVEN SPACE-TIME DIMENSION}
 The dimension of space-time $<$vecdim$ $is integer but not
even. Only even numeric dimensions are allowed.

\item{NOSPUR NOT YET IMPLEMENTED}
 Attempt to calculate trace when NOSPUR switch is on.  This facility
is not implemented now.

\item{G5 INVALID FOR VECDIM NEQ 4}
 Attempt to calculate trace with gamma5-matrix for space-time
dimension not equal to 4.

\item{CALC\_SPUR: $<$expr$>$ HAS NON-UNIT DENOMINATOR}
The <expr> has non-unit denominator.

\item{THREE INDICES HAVE NAME $<$name$>$}
 There are three indices with equal names in evaluated expression.

\end{itemize}

\begin{verbatim}
                  List of switches
------------------------------------------------------------
 switch           default    comment
------------------------------------------------------------
 CVIT             ON         If it is on then use Kennedy-
                             Cvitanovic algorithm else use
                             standard facilities.
 CVITOP           OFF        Fierz optimization switch
 CVITBTR          ON         Bubbles and triangles
                             factorisation switch
 CVITRACE         OFF        Backtracking tracing switch
------------------------------------------------------------
\end{verbatim}


\begin{verbatim}
                 Functions cross references*.

CALC_SPUR
 |
 +-->SIMP!* (REDUCE)
     |
     +-->CALC_SPUR0
         |
         |--->TRANSFORM_MAP_
         |    |
         |    |--->MK_SIMPLE_VERTEX
         |    +--->MK_SIMPLE_MAP_
         |         |
         |         +--->MK_SIMPLE_MAP_1
         |              |
         |              +--->MK_FIERS_OP
         |
         |--->WORLD_FROM_ATLAS
         |    |
         |    +--->CONSTR_WORLDS
         |         |
         |         +---->MK_WORLD1
         |               |
         |               +--->MAP_2_FROM_MAP_1
         |                    |
         |                    |--->MARK_EDGES
         |                    +--->MAP_1_TO_STRAND
         |                         |
         |                         +-->PRE!-CALC!-MAP_
         |                             (INTERFIRZ)
         |
         |--->CALC_WORLD
         |    |
         |    |--->CALC!-MAP_TAR (INTERFIRZ)
         |    |--->CALC!-DENTAR (INTERFIRZ)
         |    +--->REVAL (REDUCE)
         |
         +--->FIND_BUBLTR
              |
              +--->FIND_BUBLTR0
                   |
                   |--->SORT_ATLAS
                   +--->FIND_BUBLTR1
                        |
                        |--->FIND_BUBLES1
                        +--->FIND_TRIANGLES1
*Unmarked functions are from CVITMPPING module.

\end{verbatim}
