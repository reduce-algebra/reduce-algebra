
\hypertarget{sec:appendixd}{\chapter{Description of Algorithms}}
\label{sec:algorithms}

\section{Definite Integration}
\label{sec:definiteIntegration}
%% For MathJax, must be part of a paragraph to avoid extra space:
\ifdefined\VerbMath{\makehashletter
\(
\newcommand{\MeijerGparams}[2]{\genfrac{}{}{0pt}{}{#1}{#2}}
\)%
}\else
\newcommand{\MeijerGparams}[2]{\genfrac{}{}{0pt}{}{#1}{#2}}
\fi
This section  describes part of \REDUCE's definite
integration package that is able to calculate the definite integrals of
many functions, including several special functions.  There are other
parts of this package, such as Stan Kameny's code for contour integration,
that are not included here.  The integration process described here is not
the more normal approach of initially calculating the indefinite integral,
but is instead the rather unusual idea of representing each function as a
Meijer G-function (a formal definition of the Meijer G-function can be
found in \cite {Prudnikov:90c}), and then calculating the integral by using
the following Meijer G integration formula.
\index{Meijer's G function!use for definite integration}

\begin{equation}
\int_{0}^{\infty} x^{\alpha-1} G^{s t}_{u v}
\left( \sigma x \  \Bigg\vert \  \MeijerGparams{( c_u)}{(d_v)} \right)
G^{m n}_{p q} \left( \omega x^{l/k} \  \Bigg\vert \ \MeijerGparams{(a_p)}{(b_q)}
\right) dx = k G^{i j}_{k l} \left( \xi \ \Bigg\vert \
\MeijerGparams{(g_k)}{(h_l)} \right) \label{defint:eq1}
\end{equation}

The resulting Meijer G-function is then retransformed, either directly
or via a hypergeometric function simplification, to give
the answer. A more detailed account of this theory can be found in
\cite {Adamchik90}.

\subsection{Integration between zero and infinity}

As an example, if one wishes to calculate the following integral

\[
\int_{0}^{\infty} x^{-1} e^{-x} \sin(x) \, dx
\]

then initially the  correct Meijer G-functions are found, via a
pattern matching
process, and are substituted into eq.~\eqnref{defint:eq1} to give

\[
\sqrt{\pi} \int_{0}^{\infty} x^{-1} G^{1 0}_{0 1} \left(x
\ \Bigg\vert \
\MeijerGparams{.}{0}\right) G^{1 0}_{0 2}\left(\frac{x^{2}}{4}
\ \Bigg\vert \ \MeijerGparams{. \; . }{\frac{1}{2} \; 0} \right) dx
\]

The cases for validity of the integral are then checked. If these
are found to be satisfactory then the formula is calculated and we
obtain the following Meijer G-function

\[
G^{1 2}_{2 2} \left(1 \ \Bigg\vert \ \MeijerGparams{\frac{1}{2} \; 1}{\frac{1}{2} \; 0} \right)
\]

This is reduced to the following hypergeometric function

\[
_2F_1 (\frac{1}{2},1;\frac{3}{2};-1)
\]

which is then calculated to give the correct answer of

\[
\frac{\pi}{4}
\]

The above formula (\eqnref{defint:eq1}) is also true for the
integration of a single Meijer G-function by replacing the second
Meijer G-function with a trivial Meijer G-function.

A list of numerous particular Meijer G-functions is available in
\cite {Prudnikov:90c}.

\subsection{Integration over other ranges}

Although the description so far has been limited to the computation of
definite integrals between 0 and infinity, it can also be extended to
calculate integrals between 0 and some specific upper bound, and
by further extension, integrals between any two bounds.  One approach is
to use the Heaviside function, i.e.

\[
\int_{0}^{\infty} x^{2} e^{-x} H(1-x)\,dx = \int_{0}^{1} x^{2} e^{-x}dx
\]

Another approach, again not involving the normal indefinite integration
process, again uses Meijer G-functions, this time by means of the
following formula

\begin{equation}
\int_{0}^{y} x^{\alpha-1} G^{m n}_{p q}
\left( \sigma x \  \Bigg\vert \ \MeijerGparams{(a_u)}{(b_v)} \right) dx=%
y^{\alpha}\,G^{m \; n+1}_{p+1 \; q+1} \left( \sigma y \  \Bigg\vert \
\MeijerGparams{( a_1..a_n,1-\alpha,a_{n+1}..a_p)}{(b_1..b_m,-\alpha,b_{m+1}..b_q)} \right) \label{defint:eq2}
\end{equation}

For a more detailed look at the theory behind this see
\cite{Adamchik90}.

For example, if one wishes to calculate the following integral

\[
\int_{0}^{y} \sin(2 \sqrt{x}) \, dx
\]

then initially the correct Meijer G-function is found, by a pattern
matching process, and is substituted
into eq.~\eqnref{defint:eq2} to give

\[
\int_{0}^{y} G^{1 0}_{0 2}\left(x
\ \Bigg\vert \ \MeijerGparams{. \; .}{\frac{1}{2} \; 0} \right) dx
\]

which then in turn gives

\[
y \; G^{1 1}_{1 3}\left(y \ \Bigg\vert \ \MeijerGparams{0}{\frac{1}{2} -\!1 \; 0} \right) dx
\]

and returns the result

\[
\frac{\sqrt{\pi} \, J_{3/2}(2 \, \sqrt{\,y}) \, y}{y^{1/4}}
\]

\subsubsection{Examples}

\[
\int_{0}^{\infty} e^{-x} dx
\]


\begin{verbatim}
 int(e^(-x),x,0,infinity);

 1
\end{verbatim}

\[
\int_{0}^{\infty} x \sin(1/x) \, dx
\]

\begin{verbatim}
 int(x*sin(1/x),x,0,infinity);

           1
int(x*sin(---),x,0,infinity)
           x
\end{verbatim}

\[
\int_{0}^{\infty} x^2 \cos(x) \, e^{-2x} dx
\]

\begin{verbatim}
 int(x^2*cos(x)*e^(-2*x),x,0,infinity);

   4
 -----
  125
\end{verbatim}

\[
\int_{0}^{\infty} x e^{-1/2x} H(1-x) \,dx = \int_{0}^{1} x e^{-1/2x} dx
\]

\begin{verbatim}
 int(x*e^(-1/2x)*Heaviside(1-x),x,0,infinity);

  2*(2*sqrt(e) - 3)
 -------------------
       sqrt(e)
\end{verbatim}

\[
\int_{0}^{1} x \log(1+x) \,dx
\]

\begin{verbatim}
 int(x*log(1+x),x,0,1);

  1
 ---
  4
\end{verbatim}

\[
\int_{0}^{y} \cos(2x) \,dx
\]

\begin{verbatim}
 int(cos(2x),x,y,2y);

 sin(4*y) - sin(2*y)
---------------------
          2
\end{verbatim}
\subsection{Integral Transforms}

A useful application of the definite integration package is in the
calculation of various integral transforms. The transforms
available are as follows:

\begin{itemize}
\item Laplace transform
\item Hankel transform
\item Y-transform
\item K-transform
\item StruveH transform
\item Fourier sine transform
\item Fourier cosine transform
\end{itemize}

\subsubsection{Laplace transform}
\index{Laplace transform}
\index{Integral transform!Laplace transform}
\hypertarget{operator:LAPLACE_TRANSFORM}{}
\ttindextype{laplace\_transform}{operator}

The Laplace transform

\[
f(s) = \mathcal{L}\{\mathrm{F}(t)\} = \int_{0}^{\infty} e^{-st}F(t)\,dt
\]

can be calculated by using the \f{laplace\_transform} operator.

This requires as parameters
\begin{itemize}
\item the function to be integrated
\item the integration variable.
\end{itemize}

For example

\[
\mathcal{L}\{e^{-at}\}
\]

is entered as

\begin{verbatim}
        laplace_transform(e^(-a*x),x);
\end{verbatim}

and returns the result

\[
\frac{1}{s+a}
\]

\subsubsection{Hankel transform}
\index{Hankel transform}
\index{Integral transform!Hankel transform}
\hypertarget{operator:HANKEL_TRANSFORM}{}
\ttindextype{hankel\_transform}{operator}

The Hankel transform
\[
f(\omega) = \int_{0}^{\infty} F(t) \,J_{\nu}(2\sqrt{\omega t}) \,dt
\]

can be calculated by using the \f{hankel\_transform} operator, e.g.,

\begin{verbatim}
        hankel_transform(f(x),x);
\end{verbatim}

This is used in the same way as the \f{laplace\_transform} operator.

\subsubsection{Y-transform}
\index{Y-transform}
\index{Integral transform!Y-transform}
\hypertarget{operator:Y_TRANSFORM}{}
\ttindextype{Y\_transform}{operator}

The Y-transform

\[
f(\omega) = \int_{0}^{\infty} F(t) \,Y_{\nu}(2\sqrt{\omega t}) \,dt
\]

can be calculated by using the \f{Y\_transform} operator, e.g.,

\begin{verbatim}
        Y_transform(f(x),x);
\end{verbatim}

This is used in the same way as the \f{laplace\_transform} operator.

\subsubsection{K-transform}
\index{K-transform}
\index{Integral transform!K-transform}
\hypertarget{operator:K_TRANSFORM}{}
\ttindextype{K\_transform}{operator}

The K-transform

\[
f(\omega) = \int_{0}^{\infty} F(t) \,K_{\nu}(2\sqrt{\omega t}) \,dt
\]

can be calculated by using the \f{K\_transform} operator, e.g.,

\begin{verbatim}
        K_transform(f(x),x);
\end{verbatim}

This is used in the same way as the \f{laplace\_transform} operator.

\subsubsection{StruveH transform}
\index{StruveH transform}
\index{Integral transform!StruveH transform}
\hypertarget{operator:STRUVEH_TRANSFORM}{}
\ttindextype{struveh\_transform}{operator}

The StruveH transform

\[
f(\omega) = \int_{0}^{\infty} F(t) \,StruveH(\nu,2\sqrt{\omega t}) \,dt
\]

can be calculated by using the \f{struveh\_transform} operator, e.g.,

\begin{verbatim}
        struveh_transform(f(x),x);
\end{verbatim}

This is used in the same way as the \f{laplace\_transform} operator.

\subsubsection{Fourier sine transform}
\index{Fourier sine transform}
\index{Integral transform!Fourier sine transform}
\hypertarget{operator:FOURIER_SIN}{}
\ttindextype{fourier\_sin}{operator}

The Fourier sine transform

\[
f(s) = \int_{0}^{\infty} F(t) \sin(st) \,dt
\]

can be calculated by using the \f{fourier\_sin} operator, e.g.,
\begin{verbatim}
        fourier_sin(f(x),x);
\end{verbatim}

This is used in the same way as the \f{laplace\_transform} operator.

\subsubsection{Fourier cosine transform}
\index{Fourier cosine transform}
\index{Integral transform!Fourier cosine transform}
\hypertarget{operator:FOURIER_COS}{}
\ttindextype{fourier\_cos}{operator}

The Fourier cosine transform

\[
f(s) = \int_{0}^{\infty} F(t) \cos(st) \,dt
\]

can be calculated by using the \f{fourier\_cos} operator, e.g.,
\begin{verbatim}
        fourier_cos(f(x),x);
\end{verbatim}

This is used in the same way as the \f{laplace\_transform} operator.

\subsubsection{The print\_conditions function}
\hypertarget{operator:PRINT_CONDITIONS}{}
\ttindextype{PRINT\_CONDITIONS}{operator}

The required conditions for the validity of the transform integrals
can be viewed using the following command:

\begin{verbatim}
        print_conditions().
\end{verbatim}

For example after calculating the following laplace transform

\begin{verbatim}
        laplace_transform(x^k,x);
\end{verbatim}

using the \f{print\_conditions} operator would produce

\begin{verbatim}
repart(sum(ai) - sum(bj)) + 1/2 (q + 1 - p)>(q - p) repart(s)

 and ( - min(repart(bj))<repart(s))<1 - max(repart(ai))

 or mod(arg(eta))=pi*delta

 or ( - min(repart(bj))<repart(s))<1 - max(repart(ai))

 or mod(arg(eta))<pi*delta
\end{verbatim}

where
\[
\begin{array}{l}
delta = s+t-\frac{u-v}{2}\\
eta = 1-\alpha(v-u)-\mu-\rho\\
\mu = \sum_{j=1}^{q} b_{j} - \sum_{i=1}^{p} a_{i} + \frac{p-q}{2} + 1\\
\rho = \sum_{j=1}^{v} d{j} - \sum_{i=1}^{u} c_{i} + \frac{u-v}{2} + 1\\
s,t,u,v,p,q,\alpha \; \text{as in eq.~\ref{defint:eq1}}
\end{array}
\]

\subsection{Extending the Tables}

The relevant Meijer G representation for any function is found by a
pattern-matching process which is carried out on a list of Meijer
G-function definitions. This list, although extensive, can never hope
to be complete and therefore the user may wish to add more definitions.
Definitions can be added by adding the following lines:

\begin{verbatim}
  defint_choose(f(~x),~var => f1(n,x);

  symbolic putv(mellin!-transforms!*,n,'
                (() (m n p q) (ai) (bj) (C) (var)));

\end{verbatim} 
where $f(x)$ is the new function, $i = 1,\ldots, p$, $j=1,\ldots,q$, $C$ a constant,
%where i = 1..p, j=1..q, C = a constant,
var = variable, n = an indexing number, where all expression must be in internal prefix form.

For example when considering $\cos(x)$ we have
the \textit{Meijer G representation --}
\[
\cos x = \sqrt{\pi} \,G^{1 0}_{0 2}\left(\frac{x^{2}}{4} \ \Bigg\vert
\ \MeijerGparams{ . \; . }{ 0 \; \frac{1}{2}} \right) dx
\]
i.e. $m=1$, $n=0$, $p=0$, $q=2$, $a_{i}=\{\}$, $b_{j} = \{0,1/2)$, $C=\sqrt{\pi}$.
The corresponding \textit{internal definite integration package representation} ist
\begin{verbatim}
  defint_choose(cos(~x),~var)   => f1(3,x);
\end{verbatim}
where 3 is the indexing number corresponding to the 3
in the following formula
\begin{verbatim}
  symbolic putv(mellin!-transforms!*,3,'
                (() (1 0 0 2) () (nil (quotient 1 2))
                (sqrt pi) (quotient (expt x 2) 4)));
\end{verbatim} 

or the more interesting example of the Bessel function $J_{n}(x)$:

\textit{Meijer G representation --}
\[
  J_{n}(x)
G^{1 0}_{0 2} \left(\frac{x^{2}}{4} \ \Bigg\vert
\ \MeijerGparams{. \; .}{\frac{n}{2} \; {\frac{-n}{2}}} \right) dx
\]

\textit{Internal definite integration package representation --}

\begin{verbatim}
  defint_choose(besselj(~n,~x),~var) => f1(50,x,n);

  symbolic putv(mellin!-transforms!*,50,'
                ((n) (1 0 0 2) () ((quotient n 2)
                                   (minus quotient n 2)) 1
                                   (quotient (expt x 2) 4)));
\end{verbatim} 



\subsection{Acknowledgements}
\index{Adamchik, Victor}\index{People!Adamchik, Victor}
Kerry Gaskell would like to thank Victor Adamchik whose implementation of the
definite integration package for {\REDUCE} is vital to this
interface.

\chapter[CVIT:Dirac gamma matrix traces]%
        {CVIT: Fast calculation of Dirac gamma matrix traces}
\label{CVIT}
\typeout{[CVIT:Dirac gamma matrix traces]}

{\footnotesize
\begin{center}
V. Ilyin, A. Kryukov, A. Rodionov and A. Taranov \\
Institute for Nuclear Physics \\
Moscow State University  \\
Moscow, 119899 Russia
\end{center}
}

\ttindex{CVIT}

The package consists of 5 sections, and provides an alternative to the
\REDUCE\ high-energy physics system.  Instead of being based on
$\Gamma$-matrices as a basis for a Clifford algebra, it is based on
treating $\Gamma$-matrices as 3-j symbols, as described by
Cvitanovic.

The functions it provides are the same as those of the standard
package.  It does have four switches which control its behaviour.

\noindent{\tt CVIT}\ttindex{CVIT}

If it is on then use Kennedy-Cvitanovic algorithm else use standard
facilities.

\noindent{\tt CVITOP}\ttindex{CVITOP}

Switches on Fierz optimisation.  Default is off;

\noindent{\tt CVITBTR}\ttindex{CVITBTR}

Switches on the bubbles and triangles factorisation.  The default is
on.


\noindent{\tt CVITRACE}\ttindex{CVITRACE}

Controls internal tracing of the CVIT package.  Default is off.

\begin{verbatim}
index j1,j2,j3,;

vecdim n$

g(l,j1,j2,j2,j1);


 2
n


g(l,j1,j2)*g(l1,j3,j1,j2,j3);


 2
n

g(l,j1,j2)*g(l1,j3,j1,j3,j2);


n*( - n + 2)
\end{verbatim}

