\subsection{Purpose}

The procedures \texttt{CONLAW1}, \texttt{CONLAW2}, \texttt{CONLAW3},
\texttt{CONLAW4} try to find conservation laws for a given
single/system of differential equation(s) (ODEs or PDEs)
\begin{equation}
     u^{\alpha}_J = w^{\alpha}(x,u^{\beta},\ldots,u^{\beta}_K,\ldots) \label{conlaw-a1}
\end{equation}
\texttt{CONLAW1} tries to find the conserved current $P^i$ by solving
\begin{equation}
   \text{Div}\,P = 0 \quad \text{modulo} \quad (\ref{conlaw-a1})
\label{conlaw-a2}
\end{equation}
directly.  \texttt{CONLAW3} tries to find $P^i$ and the characteristic
functions (integrating factors) $Q^{\nu}$ by solving
\begin{equation}
\mbox{Div}\,P = \sum_{\nu} Q^{\nu}\cdot (u^{\nu}_J - w^{\nu}) \label{conlaw-a3}
\end{equation}
identically in all $u$-derivatives.  Applying the Euler operator
(variational derivative) for each $u^{\nu}$ on (\ref{conlaw-a3}) gives
a zero left hand side and therefore conditions involving only
$Q^{\nu}$.  \texttt{CONLAW4} tries to solve these conditions
identically in all $u$-derivatives and to compute $P^i$ afterwards.
\texttt{CONLAW2} does substitutions based on (\ref{conlaw-a1}) before
solving these conditions on $Q^{\nu}$ and therefore computes adjoined
symmetries.  These are completed, if possible, to conservation laws by
computing $P^i$ from the $Q^{\nu}$.

All four procedures have the same syntax.  They have two parameters,
both lists.  The first parameter specifies the equations
(\ref{conlaw-a1}), the second specifies the computation to be done.
One can either specify an ansatz for $P^i, Q^{\nu}$ or investigate a
general situation, only specifying the order of the characteristic
functions or the conserved current.

\subsection{Syntax}

The procedures \texttt{CONLAW\textit{i}}, $i=1,2,3,4$, are called
through
\begin{center}
  \texttt{CONLAW\textit{i}(\textit{problem}, \textit{runmode});}
\end{center}
where both parameters are lists.

The first parameter \textit{problem} specifies the DEs to be
investigated.  It has the form \texttt{\{\textit{equations},
  \textit{ulist}, \textit{xlist}\}} where \ldots
\begin{description}
\item{\textit{equations}} is a list of equations, each of the form
  \texttt{df(ui,\ldots)=\ldots} where the left-hand side
  \texttt{df(ui,\ldots)} is selected such that
  \begin{itemize}
  \item the right-hand side of an equation must not include the
    derivative on the left-hand side nor a derivative of it;
  \item the left-hand side of any equation should not occur in any
    other equation nor any derivative of the left-hand side.
  \end{itemize}
  If \texttt{CONLAW3} or \texttt{CONLAW4} are run where no
  substitutions are made then the left-hand sides of equations can be
  \texttt{df(ui,\ldots)**n=\ldots} where \texttt{n} is a number.  No
  distinction is made between equations and constraints.
\item{\textit{ulist}} is a list of function names, which can be chosen
  freely; the number of functions and equations need not be equal.
\item{\textit{xlist}} is a list of variable names, which can be chosen
  freely.
\end{description}

The second parameter \textit{runmode} specifies the calculation to be
done.  It has the form \texttt{\{\textit{minord}, \textit{maxord},
  \textit{expl}, \textit{flist}, \textit{inequ}\}} where \ldots
\begin{description}
\item{\textit{minord} and \textit{maxord}} are respectively the
  minimum and maximum of the highest order of derivatives of $u$
  \begin{itemize}
  \item in \texttt{p\_t} for \texttt{CONLAW1}, where \texttt{t} is the
    first variable in \textit{xlist};
  \item in \texttt{q\_j} for \texttt{CONLAW2}, \texttt{CONLAW3} and
    \texttt{CONLAW4}.
  \end{itemize}
\item{\textit{expl}} is \texttt{t} or \texttt{nil} to indicate whether
  or not the characteristic functions \texttt{q\_i} or conserved
  current may depend explicitly on the variables of \textit{xlist}.
\item{\textit{flist}} is a list of unknown functions in any ansatz for
  \texttt{p\_i}, \texttt{q\_j}, also all parameters and parametric
  functions in the equation that are to be calculated such that
  conservation laws exist; if there are no such unknown functions then
  \textit{flist} is the empty list \texttt{\{\}}.
\item{\textit{inequ}} is a list of expressions none of which may be
  identically zero for the conservation law to be found; if there is
  no such expression then \textit{inequ} is an empty list
  \texttt{\{\}}.
\end{description}

The procedures \texttt{CONLAW\textit{i}} return a list of conservation
laws \texttt{\{$C_1,C_2,\ldots$\}}; if no non-trivial conservation law
is found they return the empty list \texttt{\{\}}.  Each $C_i$
representing a conservation law has the form
\texttt{\{\{$P^1,P^2,\ldots$\},\{$Q^1,Q^2,\ldots$\}\}}.

An ansatz for a conservation law can be formulated by specifying one
or more of the components $P^i$ for \texttt{CONLAW1}, one or more of
the functions $Q^{\mu}$ for \texttt{CONLAW2} and \texttt{CONLAW4}, or
one or more of $P^i, Q^{\mu}$ for \texttt{CONLAW3}.  The $P^i$ are
input as \texttt{p\_i} where \texttt{i} stands for a variable name,
and the $Q^{\mu}$ are input as \texttt{q\_i} where \texttt{i} stands
for an index, which is the number of the equation in the input list
\textit{equations} with which \texttt{q\_i} is multiplied.

There is a restriction in the structure of all the expressions for
\texttt{p\_i} and \texttt{q\_j} that are specified: they must be
homogeneous linear in the unknown functions or constants which appear
in these expressions.  The reason for this restriction is not for
\package{CRACK} to be able to solve the resulting overdetermined
system but for \texttt{CONLAW\textit{i}} to be able afterwards to
extract the individual conservation laws from the general solution of
the determining conditions.

All such unknown functions and constants must be listed in
\textit{flist} (see above).  The dependencies of such functions must
be defined before calling \texttt{CONLAW\textit{i}}.  This is done
with the command \texttt{DEPEND}, e.g.
\begin{verbatim}
     DEPEND f,t,x,u$
\end{verbatim}
to specify $f$ as a function of $t,x,u$.  If one wants to have $f$ as
a function of derivatives of $u(t,x)$, say $f$ depending on $u_{txx}$,
then one \emph{cannot} write
\begin{verbatim}
     DEPEND f, df(u,t,x,2)$
\end{verbatim}
but instead must write
\begin{flushleft}\tt
~~~~~DEPEND f, u!\textasciigrave 1!\textasciigrave 2!\textasciigrave 2\$
\end{flushleft}
if \textit{xlist} has been specified as \texttt{\{t,x\}}, because
\texttt{t} is the first variable and \texttt{x} is the second variable
in \textit{xlist} and \texttt{u} is differentiated once
wrt.\ \texttt{t} and twice wrt.\ \texttt{x}; we therefore get
\texttt{u!\textasciigrave 1!\textasciigrave 2!\textasciigrave 2}.  The
character \texttt{!} is the escape character to allow special
characters like \textasciigrave{} to occur in an identifier name.

It is possible to add extra conditions like PDEs for $P^i, Q^{\mu}$ as
a list \texttt{cl\_condi} of expressions that shall vanish.

\subsubsection{Remarks}

\begin{itemize}

\item The input to each of \texttt{CONLAW1}, \texttt{CONLAW2},
  \texttt{CONLAW3} and \texttt{CONLAW4} is the same apart from:
  \begin{itemize}
  \item an ansatz for $Q^{\nu}$ is ignored in \texttt{CONLAW1};
  \item an ansatz for $P^i$ is ignored in \texttt{CONLAW2} and
    \texttt{CONLAW4};
  \item the meaning of \texttt{mindensord}, \texttt{maxdensord} is
    different in \texttt{CONLAW1} on the one hand and
    \texttt{CONLAW2}, \texttt{CONLAW3}, and \texttt{CONLAW4} on the
    other (see above).
  \end{itemize}

\item It matters how the differential equations are input, i.e.\ which
  derivatives are eliminated.  For example, the Korteweg-de Vries
  equation if input in the form $u_{xxx}=-uu_x-u_t$ instead of
  $u_t=-uu_x-u_{xxx}$ in \texttt{CONLAW1} and choosing
  \texttt{maxdensord}=1 then $P^i$ will be of at most first order,
  Div\,$P$ of second order and $u_{xxx}$ will not be substituted and
  no non-trival conservation laws can be found.  This does not mean
  that one will not find low order conservation laws at all with the
  substitution $u_{xxx}$; one only has to go to \texttt{maxdensord}=2
  with longer computations as a consequence compared to the input
  $u_t=-uu_x-u_{xxx}$ where \texttt{maxdensord}=0 is enough to find
  non-trivial conservation laws.

\item The drawback of using $u_t=\ldots$ compared with
  $u_{xxx}=\ldots$ is that when seeking all conservation laws up to
  some order then one has to investigate a higher-order ansatz,
  because with each substitution $u_t=-u_{xxx}+\ldots$ the order
  increases by 2.  For example, if all conservation laws of order up
  to two in $Q^{\nu}$ are to be determined then in order to include a
  $u_{tt}$-dependence the dependence of $Q^{\nu}$ on $u_x$ up to
  $u_{6x}$ has to be considered.

\item Although for any equivalence class of conserved currents with
  $P^i$ differing only by a curl there exist characteristic functions
  $Q^{\mu}$, this need not be true for any particular $P^i$.
  Therefore one cannot specify a known density $P^i$ for
  \texttt{CONLAW3} and hope to calculate the remaining $P^j$ and the
  corresponding $Q^{\mu}$ with \texttt{CONLAW3}.  What one can do is
  to use \texttt{CONLAW1} to calculate the remaining components $P^j$,
  and from this a trivial conserved density $R$ and characteristic
  functions $Q^{\nu}$ are computed such that
\[ \text{Div}\,(P-R) = \sum_{\nu} Q^{\nu}\cdot (u^{\nu}_J - w^{\nu}). \]

\item The $Q^{\mu}$ are uniquely determined only modulo $\Delta=0$.
  If one makes an ansatz for $Q^{\mu}$ then this freedom should be
  removed by having the $Q^{\mu}$ independent of the left-hand sides
  of the equations and their derivatives.  If the $Q^{\mu}$ were
  allowed to depend on anything, then (\ref{conlaw-a3}) could simply
  be solved for one $Q^{\nu}$ in terms of arbitrary $P^j$ and other
  arbitrary $Q^{\rho}$, giving $Q^{\nu}$ that are singular for
  solutions of the differential equations as the expression of the
  differential equation would appear in the denominator of $Q^{\nu}$.

\item Any ansatz for $P^i,Q^{\nu}$ should as well be independent of
  the left-hand sides of equations (\ref{conlaw-a1}) and their
  derivatives.

\item If in equation (\ref{conlaw-a3}) the right-hand side is of order
  $m$ then from the conserved current $P^i$ a trivial conserved
  current can be subtracted such that the remaining conserved current
  is of order at most $m$.  If the right hand side is linear in the
  highest derivatives of order $m$ then subtraction of a trivial
  conserved current can even achieve a conserved current of order
  $m-1$.  The relevance of this result is that if the right-hand side
  is known to be linear in the highest derivatives then for $P^i$ an
  ansatz of order only $m-1$ is necessary.  To take advantage of this
  relation if the right-hand side is known to be linear in the highest
  derivatives, a flag \texttt{quasilin\_rhs} can be set to \texttt{t}
  (see below).

\end{itemize}

\subsection{Flags and Parameters}

\begin{verbatim}
     lisp(print_ := nil/0/1/ ...)$
\end{verbatim}
\texttt{print\_:=nil} suppresses all \package{CRACK} output; for
\texttt{print\_:=}$n$ ($n$ an integer) \package{CRACK} prints only
equations with at most $n$ factors in their terms.

\begin{verbatim}
     crackhelp()$
\end{verbatim}
shows other flags controlling the solution of the overdetermined
PDE-system.

\begin{verbatim}
     off batch_mode$
\end{verbatim}
solves the system of conditions with \package{CRACK} interactively.

\begin{verbatim}
     lisp(quasilin_rhs := t)$
\end{verbatim}
reduces in the ansatz for $P^i$ the order to $m-1$ if the order of the
right-hand side is $m$.  This can be used to speed up computations if
the right-hand side is known to be linear in the highest derivatives
(see the note above).

\subsection{Requirements}

\begin{verbatim}
     load_package crack, conlaw0, conlaw1$
\end{verbatim}
where \texttt{conlaw1} can be replaced by \texttt{conlaw2},
\texttt{conlaw3} or \texttt{conlaw4} as appropriate.

\subsection{Examples}

Below a \package{CRACK} procedure \texttt{nodepnd} is used to clean up
after each run and delete all dependencies of each function in the
list of functions in the argument of \texttt{nodepnd}.  More details
concerning these examples are given when running the file
\texttt{conlaw.tst} (in the REDUCE \texttt{packages/crack} directory).

\begin{verbatim}
     lisp(print_:=nil);  % to suppress output from CRACK
\end{verbatim}

\begin{itemize}
\item A single PDE:
\begin{verbatim}
   depend u,x,t$
   conlaw1({{df(u,t)=-u*df(u,x)-df(u,x,3)},
            {u}, {t,x}},
           {0, 1, t, {}, {}})$
   nodepnd {u}$
\end{verbatim}

\item A system of equations:
\begin{verbatim}
   depend u,x,t$
   depend v,x,t$
   conlaw1({{df(u,t)=df(u,x,3)+6*u*df(u,x)+
                2*v*df(v,x),
             df(v,t)=2*df(u,x)*v+2*u*df(v,x)},
            {u,v}, {t,x}},
           {0, 1, t, {}, {}})$
   nodepnd {u,v}$
\end{verbatim}

\item A system of equations with ansatz:
\begin{verbatim}
   depend u,x,t$
   depend v,x,t$
   depend r,t,x,u,v,u!`2,v!`2$
   q_1:=r*df(u,x,2)$
   conlaw2({{df(u,t)=df(v,x),
             df(v,t)=df(u,x) }, {u,v}, {t,x}},
           {2, 2, t, {r}, {r}})$
   nodepnd {u,v,r}$
\end{verbatim}

\item For the determination of parameters, such that conservation laws
  exist:
\begin{verbatim}
   depend u,x,t;
   conlaw1({{df(u,t)=-df(u,x,5)-a*u**2*df(u,x)-
                b*df(u,x)*df(u,x,2)-c*u*df(u,x,3)},
            {u}, {t,x}},
           {0, 1, t, {a,b,c}, {}});
   nodepnd {u};
\end{verbatim}

\item For first integrals of an ODE-system including the determination
  of parameter values $s,b,r$ such that conservation laws exist:
\begin{verbatim}
   depend {x,y,z},t;
   depend a1,x,t;
   depend a2,y,t;
   depend a3,z,t;

   p_t:=a1+a2+a3;
   conlaw2({{df(x,t) = - s*x + s*y,
             df(y,t) = x*z + r*x - y,
             df(z,t) = x*y - b*z},
            {x,y,z},{t}
           },
           {0,0,t,{a1,a2,a3,s,r,b},{}});
   nodepnd {x,y,z,a1,a2,a3};
\end{verbatim}

\end{itemize}
