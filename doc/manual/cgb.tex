% ----------------------------------------------------------------------
% $Id: cgb.tex 84 2009-02-07 07:53:22Z thomas-sturm $
% ----------------------------------------------------------------------
% Copyright (c) 1999-2009 Andreas Dolzmann and Thomas Sturm
% ----------------------------------------------------------------------
% Redistribution and use in source and binary forms, with or without
% modification, are permitted provided that the following conditions
% are met:
%
%    * Redistributions of source code must retain the relevant
%      copyright notice, this list of conditions and the following
%      disclaimer.
%    * Redistributions in binary form must reproduce the above
%      copyright notice, this list of conditions and the following
%      disclaimer in the documentation and/or other materials provided
%      with the distribution.
%
% THIS SOFTWARE IS PROVIDED BY THE COPYRIGHT HOLDERS AND CONTRIBUTORS
% "AS IS" AND ANY EXPRESS OR IMPLIED WARRANTIES, INCLUDING, BUT NOT
% LIMITED TO, THE IMPLIED WARRANTIES OF MERCHANTABILITY AND FITNESS FOR
% A PARTICULAR PURPOSE ARE DISCLAIMED. IN NO EVENT SHALL THE COPYRIGHT
% OWNERS OR CONTRIBUTORS BE LIABLE FOR ANY DIRECT, INDIRECT, INCIDENTAL,
% SPECIAL, EXEMPLARY, OR CONSEQUENTIAL DAMAGES (INCLUDING, BUT NOT
% LIMITED TO, PROCUREMENT OF SUBSTITUTE GOODS OR SERVICES; LOSS OF USE,
% DATA, OR PROFITS; OR BUSINESS INTERRUPTION) HOWEVER CAUSED AND ON ANY
% THEORY OF LIABILITY, WHETHER IN CONTRACT, STRICT LIABILITY, OR TORT
% (INCLUDING NEGLIGENCE OR OTHERWISE) ARISING IN ANY WAY OUT OF THE USE
% OF THIS SOFTWARE, EVEN IF ADVISED OF THE POSSIBILITY OF SUCH DAMAGE.
% 

\index{Dolzmann, Andreas}\index{Sturm, Thomas}\index{Neun, Winfried}
\index{People!Dolzmann, Andreas}\index{People!Sturm, Thomas}\index{People!Neun, Winfried}

\subsection{Introduction}
Consider the ideal basis $F=\{ax,x+y\}$. Treating $a$ as a parameter,
the calling sequence
\begin{verbatim}
torder({x,y},lex)$
groebner{a*x,x+y};

  {x,y}
\end{verbatim}
yields $\{x,y\}$ as reduced Gr\"obner basis. This is, however, not
correct under the specialization $a=0$. The reduced Gr\"obner basis
would then be $\{x+y\}$. Taking these results together, we obtain
$C=\{x+y,ax,ay\}$, which is correct wrt.~\emph{all} specializations for
$a$ including zero specializations. We call this set $C$ a
\emph{comprehensive Gr\"obner basis} (\textsc{cgb}).

The notion of a \textsc{cgb} and a corresponding algorithm has been
introduced bei Weispfenning \cite{Weispfenning:92}. This algorithm
works by performing case distinctions wrt.~parametric coefficient
polynomials in order to find out what the head monomials are under all
possible specializations. It does thus not only determine a \textsc{cgb},
but even classifies the contained polynomials wrt.~the specializations
they are relevant for. If we keep the Gr\"obner bases for all cases
separate and associate information on the respective specializations
with them, we obtain a \emph{Gr\"obner system}. For our example, the
Gr\"obner system is the following;
\[
\left[
\begin{array}{c|c}
a\neq0 & \{x+y,ax,ay\}\\
a=0 & \{x+y\}
\end{array}
\right].
\]
A \textsc{cgb} is obtained as the union of the single Gr\"obner bases in
a Gr\"obner system. It has also been shown that, on the other hand, a
Gr\"obner system can easily be reconstructed from a given \textsc{cgb}
\cite{Weispfenning:92}.

The CGB package provides functions for computing both \textsc{cgb}'s and
Gr\"obner systems, and for turning Gr\"obner systems into \textsc{cgb}'s.
%
\subsection{Using the REDLOG Package}
For managing the conditions occurring with the \textsc{cgb} computations,
the CGB package uses the package REDLOG implementing first-order
formulas, \cite{DolzmannSturm:97a,Dolzmann:99}, which is also
part of the \textsc{reduce} distribution.
%
\subsection{Term Ordering Mode}
\ttindextype{torder}{operator}\ttindex{lex}\ttindex{revgradlex}
The CGB package uses the settings made with the function \texttt{torder}
of the GROEBNER package. This includes in particular the choice of the
main variables. All variables not mentioned in the variable list
argument of \texttt{torder} are parameters. The only term ordering modes
recognized by \textsc{cgb} are \texttt{lex} and \texttt{revgradlex}.
%
\subsection{CGB: Comprehensive Gr\"ob\-ner Basis}
\ttindextype{CGB}{operator}
The function \texttt{cgb} expects a list $F$ of expressions. It returns a
\textsc{cgb} of $F$ wrt.~the current \texttt{torder} setting.
%
\subsubsection*{Example}
\begin{verbatim}
torder({x,y},lex)$
cgb{a*x+y,x+b*y};

{x + b*y,a*x + y,(a*b - 1)*y}

ws;

{b*y + x,

 a*x + y,

 y*(a*b - 1)}
\end{verbatim}
Note that the basis returned by the \texttt{cgb} call has not undergone
the standard evaluation process: The returned polynomials are ordered
wrt.~the chosen term order. Reevaluation changes this as can be seen
with the output of \texttt{ws}.
%
\subsection{GSYS: Gr\"obner System}
\ttindextype{GSYS}{operator} The function \texttt{gsys} follows the
same calling conventions as \texttt{cgb}. It returns the complete
Gr\"obner system represented as a nested list
\begin{center}
\(
\bigl\{\bigl\{c_1,\{g_{11},\ldots,g_{1n_1}\}\bigr\},\dots,
\bigl\{c_m,\{g_{m1},\dots,g_{1n_m}\}\bigr\}\bigr\}.
\)
\end{center}
The $c_i$ are conditions in the parameters represented as
quantifier-free REDLOG formulas. Each choice of parameters will obey
at least one of the $c_i$. Whenever a choice of parameters obeys
some $c_i$, the corresponding $\{g_{i1},\ldots,g_{in_i}\}$
is a Gr\"obner basis for this choice.
%
\subsubsection*{Example}
\begin{verbatim}
torder({x,y},lex)$
gsys {a*x+y,x+b*y};

{{a*b - 1 <> 0 and a <> 0,

  {a*x + y,x + b*y,(a*b - 1)*y}},

 {a <> 0 and a*b - 1 = 0,

  {a*x + y,x + b*y}},

 {a = 0,{a*x + y,x + b*y}}}
\end{verbatim}
As with the function \texttt{cgb}, the contained polynomials remain
unevaluated.

Computing a Gr\"obner system is not harder than computing a \textsc{cgb}.
In fact, \texttt{cgb} also computes a Gr\"obner system and then turns it
into a \textsc{cgb}.

\subsubsection{Switch CGBGEN: Only the Generic Case}
\ttindextype{CGBGEN}{switch}
If the switch \texttt{cgbgen} is turned on, both \texttt{gsys} and {\tt
cgb} will assume all parametric coefficients to be non-zero ignoring
the other cases. For \texttt{cgb} this means that the result equals---up
to auto-reduction---that of \texttt{groebner}. A call to \texttt{gsys} will
return this result as a single case including the assumptions made
during the computation:
%
\subsubsection*{Example}
\begin{verbatim}
torder({x,y},lex)$
on cgbgen;
gsys{a*x+y,x+b*y};

{{a*b - 1 <> 0 and a <> 0,

  {a*x + y,x + b*y,(a*b - 1)*y}}}

off cgbgen;
\end{verbatim}
%
\subsection{GSYS2CGB: Gr\"obner System to CGB}
\ttindextype{GSYS2CGB} {operator}
The call \texttt{gsys2cgb} turns a given Gr\"obner system into a \textsc{cgb}
by constructing the union of the Gr\"obner bases of the single
cases.
%
\subsubsection*{Example}
\begin{verbatim}
torder({x,y},lex)$
gsys{a*x+y,x+b*y}$
gsys2cgb ws;

{x + b*y,a*x + y,(a*b - 1)*y}
\end{verbatim}
%
\subsection{Switch CGBREAL: Computing over the Real Numbers}
\label{cgbreal}
\ttindextype{CGBREAL}{switch}
All computations considered so far have taken place over the complex
numbers, more precisely, over algebraically closed fields. Over the
real numbers, certain branches of the \textsc{cgb} computation can become
inconsitent though they are not inconsistent over the complex numbers.
Consider, e.g., a condition $a^2+1=0$.

When turning on the switch \texttt{cgbreal}, all simplifications of conditions
are performed over the real numbers. The methods used for this are
described in \cite{Dolzmann:97b}.
%
\subsubsection*{Example}
\begin{verbatim}
torder({x,y},lex)$
off cgbreal;
gsys {a*x+y,x-a*y};

   2
{{a  + 1 <> 0 and a <> 0,

                     2
  {a*x + y,x - a*y,(a  + 1)*y}},

              2
 {a <> 0 and a  + 1 = 0,{a*x + y,x - a*y}},

 {a = 0,{a*x + y,x - a*y}}}

on cgbreal;
gsys({a*x+y,x-a*y});

{{a <> 0,

                     2
  {a*x + y,x - a*y,(a  + 1)*y}},

 {a = 0,{a*x + y,x - a*y}}}
\end{verbatim}

\subsection{Switches}
\ttindextype{CGBGS}{switch}
\ttindextype{CGBSTAT}{switch}
\ttindextype{CGBFULLRED}{switch}
\begin{description}
\item[cgbreal] Compute over the real numbers. See
Section~\ref{cgbreal} for details.

\item[cgbgs] Gr\"obner simplification of the condition. The switch
\texttt{cgbgs} can be turned on for applying advanced algebraic
simplification techniques to the conditions. This will, in general,
slow down the computation, but lead to a simpler Gr\"obner system.

\item[cgbstat] Statistics of the CGB run. The switch \texttt{cgbstat}
toggles the creation and output of statistical information on the CGB
run. The statistical information is printed at the end of the run.

\item[cgbfullred] Full reduction. By default, the CGB functions
perform full reductions in contrast to pure top reductions. By turning
off the switch \texttt{cgbfullred}, reduction can be restricted to top
reductions.
\end{description}
