
\subsection{Basic Usage}


Similarly to the Reduce cross referencer, it is used via switches as follows:
\begin{verbatim}
load_package gcref;
on gcref;
in "<filename>.red";
off gcref;
\end{verbatim}
At \texttt{off gcref;} the graph is printed to the screen in TGF format. 
To redirect this output to a file, use the following:
\begin{verbatim}
load_package gcref;
on gcref;
in "<filename>.red";
out "<filename>.tgf";
off gcref;
shut "<filename>.tgf";
\end{verbatim}

\subsection{Shell Script "gcref"}

There is a shell script "gcref" in this directory automizing this like
\begin{verbatim}
./gcref filename.red
\end{verbatim}
"gcref" is configured to use CSL Reduce. To use PSL Reduce instead, set
\$REDUCE in the environment. To use PSL by default, define
\begin{verbatim}
REDUCE=redpsl
\end{verbatim}
in line 3 of "gcref".


\subsection{Rendering with yED}

The obtained TGF file can be viewed with a graph editor. I recommend using the
free software yED, which is written in Java and available for many platforms.
\begin{verbatim}
http://www.yworks.com/en/products_yed_about.html
\end{verbatim}
Note that TGF is not suitable for storing rendering information. After opening
the TGF file with yED, the graph has to be rendered explicitly as follows:

* From menu "Layout" choose "Hierarchical Layout".

To resize the nodes to the procedure names

* from menu "Tools" choose "Fit Node to Label".

Feel free to experiment with yED and use other layout and layout options, which
might be suitable for your particular software.

For saving your particular layout at the end, use the GRAPHML format instead of
TGF.
