
\index{Hartley, David}\index{People!Hartley, David}

\subsection{Description}

The method of Gr{\"o}bner bases in commutative polynomial rings introduced by
Buchberger (e.g.~\cite{Buchberger:85}) is a well-known and very important tool
in polynomial ideal theory, for example in solving the ideal membership
problem. XIDEAL extends the method to exterior algebras using
algorithms from \cite{HartleyTuckey:93} and \cite{Apel:92}.

There are two main departures from the commutative polynomial case. First,
owing to the non-commutative product in exterior algebras, ideals are no
longer automatically two-sided, and it is necessary to distinguish between
left and right ideals. Secondly, because there are zero divisors, confluent
reduction relations are no longer sufficient to solve the ideal membership
problem: a unique normal form for every polynomial does not guarantee that
all elements in the ideal reduce to zero. This leads to two possible
definitions of Gr{\"o}bner bases as pointed out by Stokes \cite{Stokes:90}.

\package{XIDEAL} constructs Gr{\"o}bner bases for solving the left ideal membership
problem: Gr{\"o}bner left ideal bases or GLIBs. For graded ideals, where each
form is homogeneous in degree, the distinction between left and right
ideals vanishes. Furthermore, if the generating forms are all homogeneous,
then the Gr{\"o}bner bases for the non-graded and graded ideals are
identical. In this case, \package{XIDEAL} is able to save time by truncating the
Gr{\"o}bner basis at some maximum degree if desired.

\package{XIDEAL} uses the exterior calculus package \package{EXCALC} of E.~Schr{\"u}fer
\cite{Schruefer:85} to provide the exterior algebra definitions. \package{EXCALC} is loaded
automatically with \package{XIDEAL}.
% The basis 1-forms for the exterior algebra
% are automatically extracted from the input. Consequently, each expression
% must be written in terms of these 1-forms -- $p$-form kernels with $p>1$
% are not allowed. Similarly all distinct 1-forms in the input are assumed to
% be linearly independent -- if a dimension has been fixed (using the EXCALC
% \f{SPACEDIM} or \f{COFRAME} statements), then input containing more than this
% number of distinct 1-forms will generate an error. The exterior algebra can
% be based on either an abstract vector space or the cotangent space at some
% fixed point on a manifold. Any functions or 0-forms are treated as constant
% non-vanishing coefficients.
The exterior variables may be specified explicitly, or extracted
automatically from the input polynomials.  All distinct exterior variables
in the input are assumed to be linearly independent -- if a dimension has
been fixed (using the \package{EXCALC} \f{spacedim}\ttindextype{spacedim}{command}
or \f{coframe}\ttindextype{coframe}{command} statements), then
input containing distinct exterior variables with degrees totaling more
than this number will generate an error.

% The term ordering used is the graded lexicographical ordering based on the
% prevailing EXCALC kernel ordering for the basis 1-forms. This puts
% highest degree first and then sorts terms of the same degree
% lexicographically. The EXCALC kernel ordering can be changed with the
% \REDUCE{} \f{KORDER} or EXCALC \f{FORDER} statements.



\subsection{Declarations}


\subsubsection*{xorder}

\hypertarget{command:XORDER}{}
\f{xorder}\ttindextype[XIDEAL]{xorder}{declaration}
 sets the term ordering for all other calculations. The syntax is
\begin{verbatim}
     xorder k
\end{verbatim}
where \f{k} is one of \f{lex}, \f{gradlex} or \f{deglex}. The
lexicographical ordering \f{lex} is based on the prevailing EXCALC kernel
ordering for the exterior variables. The \package{EXCALC} kernel ordering can be
changed with the \REDUCE{} \f{korder} or \package{EXCALC} \f{forder} declarations. The
graded lexicographical ordering \f{gradlex} puts terms with more factors
first (irrespective of their exterior degrees) and sorts terms of the same
grading lexicographically. The degree lexicographical ordering \f{deglex}
takes account of the exterior degree of the variables, putting highest
degree first and then sorting terms of the same degree lexicographically.
The default ordering is \f{deglex}.


\subsubsection*{xvars}

\hypertarget{command:XVARS}{}
It is possible to consider scalar and 0-form variables in exterior
polynomials in two ways: as variables or as coefficient parameters. This
difference is crucial for Gr{\"o}bner basis calculations. By default, all
scalar variables which have been declared as 0-forms are treated as
exterior variables, along with any \package{EXCALC} kernels of degree 0. This
division can be changed with the \f{xvars}\ttindextype[XIDEAL]{xvars}{declaration}
declaration. The syntax is
\begin{verbatim}
     xvars U,V,W,...
\end{verbatim}
where the arguments are either kernels or lists of kernels. All variables
specified in the \f{xvars} declaration are treated as exterior variables in
subsequent \package{XIDEAL} calculations with exterior polynomials, and any other
scalars are treated as parameters. This is true whether or not the
variables have been declared as 0-forms. The declaration
\begin{verbatim}
     xvars {}
\end{verbatim}
causes all degree 0 variables to be treated as parameters, and
\begin{verbatim}
     xvars nil
\end{verbatim}
restores the default. Of course, $p$-form kernels with $p\not=0$ are always
considered as exterior variables. The order of the variables in an
\f{xvars} declaration has no effect on the \REDUCE{} kernel ordering or
\package{XIDEAL} term ordering.



\subsection{Operators}


\subsubsection*{xideal}

\hypertarget{operator:XIDEAL}{}
\f{xideal}\ttindextype[XIDEAL]{xideal}{operator} calculates a Gr{\"o}bner left ideal basis in
an exterior algebra. The syntax is
\begin{syntax}
     \f{xideal}(\meta{S:list of forms}[,\meta{V:list of kernels}][,\meta{R:integer}])
           :list of forms.
\end{syntax}
\f{xideal} calculates a Gr{\"o}bner basis for the left ideal generated by
\meta{S} using the current term ordering. The resulting list can be used for
subsequent reductions with \f{xmod} as long as the term ordering is not
changed. Which 0-form variables are to be regarded as exterior variables
can be specified in an optional argument \meta{V} (just like an \f{xvars}
declaration). The order of variables in \meta{V} has no effect on the term
ordering. If the set of generators \meta{S} is graded, an optional parameter
\meta{R} can be given, and \f{xideal} produces a truncated basis suitable for
reducing exterior forms of degree less than or equal to \meta{R} in the left
ideal. This can save time and space with large problems, but the result
cannot be used for exterior forms of degree greater than \meta{R}. The forms
returned by \f{xideal} are sorted in increasing order. See also the
switches \sw{trxideal} and \sw{xfullreduction}.


\subsubsection*{xmodideal}

\hypertarget{operator:XMODIDEAL}{}
\f{xmodideal}\ttindextype[XIDEAL]{xmodideal}{operator} reduces exterior forms to their (unique) normal forms modulo
a left ideal. The syntax is
\begin{syntaxtable}
     \f{xmodideal}(\meta{F:form}, \meta{S:list of forms}):form \\
     \intertext{or}\\
     \f{xmodideal}(\meta{F:list of forms}, \meta{S:list of forms})
              :list of forms.
\end{syntaxtable}
An alternative infix syntax is also available:
\begin{syntax}
     \meta{F} xmodideal \meta{S} .
\end{syntax}
\f{xmodideal}(\meta{F},\meta{S}) first calculates a Gr{\"o}bner basis for the left ideal
generated by \meta{S}, and then reduces \meta{F}. \meta{F} may be either a single
exterior form, or a list of forms, and \meta{S} is a list of forms. If \meta{F}
is a list of forms, each element is reduced, and any which vanish are
deleted from the result. 
% If this operator is used more than once, and \meta{S} does not change
% between calls, then the Gr{\"o}bner basis is not recalculated. 
If the set of generators \meta{S} is graded, then a truncated Gr{\"o}bner basis
is calculated using the degree of \meta{F} (or the maximal degree in
\meta{F}). See also \sw{trxmod}.


\subsubsection*{xmod}

\hypertarget{operator:XMOD}{}
\f{xmod}\ttindextype[XIDEAL]{xmod}{operator}
 reduces exterior forms to their (not necessarily unique) normal
forms modulo a set of exterior polynomials. The syntax is
\begin{syntaxtable}
     \f{xmod}(\meta{F:form}, \meta{S:list of forms}):form\\
     \intertext{or}\\
     \f{xmod}(\meta{F:list of forms}, \meta{S:list of forms}):list of forms.
\end{syntaxtable}
An alternative infix syntax is also available:
\begin{syntax}
      \meta{F} \f{xmod} \meta{S} .
\end{syntax}
\f{xmod}(\meta{F},\meta{S}) reduces \meta{F} with respect to the set of exterior polynomials
\meta{S}, which is not necessarily a Gr{\"o}bner basis. \meta{F} may be either a
single exterior form, or a list of forms, and \meta{S} is a list of
forms. This operator can be used in conjunction with \f{xideal} to produce
the same effect as \f{xmodideal}: for a single homogeneous form \meta{F} and a
set of exterior forms \meta{S}, \meta{F} \f{xmodideal} \meta{S} is equivalent to
\meta{F} \f{xmod} \f{xideal}(\meta{S},\f{exdegree} \meta{F}). See also \sw{trxmod}.


\subsubsection*{xauto}

\hypertarget{operator:XAUTO}{}
\f{xauto}\ttindextype[XIDEAL]{xauto}{operator} autoreduces a set of exterior forms. The syntax is
\begin{syntax}
     \f{xauto}(\meta{S:list of forms}):list of forms.
\end{syntax}
\f{xauto} \meta{S} returns a set of exterior polynomials which generate the same
left ideal, but which are in normal form with respect to each other. For
linear expressions, this is equivalent to finding the reduced row echelon
form of the coefficient matrix.


\subsubsection*{excoeffs}

\hypertarget{operator:EXCOEFFS}{}
The operator \f{excoeffs}\ttindextype[XIDEAL]{excoeffs}{operator}, with syntax
\begin{syntax}
     \f{excoeffs}(\meta{F:form}):list of expressions
\end{syntax}
returns the coefficients from an exterior form as a list. The coefficients
are always scalars, but which degree 0 variables count as coefficient
parameters is controlled by the command \f{xvars}.


\subsubsection*{exvars}

\hypertarget{operator:EXVARS}{}
The operator \f{exvars}\ttindextype[XIDEAL]{xvars}{operator}, with syntax
\begin{syntax}
     \f{exvars}(\meta{F:form}):list of kernels
\end{syntax}
returns the exterior powers from an exterior form as a list. All non-scalar
variables are returned, but which degree 0 variables count as coefficient
parameters is controlled by the command \f{xvars}.


\subsection{Switches} 


\subsubsection*{xfullreduce}

\hypertarget{switch:XFULLREDUCE}{}
\texttt{on} \sw{xfullreduce}\ttindexswitch[XIDEAL]{xfullreduce} allows \f{xideal} and \f{xmodideal} to calculate
reduced, monic Gr{\"o}bner bases, which speeds up subsequent reductions, and
guarantees a unique form for the Gr{\"o}bner basis. \texttt{off} \sw{xfullreduce} turns
of this feature, which may speed up calculation of some Gr{\"o}bner
basis. \sw{xfullreduce} is \f{on} by default.


\subsubsection*{trxideal}

\hypertarget{switch:TRXIDEAL}{}
\f{on} \sw{trxideal}\ttindexswitch[XIDEAL]{trxideal} produces a trace of the calculations done by \f{xideal} and
\f{xmodideal}, showing the basis polynomials and the results of the
critical element calculations. This can generate profuse amounts of output.
\sw{trxideal} is \f{off} by default.


\subsubsection*{trxmod}

\hypertarget{switch:TRXMOD}{}
\f{on} \sw{trxmod}\ttindexswitch[XIDEAL]{trxmod} produces a trace of reductions to normal form during
calculations by \package{XIDEAL} operators. \sw{trxmod} is \f{off} by default.


% \subsubsection*{xstats}

% \f{on xstats} produces counting and timing information. As \f{XIDEAL} is
% running, a hash mark (\texttt{\#}) is printed for each form taken from the
% input list, followed by a sequences of carets (\texttt{\textasciicricum}) and dollar signs
% (\texttt{\$}). Each caret represents a new basis element obtained by a simple
% wedge product, and each dollar sign represents a new basis element obtained
% from an S-polynomial. At the end, a table is printed summarising the
% calculation. \f{xstats} is \f{off} by default.

\subsection{Examples}

Suppose \package{XIDEAL} has been loaded, the switches are at their default settings,
and the following exterior variables have been declared:

\begin{verbatim}
     pform x=0,y=0,z=0,t=0,f(i)=1,h=0,hx=0,ht=0;
\end{verbatim}

In a commutative polynomial ring, a single polynomial is its own Gr{\"o}bner
basis. This is no longer true for exterior algebras because of the presence
of zero divisors, and can lead to some surprising reductions:
\begin{verbatim}
     xideal {d x^d y - d z^d t};

          {d t^d z + d x^d y,

           d x^d y^d z,

           d t^d x^d y}

     f(3)^f(4)^f(5)^f(6) 
       xmodideal {f(1)^f(2) + f(3)^f(4) + f(5)^f(6)};

          0
\end{verbatim}

The heat equation, $h_{xx}=h_t$ can be represented by the following
exterior differential system. 
\begin{verbatim}
     S := {d h - ht*d t - hx*d x,
           d ht^d t + d hx^d x,
           d hx^d t - ht*d x^d t};
\end{verbatim}
\f{xmodideal} can be used to check that the exterior differential system is
closed under exterior differentiation.
\begin{verbatim}
     d S xmodideal S;

          {}
\end{verbatim}

\f{xvars}, or a second argument to \f{xideal} can be used to change the
division between exterior variables of degree 0 and parameters.
\begin{verbatim}
        xideal {a*d x+y*d y};

                  d x*a + d y*y
                {---------------}
                        a

        xvars {a};
        xideal {a*d x+y*d y};
 
                {d x*a + d y*y,d x^d y}
 
        xideal({a*d x+y*d y},{a,y});
 
                {d x*a + d y*y,
 
                 d x^d y*y}
 
        xvars {};       % all 0-forms are coefficients
        excoeffs(d u - (a*p - q)*d y);
        
                {1, - a*p + q}
        
        exvars(d u - (a*p - q)*d y);

                {d u,d y}

        xvars {p,q};    % p,q are no longer coefficients
        excoeffs(d u - (a*p - q)*d y);
        
                { - a,1,1}

        exvars(d u - (a*p - q)*d y);

                {d y*p,d y*q,d u}

        xvars nil;
\end{verbatim}


Non-graded left and right ideals are no longer the same:
\begin{verbatim}
     d t^(d z+d x^d y) xmodideal {d z+d x^d y};

          0

     (d z+d x^d y)^d t xmodideal {d z+d x^d y};

           - 2*d t^d z
\end{verbatim}
Any form containing a 0-form term generates the whole ideal:
\begin{verbatim}
     xideal {1 + f(1) + f(1)^f(2) + f(2)^f(3)^f(4)};

          {1}
\end{verbatim}

