
\index{Melenk, Herbert}\index{People!Melenk, Herbert}

\subsection{Introduction}

This package implements the basic arithmetic for polynomial ideals
by exploiting the Gr\"obner bases package of \REDUCE.
In order to save computing time all intermediate Gr\"obner bases
are stored internally such that time consuming repetitions
are inhibited. A uniform setting facilitates the access.

\subsection{Initialization}

\ttindextype[IDEALS]{I\_setting}{operator}
\hypertarget{operator:I_SETTING}{}
Prior to any computation the set of variables has to be declared
by calling the operator \f{I\_setting} . E.g. in order to initiate
computations in the polynomial ring $Q[x,y,z]$ call
\begin{verbatim}
    I_setting(x,y,z);
\end{verbatim}
A subsequent call to \f{I\_setting} allows one to select another set
of variables; at the same time the internal data structures
are cleared in order to free memory resources.

\subsection{Bases}

\ttindextype[IDEALS]{I}{reserved symbol}
\hypertarget{reserved:IDEALS_I}{}
\ttindextype[IDEALS]{ideal2list}{operator}
\hypertarget{operator:IDEAL2LIST}{}
An ideal is represented by a basis (set of polynomials) tagged
with the symbol \texttt{I}, e.g.,
\begin{verbatim}
   u := I(x*z-y**2, x**3-y*z);
\end{verbatim}
Alternatively a list of polynomials can be used as input basis; however,
all arithmetic results will be presented in the above form. The
operator \f{ideal2list} allows one to convert an ideal basis into a
conventional \REDUCE list.

\subsubsection{Operators}

Because of syntactical restrictions in \REDUCE, special operators
have to be used for ideal arithmetic: 
\ttindextype[IDEALS]{.+ (ideal sum)}{infix operator}
\ttindextype[IDEALS]{.* (ideal product)}{infix operator}
\ttindextype[IDEALS]{.: (ideal quotient)}{infix operator}
\ttindextype[IDEALS]{./ (ideal quotient)}{infix operator}
\ttindextype[IDEALS]{.= (ideal equality)}{infix operator}
\ttindextype[IDEALS]{subset}{(ideal inclusion test) infix operator}
\ttindextype[IDEALS]{intersection}{(ideal intersection) operator}
\ttindextype[IDEALS]{member}{(ideal membership test) infix operator}
\ttindextype[IDEALS]{gb}{operator}
\hypertarget{operator:IDEALSUM}{}
\hypertarget{operator:IDEALPRODUCT}{}
\hypertarget{operator:IDEALQUOTIENT}{}
\hypertarget{operator:IDEALQUOTIENTALT}{}
\hypertarget{operator:IDEALEQUALITY}{}
\hypertarget{operator:IDEALSUBSET}{}
\hypertarget{operator:IDEALINTERSECTION}{}
\hypertarget{operator:IDEALMEMBER}{}
\hypertarget{operator:IDEALGB}{}
\begin{verbatim}
         .+            ideal sum (infix)
         .*            ideal product (infix)
         .:            ideal quotient (infix)
         ./            ideal quotient (infix)
         .=            ideal equality test (infix)
         subset        ideal inclusion test (infix)
         intersection  ideal intersection (prefix,binary)
         member        test for membership in an ideal
                         (infix: polynomial and ideal)
         gb            Groebner basis of an ideal (prefix, unary)
         ideal2list    convert ideal basis to polynomial list 
                         (prefix,unary)
\end{verbatim}

Example:

\begin{verbatim}
    I(x+y,x^2) .* I(x-z);

      2                      2    2
   i(x  + x*y - x*z - y*z,x*y  - y *z)
\end{verbatim}

The test operators return the values 1 (=true) or 0 (=false)
such that they can be used in \REDUCE \texttt{if}-\texttt{then}-\texttt{else} statements
directly.

The results of \texttt{.+}, \texttt{.*}, \texttt{.:}/\texttt{./}, and \f{intersection} are ideals
represented by their Gr\"obner basis in the current setting and
term order. The term order can be modified using the operator
\f{torder}\ttindextype{torder}{opreator} from the Gr\"obner package. Note that ideal equality 
cannot be tested with the \REDUCE equal sign:

\begin{verbatim}

   I(x,y)  = I(y,x)       is false
   I(x,y) .= I(y,x)       is true

\end{verbatim}

\subsection{Algorithms}

The operators \f{groebner}, \f{preduce}, and \f{idealquotient} of the
\ttindextype{groebner}{operator}%
\ttindextype{preduce}{operator}%
\ttindextype{idealquotient}{operator}%
\REDUCE Gr\"obner package support the basic algorithms:

$\mathtt{gb}(Iu_1,u_2...) \rightarrow \mathtt{groebner}(\{u_1,u_2...\},\{x,...\})$

$p \in I_1 \rightarrow p=0 \ mod \ I_1$

$I_1 : I(p) \rightarrow (I_1 \bigcap I(p)) / p \ elementwise$

\noindent
On top of these the \textsc{Ideals} package implements the following 
operations:


$I(u_1,u_2...)+I(v_1,v_2...) \rightarrow GB(I(u_1,u_2...,v_1,v_2...))$


$I(u_1,u_2...)*I(v_1,v_2...)\rightarrow 
 GB(I(u_1*v_1,u_1*v2,...,u_2*v_1,u_2*v_2...))$


$I_1 \bigcap I_2 \rightarrow
  Q[x,...] \bigcap GB_{lex}(t*I_1 + (1-t)*I_2,\{t,x,..\}) $


$I_1 : I(p_1,p_2,...) \rightarrow I_1 : I(p_1) \bigcap I_1 : I(p_2)
\bigcap ...$

$I_1 = I_2 \rightarrow GB(I_1)=GB(I_2)$

$I_1 \subseteq I_2
   \rightarrow \ u_i \in I_2 \ \forall \ u_i \in I_1=I(u_1,u_2...)$

\subsection{Examples}

Please consult the file \texttt{ideals.tst}.
