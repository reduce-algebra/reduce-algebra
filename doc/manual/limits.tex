\section{LIMIT operator}
\index{Computing limits}
\index{Kameny, Stanley L.}\index{People!Kameny, Stanley L.}
\index{Cohen, Ian}\index{People!Cohen, Ian}
\index{Fitch, John P.}\index{People!Fitch, John P.}

LIMITS is a fast limit package for REDUCE for functions which are
continuous except for computable poles and singularities, written by Stanley L.~Kameny, based on some
earlier work by Ian Cohen and John P. Fitch.  The Truncated Power Series
package is used for non-critical points, at which the value of the
function is the constant term in the expansion around that point.
\index{l'H\^opital's rule}
l'H\^opital's rule is used in critical cases, with preprocessing of
$\infty - \infty$ forms and reformatting of product forms in order
to apply l'H\^opital's rule.  A limited amount of bounded arithmetic
is also employed where applicable.

The standard way of calling limit, applying all of the methods, is
\ttindextype{limit}{operator}\hypertarget{operator:LIMIT}{}
\begin{syntax}
  \f{limit(}\meta{exprn:algebraic},\,\meta{var:kernel},\,%
    \meta{limpoint:algebraic})\,:\,\textit{algebraic}
\end{syntax}
The result is the limit of exprn as var approaches limpoint.
To compute the of \(\sin(x)/x\) at the point $0$, enter
\begin{verbatim}
limit(sin(x)/x,x,0);

1
\end{verbatim}
If the limit depends upon the direction of approach to the \texttt{limpoint},                                                                                                        the onesided limit functions \texttt{limit!+} and \texttt{limit!-} may be used:
\ttindextype{limit"!+}{operator}\ttindextype{limit"!-}{operator}
\hypertarget{operator:LIMIT+}{}
\hypertarget{operator:LIMIT-}{}
\begin{syntaxtable}
  \texttt{limit!+(}\meta{exprn:algebraic},\,\meta{var:kernel},\,%
    \meta{limpoint:algebraic}\texttt{)}\,:\,\textit{algebraic} \\
  \texttt{limit!-(}\meta{exprn:algebraic},\,\meta{var:kernel},\,%
    \meta{limpoint:algebraic}\texttt{)}\,:\,\textit{algebraic}
\end{syntaxtable}
they are defined by:
\begin{quote}
\begin{tabular}{l}
 \texttt{limit!+ (limit!-)} (exp,var,limpoint) $\rightarrow$\texttt{limit}(exp*,$\epsilon$,0), \\
  \qquad exp*=sub(var=var+(-)$\epsilon^2$,exp)
\end{tabular}
\end{quote}
for example,
\begin{verbatim}
limit!+(sqrt x/sin x,x,0);

infinity;
\end{verbatim}


\iffalse
\subsection{Diagnostic Functions}

\ttindex{LIMIT0}
\hypertarget{operator:LIMIT0}{}
\vspace{.1in}
\noindent \texttt{LIMIT!-}(EXPRN:{\em algebraic}, VAR:{\em kernel},
LIMPOINT:{\em algebraic}):{\em algebraic}
\noindent \texttt{LIMIT0}(EXPRN:{\em algebraic}, VAR:{\em kernel},
LIMPOINT:{\em algebraic}):{\em algebraic}
\vspace{.1in}

This function will use all parts of the limits package, but it does not
combine log terms before taking limits, so it may fail if there is a sum
of log terms which have a removable singularity in some of the terms.

\ttindex{LIMIT1}
\vspace{.1in}
\noindent \texttt{LIMIT1}(EXPRN:{\em algebraic}, VAR:{\em kernel},
LIMPOINT:{\em algebraic}):{\em algebraic}
\vspace{.1in}

\index{TPS package}
This function uses the TPS branch only, and will fail if the limit point is
singular.

\ttindex{LIMIT2}
\hypertarget{operator:LIMIT2}{}
\begin{quote}
\begin{tabular}{l@{}l}
\texttt{LIMIT2(} & TOP:{\em algebraic}, \\
&BOT:{\em algebraic}, \\
&VAR:{\em kernel}, \\
&LIMPOINT:{\em algebraic}):{\em algebraic}
\end{tabular}
\end{quote}

This function applies l'H\^opital's rule to the quotient (TOP/BOT).

\fi
