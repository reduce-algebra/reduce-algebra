% $Id$
\index{Wright, Francis J.}\index{People!Wright, Francis J.}

The operator \texttt{with} allows an expression to be evaluated and
its value displayed subject to switch settings that apply only locally
during the evaluation and display of this expression.  Its syntax is
\begin{center}
  \textit{expression} \texttt{with}
  \textit{on/off} \textit{switches}\texttt{,}
  \textit{on/off} \textit{switches}\texttt{,}
  ...
\end{center}
where \textit{on/off} is either \texttt{on} or \texttt{off}, and
\textit{switches} is a single switch name or a comma-separated
sequence of switch names (as for the \texttt{on} and \texttt{off}
commands).  It is intended primarily for interactive use and provides
a convenient way to experiment with the effects of different switches.
Here are some examples, assuming default switch settings:
\begin{verbatim}
(a+b)^2/2 with off exp, on div;

 1         2
---*(a + b)
 2

pi with on rounded;

3.14159265359

fix(sqrt 10) with on rounded;

3
\end{verbatim}

The \texttt{with} operator has precedence immediately above
\texttt{:=}, so \texttt{with} binds tighter than \texttt{:=} but looser
than almost every other infix operator.  Therefore,
\begin{verbatim}
x := pi + e with on rounded;
\end{verbatim}
parses as
\begin{verbatim}
x := ((pi + e) with on rounded);
\end{verbatim}
Hence,
\begin{verbatim}
x := pi + e with on rounded;

5.85987448205

x;

 103088002085129
-----------------
 17592186044416

x with on rounded;

5.85987448205
\end{verbatim}
In the above example, $x$ is assigned a floating-point number, but it
is displayed as a rational number when the default switch settings are
in effect.

The keywords \texttt{on} and \texttt{off} can appear as many times as
desired in the right operand of \texttt{with}.  If a switch already
has the setting specified in the right operand then it is not changed,
but if it has the opposite setting then it is changed before the left
operand is evaluated, and displayed if required, and changed back
afterwards.  Repeated identical switch settings are ignored but
conflicting settings cause an error.  The order of switch settings is
preserved.

Not all switches work sensibly locally, but for example
\begin{verbatim}
int(sin x, x) with on trint;
\end{verbatim}
turns on integration tracing temporarily.
