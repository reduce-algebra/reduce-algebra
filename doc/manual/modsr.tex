\section{Modular solve and roots}

\ttindextype{m\_solve}{operator}\ttindextype{m\_roots}{operator}
\hypertarget{operator:M_SOLVE}{}
\hypertarget{operator:M_ROOTS}{}
The operators \f(m\_solve) and \f(m\_roots) are for
modular polynomials and modular polynomial systems.\footnote{This code was written by Herbert Melenk.}  The moduli need not
be primes. \f{m\_solve} requires a modulus to be set.  \f{m\_roots} takes the
modulus as a second argument. For example:

\begin{verbatim}
on modular; setmod 8;
m_solve(2x=4);            ->  {{X=2},{X=6}}
m_solve({x^2-y^3=3});
    ->  {{X=0,Y=5}, {X=2,Y=1}, {X=4,Y=5}, {X=6,Y=1}}
m_solve({x=2,x^2-y^3=3}); ->  {{X=2,Y=1}}
off modular;
m_roots(x^2-1,8);         ->  {1,3,5,7}
m_roots(x^3-x,7);         ->  {0,1,6}
\end{verbatim}


