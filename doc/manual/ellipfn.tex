\subsection{Elliptic Functions: Introduction}
\index{Elliptic functions}
\index{Jacobi Elliptic functions}
\index{Elliptic Integrals}
\index{Nome and Related functions}
\index{Jacobi Theta functions}
\index{Theta function derivatives}
\index{Weierstrass Elliptic functions}
\index{Sigma functions}
\index{Inverse Elliptic functions}
\hypertarget{ELLIPFNS}{}
%% For MathJax, must be part of a paragraph to avoid extra space:
\ifdefined\VerbMath{\makehashletter
\(\newcommand{\f}[1]{\texttt{#1}}\)%
}\fi
%
The package ELLIPFN is designed to provide algebraic and numeric manipulations of
many elliptic functions, namely:

\begin{itemize}
\item \hyperlink{JACEF}{Jacobi's Elliptic Functions};
\item \hyperlink{ELLIPI}{Elliptic Integrals};
\item \hyperlink{ELLIPNOME}{Nome and Related Functions};
\item \hyperlink{JACTF}{Jacobi's Theta Functions} and their
\hyperlink{THETAD}{ derivatives};
\item \hyperlink{WEIERSTRASS}{Weierstrass Elliptic Functions} and the
\hyperlink{SIGMA}{Sigma Function};
\item \hyperlink{SIGMA1}{Other Sigma Functions};
\item \hyperlink{ETA}{Period Lattice and Related Functions};
\item \hyperlink{INVELL}{Inverse Elliptic Functions}.
\end{itemize}

\subsection{Elliptic Functions}
\index{Elliptic functions}
The implementation of the functions in this and the next two subsections have
been substantially revised by Alan Barnes in 2019. This is to bring the
notation more into line with standard (British) texts such as Whittaker
\& Watson \cite{WhittakerWatson:69} and Lawden \cite{Lawden:89} and also to correct
a number of errors and omissions. These changes and additions will be itemised in the relevant
sections below.  New subsections has been added in 2021 to support  Weierstrassian Elliptic
functions, Sigma functions and inverse Jacobi elliptic functions.

The functions in this subsection are for the most part autoloading;
the exceptions being the subsidiary utility functions such as the AGM function,
the quasi-period factors, lattice functions and derivatives of the theta functions.

\hypertarget{JACEF}{}
\subsection{Jacobi Elliptic  Functions}
The following functions have been implemented:
\begin{itemize}
\item The 12 Jacobi Elliptic Functions
\item The Jacobi Amplitude Function
\item Arithmetic Geometric Mean
\item Descending Landen Transformation
\end{itemize}

\index{Jacobi Elliptic functions}
The following Jacobi elliptic functions are available:-
\hypertarget{operator:JACOBISN}{}
\hypertarget{operator:JACOBICN}{}
\hypertarget{operator:JACOBIDN}{}
\hypertarget{operator:JACOBISD}{}
\hypertarget{operator:JACOBIND}{}
\hypertarget{operator:JACOBIDC}{}
\hypertarget{operator:JACOBINC}{}
\hypertarget{operator:jACOBISC}{}
\hypertarget{operator:JACOBINS}{}
\hypertarget{operator:JACOBIDS}{}
\hypertarget{operator:JACOBICS}{}
\ttindextype[ELLIPFN]{jacobisn}{operator}\ttindextype[ELLIPFN]{jacobicn}{operator}
\ttindextype[ELLIPFN]{jacobidn}{operator}\ttindextype[ELLIPFN]{jacobicd}{operator}
\ttindextype[ELLIPFN]{jacobisd}{operator}\ttindextype[ELLIPFN]{jacobind}{operator}
\ttindextype[ELLIPFN]{jacobidc}{operator}\ttindextype[ELLIPFN]{jacobinc}{operator}
\ttindextype[ELLIPFN]{jacobisc}{operator}\ttindextype[ELLIPFN]{jacobins}{operator}
\ttindextype[ELLIPFN]{jacobids}{operator}\ttindextype[ELLIPFN]{jacobics}{operator}
\begin{itemize}
\item jacobisn(u,k)
\item jacobidn(u,k)
\item jacobicn(u,k)
\item jacobicd(u,k)
\item jacobisd(u,k)
\item jacobind(u,k)
\item jacobidc(u,k)
\item jacobinc(u,k)
\item jacobisc(u,k)
\item jacobins(u,k)
\item jacobids(u,k)
\item jacobics(u,k)
\end{itemize}

These differ somewhat from the originals implemented by Lisa Temme in that
the second argument is now the modulus (usually denoted by $k$ in most texts
rather than its square $m$).  The notation for the most part follows  Lawden
\cite{Lawden:89}. The last nine Jacobi functions are related to the three
basic ones: \f{jacobisn(u,k)}, \f{jacobicn(u,k)} and \f{jacobidn(u,k)} and
use Glaisher's notation. For example
\[ \mathrm{ns}(x,k) = \frac{1}{\mathrm{sn}(u,k)}, \qquad
\mathrm{cs}(x,k) = \frac{\mathrm{cn}(u,k)}{\mathrm{sn}(u,k)}, \qquad
\mathrm{cd}(x,k) = \frac{\mathrm{cn}(u,k)}{\mathrm{dn}(u,k)}. \]

Extensive rule lists are provided for differentiation of these functions with
respect to either argument, for argument shifts by multiples of the two
quarter-periods $K$ and $iK'$ and finally Jacobi's transformations for a
purely imaginary first argument.

Complex split functions have recently been implemented for the 12 Jacobi
functions for the cases when the modulus $k$ is either real or imaginary.
Thus \texttt{REPART} and \texttt{IMPART} will now return rational expressions
involving the twelve Jacobi functions with real arguments and a real modulus
$|k|<1$. As far as I (AB) am aware there are no known methods of implementing
split functions for general complex values of the modulus.

\hypertarget{reserved:TO_SN}{}
\hypertarget{reserved:TO_CN}{}
\hypertarget{reserved:TO_DN}{}
\hypertarget{reserved:NO_GLAISHER}{}
\hypertarget{reserved:JacobiAdditionRules}{}
\ttindextype[ELLIPFN]{to\_sn}{rule list}
\ttindextype[ELLIPFN]{to\_cn}{rule list}
\ttindextype[ELLIPFN]{to\_dn}{rule list}
\ttindextype[ELLIPFN]{no\_glaisher}{rule list}
\ttindextype[ELLIPFN]{jacobiadditionrules}{rule list}
Four useful rule lists \texttt{TO\_SN}, \texttt{TO\_CN}, \texttt{TO\_DN} and
\texttt{NO\_GLAISHER} are provided for user convenience. The first
\texttt{TO\_SN} replaces occurences of the squares of $\mathrm{cn}$ and
$\mathrm{dn}$ in favour of the square of $\mathrm{sn}$ using the
`Pythagorean' identities. \texttt{TO\_CN} and \texttt{TO\_DN} apply these
identities to replace squares in favour of $\mathrm{cn}$ and $\mathrm{dn}$
respectively. At most one of these rule sets should be active at any one time
to avoid a potential infinite recursion in the simplification.
The fourth rule list \texttt{NO\_GLAISHER} replaces all occurences of the
nine subsidiary Jacobi elliptic functions ($\mathrm{ns}$, $\mathrm{sc}$,
$\mathrm{sd}$, \ldots by reciprocals and quotients of the `basic' Jacobi
elliptic functions $\mathrm{sn}$, $\mathrm{cn}$ and $\mathrm{dn}$.

A fifth rule list \texttt{JACOBIADDITIONRULES} applies addition rules when the
first argument of any Jacobi function is a sum.  Note that this rule list is NO
LONGER active by default. There are many equivalent ways of expressing the
right-hand sides of these rules. Previously all the rules were expressed using
only the `basic' Jacobi functions: $\mathrm{sn}$, $\mathrm{cn}$ and
$\mathrm{dn}$. Currently, however, the right-hand sides of the rules for the
reciprocal and quotient Glaisher functions $\mathrm{ns}$, $\mathrm{cs}$,
$\mathrm{sd}$ etc are expressed using the three Glaisher functions with
the same `denominator' as on the left-hand side. Thus the rule for $\mathrm{ns}$
is expressed in terms of $\mathrm{ns}$, $\mathrm{cs}$ and $\mathrm{ds}$
whilst that for $\mathrm{cd}$ is expressed in terms of $\mathrm{nd}$,
$\mathrm{cd}$ and $\mathrm{sd}$ whereas the rules for the three `basic'
functions are unchanged and use only $\mathrm{sn}$, $\mathrm{cn}$ and
$\mathrm{dn}$.  Note, however, that the previous form of these rules will be
obtained if the rule-list \texttt{NO\_GLAISHER} is also active. 

If both the switches \sw{rounded} and \sw{complex} are ON and both arguments
of a Jacobi function are purely numerical, it will be evaluated numerically 
The numerical evaluation of the Jacobi functions now uses their definitions
in terms of theta functions as these are valid for all complex values of the
argument and modulus.  Note that as a consequence it is now necessary that the
switch \sw{complex} is ON even if both the argument and the modulus are real.

The traditional AGM and  $\phi$-function based algorithm which was previously
used when the argument and modulus are both real and $|k| < 1$ as been
superceded. It appears that there was a problem with the numerical evaluation
of \texttt{dn, nd, sd, ds, cd, dc} resulting in largish rounding errors. There
was no problem with previous evaluation of \texttt{sn, ns, cn, nc, sc, cs} and
these produced the same results (modulo acceptable rounding errors) as the
theta function based method now used.

\subsubsection{Jacobi Amplitude Function}
\hypertarget{operator:JACOBIAM}{}
\ttindextype[ELLIPFN]{jacobiam}{operator}
The amplitude of $u$ can be evaluated using the \f{jacobiam(u,k)}
command.  A rule list is provided for differentiation of this
function with respect to either argument.

\subsection{Some Numerical Procedures}
This section briefly describes several procedures which are primarily
intended for use in the numerical evaluation of the various elliptic
functions and integrals rather than for direct use by users.

\subsubsection{Arithmetic Geometric Mean (AGM)}
\hypertarget{operator:AGM_FUNCTION}{}
\ttindextype[ELLIPFN]{AGM\_function}{operator}
A procedure to evaluate the AGM of initial values \(a_0,b_0,c_0\)
exists as \\
\texttt{AGM\_function(\(a_0,b_0,c_0\))} and will return \\
$\{ N, AGM, \{ a_N, \ldots ,a_0\}, \{ b_N, \ldots ,b_0\},
\{c_N, \ldots ,c_0\}\}$,
where $N$ is the number of steps to compute the AGM to the
desired accuracy.

To determine the Elliptic Integrals K($m$), E($m$) we use initial values
\(a_0 = 1\); \(b_0 = \sqrt{1-k^2}\) ; \(c_0 = k\).

\subsubsection{Descending Landen Transformation}
The procedure to evaluate the Descending Landen Transformation of
$\phi$ and $\alpha$ uses the following equations:
\begin{align*}
 (1+\sin \alpha_{n+1})(1+\cos \alpha_n)=2 &\text{ where } \alpha_{n+1}<\alpha_n, \\
  \tan(\phi_{n+1}-\phi_n)=\cos \alpha_n \tan \phi_n & \text{ where } \phi_{n+1}>\phi_n.
\end{align*}
It can be called using \f{landentrans}($\phi_0$, $\alpha_0$)
and will return \\
$\{\{\phi_0, \ldots ,\phi_n\},\{\alpha_0, \ldots ,\alpha_n\}\}$.

\subsubsection{Symmetric Elliptic Integrals}
TO BE COMPLETED.

\subsection{Elliptic Integrals}
\hypertarget{ELLIPI}{}
\index{Elliptic Integrals}
The following functions have been implemented:

\begin{itemize}
\item Complete \& Incomplete Elliptic Integrals of the First Kind
\item Complete \& Incomplete Elliptic Integrals of the Second Kind
%\item Ellpitic Integrals of the Third Kind
\item Jacobi's Zeta Function
\end{itemize}

These again differ somewhat from the originals implemented by Lisa Temme
as the second argument is now the modulus $k$ rather that its square.
Also in the original implementation  there was some confusion between
Legendre's form and Jacobi's form of the incomplete elliptic integrals of
the second kind; $E(u,k)$ denoted the first in numerical
evaluations and the second in the derivative formulae for the Jacobi
elliptic functions with respect to their second argument.
This confusion was perhaps understandable
as in the literature some authors use the notation $\mathrm{E}(u, k)$ for
the Legendre form and others for Jacobi's form.

To bring the notation more into line with that in the NIST Digital Library of
Mathematical Functions and avoid any possible confusion, $\mathrm{E}(u, k)$ is used for
the Legendre form and $\mathcal{E}(u, k)$ for Jacobi's form.
This differs from the 2019 version of this section which followed Lawden \cite{Lawden:89},
where the notation $\mathrm{D}(\phi, k)$  and $\mathrm{E}(u, k)$ were used for the
Legendre and Jacobi forms respectively.

A number of rule lists have been provided to implement, where appropriate,
derivatives of these functions, addition rules and periodicity and
quasi-periodicity properties and to provide simplifications for special values
of the arguments.

\subsubsection{Elliptic F}
\hypertarget{operator:ELLIPTICF}{}
\ttindextype[ELLIPFN]{EllipticF}{operator}

The Elliptic F function can be used as \f{EllipticF(phi,k)} and
will return the value of the \emph{Incomplete Elliptic Integral of the
First Kind}:
\[\mathrm{F}(\phi, k)=\int_0^\phi(1-k^2 \sin^2 \theta)^{-1/2} \mathrm{d}\theta.\]

\subsubsection{Elliptic K}
\hypertarget{operator:ELLIPTICK}{}
\hypertarget{operator:ELLIPTICK'}{}
\ttindextype[ELLIPFN]{EllipticK}{operator}
\ttindextype[ELLIPFN]{EllipticK'}{operator}

The Elliptic K function can be used as \f{EllipticK(k)} and will
return the value of the \emph{Complete Elliptic Integral of the
First Kind}:
\[\mathrm{K}(k)=\mathrm{F}(\pi/2, k) =\int_0^{\pi/2}(1-k^2 \sin^2 \theta)^{-1/2}\mathrm{d}\theta.\]
This is one of the quarter periods of the Jacobi elliptic
functions and is often used in the calculation of other elliptic functions.

The complementary Elliptic K$'$ function can be used as \f{EllipticK!$'$(k)}
and will return the value
\[\mathrm{K}(k')=\mathrm{K}(\sqrt{1-k^2})\]
which is the other quarter-period of the Jacobi elliptic functions.

\subsubsection{Elliptic E}
\hypertarget{operator:ELLIPTICE}{}
\hypertarget{operator:ELLIPTICE'}{}
\ttindextype[ELLIPFN]{EllipticE}{operator}
\ttindextype[ELLIPFN]{EllipticE'}{operator}

The Elliptic E function comes with either one or two arguments;
used with two arguments as \f{EllipticE(u,k)}
it will return the value of Legendre's form of
the \emph{Incomplete Elliptic Integral of the Second Kind}:
\[\mathrm{E}(\phi, k)=\int_0^\phi \sqrt{1-k^2 \sin^2 \theta} \,\mathrm{d}\theta.\]
 When called with one argument \f{EllipticE(k)} will return the value of the
\emph{Complete Elliptic Integral of the Second Kind}:
\[\mathrm{E}(k)=\mathrm{E}(\pi/2, k) =
\int_0^{\pi/2} \sqrt{1-k^2 \sin^2 \theta} \,\mathrm{d}\theta.\]

The complementary Elliptic E$'$ function can be used as \f{EllipticE!$'$(k)}
and will return the value
\[\mathrm{E}(k') = \mathrm{E}(\sqrt{1-k^2}).\]

The numerical evaluation of the complete integrals is more robust as it is
now via symmetric elliptic integrals. It should now work for all complex values
of the modulus. 

\subsubsection{Jacobi E}
\hypertarget{operator:JACOBIE}{}
\ttindextype[ELLIPFN]{JacobiE}{operator}
The Jacobi E function can be used as  \f{jacobiE(u,k)};
it will return the value of Jacobi's form of
the \emph{Incomplete Elliptic Integral of the Second Kind}:
\[\mathcal{E}(u, k)=\int_0^u \mathrm{dn}^2 (v, k) \,\mathrm{d}v.\]

The relationship between the two forms of incomplete elliptic integrals can
be expressed as
\[\mathcal{E}(u, k) = \mathrm{E}(\mathrm{am}(u), k).\]
Note that
\[\mathrm{E}(k)=\mathcal{E}(\mathrm{K}(k), k)
=\int_0^{\mathrm{K}(k)} \mathrm{dn}^2(v, k) \,\mathrm{d}v.\]

On a GUI that supports calligraphic characters (NB.\ this is now the case with the
CSL GUI), there is no problem and it is rendered as $\mathcal{E}(u,k)$
in accordance with NIST usage.
On non-GUI interfaces the Jacobi E function is rendered as E\_j.

%\subsection{Ellpitic $\Pi$}
%
%The Elliptic $\pi$ function can be used as \f{EllipticPi( )} and
%will return the value of the {\underline {Elliptic Integral of the
%Third Kind}}.
%

\subsubsection{Jacobi's Zeta Function}
\hypertarget{operator:JACOBIZETA}{}
\ttindextype[ELLIPFN]{jacobiZeta}{operator}\index{Zeta function of Jacobi!\package{Ellipfn} package}

This can be called as \f{jacobiZeta(u,k)} and refers to Jacobi's (elliptic)
Zeta function $\mathrm{Z}(u,k)$ whereas the operator \f{Zeta} will invoke
Riemann's $\zeta$ function.


\subsection{Some Numerical Utility Functions}
\hypertarget{operator:NOME}{}
\hypertarget{operator:NOME2K}{}
\hypertarget{operator:NOME2K'}{}
\hypertarget{operator:NOME2MOD}{}
\hypertarget{operator:NOME2MOD'}{}
\hypertarget{ELLIPNOME}{}
\index{Nome and Related functions}
\ttindextype[ELLIPFN]{nome}{operator}
\ttindextype[ELLIPFN]{nome2"!K}{operator}
\ttindextype[ELLIPFN]{nome2"!K"!'}{operator}
\ttindextype[ELLIPFN]{nome2mod}{operator}
\ttindextype[ELLIPFN]{nome2mod"!'}{operator}

Five utility functions are provided:
\begin{itemize}
\item \f{nome2mod(q)}
\item \f{nome2mod!$'$(q)}
\item \f{nome2!K(q)}
\item \f{nome2!K!$'$(q)}
\item \f{nome(k)}
\end{itemize}

These are only operative when the switch \sw{rounded} is on and their
argument is numerical. The first pair relate the nome $q$ of the theta
functions with the moduli $k$ and $k'=\sqrt{1-k^2}$ of the associated Jacobi
elliptic functions.

The second pair return the quarter-periods K and K$'$ respectively of
the Jacobi elliptic functions associated with the nome $q$.

Finally, \f{nome(k)} returns the nome $q$ associated with the modulus $k$ of
a Jacobi elliptic function and is essentially the inverse of \f{nome2mod}.

\subsection{Jacobi Theta Functions}
\hypertarget{JACTF}{}
\index{Jacobi Theta functions}
These theta functions differ from those originally defined by Lisa Temme
in a number of respects.
Firstly four separate functions of two arguments are defined:
\hypertarget{operator:ELLIPTICTHETA1}{}
\hypertarget{operator:ELLIPTICTHETA2}{}
\hypertarget{operator:ELLIPTICTHETA3}{}
\hypertarget{operator:ELLIPTICTHETA4}{}
\ttindextype[ELLIPFN]{elliptictheta1}{operator} \ttindextype[ELLIPFN]{elliptictheta2}{operator}
\ttindextype[ELLIPFN]{elliptictheta3}{operator} \ttindextype[ELLIPFN]{elliptictheta4}{operator}
\begin{itemize}
\item \f{elliptictheta1(u,tau)} $\qquad \vartheta_1(u, \tau)$
\item \f{elliptictheta2(u,tau)} $\qquad \vartheta_2(u, \tau)$
\item \f{elliptictheta3(u,tau)} $\qquad \vartheta_3(u, \tau)$
\item \f{ellipticthetas(u,tau)} $\qquad \vartheta_4(u, \tau)$
\end{itemize}

rather than a single function with three arguments (with the first argument
taking integer values in the range 1 to 4).
Secondly the periods are $2\pi, 2\pi, \pi$ and $\pi$ respectively
(NOT 4K, 4K, 2K and 2K).
Thirdly the second argument is the modulus $\tau = a+i b$ where $b=\Im\tau>0$
and hence the quasi-period is $\pi\tau$.

The second parameter was previously the nome $q$
where $|q|<1$. As a consequence \f{elliptictheta1} and \f{elliptictheta2} were
multi-valued owing to the appearance of $q^{1/4}$ in their defining expansions.
\f{elliptictheta3} and \f{elliptictheta4} were, however, single-valued
functions of $q$.

Regarded as functions of $\tau$,
\f{elliptictheta1} and \f{elliptictheta2} are single-valued functions. The nome
is given by $q = \exp(i\pi\tau)$  so that the condition $\Im(\tau)>0$ ensures
that $|q| < 1$. Note also  in this case $q^{1/4}$ is interpreted as
$\exp(i\pi\tau/4)$ rather than the principal value of $q^{1/4}$.
Thus, $\tau$, $2+\tau$, $4+\tau$ and $6+\tau$ produce four different values of
both \f{elliptictheta1} and \f{elliptictheta2} although they all correspond to
the same nome $q$.

The four theta functions are defined by their Fourier series:
\begin{align*}
  \vartheta_1(z,\tau) & = 2 e^{i\pi\tau/4}\sum_{n=0}^\infty (-1)^nq^{n^2+n} \sin(2n+1)z\\
\vartheta_2(z,\tau) & = 2 e^{i\pi\tau/4}\sum_{n=0}^\infty q^{n^2+n} \cos(2n+1)z\\
\vartheta_3(z,\tau) & = 1 +2\sum_{n=1}^\infty q^{n^2} \cos 2n z\\
\vartheta_4(z,\tau) & = 1 +2\sum_{n=1}^\infty (-1)^n q^{n^2} \cos 2n z.
\end{align*}

Utilising the periodicity and quasi-periodicity of the theta functions
some generalised shift rules are implemented to shift their first argument
into the base period parallelogram with vertices
\[(\pi/2, \pi\tau/2),\quad (-\pi/2, \pi\tau/2),\quad (-\pi/2, -\pi\tau/2),
\quad (\pi/2, -\pi\tau/2).\]
Together with the relation $\vartheta_1(0,\tau)=0$,  these shift rules serve to
simplify all four theta functions to zero when appropriate.

When the switches \sw{rounded} and \sw{complex} are on and the arguments are
purely numerical and the imaginary part of $\tau$ positive,
the theta functions are evaluated numerically. Note that as $\tau$ is
necessarily complex, the switch \sw{complex} \emph{must} be on.

In what follows $a$ and $b$ will denote the real and imaginary parts of
$\tau$ respectively and so $|q| = \exp(-\pi b)$ and $\arg q =\pi a$.
The series for the theta functions are fairly rapidly convergent
due to the quadratic growth of the exponents of the nome $q$ -- except
for values of $q$ for which $|q|$ is near to 1
(i.e. $b=\Im \tau $ close to zero).
In such cases the direct algorithm would suffer from slow convergence and
rounding errors.
For such values of $|q|$, Jacobi's transformation $\tau'=-1/\tau$ can be
used to produce a smaller value of the nome and so increase the rate of
convergence.
This works very well for real values of $q$, or equivalently for $\tau$ purely
imaginary since $q'= q^{1/b^2}$, but for complex
values the gains are somewhat smaller. The Jacobi transformation produces a
nome $q'$ for which $|q'| =  |q|^{1/(a^2+b^2)}$.

When $\Re q < 0$, the Jacobi transformation is preceded by either the
modular transformation $\tau' = \tau+1$ when $\Im q < 0$, or $\tau' = \tau-1$
when $\Im q > 0$, which both have the effect  of multiplying $q$ by $-1$,
so that the new nome has a non-negative real part and $|a| \leq 1/2$.
Thus the worst case occurs for values of the nome $q$ near to $\pm i$ where
$|q'| \approx |q|^4$.

By using a series of Jacobi transformations preceded, if necessary by
$\tau$-shifts to ensure $|a| <= 1/2$, $|q|$ may be reduced to an acceptable
level. Somewhat arbitrarily these Jacobi's transformations are used
until $b > 0.6$ (i.e.~$|q| < 0.15$). This seems to produce reasonable
behaviour. In practice more than two applications of Jacobi transformations
are rarely necessary.

The previous version of the numerical code returned the principal values
of $\vartheta_1$ and $\vartheta_2$, that is the ones obtained by taking
the principal value of $q^{1/4}$ in their series expansions. The current version replaces
$q^{1/4}$ by $\exp(i\pi\tau/4)$.  If the principal value is required, it is easily obtained
by multiplying by the `correcting' factor $q^{1/4}\exp(-i\pi\tau/4)$.

\subsubsection{Derivatives of Theta Functions}

\hypertarget{THETAD}{}
\hypertarget{operator:THETA1D}{}
\hypertarget{operator:THETA2D}{}
\hypertarget{operator:THETA3D}{}
\hypertarget{operator:THETA4D}{}
\index{Theta function derivatives}
\ttindextype[ELLIPFN]{theta1d}{operator} \ttindextype[ELLIPFN]{theta2d}{operator}
\ttindextype[ELLIPFN]{theta3d}{operator}\ttindextype[ELLIPFN]{theta4d}{operator}
Four functions are provided:
\begin{itemize}
\item \f{theta1d(u,ord,tau)}
\item \f{theta2d(u,ord,tau)}
\item \f{theta3d(u,ord,tau)}
\item \f{theta4d(u,ord,tau)}
\end{itemize}
These return the $d$th derivatives of the respective theta functions
with respect to their first argument $u$; $\tau$ is as usual the modulus
of the theta function. These functions are only operative when the switches
\sw{rounded} and \sw{complex} are ON and their arguments are numeric with
$d$ being a positive integer.  They are provided mainly to support the implementation
the Weierstrassian and Sigma functions discussed in the following subsection.

The numeric code simply sums the Fourier series for the required derivatives.
Unlike the theta functions themselves the code does not use the quasi-periodicity nor
modular transformations to speed  up the convergence of the series by reducing the sizes
of $\Im u$ and $|q|$.  In the numerical evaluation of the Weierstrassian and Sigma functions
these functions are only called after the necessary shifts of the argument $u$ and modular
transformations of $\tau$ have been performed. These are much simpler in this context.

Nevertheless they may be used from top level and numerical experiments reveal that the rounding
errors are not significant provided $|q|$ is not near one (say $|q|<0.9$)
and $u$ is real or at least has a relatively small imaginary part.

\subsection{Weierstrass Elliptic \& Sigma Functions}
\index{Weierstrass Elliptic functions}
\index{Sigma functions}
Three main functions of three arguments are defined:
\hypertarget{WEIERSTRASS}{}
\hypertarget{WEIERSTRASSZETA}{}
\hypertarget{SIGMA}{}
\hypertarget{operator:WEIERSTRASS_SIGMA}{}
\hypertarget{operator:WEIERSTRASS}{}
\hypertarget{operator:WEIERSTRASSZETA}{}
\ttindextype[ELLIPFN]{weierstrass}{operator} \ttindextype[ELLIPFN]{weierstrassZeta}{operator}
\ttindextype[ELLIPFN]{weierstrass\_sigma}{operator}
\begin{itemize}
\item  $\wp(u, \omega_1, \omega_3)$ \ --- \ \f{weierstrass(u,omega1,omega3)}
\item $\zeta_w(u, \omega_1, \omega_3)$ \ --- \ \f{weierstrassZeta(u,omega1,omega3)}
\item $\sigma(u, \omega_1, \omega_3)$ \ --- \ \f{weierstrass\_sigma(u,omega1,omega3)}
\end{itemize}

The notation used is broadly similar used by Lawden \cite{Lawden:89} which is also used in the
NIST Digital Library of Mathematical Functions \href{https://dlmf.nist.gov/}{DLMF:NIST}. However,
$\zeta_w$ is used for the Weierstrassian Zeta function to distinguish it from the Riemann Zeta
function and the usual symbol $\wp$ is used for the Weierstrassian elliptic function itself.

The two primitive periods of the Weierstrass function are $2\omega_1$ and $2\omega_3$ and these must satisfy
$\Im(\omega_3/\omega_1) \neq 0$. The two periods are normally numbered so that $\tau = \omega_3/\omega_1$ has
a positive imaginary part and hence the nome $q = exp(i\pi\tau)$ satisfies $|q| <1$.

Any linear combination $\Omega_{m,n} = 2m\omega_1 +2n\omega_3$ where $m$ and $n$ are
integers (not both zero) is also a period. The set of all such periods plus the origin form a lattice. In the literature
$-(\omega_1+\omega_3)$ is often denoted by $\omega_2$ and $2\omega_2$ is clearly also a period; this
accounts for the gap in the numbering of primitive periods. The period $\omega_2$ is not used in \REDUCE the rule sets for
the Weierstrassian elliptic and related functions.

The primitive periods are not unique;
indeed any periods $2\Omega_1$ and $2\Omega_3$ defined by the unimodular integer bilinear transformation:
\[\Omega_1 = a\omega_1 + b\omega_3,\qquad\Omega_3 = c\omega_1 + d\omega_3,\qquad\text{ where }ad-bc = 1\]
are also primitive. This fact is very useful in the numerical evaluation of the Weierstrassian and Sigma
functions as a sequence of such transformations may be used to increase the size $\Im \tau$ and so reduce
the size of $|q|$. Thus the Fourier series for the theta functions and their derivatives will converge rapidly.
In theory these transformations may be used to reduce the size of $|q|$ until $\Im \tau \geq \sqrt 3/2$ when
$|q|<0.06$. However, in numerical evaluations in \REDUCE it is sufficient to use these transformations only until
$\Im \tau > 0.7$, i.e.~until $|q| < 0.11$. In practice only two or three iterations are required
and usually very much smaller values of $|q|$ are achieved particularly when $\tau$ is purely imaginary i.e.~$q$ is real.

In the numerical evaluations, if a result is real (or purely imaginary) it may
happen that the result returned has a very small imaginary part
(resp. real part). The ratio of the `deliquent' part to the actual result is
invariably smaller than current PRECISION and is due to rounding. Similarly if
the true result is actually zero the result returned may have a very small
absolute value -- again smaller than the current PRECISION.

The Weierstrassian function is even and has a pole of order 2 at all lattice points.
The Zeta and Sigma functions are only quasi-periodic on the lattice. Zeta is odd and has simple poles of residue
1 at all lattice points. The basic Sigma function $\sigma(u,\omega_1,\omega_3)$ is odd and regular everywhere as is
the function $\vartheta_1(u,\tau)$ to which it is closely related. It has zeros at all lattice points. All three functions
$\wp$, $\zeta_w$ and $\sigma$ are homogenous of degrees -2, -1 and +1 respectively. The functions are related by
\[ \wp(u) = -\zeta_w^\prime(u),\qquad\qquad \zeta_w(u) = \sigma^\prime(u)/\sigma(u),\]
where the lattice parameters have been omitted for conciseness.

Rule sets are provided which implement all the properties such as double periodicity discussed above. For numerical evaluation
the switches \sw{rounded} and \sw{complex} must both be ON and all three parameters must be numeric. It is not, however,
necessary to ensure $\Im(\omega_3/\omega_1) >0$ as the second and third parameters will be swapped if required.

\subsubsection{Alternative forms of the Weierstrass Functions}
\hypertarget{WEIERSTRASS1}{}
\hypertarget{WEIERSTRASSZETA1}{}
\hypertarget{operator:WEIERSTRASS1}{}
\hypertarget{operator:WEIERSTRASSZETA1}{}
\ttindextype[ELLIPFN]{weierstrass1}{operator} \ttindextype[ELLIPFN]{weierstrassZeta1}{operator}

Two commonly used alternative forms of the Weierstrassian functions in which
they are regarded as functions of the lattice invariants $g_2$ and $g_3$
rather than the primitive periods $\omega_1$ and $\omega_3$ are provided:
\begin{itemize}
\item  $\wp(u \mid g_2, g_3)$ \ --- \ \f{weierstrass1(u,g2,g3)}
\item $\zeta_w(u \mid g_2, g_3)$ \ --- \ \f{weierstrassZeta1(u,g2,g3)}.
\end{itemize}
Note that for output they are distinguished from the two discussed above
by separating the first and
second arguments by a vertical bar rather than a comma. The rule for
differentiation of the Weierstrass function is simpler when expressed in
this  alternative:
\[ \wp^\prime(u \mid g_2,g_3)^2 = 4\,\wp(u \mid g_2,g_3)^3
   - g_2\, \wp(u \mid g_2,g_3) -g_3. \]

\subsubsection{Other Sigma Functions}
\hypertarget{SIGMA1}{}
\hypertarget{operator:WEIERSTRASS_SIGMA1}{}
\hypertarget{operator:WEIERSTRASS_SIGMA2}{}
\hypertarget{operator:WEIERSTRASS_SIGMA3}{}

Three further Sigma functions are also provided:
\index{Sigma functions}
\ttindextype[ELLIPFN]{weierstrass\_sigma1}{operator}\ttindextype[ELLIPFN]{weierstrass\_sigma2}{operator}
\ttindextype[ELLIPFN]{weierstrass\_sigma3}{operator}
\begin{itemize}
\item $\sigma_1(u, \omega_1, \omega_3)$ \ --- \ \f{weierstrass\_sigma1(u,omega1,omega3)}
\item $\sigma_2(u, \omega_1, \omega_3)$ \ --- \ \f{weierstrass\_sigma2(u,omega1,omega3)}
\item $\sigma_3(u, \omega_1, \omega_3)$ \ --- \ \f{weierstrass\_sigma3(u,omega1,omega3)}
\end{itemize}
These are all even functions, regular everywhere, homogenous of degree zero and doubly quasi-periodic. They are closely related to the
theta functions $\vartheta_2$, $\vartheta_3$ and $\vartheta_4$ respectively; but \emph{note the difference in numbering}.
For more information on the properties these sigma functions, see Lawden \cite{Lawden:89};
they do not appear in the NIST Digital Library of Mathematical Functions, but are included here for completeness.

\subsubsection{ Quasi-Period Factors \& Lattice Functions}
\hypertarget{ETA}{}
\hypertarget{operator:LATTICE_E1}{}
\hypertarget{operator:LATTICE_E2}{}
\hypertarget{operator:LATTICE_E3}{}
\hypertarget{operator:LATTICE_G}{}
\hypertarget{operator:LATTICE_DELTA}{}
\hypertarget{operator:LATTICE_G2}{}
\hypertarget{operator:LATTICE_G3}{}
\ttindextype[ELLIPFN]{lattice\_e1}{operator}\ttindextype[ELLIPFN]{lattice\_e2}{operator}
\ttindextype[ELLIPFN]{lattice\_e3}{operator}\ttindextype[ELLIPFN]{lattice\_g}{operator}
\ttindextype[ELLIPFN]{lattice\_g2}{operator}\ttindextype[ELLIPFN]{lattice\_g3}{operator}\ttindextype[ELLIPFN]{lattice\_delta}{operator}
\index{Lattice roots}\index{Lattice invariants}\index{Quasi-period factors}

Ten functions are provided:
\begin{itemize}
\item $e_1(\omega_1, \omega_3)$ \ --- \ \f{lattice\_e1(omega1, omega3)};
\item $e_2(\omega_1, \omega_3)$ \ --- \ \f{lattice\_e2(omega1, omega3)};
\item $e_3(\omega_1, \omega_3)$ \ --- \ \f{lattice\_e3(omega1, omega3)};
\item $g_2(\omega_1, \omega_3)$ \ --- \ \f{lattice\_g2(omega1, omega3)};
\item $g_3(\omega_1, \omega_3)$ \ --- \ \f{lattice\_g3(omega1, omega3)};
\item $\Delta(\omega_1, \omega_3)$ \ --- \ \f{lattice\_delta(omega1, omega3)};
\item $\mathrm{G}(\omega_1, \omega_3)$ \ --- \ \f{lattice\_g(omega1, omega3)};
\item $\eta_1(\omega_1, \omega_3)$ \ --- \ \f{eta\_1(omega1, omega3)};
\item $\eta_2(\omega_1, \omega_3)$ \ --- \ \f{eta\_2(omega1, omega3)};
\item $\eta_3(\omega_1, \omega_3)$ \ --- \ \f{eta\_3(omega1, omega3)}.
\end{itemize}

These are operative when the switches \sw{rounded} and \sw{complex} are ON
and their arguments are numerical. The first three are referred to as lattice
roots and are related to the invariants
$g_2, g_3$, the discriminant $\Delta = g_2^3-27g_3^2$ and a closely related
invariant $\mathrm{G} = g_2^3/(27 g_3^2)$ of the Weierstrassian
elliptic function $\wp$. The lattice roots also appear in the numerical
evaluation of the Weierstrass function. These lattice roots satisfy:
\[e_1+e_2+e_3=0,\qquad g_2=2(e_1^2+e_2^2+e_3^2),\qquad g_3= 4e_1e_2e_3.\]
If the discriminant $\Delta$ vanishes or equivalently if $\mathrm{G} = 1$,
there are at most two distinct lattice roots and the elliptic function
degenerates to an elementary one. The advantage of the invariant
$\mathrm{G}$ is that it is a function of $\tau = \omega_3/\omega_1$ only.

\hypertarget{operator:ETA_1}{}
\hypertarget{operator:ETA_2}{}
\hypertarget{operator:ETA_3}{}
\ttindextype[ELLIPFN]{eta\_1}{operator}\ttindextype[ELLIPFN]{eta\_2}{operator}\ttindextype[ELLIPFN]{eta\_3}{operator}
The remaining three functions \f{eta\_1}, \f{eta\_2} \& \f{eta\_3} appear in
the rules for the quasi-periodicity of the four sigma functions and of the
Weierstrassian Zeta function. They are also used in the numerical
evaluation of these functions when the switches \sw{rounded} and \sw{complex}
are ON. The quasi-period relations are:
\begin{align*}
  \zeta_w(u+2\omega_j) & = \zeta_w(u)+2\eta_j\\
  \sigma(u+2\omega_j) & = -exp(2\eta_j(u+\omega_j))\sigma(u)\\
  \sigma_k(u+2\omega_j) & =  exp(2\eta_j(u+\omega_j))\sigma_k(u) \quad\text{  if  }j\neq k\\
  \sigma_j(u+2\omega_j) & = -exp(2\eta_j(u+\omega_j))\sigma_j(u)\\
  \zeta_w(\omega_j) & = \eta_j\\
  \sigma_j(\omega_j) & = 0,
\end{align*}
where the lattice parameters have been omitted for conciseness and $j,k = 1\ldots 3$.
The quasi-period factors satisfy
\[\eta_1+\eta_2+\eta_3=0,\qquad
   \eta_1\omega_3-\eta_3\omega_1=\eta_2\omega_1-\eta_1\omega_2=\eta_3\omega_2-\eta_2\omega_3=i\pi/2.\]
As well as the scalar-valued functions discussed above in this section,
there are four functions which return a list as their value:
\hypertarget{operator:LATTICE_ROOTS}{}
\hypertarget{operator:LATTICE_INVARIANTS}{}
\hypertarget{operator:LATTICE_GENERATORS}{}
\hypertarget{operator:QUASI_PERIOD_FACTORS}{}
\ttindextype[ELLIPFN]{lattice\_roots}{operator}\ttindextype[ELLIPFN]{lattice\_invariants}{operator}
\ttindextype[ELLIPFN]{lattice\_generators}{operator}\ttindextype[ELLIPFN]{quasi\_period\_factors}{operator}
\begin{itemize}
\item \f{lattice\_roots(omega1, omega3)} --- returns $\{e_1,\ e_2,\ e_3\}$;
\item \f{lattice\_invariants(omega1, omega3)} --- returns
  $\{g_2,\ g_3,\ \Delta,\ \mathrm{G}\}$;
\item \f{quasi\_period\_factors(omega1, omega3)}
  --- returns $\{\eta_1,\ \eta_2,\ \eta_3\}$;
\item \f{lattice\_generators(g2, g3)}  --- returns $\{\omega_1,\ \omega_3\}$.
\end{itemize}
The first three are actually more efficient than calling the requisite
scalar-valued functions individually and the fourth is used in the numerical
evaluation of the Weierstrass functions regarded as functions of the
invariants. These functions are only useful when the switches \sw{rounded} and
\sw{complex} are ON and their arguments are all numerical.
Note that the call sequence:
\begin{verbatim}
  lattice_generators(g2,g3);
  lattice_invariants(first ws, second ws);
  {first ws, second ws};
\end{verbatim}
should reproduce the list \{g2, g3\}, perhaps with small rounding errors. The
corresponding sequence with the calls to \f{lattice\_generators} and
\f{lattice\_invariants} interchanged (and g2 \& g3 replaced by w1 \& w3),
in general, will not produce the same pair of lattice generators since the
generators are only defined up to a unimodular bilinear transformation.

For details of the algorithm used to calculate the lattice generators from the
invariants see the DLMF:NIST chapter on
\href{https://dlmf.nist.gov/23.22#ii}{Lattice Calculations}.

\subsection{Inverse Jacobi Elliptic Functions}
\index{Inverse Jacobi Elliptic functions}
The following inverses of the 12 Jacobi elliptic functions are available:-
\hypertarget{INVELL}{}
\hypertarget{operator:ARCSN}{}
\hypertarget{operator:ARCCN}{}
\hypertarget{operator:ARCDN}{}
\hypertarget{operator:ARCCD}{}
\hypertarget{operator:ARCSD}{}
\hypertarget{operator:ARCND}{}
\hypertarget{operator:ARCDC}{}
\hypertarget{operator:ARCNC}{}
\hypertarget{operator:ARCSC}{}
\hypertarget{operator:ARCNS}{}
\hypertarget{operator:ARCDS}{}
\hypertarget{operator:ARCCS}{}
\ttindextype[ELLIPFN]{arcsn}{operator}\ttindextype[ELLIPFN]{arccn}{operator}
\ttindextype[ELLIPFN]{arcdn}{operator}\ttindextype[ELLIPFN]{arccd}{operator}
\ttindextype[ELLIPFN]{arcsd}{operator}\ttindextype[ELLIPFN]{arcnd}{operator}
\ttindextype[ELLIPFN]{arcdc}{operator}\ttindextype[ELLIPFN]{arcnc}{operator}
\ttindextype[ELLIPFN]{arcsc}{operator}\ttindextype[ELLIPFN]{arcns}{operator}
\ttindextype[ELLIPFN]{arcds}{operator}\ttindextype[ELLIPFN]{arccs}{operator}
\begin{itemize}
\item arcsn(u,k)
\item arcdn(u,k)
\item arccn(u,k)
\item arccd(u,k)
\item arcsd(u,k)
\item arcnd(u,k)
\item arcdc(u,k)
\item arcnc(u,k)
\item arcsc(u,k)
\item arcns(u,k)
\item arcds(u,k)
\item arccs(u,k)
\end{itemize}

Thus, for example,
\begin{verbatim}
   jacobisn(arcsn(x, k), k)   --> x
   jacobisc(arcsc(x, k), k)   --> x
\end{verbatim}

A rule list is provided to simplify these functions for special values of their
arguments such $x=0$, $k=0$ and $k=1$, to implement the inverse function
simplification formulae illustrated immediately above and for differentiation
of these functions with respect to their two arguments.

Note that for simplicity $ \mathrm{arccs}$ is now defined to be an odd
function of its first argument like $\mathrm{cs}$.
This choice means that the range of (real) principal values is
\emph{no longer connected} as it is the open set
$(-\mathrm{K}(k), \mathrm{K}(k)) \setminus 0$.
It brings it into line with $\mathrm{arcns}$ and
$\mathrm{arcds}$ whose principal value ranges are necessarily not connected
as they have poles at zero;
similarly for the range of principal values of $\mathrm{arcdc}$ which has
a pole at $\mathrm{K}(k)$.
  
In earlier versions of the package it was taken to satisfy:
\[ \mathrm{arccs}(-x, k) = 2\mathrm{K}(k)-\mathrm{arccs}(x, k).\]
This was analogous to the situation in Reduce for $\mathrm{acot}$ where
\[ \mathrm{arctan}(-x) = -\mathrm{arctan}(x),\qquad
  \mathrm{arccot}(-x) = \pi -\mathrm{arccot}(x). \]

When their arguments are \emph{numerical and real}, these functions will be
evaluated numerically if the \sw{rounded} switch is ON provided, of course that
the defining elliptic integral is convergent. Note that in some cases the result
may be imaginary even if both arguments are real. In these cases if further computation
is required involving the result, it is better if the switch \sw{complex} is also ON

Note also that for $\mathrm{arcdn}$ and $\mathrm{arcnd}$ a zero value of the modulus $k$
is excluded (since $\mathrm{dn}(x,0) = \mathrm{nd}(x,0) = 1 \quad \forall x$).

As the Jacobi elliptic functions are doubly periodic, the inverse functions
are multi-valued. When both arguments are real and $|k|<=1$ and when the
result exists and is real there are certain restrictions on the range of
acceptable values of the first parameter $x$ (see the table below).

The numerical value returned is the principal value which
lies in the range  given in the third column of the table below. 
(c.f. the inverse trigonometric functions). Other \emph{real} values of the
inverse functions are indicated in the fourth column of the table below where
$n$ is an arbitrary integer.

\begin{tabular}{llll}
  Fn & Domain & Principal Value $v$ & Other real values\\
$\mathop{\mathrm{arcsn}}$: & $ |x| <=1 $ &
  $-\mathrm{K}(k) <= v <= \mathrm{K}(k)$ &
  $2 n\mathrm{K}(k)+(-1)^nv$ \\
$\mathop{\mathrm{arccn}}$: &  $ |x| <=1 $ &
  $0 <= v <= 2\mathrm{K}(k)$ & 
  $4 n\mathrm{K}(k) \pm v$ \\
$\mathop{\mathrm{arccd}}$: & $ |x| <=1 $ &
  $0 <= v <= 2\mathrm{K}(k)$ &
  $4 n\mathrm{K}(k) \pm v$ \\
$\mathop{\mathrm{arcns}}$: & $ |x| >=1 $ &
  $-\mathrm{K}(k) <= v <= \mathrm{K}(k)$ \& $v \neq 0$ &
  $2 n\mathrm{K}(k)+(-1)^nv$ \\
$\mathop{\mathrm{arcnc}}$: & $ |x| >=1 $ &
  $0 <= v <= 2\mathrm{K}(k)$ \& $v \neq \mathrm{K}(k)$  &
  $4 n\mathrm{K}(k) \pm v$ \\
$\mathop{\mathrm{arcdc}}$: & $ |x| >=1 $ &
  $0 <= v <= 2\mathrm{K}(k)$ \& $v \neq \mathrm{K}(k)$ &
  $4 n\mathrm{K}(k) \pm v$ \\
$\mathop{\mathrm{arcdn}}$: & $ k' <= x <= 1$ &
  $0 <= v <= \mathrm{K}(k)$ &
  $2 n\mathrm{K}(k) \pm v$ \\  
$\mathop{\mathrm{arcnd}}$: & $ 1 <= x <= 1/k'$ &
  $0 <= v <= \mathrm{K}(k)$ &
  $2 n\mathrm{K}(k) \pm v$ \\   
$\mathop{\mathrm{arcds}}$: & $ |x| >= k'$ &
  $-\mathrm{K}(k) <= v <= \mathrm{K}(k)$ \& $v \neq 0$ &
  $2 n\mathrm{K}(k)+(-1)^nv $ \\  
$\mathop{\mathrm{arcsd}}$: & $  |x| <= 1/k'$ &
  $-\mathrm{K}(k) <= v <= \mathrm{K}(k)$ &
  $2 n\mathrm{K}(k)+(-1)^nv $ \\
$\mathop{\mathrm{arcsc}}$: & $ x \in \mathbb{R}$ &
  $-\mathrm{K}(k) < v < \mathrm{K}(k)$ &
  $2 n\mathrm{K}(k) + v$ \\  
$\mathop{\mathrm{arccs}}$: & $ x \in \mathbb{R}$ &
$-\mathrm{K}(k) < v < \mathrm{K}(k)$ \& $v \neq 0$ 
  $2 n\mathrm{K}(k) + v$ \\
\end{tabular}

If one or both arguments are complex, the current version of the package
attempts to evaluate the result numerically \emph{provided that} the switch
\sw{complex} is also ON. This code is still under development and whilst
it succeeds in many cases, it may fail with a \emph{divergent integral error
message}. If the function in question has a branch point or pole at the
evaluation point, this is of course correct, but in other cases the value may
exist, but the current choice of symmetric integral in the algorithm may
diverge. This can occur when the first parameter is a subrange of the real or
imaginary axis.

The value currently returned may not always be the principal value
(ie. the value in the fundamental period parallelogram of the elliptic
function concerned). However, the value returned should be \emph
{an inverse value}

(ie. $\mathrm{jacobipq}(\mathrm{arcpq}(z,k),k)) = z$ where
$\mathrm{pq} = \mathrm{sn, cn, dn, cs} \ldots $).

The numerical values of the inverse functions are calculated by
expressing them in terms of the symmetric elliptic integral:
\[ R_F(x,y,z)=\int_0^\infty 1/\sqrt{(t-x)(t-y)(t-z)}\,\mathrm{d}t. \]
This is then evaluated using a sequence of quadratic transformations 
which converge rapidly to the elementary hyperbolic integral:
\[R_c(X^2+Y^2,X^2) = R_F(X^2,X^2,X^2+Y^2) = \mathrm{arctanh}
(Y/(X^2+Y^2))/Y.\]
More information on this method due to Carlson in the 1990's
may be found on the DLMF website:
\href{https://dlmf.nist.gov/19.36#ii}{Quadratic Transformations}
and \href{https://dlmf.nist.gov/19.25#v}{Inverse Jacobian
Elliptic Functions}.

\subsection{Table of Elliptic Functions and Integrals}
\hypertarget{ELLIPFNTAB}{}

\fbox{
\begin{tabular}{r l}
\rule{0pt}{16pt}Function & Operator\\[5pt]
$\mathrm{am}(u,k)$ & \f{jacobiam(u,k)}\\
$\mathrm{sn}(u,k)$ & \f{jacobisn(u,k)}\\
$\mathrm{dn}(u,k)$ & \f{jacobidn(u,k)}\\
$\mathrm{cn}(u,k)$ & \f{jacobicn(u,k)}\\
$\mathrm{cd}(u,k)$ & \f{jacobicd(u,k)}\\
$\mathrm{sd}(u,k)$ & \f{jacobisd(u,k)}\\
$\mathrm{nd}(u,k)$ & \f{jacobind(u,k)}\\
$\mathrm{dc}(u,k)$ & \f{jacobidc(u,k)}\\
$\mathrm{nc}(u,k)$ & \f{jacobinc(u,k)}\\
$\mathrm{sc}(u,k)$ & \f{jacobisc(u,k)}\\
$\mathrm{ns}(u,k)$ & \f{jacobins(u,k)}\\
$\mathrm{ds}(u,k)$ & \f{jacobids(u,k)}\\
$\mathrm{cs}(u,k)$ & \f{jacobics(u,k)}\\
Inverse Functions of the above:\\
$\mathrm{arcsn}(u,k)$ & \f{arcsn(u,k)}\\
$\mathrm{arccn}(u,k)$ & \f{arccn(u,k)}\\
   ... \\
$\mathrm{arccs}(u,k)$ & \f{arccs(u,k)}\\

Complete Integral (1st kind) $\mathrm{K}(k)$ & \f{ellipticK(k)}\\
$\mathrm{K}^\prime(k)$ & \f{ellipticK!'(k)}\\
Incomplete Integral (1st kind) $\mathrm{F}(\phi,k)$ & \f{ellipticF(phi,k)}\\
Complete Integral (2nd kind) $\mathrm{E}(k)$ & \f{ellipticE(k)}\\
$\mathrm{E}^\prime(k)$ & \f{ellipticE!'(k)}\\
Legendre Incomplete Int (2nd kind) $\mathrm{E}(u,k)$ & \f{ellipticE(u,k)}\\
Jacobi Incomplete Int (2nd kind) $\mathcal{E}(u,k)$ & \f{jacobiE(u,k)}\\
Jacobi's Zeta $\mathrm{Z}(u,k)$ & \f{jacobiZeta(u,k)}\\
$\vartheta_1(u,\tau)$ & \f{elliptictheta1(u,tau)}\\
$\vartheta_2(u,\tau)$ & \f{elliptictheta2(u,tau)}\\
$\vartheta_3(u,\tau)$ & \f{elliptictheta3(u,tau)}\\
$\vartheta_4(u,\tau)$ & \f{elliptictheta4(u,tau)}\\
$\wp(u,\omega_1, \omega_3)$ & \f{weierstrass(u,omega1,omega3)}\\
$\zeta_w(u,\omega_1, \omega_3)$ & \f{weierstrassZeta(u,omega1,omega3)}\\
$\sigma(u,\omega_1, \omega_3)$ & \f{weierstrass\_sigma(u,omega1,omega3)}\\
$\sigma_1(u,\omega_1, \omega_3)$ & \f{weierstrass\_sigma1(u,omega1,omega3)}\\
$\sigma_2(u,\omega_1, \omega_3)$ & \f{weierstrass\_sigma2(u,omega1,omega3)}\\
$\sigma_3(u,\omega_1, \omega_3)$ & \f{weierstrass\_sigma3(u,omega1,omega3)}\\
$\wp(u \mid g_2, g_3)$ & \f{weierstrass1(u,g2,g3)}\\
$\zeta_w(u \mid g_2, g_3)$ & \f{weierstrassZeta1(u,g2,g3)}\\[5pt]
\end{tabular}}
