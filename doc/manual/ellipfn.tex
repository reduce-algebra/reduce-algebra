
\subsection{Elliptic Functions: Introduction}
\index{Elliptic functions}
\index{Jacobi Elliptic functions}
\index{Elliptic Integrals}
\index{Nome and Related functions}
\index{Jacobi Theta functions}
\index{Theta function derivatives}
\index{Weierstrass Elliptic functions}
\index{Sigma functions}
\index{Inverse Elliptic functions}
\hypertarget{ELLIPFNS}{}
%% For MathJax, must be part of a paragraph to avoid extra space:
\ifdefined\VerbMath{\makehashletter
\(\newcommand{\f}[1]{\texttt{#1}}\)%
}\fi
%
The package ELLIPFN is designed to provide algebraic and numeric manipulations of
many elliptic functions, namely:

\begin{itemize}
\item \hyperlink{JACEF}{Jacobi's Elliptic Functions};
\item \hyperlink{ELLIPI}{Elliptic Integrals};
\item \hyperlink{ELLIPNOME}{Nome and Related Functions};
\item \hyperlink{JACTF}{Jacobi's Theta Functions} and their
\hyperlink{THETAD}{ derivatives};
\item \hyperlink{WEIERSTRASS}{Weierstrass Elliptic Functions} and the
\hyperlink{SIGMA}{Sigma Function};
\item \hyperlink{SIGMA1}{Other Sigma Functions};
\item \hyperlink{ETA}{Period Lattice and Related Functions};
\item \hyperlink{INVELL}{Inverse Elliptic Functions}.
\end{itemize}

The implementation of the functions in this and the next two subsections have
been substantially revised by Alan Barnes in 2019. This is to bring the
notation more into line with standard (British) texts such as Whittaker
\& Watson \cite{WhittakerWatson:69} and Lawden \cite{Lawden:89} and also to correct
a number of errors and omissions. These changes and additions will be itemised in the relevant
sections below.  New subsections has been added starting in 2021 to support  Weierstrassian Elliptic
functions, Sigma functions, inverse Jacobi elliptic functions and finally symmetric elliptic integrals.

The functions in this subsection are for the most part autoloading;
the exceptions being the subsidiary utility functions such as the AGM function,
the quasi-period factors, lattice functions, derivatives of the theta functions and symmetric elliptic integrals.

\hypertarget{JACEF}{}
\subsection{Jacobi Elliptic  Functions}
The following functions have been implemented:
\begin{itemize}
\item The 12 Jacobi Elliptic Functions
\item The Jacobi Amplitude Function
\item Arithmetic Geometric Mean
\item Descending Landen Transformation
\end{itemize}

\index{Jacobi Elliptic functions}
The following Jacobi elliptic functions are available:-
\hypertarget{operator:JACOBISN}{}
\hypertarget{operator:JACOBICN}{}
\hypertarget{operator:JACOBIDN}{}
\hypertarget{operator:JACOBISD}{}
\hypertarget{operator:JACOBIND}{}
\hypertarget{operator:JACOBIDC}{}
\hypertarget{operator:JACOBINC}{}
\hypertarget{operator:jACOBISC}{}
\hypertarget{operator:JACOBINS}{}
\hypertarget{operator:JACOBIDS}{}
\hypertarget{operator:JACOBICS}{}
\ttindextype[ELLIPFN]{jacobisn}{operator}\ttindextype[ELLIPFN]{jacobicn}{operator}
\ttindextype[ELLIPFN]{jacobidn}{operator}\ttindextype[ELLIPFN]{jacobicd}{operator}
\ttindextype[ELLIPFN]{jacobisd}{operator}\ttindextype[ELLIPFN]{jacobind}{operator}
\ttindextype[ELLIPFN]{jacobidc}{operator}\ttindextype[ELLIPFN]{jacobinc}{operator}
\ttindextype[ELLIPFN]{jacobisc}{operator}\ttindextype[ELLIPFN]{jacobins}{operator}
\ttindextype[ELLIPFN]{jacobids}{operator}\ttindextype[ELLIPFN]{jacobics}{operator}
\begin{itemize}
\item jacobisn(u,k)
\item jacobidn(u,k)
\item jacobicn(u,k)
\item jacobicd(u,k)
\item jacobisd(u,k)
\item jacobind(u,k)
\item jacobidc(u,k)
\item jacobinc(u,k)
\item jacobisc(u,k)
\item jacobins(u,k)
\item jacobids(u,k)
\item jacobics(u,k)
\end{itemize}

These differ somewhat from the originals implemented by Lisa Temme in that
the second argument is now the modulus (usually denoted by $k$ in most texts
rather than its square $m$).  The notation for the most part follows  Lawden
\cite{Lawden:89}. The last nine Jacobi functions are related to the three
basic ones: \f{jacobisn(u,k)}, \f{jacobicn(u,k)} and \f{jacobidn(u,k)} and
use Glaisher's notation. For example
\[ \mathrm{ns}(x,k) = \frac{1}{\mathrm{sn}(u,k)}, \qquad
\mathrm{cs}(x,k) = \frac{\mathrm{cn}(u,k)}{\mathrm{sn}(u,k)}, \qquad
\mathrm{cd}(x,k) = \frac{\mathrm{cn}(u,k)}{\mathrm{dn}(u,k)}. \]

All twelve functions are doubly periodic in the complex plane.
The primitive periods and the positions of the zeros and poles of the
functions are conveniently expressed in terms of the so-called quarter periods
$\mathrm{K}(k)$ and $i\mathrm{K}'(k)$ which are complete elliptic
integrals of the first kind. The details are displayed in the table below;
in all cases the pole is single and the corresponding residue is given in
the last column of the table.
As the functions are doubly-periodic, there are, of course, infinitely many
other zeros and poles. These occur at points `congruent' to the points
given in the table obtained by translating by
$2m\mathrm{K} +2i n\mathrm{K}'$ where $m$ and $n$ are arbitrary integers.

\begin{tabular}{llllll}
  Function & Period1 & Period2 & Zero & Pole & Residue\\
$\mathop{\mathrm{sn}}$&$4\mathrm{K}$&$2i\mathrm{K}'$&$0$&$i\mathrm{K}'$&$1/k$\\
$\mathop{\mathrm{ns}}$&$4\mathrm{K}$&$2i\mathrm{K}'$&$i\mathrm{K}'$&$0$&$1$\\
$\mathop{\mathrm{cd}}$&$4\mathrm{K}$&$2i\mathrm{K}'$&$\mathrm{K}$&$\mathrm{K}+i\mathrm{K}'$&$-1/k$\\  
$\mathop{\mathrm{dc}}$&$4\mathrm{K}$&$2i\mathrm{K}'$&$\mathrm{K}+i\mathrm{K}'$&$\mathrm{K}$&$-1$\\ 

$\mathop{\mathrm{cn}}$&$4\mathrm{K}$&$2(\mathrm{K}+i\mathrm{K}')$&$\mathrm{K}$&$i\mathrm{K}'$&$-i/k$\\
$\mathop{\mathrm{nc}}$&$4\mathrm{K}$&$2(\mathrm{K}+i\mathrm{K}')$&$i\mathrm{K}'$&$\mathrm{K}$&$-1/k'$\\
$\mathop{\mathrm{sd}}$&$4\mathrm{K}$&$2(\mathrm{K}+i\mathrm{K}')$&$0$&$\mathrm{K}+i\mathrm{K}'$&$-i/(k k')$\\
$\mathop{\mathrm{ds}}$&$4\mathrm{K}$&$2(\mathrm{K}+i\mathrm{K}')$&$\mathrm{K}+i\mathrm{K}'$&$0$&$1$\\

$\mathop{\mathrm{dn}}$&$4i\mathrm{K}'$&$2\mathrm{K}$&$\mathrm{K}+i\mathrm{K}'$&$i\mathrm{K}'$&$-i$\\
$\mathop{\mathrm{nd}}$&$4i\mathrm{K}'$&$2\mathrm{K}$&$i\mathrm{K}'$&$\mathrm{K}+i\mathrm{K}'$&$-i/k'$\\
$\mathop{\mathrm{sc}}$&$4i\mathrm{K}'$&$2\mathrm{K}$&$0$&$\mathrm{K}$&$-1/k'$\\
$\mathop{\mathrm{cs}}$&$4i\mathrm{K}'$&$2\mathrm{K}$&$\mathrm{K}$&$0$&$1$\\
\end{tabular}

All other periods of the Jacobi functions can be expressed as linear
combinations of the two primitive periods with integer coefficients.
Thus, for example, $4i\mathrm{K}'$ is a period of $\mathrm{cn}$ as
\[4i\mathrm{K}' = 2(\mathrm{K}+\mathrm{K}') -4\mathrm{K}\]

Extensive rule lists are provided for differentiation of these functions with
respect to either argument, for argument shifts by integer multiples of the
two quarter periods $\mathrm{K}(k)$ and $i\mathrm{K}'(k)$ and finally
Jacobi's transformations for a purely imaginary first argument.

Rules are also provided for the values of the twelve Jacobi functions at the
`eighth' period values: $\mathrm{K}/2$, $i\mathrm{K}'/2$ and
$(\mathrm{K}+i\mathrm{K}')/2$. For these rules to yield correct values it
is essential that the switch \sw{precise\_complex} is ON, otherwise Reduce will
often incorrectly simplify the results if they involve complex values of $k$ or
$k'$. It should also be noted that the rule for
$\mathrm{dn}((\mathrm{K}(k)+i\mathrm{K}'(k))/2, k)$ differs from that on
\href{https://dlmf.nist.gov/22.5#i}{DLMF:NIST} which appears to give incorrect
values for some values of $k$. The rule used in \REDUCE is not only simpler,
but appears to give correct values in all cases as do the derived rules for
$\mathrm{cd}$, $\mathrm{sd}$, $\mathrm{nd}$, $\mathrm{ds}$ and $\mathrm{dc}$.
It is derived from the third of the identities (2.2.19) of Lawden
\cite{Lawden:89} with $u =-(\mathrm{K}(k)+i\mathrm{K}'(k))/2$ and
the corresponding identities for $\mathrm{sn}$ and $\mathrm{cn}$ on the
DLMF NIST site.

Some care must be exercised when applying the `eighth' period rules when
the modulus $k$ belongs to the negative real or imaginary axes as
$\mathrm{K}(k)$ and all twelve Jacobi functions $\mathrm{sn}(x,k)$ etc. are
even functions of the modulus $k$. The rule should be applied when the modulus
has no assigned value and then resimplifying the result after assigning the
modulus its required value.

Complex split functions have recently been implemented for the 12 Jacobi
functions for the cases when the modulus $k$ is either real or imaginary.
Thus \texttt{REPART} and \texttt{IMPART} will now return rational expressions
involving the twelve Jacobi functions with real arguments and a real modulus
$|k|<1$. As far as I (AB) am aware there are no known methods of implementing
split functions for general complex values of the modulus.

\hypertarget{reserved:TO_SN}{}
\hypertarget{reserved:TO_CN}{}
\hypertarget{reserved:TO_DN}{}
\hypertarget{reserved:NO_GLAISHER}{}
\hypertarget{reserved:JacobiAdditionRules}{}
\ttindextype[ELLIPFN]{to\_sn}{rule list}
\ttindextype[ELLIPFN]{to\_cn}{rule list}
\ttindextype[ELLIPFN]{to\_dn}{rule list}
\ttindextype[ELLIPFN]{no\_glaisher}{rule list}
\ttindextype[ELLIPFN]{jacobiadditionrules}{rule list}
Four useful rule lists \texttt{TO\_SN}, \texttt{TO\_CN}, \texttt{TO\_DN} and
\texttt{NO\_GLAISHER} are provided for user convenience. The first
\texttt{TO\_SN} replaces occurences of the squares of $\mathrm{cn}$ and
$\mathrm{dn}$ in favour of the square of $\mathrm{sn}$ using the
`Pythagorean' identities. \texttt{TO\_CN} and \texttt{TO\_DN} apply these
identities to replace squares in favour of $\mathrm{cn}$ and $\mathrm{dn}$
respectively. At most one of these rule sets should be active at any one time
to avoid a potential infinite recursion in the simplification.
The fourth rule list \texttt{NO\_GLAISHER} replaces all occurences of the
nine subsidiary Jacobi elliptic functions $\mathrm{ns}$, $\mathrm{sc}$,
$\mathrm{sd}$, \ldots by reciprocals and quotients of the `basic' Jacobi
elliptic functions $\mathrm{sn}$, $\mathrm{cn}$ and $\mathrm{dn}$.

A fifth rule list \texttt{JACOBIADDITIONRULES} applies addition rules when the
first argument of any Jacobi function is a sum.  Note that this rule list is NO
LONGER active by default. There are many equivalent ways of expressing the
right-hand sides of these rules. Previously all the rules were expressed using
only the `basic' Jacobi functions: $\mathrm{sn}$, $\mathrm{cn}$ and
$\mathrm{dn}$. Currently, however, the right-hand sides of the rules for the
reciprocal and quotient Glaisher functions $\mathrm{ns}$, $\mathrm{cs}$,
$\mathrm{sd}$ etc are expressed using the three Glaisher functions with
the same `denominator' as on the left-hand side. Thus the rule for $\mathrm{ns}$
is expressed in terms of $\mathrm{ns}$, $\mathrm{cs}$ and $\mathrm{ds}$
whilst that for $\mathrm{cd}$ is expressed in terms of $\mathrm{nd}$,
$\mathrm{cd}$ and $\mathrm{sd}$ whereas the rules for the three `basic'
functions are unchanged and use only $\mathrm{sn}$, $\mathrm{cn}$ and
$\mathrm{dn}$.  Note, however, that the previous form of these rules will be
obtained if the rule-list \texttt{NO\_GLAISHER} is also active. 

If the switch \sw{rounded} is ON and both arguments of a Jacobi function are
purely numerical, it will be evaluated numerically (previously it was
necessary for both the switches \sw{rounded} and \sw{complex} to be ON).
Of course, if either argument is a complex number, the switch \sw{complex} must
also be ON for numerical evaluation to occur.

The numerical evaluation of the Jacobi functions uses their definitions
in terms of theta functions which are valid for all complex values of the
argument and modulus.  Since theta functions are inherently complex-valued,
it is necessary to turn the switch \sw{complex} ON during the evaluation even
if both the argument and the modulus of the Jacobi function are real. The
switch \sw{complex} is however returned to its old value as the computation
completes.  One consequence of the use of theta functions is that occasionally
the value of a Jacobi function with real arguments is returned with an
insignificantly small imaginary part due to rounding errors. Since all Jacobi
functions are real-valued for real arguments the imaginary part may be ignored.
Similarly in some case where the first argument is purely imaginary and the
modulus k<1, the result should be purely imaginary, but due to rounding the
result contains an insignificantly small real part which may be safely ignored.

The traditional AGM and  $\phi$-function based algorithm which was previously
used when the argument and modulus are both real and $|k| < 1$ as been
superceded. It appears that there was a problem with the numerical evaluation
of \texttt{dn, nd, sd, ds, cd, dc} resulting in largish rounding errors. There
was no problem with previous evaluation of \texttt{sn, ns, cn, nc, sc, cs} and
these produced the same results (modulo acceptable rounding errors) as the
theta function based method now used.

\subsubsection{Jacobi Amplitude Function}
\hypertarget{operator:JACOBIAM}{}
\ttindextype[ELLIPFN]{jacobiam}{operator}
This function is defined as
\[ \mathrm{am}(u,k) = \int_0^u\mathrm{dn}(z,k)\mathrm{d}z.\]

As $\mathrm{dn}(z,k)$ has poles with residue $-i$ at points
$z=(4m+1)i\mathrm{K}'(k)+2n\mathrm{K}(k)$ and others with residue $+i$ when
$z=(4m+3)i\mathrm{K}'(k)+2n\mathrm{K}(k)$ where $m$ and $n$ are arbitrary
integers, $\mathrm{am}(z,k)$ has logarithmic singularities at these points.
In fact the amplitude function is multivalued with its principal value $v$, say,
being given by choosing the contour in the defining integral to be the
straight line segment between $0$ and $u$.  Other values are given by $v+2n\pi$
where $n$ is an arbitrary integer and depend on how the chosen contour winds
around the logarithmic branch points.  To obtain a single valued function it is
necessary to introduce branch cuts between the points $i\mathrm{K}'(k)$ and
$3i\mathrm{K}'(k)$ and between corresponding congruent branch points.

A rule list is provided for to simplify this function for special values of the
arguments and for differentiation with respect to either argument.
When the switches \sw{ROUNDED} and \sw{COMPLEX} are both ON and both arguments
are numeric, \REDUCE will attempt to evaluate \f{jacobiam(z,k)} numerically.
Several possible methods are available to
evalaute the function, for example summation of its Fourier series and an
AGM-based method due to Sala (see \href{https://dlmf.nist.gov/22.20#vi}
{DLMF:NIST} for more details.
Currently the AGM-based method is used and this certainly produces reliable
results if the modulus $k$ is real and $|\Im(z)| < \mathrm{K}'(k)$. These
agree with those produced by the Fourier series method although rounding errors
become significant as $|\Im(z)|$ approaches $\mathrm{K}'(k)$.  If $k$ is complex
both methods produce the same results when the imaginary parts of $k$ and $z$
are not too large, but it is difficult to give precise bounds on the size of
these.

For general values of $z$ and $k$ the AGM-based method always produces a value
except perhaps at the logarithmic branch points but these, I believe, are
not correct as they fail to satisfy the identity
\[\mathrm{sn}(z,k) = \sin(\mathrm{am}(z,k)) \]
where $\mathrm{sn}$ is calculated using the theta function method. The Fourier
series method fails to converge in these cases and so is not used except for
comparison purposes.  Currently an alternative method of evaluation due to
Sala is being investigated; it uses the Poisson summation formula and is claimed
to be valid for all $z$ and $k$ except at branch points.

\subsection{Some Numerical Procedures}
This section briefly describes several procedures which are primarily
intended for use in the numerical evaluation of the various elliptic
functions and integrals rather than for direct use by users.

\subsubsection{Arithmetic Geometric Mean (AGM)}
\hypertarget{operator:AGM_FUNCTION}{}
\ttindextype[ELLIPFN]{AGM\_function}{operator}
A procedure to evaluate the AGM of initial values \(a_0,b_0,c_0\)
exists as \\
\texttt{AGM\_function(\(a_0,b_0,c_0\))} and will return \\
$\{ N, AGM, \{ a_N, \ldots ,a_0\}, \{ b_N, \ldots ,b_0\},
\{c_N, \ldots ,c_0\}\}$,
where $N$ is the number of steps to compute the AGM to the
desired accuracy.

To determine the Elliptic Integrals K($m$), E($m$) we use initial values
\(a_0 = 1\); \(b_0 = \sqrt{1-k^2}\) ; \(c_0 = k\).

\subsubsection{Descending Landen Transformation}
The procedure to evaluate the Descending Landen Transformation of
$\phi$ and $\alpha$ uses the following equations:
\begin{align*}
 (1+\sin \alpha_{n+1})(1+\cos \alpha_n)=2 &\text{ where } \alpha_{n+1}<\alpha_n, \\
  \tan(\phi_{n+1}-\phi_n)=\cos \alpha_n \tan \phi_n & \text{ where } \phi_{n+1}>\phi_n.
\end{align*}
It can be called using \f{landentrans}($\phi_0$, $\alpha_0$)
and will return \\
$\{\{\phi_0, \ldots ,\phi_n\},\{\alpha_0, \ldots ,\alpha_n\}\}$.

\subsection{Legendre's Elliptic Integrals}
\hypertarget{ELLIPI}{}
\index{Legendre Elliptic Integrals}
The following functions have been implemented:

\begin{itemize}
\item Complete \& Incomplete Elliptic Integrals of the First Kind
\item Complete \& Incomplete Elliptic Integrals of the Second Kind
\item Jacobi's Epsilon \& Zeta Functions
\item Complete \& Incomplete Ellpitic Integrals of the Third Kind
\item Some Utility Functions  
\end{itemize}

These again differ somewhat from the originals implemented by Lisa Temme
as the second argument is now the modulus $k$ rather that its square.
Also in the original implementation  there was some confusion between
Legendre's form and Jacobi's form of the incomplete elliptic integrals of
the second kind; $E(u,k)$ denoted the first in numerical
evaluations and the second in the derivative formulae for the Jacobi
elliptic functions with respect to their second argument.
This confusion was perhaps understandable
as in the literature some authors use the notation $\mathrm{E}(u, k)$ for
the Legendre form and others for Jacobi's form.

To bring the notation more into line with that in the NIST Digital Library of
Mathematical Functions and avoid any possible confusion, $\mathrm{E}(u, k)$ is used for
the Legendre form and $\mathcal{E}(u, k)$ for Jacobi's form.
This differs from the 2019 version of this section which followed Lawden
\cite{Lawden:89} chapter 3, where the notation $\mathrm{D}(\phi, k)$ and
$\mathrm{E}(u, k)$ were used for the Legendre and Jacobi forms respectively.

A number of rule lists have been provided to implement, where appropriate,
derivatives of these functions, addition rules and periodicity and
quasi-periodicity properties and to provide simplifications for special values
of the arguments.

\subsubsection{Elliptic F}
\hypertarget{operator:ELLIPTICF}{}
\ttindextype[ELLIPFN]{EllipticF}{operator}

The Elliptic F function can be used as \f{EllipticF(phi,k)} and
will return the value of the \emph{Incomplete Elliptic Integral of the
First Kind}:
\[\mathrm{F}(\phi, k)=\int_0^\phi(1-k^2 \sin^2 \theta)^{-1/2} \mathrm{d}\theta.\]
This is actually closely related to the inverse Jacobi function
$\mathrm{arcsn}$; in fact for $-\pi/2 <= \Re \phi <=\pi/2$:
\[ \mathrm{F}(\phi, k) = \mathrm{arcsn}(\sin \phi, k) \].
  
\subsubsection{Elliptic K}
\hypertarget{operator:ELLIPTICK}{}
\hypertarget{operator:ELLIPTICK'}{}
\ttindextype[ELLIPFN]{EllipticK}{operator}
\ttindextype[ELLIPFN]{EllipticK'}{operator}

The Elliptic K function can be used as \f{EllipticK(k)} and will
return the value of the \emph{Complete Elliptic Integral of the
First Kind}:
\[\mathrm{K}(k)=\mathrm{F}(\pi/2, k) =\int_0^{\pi/2}(1-k^2 \sin^2 \theta)^{-1/2}\mathrm{d}\theta.\]
This is one of the quarter periods of the Jacobi elliptic
functions and is often used in the calculation of other elliptic functions.
The complementary Elliptic K$'$ function can be used as \f{EllipticK!$'$(k)}.
Note that $i\mathrm{K}'(k)$ is the other quarter period of the Jacobi
elliptic functions.  

\subsubsection{Elliptic E}
\hypertarget{operator:ELLIPTICE}{}
\hypertarget{operator:ELLIPTICE'}{}
\ttindextype[ELLIPFN]{EllipticE}{operator}
\ttindextype[ELLIPFN]{EllipticE'}{operator}

The Elliptic E function comes with either one or two arguments;
used with two arguments as \f{EllipticE(u,k)}
it will return the value of Legendre's form of
the \emph{Incomplete Elliptic Integral of the Second Kind}:
\[\mathrm{E}(\phi, k)=\int_0^\phi \sqrt{1-k^2 \sin^2 \theta} \,\mathrm{d}\theta.\]
 When called with one argument \f{EllipticE(k)} will return the value of the
\emph{Complete Elliptic Integral of the Second Kind}:
\[\mathrm{E}(k)=\mathrm{E}(\pi/2, k) =
\int_0^{\pi/2} \sqrt{1-k^2 \sin^2 \theta} \,\mathrm{d}\theta.\]
The complementary Elliptic E$'$ function can be used as \f{EllipticE!$'$(k)}.

The complete integrals are actually multi-valued; to obtain single valued
functions it is necessary to introduce branch cuts. For $\mathrm{K}(k)$ and
$\mathrm{E}(k)$  these are $(-\infty, -1] \cup [1, +\infty)$.
The functions are continuous if the two components of the branch cut are
approached from the second and fourth quadrants respectively.
For $\mathrm{K}'(k)$ and $\mathrm{E}'(k)$ the branch cut is
$(-\infty, 0]$ with continuity if the cut is approached from the second
quadrant. For more details, see Lawden \cite{Lawden:89} sections 8.12 to 8.14.
The principal values of $\mathrm{K}(k)$ and $\mathrm{E}(k)$ are even functions
of $k$.

The numerical evaluation of the complete integrals is more robust and
 uses symmetric elliptic integrals. They should now work for all complex
values of the modulus and return the principal value of the integral concerned.
Note that for all complex values of $k$, the well-known identities:
\[\mathrm{K}'(k)=\mathrm{K}(\sqrt{1-k^2})\qquad 
\mathrm{E}'(k) = \mathrm{E}(\sqrt{1-k^2}).\]
are not actually valid for the principal values of the functions concerned.
It is necessary that $\Re k >=0$ with in addition if $\Re k=0$, $\Im k>=0$.
One of the consequences of this is that the principal values of
$\mathrm{K}'(k)$ and $\mathrm{E}'(k)$ are not even functions.

General values for the four complete integrals together with the principal value
when $\Re{k} <0$ are given in the table below, where
 $n$ is an arbitrary integer, $k' = \sqrt{1-k^2}$ and the expressions
in the second and third columns refer always to principal values:

\begin{tabular}  {lll}
  Function&General Value&Principal Value when $\Re(k) <0$\\
  &&and when $\Re(k)=0$ and $\Im(k) <0$\\
$\mathrm{K}(k)$& $\mathrm{K}(k)-2in\mathrm{K}(k')$&$\mathrm{K}(-k)$\\
$\mathrm{K}'(k)$& $\mathrm{K}(k')-4in\mathrm{K}(k)$&$\mathrm{K}(k')\mp 2i\mathrm{K}(-k)$\\
$\mathrm{E}(k)$& $\mathrm{E}(k)-2in(\mathrm{K}(k')-\mathrm{E}(k'))$&$\mathrm{E}(-k)$\\
  $\mathrm{E}'(k)$& $\mathrm{E}(k')-4in(\mathrm{K}(k)-\mathrm{E}(k))$&
  $\mathrm{E}(k')\mp 2i(\mathrm{K}(-k)-\mathrm{E}(-k))$\\
\end{tabular}

In the third column the upper and lower alternative signs are taken when
$k$ lies in the second and third quadrants respectively.

One quite subtle point arises: although the twelve Jacobi elliptic functions
and $\mathrm{K}(k)$ are even functions of the modulus $k$, the quarter period
$i\mathrm{K}'(k)$ is not (see the second row of the table above).
If $\Re(k)>0$, then for example
$4\mathrm{K}(k)$ and  $2i\mathrm{K}'(k)=2i\mathrm{K}(k')$ are
primitive periods of $\mathrm{sn}(x,k)$. However, if we use $-k$ as the modulus,
the first primitive period obtained is the same (since $\mathrm{K}(k)$ is an
even function) but the second is different namely:
$2i\mathrm{K}'(-k)=2i\mathrm{K}(k') \pm 4\mathrm{K}(k)$.
This explains why some of the values returned by the `eighth' period rules for
$\mathrm{sn}$ etc.\ are not even functions of k.

\subsubsection{Elliptic D}
\hypertarget{operator:ELLIPTICD}{}
\ttindextype[ELLIPFN]{EllipticD}{operator}

The Elliptic D function also comes with either one or two arguments;
used with two arguments as \f{EllipticD(u,k)}
it will return the value of an alternative form of Legendre's 
\emph{Incomplete Elliptic Integral of the Second Kind}:
\[\mathrm{D}(\phi, k)=\int_0^\phi\frac{\sin^2 \theta}{\sqrt{1-k^2\sin^2\theta}}
\,\mathrm{d}\theta.\]
When called with one argument \f{EllipticD(k)} will return the value of the
\emph{Complete Elliptic Integral of the Second Kind}:
\[\mathrm{D}(k)=\mathrm{D}(\pi/2, k) =
\int_0^{\pi/2} \frac{\sin^2 theta}{\sqrt{1-k^2 \sin^2 \theta}} \,\mathrm{d}\theta.\]
The integrals of the first and second kind are related:
\[\mathrm{E}(k) = \mathrm{K}(k)-k^2\mathrm{D}(k), \qquad
\mathrm{E}(\phi, k) = \mathrm{F}(\phi, k)-k^2\mathrm{D}(\phi, k).\]
% For numerical computations \f{EllipticD(u,k)} is preferred to
% \f{EllipticE(u,k)} as it involves only one invocation of the symmetric
%  elliptic integral $\mathrm{RD}$.

For all the elliptic integrals, rule lists are provided for simplification for
special values of the argument(s), differentiation with respect to either
argument and shift rules for the incomplete integrals so that their first
argument lies in the range  $0 <= \phi <= \pi/2$.

If the arguments of the elliptic integral function are numeric and the switches
\sw{ROUNDED} and \sw{COMPLEX} are both ON, the numerical value will be returned. The
numerical evaluation routines should now succeed for all complex values of the arguments
provided, of course, the corresponding integral exists. The incomplete integrals
$\mathrm{F}(\phi, k)$, $\mathrm{E}(\phi, k)$, $\mathrm{D}(\phi, k)$ and $\Pi(\phi, a, k)$
are all multi-valued as there are branch points in the  defining integrands  when
$\sin(\phi) =\pm 1$ and $k\sin(\phi) = \pm 1$.

For mumerical evaluation when the first argument $\phi$ is complex there are issues that
arise which are absent when $\phi$ is real. One might use the the straight line contour
from $\theta=0$ to $\theta=\phi$, but this is leads to integrals that are difficult to
evaluate. When $|\Re{\phi}| \leq \pi/2$, the change of variable $x=\sin \theta$ is applied
and the value returned uses the straight line contour between $x=0$ and $x=\sin(\phi)$.
This will  produce a different value if the loop formed by the two contours encloses a
branch point. When $|\Re{\phi}| > \pi/2$ the contour used is the straight line segment
along the real axis from 0 to the multiple of $\pi$ nearest to $\phi$ followed by the
straight line segment from between $x=0$ and $x=\sin(\phi)$. When $\phi$ is real the
contours in the $\theta$ and $x$ planes coincide. Other contours between the
two endpoints will yield different values depending on how the contour winds around the
two branch points.

Note also that if the contour contains a branch point, there is sign ambiguity as the
branch point is crossed depending on which branch of the square root in the integrand is
chosen. It seems logical to choose the principal branch. 

For example when $k>1$ \& $1/k < y <1$
\[ \int_0^y \frac{\mathrm{d}x}{\sqrt{1-x^2}\sqrt{1-k^2x^2}}\]
is taken to be
\[\int_0^{1/k} \frac{\mathrm{d}x}{\sqrt{1-x^2}\sqrt{1-k^2x^2}}
-i\int_{1/k}^y \frac{\mathrm{d}x}{\sqrt{1-x^2}\sqrt{k^2x^2-1}}.\]

\subsubsection{Elliptic $\Pi$}
\hypertarget{operator:ELLIPTICPI}{}
\ttindextype[ELLIPFN]{EllipticPi}{operator}
The Elliptic $\Pi$ function can be used as \f{EllipticPi( )} and
will return the value of the \emph{Elliptic Integral of the
Third Kind}.

The Elliptic $\Pi$ function comes with either two or three arguments;
when called with three arguments \f{EllipticPi(phi,a,k)} will return the
value of the \emph{Incomplete Elliptic Integral of the Third Kind}:
\[\Pi(\phi,a,k)=\int_0^{\phi}
\left((1-a^2\sin^2 \theta)\sqrt{1-k^2 \sin^2 \theta}\right)^{-1}
\,\mathrm{d}\theta.\]
For $-\pi/2 \leq \Re{\phi} \leq \pi/2$, the incomplete integral may be written in
the form:
\[\Pi(\phi,a,k)=\int_0^{sin \phi}
\left((1-a^2t^2)\sqrt{(1-t^2)(1-k^2t^2)}\right)^{-1}\,\mathrm{d}t.\]
When called with two arguments as \f{EllipticPi(a,k)}
it will return the value of the
\emph{Complete Elliptic Integral of the Third Kind}:
\begin{eqnarray*}
  \Pi(a, k)= \Pi(\pi/2,a,k) & = &
\int_0^{\pi/2} \left((1-a^2\sin^2 \theta)\sqrt{1-k^2 \sin^2 \theta}\right)^{-1}
\,\mathrm{d}\theta \\
& =& \int_0^1 \left((1-a^2t^2)\sqrt{(1-t^2)(1-k^2t^2)}\right)^{-1}\,\mathrm{d}t.
\end{eqnarray*}

For certain values of $a$ namely $0$, $\pm 1$ and $\pm k$ the integrals reduce
to elliptic integrals of the first and second kinds.

\subsection{Jacobi's Elliptic Integrals}
\index{Jacobi Elliptic Integrals}
\subsubsection{Jacobi E}
\hypertarget{operator:JACOBIE}{}
\ttindextype[ELLIPFN]{JacobiE}{operator}
The Jacobi E function can be used as  \f{jacobiE(u,k)};
it will return the value of Jacobi's form of
the \emph{Incomplete Elliptic Integral of the Second Kind}:
\[\mathcal{E}(u, k)=\int_0^u \mathrm{dn}^2 (v, k) \,\mathrm{d}v.\]

The relationship between the two forms of incomplete elliptic integrals can
be expressed as
\[\mathcal{E}(u, k) = \mathrm{E}(\mathrm{am}(u), k).\]
Note that
\[\mathrm{E}(k)=\mathcal{E}(\mathrm{K}(k), k)
=\int_0^{\mathrm{K}(k)} \mathrm{dn}^2(v, k) \,\mathrm{d}v.\]

On a GUI that supports calligraphic characters (NB.\ this is now the case with the
CSL GUI), there is no problem and it is rendered as $\mathcal{E}(u,k)$
in accordance with NIST usage.
On non-GUI interfaces the Jacobi E function is rendered as E\_j.

\subsubsection{Jacobi's Zeta Function}
\hypertarget{operator:JACOBIZETA}{}
\ttindextype[ELLIPFN]{jacobiZeta}{operator}
\index{Zeta function of Jacobi!\package{Ellipfn} package}

This can be called as \f{jacobiZeta(u,k)} and refers to Jacobi's (elliptic)
Zeta function $\mathrm{Z}(u,k)$ whereas the operator \f{Zeta} will invoke
Riemann's $\zeta$ function. It is closely related to Jacobi's Epsilon function
\f{jacobie}; in fact
\[\mathrm{Z}(u,k) = \mathcal{E}(u,k)-u\mathrm{E}(k)/\mathrm{K}(k).\]

\
\subsection{Symmetric Elliptic Integrals}
\index{Symmetric Elliptic Integrals}
The functions in this section are currently only intended for use for in the
numerical evaluation of the `classical' elliptic integrals discussed above and
in certain other `basic' elliptic integrals. The switches \sw{ROUNDED} and
\sw{COMPLEX} should both be ON.

Symmetric elliptic integrals are a relatively new development in the theory of
elliptic functions mainly due to Carlson and his collaborators; an extensive
bibliography is available on the NIST Digital Library of Mathematical Functions
starting at the link \href{https://dlmf.nist.gov/bib/C#bib449}
{DLMF:NIST, Bibliography}. They have a number of advantages over more
traditional approaches; see for example
\href{https://dlmf.nist.gov/19.15}{Advantages of Symmetry}.

The fundamental symmetric integrals are all integrals over the positive real
line and involve the function $s(t)=\sqrt{t+x}\sqrt{t+y}\sqrt{t+z}$ where
$x,y,z\in \mathbb{C} \setminus(-\infty, 0]$ but, except where otherwise stated, at most
one of $x,y,z$ may be zero.
\ttindextype[ELLIPFN]{RF}{operator}
\ttindextype[ELLIPFN]{RD}{operator}
\ttindextype[ELLIPFN]{RJ}{operator}
\ttindextype[ELLIPFN]{RC}{operator}
\begin{eqnarray*}
\mathrm{RF}(x,y,z) & = & \frac{1}{2}\int_0^\infty \frac{\mathrm{d}t}{s(t)}\\
\mathrm{RG}(x,y,z) & = & \frac{1}{4}\int_0^\infty \frac{1}{s(t)}\left(
     \frac{x}{t+x}+\frac{y}{t+y}+\frac{z}{t+z}\right)t\mathrm{d}t\\
\mathrm{RJ}(x,y,z,p) & = & \frac{3}{2}\int_0^\infty \frac{\mathrm{d}t}{s(t)(t+p)}\\
\mathrm{RD}(x,y,z) & = & \frac{3}{2}\int_0^\infty \frac{\mathrm{d}t}{s(t)(t+z)}
\qquad z \neq 0, \quad x+y \neq 0\\
\mathrm{RC}(x,y) & = & \frac{1}{2}\int_0^\infty\frac{\mathrm{d}t}{\sqrt{t+x)}(t+y)}\qquad y \neq 0
\end{eqnarray*}
The first three integrals defined above are symmetric in $x,y,z$ and are termed
\emph{symmetric elliptic integrals of the first, second and third kinds}
respectively. $\mathrm{RD}$ and $\mathrm{RC}$ are degnerate versions of
$\mathrm{RJ}$ and $\mathrm{RF}$ respectively as
\[\mathrm{RD}(x,y,z) =\mathrm{RJ}(x,y,z,z) \qquad 
\mathrm{RC}(x,y) =\mathrm{RF}(x,y,y) \]
$\mathrm{RD}$ is an elliptic integral of the second kind and is only symmetric
in $x,y$ whilst $\mathrm{RC}$ is an elementary integral, but can be numerically
evaluated efficiently without needing to distinguish between the trigonometric
and hyperbolic cases. For certain special values of $p$ the integral
$\mathrm{RJ}$ of the third kind degenerates into integrals of the first and
second kinds. For numeric evaluations it turns out that the use of $\mathrm{RD}$
is more convenient than that of $\mathrm{RG}$ (which is not currently
implemented in \REDUCE). For more information see
\href{https://dlmf.nist.gov/19#PT3}{\S19.15-19.29} of the DLMF:NIST.

The Legendre elliptic integrals may all be expressed in terms of the symmetric
integrals; the complete integrals satisfy
\begin{eqnarray*}
\mathrm{K}(k) &=& \mathrm{RF}(0,k^{\prime 2},1) \\
\mathrm{D}(k) &=& \mathrm{RD}(0,k^{\prime 2},1)/3 \\
\mathrm{E}(k) &=& \mathrm{RF}(0,k^{\prime 2},1)-k^2\mathrm{RD}(0,k^{\prime 2},1)/3\\
\Pi(a,k) &=&\mathrm{RF}(0,k^{\prime 2},1)+a^2\mathrm{RJ}(0,k^{\prime 2},1,1-a^2)/3
\end{eqnarray*}
where here and below $k^\prime=\sqrt{1-k^2}$. For the incomplete integrals,
defining $s=\sin \phi$ with $-\pi/2 \leq \Re{\phi} \leq \pi/2$, we have
\begin{eqnarray*}
\mathrm{F}(\phi,k) &=& s \mathrm{RF}(1-s^2, 1-k^2s^2, 1)\\
\mathrm{D}(\phi,k) &=& s^3 \mathrm{RD}(1-s^2, 1-k^2s^2, 1)/3\\
\mathrm{E}(\phi,k) &=&\mathrm{F}(\phi,k)-k^2s^3\mathrm{RD}(1-s^2,1-k^2s^2,1)/3\\
\Pi(\phi,a,k) &=&\mathrm{F}(\phi,k) +a^2s^3\mathrm{RJ}(1-s^2,1-k^2s^2,1,1-a^2s^2)/3
\end{eqnarray*}
The above formulae are only valid when the range of integration does not
include one or more branch points of the integrand (when $s$ is real and
$s^2>1$ or when $k s$ is real and $(k s)^2 >1$).  In these cases it is necessary
to split the range of integration at the branch points resulting in two or three
elliptic integrals.

Currently the Carlson's duplication method is used to evaluate the
symmetric elliptic integrals of the first, second and third kinds
$\mathrm{RF}$, $\mathrm{RD}$ and $\mathrm{RJ}$ (and also the related
elementary integral $\mathrm{RC}$).  For more details see the DLMF website:
\href{https://dlmf.nist.gov/19.36#i}{Duplication Method}.
In particular the \REDUCE code  is essentially a translation from
Fortran of the code of Carlson \& Notis \cite{CarlsonNotis:81} generalised
for complex arguments and structured to avoid GO TO.

Alternatively the symmetric integral $\mathrm{RF}$ may be evaluated
using a sequence of quadratic transformations 
which converge rapidly to the elementary hyperbolic integral:
\[\mathrm{RC}(X^2+Y^2,X^2) = \mathrm{RF}(X^2,X^2,X^2+Y^2) = \mathrm{arctanh}
(Y/(X^2+Y^2))/Y.\]
More information on this method due to Carlson in the 1990's
may be found on the DLMF website:
\href{https://dlmf.nist.gov/19.36#ii}{Quadratic Transformations}.

\subsubsection{Basic Elliptic Integrals}
\index{Basic Elliptic Integrals}
Elliptic integrals involving the square root $s(t)$ of a cubic or quartic
polynomial in which the range of integration is an arbitrary interval $[y,x]$
of the real line are considered. Again there are three basic kinds: first,
second and third. Carlson \cite{Carlson:88} \& \cite{Carlson:99} showed how each
of these can be expressed as a fundamental symmetric elliptic integral of the
corresponding kind over the range $[0,\infty)$. Let
\begin{eqnarray*}
X_\alpha  = & \sqrt{a_\alpha+b_\alpha x} \quad & 1 \leq \alpha \leq 5\\
Y_\beta  = & \sqrt{a_\beta+b_\beta y} \quad & 1 \leq \beta \leq 5\\
d_{\alpha\beta}  = & a_\alpha b_\beta - a_\beta b_\alpha \quad & d_{\alpha\beta} \neq 0\mbox{ if } \alpha \neq \beta\\
s(t)  = &  \prod^4_{\alpha=1} \sqrt{a_\alpha+b_\alpha t}
\end{eqnarray*}
and where the four line segments with end points $a_\alpha+b_\alpha y$ and
$a_\alpha+b_\alpha x$ for $1\leq\alpha\leq 4$ lie in $ \mathbb{C} \setminus (-\infty, 0)$.
Note that if $s(t)$ is a cubic then one simply chooses supplies unity
i.e. $a_\alpha=1, b_\alpha=0$ as a fourth factor of $s(t)$.

Then integrals of the first and second kinds take the form:
\ttindextype[ELLIPFN]{ellint\_1st}{operator}
\ttindextype[ELLIPFN]{ellint\_2nd}{operator}
\begin{eqnarray*}
\int_y^x \frac{\mathrm{d}t}{s(t)} &=& 2\mathrm{RF}(U_{12}^2, U_{13}^2, U_{23}^2)\\
\int_y^x\frac{(a_1+b_1t)\mathrm{d}t}{(a_4+b_4t)s(t)} &=& \frac{2}{3}d_{12}d_{14}
\mathrm{RD}(U_{12}^2, U_{13}^2, U_{23}^2) +\frac{2X_1Y_1}{X_4Y_4U_{23}}\qquad
\mbox{if }U_{23}\neq 0
\end{eqnarray*}
where in the second equation we have chosen (wlog) the distinguished factors to have
indices 1 \& 4.

For any permutation $\alpha,\beta,\gamma,\delta$ of 1,2,3,4, let
\begin{eqnarray*}
U_{\alpha\beta} & = &(X_\alpha X_\beta Y_\gamma Y_\delta + X_\gamma X_\delta Y_\alpha Y_\beta)
  /(x-y) \qquad \mbox{for } x,y \mbox{ finite}\\
  & = & \sqrt{b_\alpha}\sqrt{b_\beta}Y_\gamma Y_\delta + Y_\alpha Y_\beta
  \sqrt{b_\gamma}\sqrt{b_\delta} \qquad \qquad \mbox{for } x = \infty\\
  & = & X_\alpha X_\beta \sqrt{-b_\gamma}\sqrt{-b_\delta}+X_\gamma X_\delta
      \sqrt{-b_\alpha}\sqrt{-b_\beta}X_\gamma \qquad \mbox{for } y = -\infty
\end{eqnarray*}
Clearly $U_{\alpha\beta}=U_{\beta\alpha}=U_{\gamma\delta}=U_{\delta\gamma}$ and at most
one of $U_{12}, U_{13}, U_{23}$ is zero since
$U_{\alpha\beta}^2 - U_{\alpha\gamma}^2 = d_{\alpha\delta}d_{\beta\gamma} \neq 0$, thus
at most one of the parameters of $\mathrm{RF}$ and $\mathrm{RD}$ is zero.
If $X_4$ or $Y_4$ is zero,that is if $a_4+b_4t$ vanishes at one end of the range
of integration, the integral of the second kind diverges.

In the `awkward' case when $U_{23}=0$ or $U_{12}^2+U_{13}^2=0$, the above method breaks
down. However it is still possible to calculate the required integral; choose
$\beta =2$ or $\beta=3$ so that $b_\beta \neq 0$ and note that
\begin{eqnarray*}
b_\beta\int \frac{(a_1+b_1t)\mathrm{d}t} {(a_4+b_4t)s(t)}
-b_1\int \frac{(a_\beta+b_\beta t)\mathrm{d}t}{(a_4+b_4t)s(t)}
&=& d_{1\beta}\int \frac{\mathrm{d}t}{(a_4+b_4t)s(t)} \\
b_4\int \frac{(a_\beta+b_\beta t)\mathrm{d}t}{(a_4+b_4t)s(t)}
-b_\beta\int\frac{(a_4+b_4 t)\mathrm{d}t}{(a_4+b_4t)s(t)}
&=& d_{\beta 4}\int \frac{\mathrm{d}t}{(a_4+b_4t)s(t)}.
\end{eqnarray*}
The second term on the lhs of the second equation reduces to
$b_\beta\int 1/s(t)\mathrm{d}t$ and so is an integral of the first kind.
Thus the required integral can be expressed in terms of an integral of the
second kind and one of the first kind.

Integrals of the third kind take the form
\ttindextype[ELLIPFN]{ellint\_3rd}{operator}
\[\int_y^x\frac{(a_1+b_1t)\mathrm{d}t}{(a_5+b_5t)s(t)} = \frac{2d_{12}d_{13}
d_{14}}{3d_{15}}\mathrm{RJ}(U_{12}^2, U_{13}^2, U_{23}^2,U_{15}^2) +
\mathrm{RC}(S_{15}^2,Q_{15}^2)\]
where
\begin{eqnarray*}
S_{15}  &=& \left(\frac{X_2X_3X_4}{X_1}Y_5^2+\frac{Y_2Y_3Y_4}{Y_1}X_5^2\right)/(x-y) \qquad
\mbox{for }x,y\mbox{ finite}\\
&=& \frac{X_2X_3X_4}{X_1}Y_5^2+\frac{Y_2Y_3Y_4}{Y_1}X_5^2 \quad \qquad \mbox{for }
x = \infty \mbox{ or } y = -\infty \\
U_{15}^2 &=& U_{1\beta}^2-\frac{d_{1\gamma}d_{1\delta}d_{\beta 5}}{d_{15}}
= U_{\beta\gamma}^2-\frac{d_{1\beta}d_{1\gamma}d_{\delta 5}}{d_{15}} \neq 0\\
Q_{15}^2 &=& \frac{(X_5Y_5)^2}{(X_1Y_1)^2}U_{15}^2\qquad\mbox{ where }
S_{15}^2-Q_{15}^2 = \frac{d_{25}d_{35}d_{45}}{d_{15}} \neq 0
\end{eqnarray*}
where $\beta,\gamma,\delta$ is a permutation of 2,3,4. Again at most one of the
parameters of $\mathrm{RJ}$ is zero. This method will break down when calculating
$S_{15}$ if $a_1+b_1t$ vanishes at either the upper or lower limit of integration as
either $X_1$ or $Y_1$ is zero. However the situation can be remedied by choosing $\beta$
so that $X_\beta \neq 0$, $Y_\beta \neq 0$ and $b_\beta \neq 0$ and using similar
methods to that used for the awkward case for integrals of the second kind.

Note if the integration range is $(-\infty, +\infty)$, the above formulae for
integrals of all three kinds are not valid and the integral needs to be split
into two at, say, zero.
Similarly if the integration range is such that the integrand has branch points,
the integration range will need to be split into two at each of these branch
points.

In \REDUCE elliptic integrals of the three kinds may be evaluated numerically
when the switches \sw{ROUNDED} \& \sw{COMPLEX} are both ON by calling the functions
\f{ellint\_1st},  \f{ellint\_2nd} and \f{ellint\_3rd}. The first two
parameters are the lower and upper limits of the integration range; these are
followed by 4 (or 5 for \f{ellint\_3rd}) two-element lists
$\{a_1,b_1\}, \{a_2,b_2\} \ldots$.


\textbf{Known bug:} As pointed out by Carlson \& FitzSimmons \cite{CarlsonFitzSimmons:00},
the algorithms for $\mathrm{RF}$ \& $\mathrm{RJ}$ may break down when $s(t)$ is the product
of two quadratic factors each with complex conjugate zeros $x_1\pm i y_1$ \&
$x_2\pm i y_2$. One of the three quantities
$U_{12}, U_{13}, U_{23}$ may be negative and this causes the algorithm to return an
incorrect result. This error only arises when the crossing point (neccessarily real) of
the diagonals of the parallellogram in the complex plane formed by the four zeros of
$s(t)$  lies inside the range of integration. The algorithm also appears to produce the
correct result when the crossing point is inside the range of integration but is
`sufficiently close' to either end of the range -- the precise condition appears to be
unknown.
In the same paper Carlson \& FitzSimmons propose a modified algorithm of the same ilk
to overcome these difficulties, however it is yet to be implemented in \REDUCE.

\subsubsection{Some Examples}
In this subsection some of the advantages of symmetry are illustrated; the formula for the
integral of the first kind replaces the 28 formulae in Gradshteyn and Ryzhik. For the cubic
case one of the factors in $s(t)$ is simply taken to be unity whilst for cases of the form
\[\int_\alpha^\beta \frac{\mathrm{d}x}{\sqrt{(a+b x^2)(c+d x^2)}}\]
one must use the substitution $t=x^2$ to obtain
\[\frac{1}{2}\int_{\alpha^2}^{\beta^2} \frac{\mathrm{d}t}{\sqrt{t(a+b t)(c+d t)}}\]
Moreover the restriction that one limit of integration be a branch point of the integrand is
eliminated without doubling the number of standard integrals in the result.
Similarly the formula for the integral of the second kind replaces no less than 144 cases
in Gradshteyn and Ryzhik (72 each for the quartic and cubic cases). For cases where the
numerator in the integrand is unity one simply takes $a_1=0, b_1=1$ and if there is to be no
apparent term of the form $(a_4+b_4t)^{3/2}$ in its denominator one takes $a_4=0, b_4=1$.

When $0 \leq \phi \leq \pi/2$, the fundamental integrals of the first, second
and third kinds $\mathrm{F}(\phi,k)$, $\mathrm{E}(\phi,k)$, $\mathrm{D}(\phi,k)$
and $\Pi(\phi,a,k)$ may be written as
\begin{eqnarray*}
\mathrm{F}(\phi,k) &=& \frac{1}{2}\int_0^{s^2} \frac{\mathrm{d}t}{\sqrt{t(1-t)(1-k^2t)}}\\  
\mathrm{E}(\phi,k) &=& \frac{1}{2}\int_0^{s^2} \frac{\sqrt{1-k^2t}\mathrm{d}t}{\sqrt{t(1-t)}}\\
\mathrm{D}(\phi,k) &=& \frac{1}{2}\int_0^{s^2} \frac{\sqrt{t}\mathrm{d}t}{\sqrt{(1-t)(1-k^2t)}}\\
\Pi(\phi,a,k) &=& \frac{1}{2}\int_0^{s^2} \frac{\mathrm{d}t}{\sqrt{t(1-t)(1-k^2t)}(1-a^2t)}
\end{eqnarray*}
where in all four cases use has been made of the substitution $t=\sin^2 \theta$
and $s=\sin\phi$. In the last three equations one takes $a_4=1, b_4 =0$ so that
the fourth factor in $s(t)$ is unity and in the fourth equation one also takes
$a_1=1,b_1=0$.

\subsection{Some Numerical Utility Functions}
\hypertarget{operator:NOME}{}
\hypertarget{operator:NOME2K}{}
\hypertarget{operator:NOME2K'}{}
\hypertarget{operator:NOME2MOD}{}
\hypertarget{operator:NOME2MOD'}{}
\hypertarget{ELLIPNOME}{}
\index{Nome and Related functions}
\ttindextype[ELLIPFN]{nome}{operator}
\ttindextype[ELLIPFN]{nome2"!K}{operator}
\ttindextype[ELLIPFN]{nome2"!K"!'}{operator}
\ttindextype[ELLIPFN]{nome2mod}{operator}
\ttindextype[ELLIPFN]{nome2mod"!'}{operator}

Five utility functions are provided:
\begin{itemize}
\item \f{nome2mod(q)}
\item \f{nome2mod!$'$(q)}
\item \f{nome2!K(q)}
\item \f{nome2!K!$'$(q)}
\item \f{nome(k)}
\end{itemize}

These are only operative when the switch \sw{rounded} is on and their
argument is numerical. The first pair relate the nome $q$ of the theta
functions with the moduli $k$ and $k'=\sqrt{1-k^2}$ of the associated Jacobi
elliptic functions.

The second pair return the quarter periods K and K$'$ respectively of
the Jacobi elliptic functions associated with the nome $q$.

Finally, \f{nome(k)} returns the nome $q$ associated with the modulus $k$ of
a Jacobi elliptic function and is essentially the inverse of \f{nome2mod}.

\subsection{Jacobi Theta Functions}
\hypertarget{JACTF}{}
\index{Jacobi Theta functions}
These theta functions differ from those originally defined by Lisa Temme
in a number of respects.
Firstly four separate functions of two arguments are defined:
\hypertarget{operator:ELLIPTICTHETA1}{}
\hypertarget{operator:ELLIPTICTHETA2}{}
\hypertarget{operator:ELLIPTICTHETA3}{}
\hypertarget{operator:ELLIPTICTHETA4}{}
\ttindextype[ELLIPFN]{elliptictheta1}{operator} \ttindextype[ELLIPFN]{elliptictheta2}{operator}
\ttindextype[ELLIPFN]{elliptictheta3}{operator} \ttindextype[ELLIPFN]{elliptictheta4}{operator}
\begin{itemize}
\item \f{elliptictheta1(u,tau)} $\qquad \vartheta_1(u, \tau)$
\item \f{elliptictheta2(u,tau)} $\qquad \vartheta_2(u, \tau)$
\item \f{elliptictheta3(u,tau)} $\qquad \vartheta_3(u, \tau)$
\item \f{ellipticthetas(u,tau)} $\qquad \vartheta_4(u, \tau)$
\end{itemize}

rather than a single function with three arguments (with the first argument
taking integer values in the range 1 to 4).
Secondly the periods are $2\pi, 2\pi, \pi$ and $\pi$ respectively
(NOT 4K, 4K, 2K and 2K).
Thirdly the second argument is the modulus $\tau = a+i b$ where $b=\Im\tau>0$
and hence the quasi-period is $\pi\tau$.

The second parameter was previously the nome $q$
where $|q|<1$. As a consequence \f{elliptictheta1} and \f{elliptictheta2} were
multi-valued owing to the appearance of $q^{1/4}$ in their defining expansions.
\f{elliptictheta3} and \f{elliptictheta4} were, however, single-valued
functions of $q$.

Regarded as functions of $\tau$,
\f{elliptictheta1} and \f{elliptictheta2} are single-valued functions. The nome
is given by $q = \exp(i\pi\tau)$  so that the condition $\Im(\tau)>0$ ensures
that $|q| < 1$. Note also  in this case $q^{1/4}$ is interpreted as
$\exp(i\pi\tau/4)$ rather than the principal value of $q^{1/4}$.
Thus, $\tau$, $2+\tau$, $4+\tau$ and $6+\tau$ produce four different values of
both \f{elliptictheta1} and \f{elliptictheta2} although they all correspond to
the same nome $q$.

The four theta functions are defined by their Fourier series:
\begin{align*}
  \vartheta_1(z,\tau) & = 2 e^{i\pi\tau/4}\sum_{n=0}^\infty (-1)^nq^{n^2+n} \sin(2n+1)z\\
\vartheta_2(z,\tau) & = 2 e^{i\pi\tau/4}\sum_{n=0}^\infty q^{n^2+n} \cos(2n+1)z\\
\vartheta_3(z,\tau) & = 1 +2\sum_{n=1}^\infty q^{n^2} \cos 2n z\\
\vartheta_4(z,\tau) & = 1 +2\sum_{n=1}^\infty (-1)^n q^{n^2} \cos 2n z.
\end{align*}

Utilising the periodicity and quasi-periodicity of the theta functions
some generalised shift rules are implemented to shift their first argument
into the base period parallelogram with vertices
\[(\pi/2, \pi\tau/2),\quad (-\pi/2, \pi\tau/2),\quad (-\pi/2, -\pi\tau/2),
\quad (\pi/2, -\pi\tau/2).\]
Together with the relation $\vartheta_1(0,\tau)=0$,  these shift rules serve to
simplify all four theta functions to zero when appropriate.

When the switches \sw{rounded} and \sw{complex} are on and the arguments are
purely numerical and the imaginary part of $\tau$ positive,
the theta functions are evaluated numerically. Note that as $\tau$ is
necessarily complex, the switch \sw{complex} \emph{must} be on.

In what follows $a$ and $b$ will denote the real and imaginary parts of
$\tau$ respectively and so $|q| = \exp(-\pi b)$ and $\arg q =\pi a$.
The series for the theta functions are fairly rapidly convergent
due to the quadratic growth of the exponents of the nome $q$ -- except
for values of $q$ for which $|q|$ is near to 1
(i.e. $b=\Im \tau $ close to zero).
In such cases the direct algorithm would suffer from slow convergence and
rounding errors.
For such values of $|q|$, Jacobi's transformation $\tau'=-1/\tau$ can be
used to produce a smaller value of the nome and so increase the rate of
convergence.
This works very well for real values of $q$, or equivalently for $\tau$ purely
imaginary since $q'= q^{1/b^2}$, but for complex
values the gains are somewhat smaller. The Jacobi transformation produces a
nome $q'$ for which $|q'| =  |q|^{1/(a^2+b^2)}$.

When $\Re q < 0$, the Jacobi transformation is preceded by either the
modular transformation $\tau' = \tau+1$ when $\Im q < 0$, or $\tau' = \tau-1$
when $\Im q > 0$, which both have the effect  of multiplying $q$ by $-1$,
so that the new nome has a non-negative real part and $|a| \leq 1/2$.
Thus the worst case occurs for values of the nome $q$ near to $\pm i$ where
$|q'| \approx |q|^4$.

By using a series of Jacobi transformations preceded, if necessary by
$\tau$-shifts to ensure $|a| <= 1/2$, $|q|$ may be reduced to an acceptable
level. Somewhat arbitrarily these Jacobi's transformations are used
until $b > 0.6$ (i.e.~$|q| < 0.15$). This seems to produce reasonable
behaviour. In practice more than two applications of Jacobi transformations
are rarely necessary.

The previous version of the numerical code returned the principal values
of $\vartheta_1$ and $\vartheta_2$, that is the ones obtained by taking
the principal value of $q^{1/4}$ in their series expansions. The current version replaces
$q^{1/4}$ by $\exp(i\pi\tau/4)$.  If the principal value is required, it is easily obtained
by multiplying by the `correcting' factor $q^{1/4}\exp(-i\pi\tau/4)$.

\subsubsection{Derivatives of Theta Functions}

\hypertarget{THETAD}{}
\hypertarget{operator:THETA1D}{}
\hypertarget{operator:THETA2D}{}
\hypertarget{operator:THETA3D}{}
\hypertarget{operator:THETA4D}{}
\index{Theta function derivatives}
\ttindextype[ELLIPFN]{theta1d}{operator} \ttindextype[ELLIPFN]{theta2d}{operator}
\ttindextype[ELLIPFN]{theta3d}{operator}\ttindextype[ELLIPFN]{theta4d}{operator}
Four functions are provided:
\begin{itemize}
\item \f{theta1d(u,ord,tau)}
\item \f{theta2d(u,ord,tau)}
\item \f{theta3d(u,ord,tau)}
\item \f{theta4d(u,ord,tau)}
\end{itemize}
These return the $d$th derivatives of the respective theta functions
with respect to their first argument $u$; $\tau$ is as usual the modulus
of the theta function. These functions are only operative when the switches
\sw{rounded} and \sw{complex} are ON and their arguments are numeric with
$d$ being a positive integer.  They are provided mainly to support the implementation
the Weierstrassian and Sigma functions discussed in the following subsection.

The numeric code simply sums the Fourier series for the required derivatives.
Unlike the theta functions themselves the code does not use the quasi-periodicity nor
modular transformations to speed  up the convergence of the series by reducing the sizes
of $\Im u$ and $|q|$.  In the numerical evaluation of the Weierstrassian and Sigma functions
these functions are only called after the necessary shifts of the argument $u$ and modular
transformations of $\tau$ have been performed. These are much simpler in this context.

Nevertheless they may be used from top level and numerical experiments reveal that the rounding
errors are not significant provided $|q|$ is not near one (say $|q|<0.9$)
and $u$ is real or at least has a relatively small imaginary part.

\subsection{Weierstrass Elliptic \& Sigma Functions}
\index{Weierstrass Elliptic functions}
\index{Sigma functions}
Three main functions of three arguments are defined:
\hypertarget{WEIERSTRASS}{}
\hypertarget{WEIERSTRASSZETA}{}
\hypertarget{SIGMA}{}
\hypertarget{operator:WEIERSTRASS_SIGMA}{}
\hypertarget{operator:WEIERSTRASS}{}
\hypertarget{operator:WEIERSTRASSZETA}{}
\ttindextype[ELLIPFN]{weierstrass}{operator} \ttindextype[ELLIPFN]{weierstrassZeta}{operator}
\ttindextype[ELLIPFN]{weierstrass\_sigma}{operator}
\begin{itemize}
\item  $\wp(u, \omega_1, \omega_3)$ \ --- \ \f{weierstrass(u,omega1,omega3)}
\item $\zeta_w(u, \omega_1, \omega_3)$ \ --- \ \f{weierstrassZeta(u,omega1,omega3)}
\item $\sigma(u, \omega_1, \omega_3)$ \ --- \ \f{weierstrass\_sigma(u,omega1,omega3)}
\end{itemize}

The notation used is broadly similar used by Lawden \cite{Lawden:89} which is also used in the
NIST Digital Library of Mathematical Functions \href{https://dlmf.nist.gov/}{DLMF:NIST}. However,
$\zeta_w$ is used for the Weierstrassian Zeta function to distinguish it from the Riemann Zeta
function and the usual symbol $\wp$ is used for the Weierstrassian elliptic function itself.

The two primitive periods of the Weierstrass function are $2\omega_1$ and $2\omega_3$ and these must satisfy
$\Im(\omega_3/\omega_1) \neq 0$. The two periods are normally numbered so that $\tau = \omega_3/\omega_1$ has
a positive imaginary part and hence the nome $q = exp(i\pi\tau)$ satisfies $|q| <1$.

Any linear combination $\Omega_{m,n} = 2m\omega_1 +2n\omega_3$ where $m$ and $n$ are
integers (not both zero) is also a period. The set of all such periods plus the origin form a lattice. In the literature
$-(\omega_1+\omega_3)$ is often denoted by $\omega_2$ and $2\omega_2$ is clearly also a period; this
accounts for the gap in the numbering of primitive periods. The period $\omega_2$ is not used in \REDUCE the rule sets for
the Weierstrassian elliptic and related functions.

The primitive periods are not unique;
indeed any periods $2\Omega_1$ and $2\Omega_3$ defined by the unimodular integer bilinear transformation:
\[\Omega_1 = a\omega_1 + b\omega_3,\qquad\Omega_3 = c\omega_1 + d\omega_3,\qquad\text{ where }ad-bc = 1\]
are also primitive. This fact is very useful in the numerical evaluation of the Weierstrassian and Sigma
functions as a sequence of such transformations may be used to increase the size $\Im \tau$ and so reduce
the size of $|q|$. Thus the Fourier series for the theta functions and their derivatives will converge rapidly.
In theory these transformations may be used to reduce the size of $|q|$ until $\Im \tau \geq \sqrt 3/2$ when
$|q|<0.06$. However, in numerical evaluations in \REDUCE it is sufficient to use these transformations only until
$\Im \tau > 0.7$, i.e.~until $|q| < 0.11$. In practice only two or three iterations are required
and usually very much smaller values of $|q|$ are achieved particularly when $\tau$ is purely imaginary i.e.~$q$ is real.

In the numerical evaluations, if a result is real (or purely imaginary) it may
happen that the result returned has a very small imaginary part
(resp. real part). The ratio of the `deliquent' part to the actual result is
invariably smaller than current PRECISION and is due to rounding. Similarly if
the true result is actually zero the result returned may have a very small
absolute value -- again smaller than the current PRECISION.

The Weierstrassian function is even and has a pole of order 2 at all lattice points.
The Zeta and Sigma functions are only quasi-periodic on the lattice. Zeta is odd and has simple poles of residue
1 at all lattice points. The basic Sigma function $\sigma(u,\omega_1,\omega_3)$ is odd and regular everywhere as is
the function $\vartheta_1(u,\tau)$ to which it is closely related. It has zeros at all lattice points. All three functions
$\wp$, $\zeta_w$ and $\sigma$ are homogenous of degrees -2, -1 and +1 respectively. The functions are related by
\[ \wp(u) = -\zeta_w'(u),\qquad\qquad \zeta_w(u) = \sigma'(u)/\sigma(u),\]
where the lattice parameters have been omitted for conciseness.

Rule sets are provided which implement all the properties such as double periodicity discussed above. For numerical evaluation
the switches \sw{rounded} and \sw{complex} must both be ON and all three parameters must be numeric. It is not, however,
necessary to ensure $\Im(\omega_3/\omega_1) >0$ as the second and third parameters will be swapped if required.

\subsubsection{Alternative forms of the Weierstrass Functions}
\hypertarget{WEIERSTRASS1}{}
\hypertarget{WEIERSTRASSZETA1}{}
\hypertarget{operator:WEIERSTRASS1}{}
\hypertarget{operator:WEIERSTRASSZETA1}{}
\ttindextype[ELLIPFN]{weierstrass1}{operator} \ttindextype[ELLIPFN]{weierstrassZeta1}{operator}

Two commonly used alternative forms of the Weierstrassian functions in which
they are regarded as functions of the lattice invariants $g_2$ and $g_3$
rather than the primitive periods $\omega_1$ and $\omega_3$ are provided:
\begin{itemize}
\item  $\wp(u \mid g_2, g_3)$ \ --- \ \f{weierstrass1(u,g2,g3)}
\item $\zeta_w(u \mid g_2, g_3)$ \ --- \ \f{weierstrassZeta1(u,g2,g3)}.
\end{itemize}
Note that for output they are distinguished from the two discussed above
by separating the first and
second arguments by a vertical bar rather than a comma. The rule for
differentiation of the Weierstrass function is simpler when expressed in
this  alternative:
\[ \wp'(u \mid g_2,g_3)^2 = 4\,\wp(u \mid g_2,g_3)^3
- g_2\, \wp(u \mid g_2,g_3) -g_3. \]

\subsubsection{Numerical Evaluation}
If the arguments of the Weierstrassian, sigma and lattice functions (see below)
are all numeric, the functions will only be evaluated if both \sw{rounded} and
\sw{complex} are ON.  The only exceptions are the functions weierstrass1 and
weierstrassZeta1. These are real-valued for real arguments and if all three
arguments are real it suffices that the switch \sw{rounded} is ON.

\subsubsection{Other Sigma Functions}
\hypertarget{SIGMA1}{}
\hypertarget{operator:WEIERSTRASS_SIGMA1}{}
\hypertarget{operator:WEIERSTRASS_SIGMA2}{}
\hypertarget{operator:WEIERSTRASS_SIGMA3}{}

Three further Sigma functions are also provided:
\index{Sigma functions}
\ttindextype[ELLIPFN]{weierstrass\_sigma1}{operator}\ttindextype[ELLIPFN]{weierstrass\_sigma2}{operator}
\ttindextype[ELLIPFN]{weierstrass\_sigma3}{operator}
\begin{itemize}
\item $\sigma_1(u, \omega_1, \omega_3)$ \ --- \ \f{weierstrass\_sigma1(u,omega1,omega3)}
\item $\sigma_2(u, \omega_1, \omega_3)$ \ --- \ \f{weierstrass\_sigma2(u,omega1,omega3)}
\item $\sigma_3(u, \omega_1, \omega_3)$ \ --- \ \f{weierstrass\_sigma3(u,omega1,omega3)}
\end{itemize}
These are all even functions, regular everywhere, homogenous of degree zero and doubly quasi-periodic. They are closely related to the
theta functions $\vartheta_2$, $\vartheta_3$ and $\vartheta_4$ respectively; but \emph{note the difference in numbering}.
For more information on the properties these sigma functions, see Lawden \cite{Lawden:89};
they do not appear in the NIST Digital Library of Mathematical Functions, but are included here for completeness.

\subsubsection{ Quasi-Period Factors \& Lattice Functions}
\hypertarget{ETA}{}
\hypertarget{operator:LATTICE_E1}{}
\hypertarget{operator:LATTICE_E2}{}
\hypertarget{operator:LATTICE_E3}{}
\hypertarget{operator:LATTICE_G}{}
\hypertarget{operator:LATTICE_DELTA}{}
\hypertarget{operator:LATTICE_G2}{}
\hypertarget{operator:LATTICE_G3}{}
\ttindextype[ELLIPFN]{lattice\_e1}{operator}\ttindextype[ELLIPFN]{lattice\_e2}{operator}
\ttindextype[ELLIPFN]{lattice\_e3}{operator}\ttindextype[ELLIPFN]{lattice\_g}{operator}
\ttindextype[ELLIPFN]{lattice\_g2}{operator}\ttindextype[ELLIPFN]{lattice\_g3}{operator}\ttindextype[ELLIPFN]{lattice\_delta}{operator}
\index{Lattice roots}\index{Lattice invariants}\index{Quasi-period factors}

Ten functions are provided:
\begin{itemize}
\item $e_1(\omega_1, \omega_3)$ \ --- \ \f{lattice\_e1(omega1, omega3)};
\item $e_2(\omega_1, \omega_3)$ \ --- \ \f{lattice\_e2(omega1, omega3)};
\item $e_3(\omega_1, \omega_3)$ \ --- \ \f{lattice\_e3(omega1, omega3)};
\item $g_2(\omega_1, \omega_3)$ \ --- \ \f{lattice\_g2(omega1, omega3)};
\item $g_3(\omega_1, \omega_3)$ \ --- \ \f{lattice\_g3(omega1, omega3)};
\item $\Delta(\omega_1, \omega_3)$ \ --- \ \f{lattice\_delta(omega1, omega3)};
\item $\mathrm{G}(\omega_1, \omega_3)$ \ --- \ \f{lattice\_g(omega1, omega3)};
\item $\eta_1(\omega_1, \omega_3)$ \ --- \ \f{eta\_1(omega1, omega3)};
\item $\eta_2(\omega_1, \omega_3)$ \ --- \ \f{eta\_2(omega1, omega3)};
\item $\eta_3(\omega_1, \omega_3)$ \ --- \ \f{eta\_3(omega1, omega3)}.
\end{itemize}

These are operative when the switches \sw{rounded} and \sw{complex} are ON
and their arguments are numerical. The first three are referred to as lattice
roots and are related to the invariants
$g_2, g_3$, the discriminant $\Delta = g_2^3-27g_3^2$ and a closely related
invariant $\mathrm{G} = g_2^3/(27 g_3^2)$ of the Weierstrassian
elliptic function $\wp$. The lattice roots also appear in the numerical
evaluation of the Weierstrass function. These lattice roots satisfy:
\[e_1+e_2+e_3=0,\qquad g_2=2(e_1^2+e_2^2+e_3^2),\qquad g_3= 4e_1e_2e_3.\]
If the discriminant $\Delta$ vanishes or equivalently if $\mathrm{G} = 1$,
there are at most two distinct lattice roots and the elliptic function
degenerates to an elementary one. The advantage of the invariant
$\mathrm{G}$ is that it is a function of $\tau = \omega_3/\omega_1$ only.

\hypertarget{operator:ETA_1}{}
\hypertarget{operator:ETA_2}{}
\hypertarget{operator:ETA_3}{}
\ttindextype[ELLIPFN]{eta\_1}{operator}\ttindextype[ELLIPFN]{eta\_2}{operator}\ttindextype[ELLIPFN]{eta\_3}{operator}
The remaining three functions \f{eta\_1}, \f{eta\_2} \& \f{eta\_3} appear in
the rules for the quasi-periodicity of the four sigma functions and of the
Weierstrassian Zeta function. They are also used in the numerical
evaluation of these functions when the switches \sw{rounded} and \sw{complex}
are ON. The quasi-period relations are:
\begin{align*}
  \zeta_w(u+2\omega_j) & = \zeta_w(u)+2\eta_j\\
  \sigma(u+2\omega_j) & = -exp(2\eta_j(u+\omega_j))\sigma(u)\\
  \sigma_k(u+2\omega_j) & =  exp(2\eta_j(u+\omega_j))\sigma_k(u) \quad\text{  if  }j\neq k\\
  \sigma_j(u+2\omega_j) & = -exp(2\eta_j(u+\omega_j))\sigma_j(u)\\
  \zeta_w(\omega_j) & = \eta_j\\
  \sigma_j(\omega_j) & = 0,
\end{align*}
where the lattice parameters have been omitted for conciseness and $j,k = 1\ldots 3$.
The quasi-period factors satisfy
\[\eta_1+\eta_2+\eta_3=0,\qquad
   \eta_1\omega_3-\eta_3\omega_1=\eta_2\omega_1-\eta_1\omega_2=\eta_3\omega_2-\eta_2\omega_3=i\pi/2.\]
As well as the scalar-valued functions discussed above in this section,
there are four functions which return a list as their value:
\hypertarget{operator:LATTICE_ROOTS}{}
\hypertarget{operator:LATTICE_INVARIANTS}{}
\hypertarget{operator:LATTICE_GENERATORS}{}
\hypertarget{operator:QUASI_PERIOD_FACTORS}{}
\ttindextype[ELLIPFN]{lattice\_roots}{operator}\ttindextype[ELLIPFN]{lattice\_invariants}{operator}
\ttindextype[ELLIPFN]{lattice\_generators}{operator}\ttindextype[ELLIPFN]{quasi\_period\_factors}{operator}
\begin{itemize}[left=3mm]
\item \f{lattice\_roots(omega1, omega3)} --- returns $\{e_1,\ e_2,\ e_3\}$;
\item \f{lattice\_invariants(omega1, omega3)} --- returns
  $\{g_2,\ g_3,\ \Delta,\ \mathrm{G}\}$;
\item \f{quasi\_period\_factors(omega1, omega3)}
  --- returns $\{\eta_1,\ \eta_2,\ \eta_3\}$;
\item \f{lattice\_generators(g2, g3)}  --- returns $\{\omega_1,\ \omega_3\}$.
\end{itemize}
The first three are actually more efficient than calling the requisite
scalar-valued functions individually and the fourth is used in the numerical
evaluation of the Weierstrass functions regarded as functions of the
invariants. These functions are only useful when the switches \sw{rounded} and
\sw{complex} are ON and their arguments are all numerical.
Note that the call sequence:
\begin{verbatim}
  lattice_generators(g2,g3);
  lattice_invariants(first ws, second ws);
  {first ws, second ws};
\end{verbatim}
should reproduce the list \{g2, g3\}, perhaps with small rounding errors. The
corresponding sequence with the function \f{lattice\_generators} being called after
\f{lattice\_invariants} (and g2 \& g3 replaced by w1 \& w3),
in general, will not produce the same pair of lattice generators since the
generators are only defined up to a unimodular bilinear transformation.

For details of the algorithm used to calculate the lattice generators from the
invariants see the DLMF:NIST chapter on
\href{https://dlmf.nist.gov/23.22#ii}{Lattice Calculations}.

\subsection{Inverse Jacobi Elliptic Functions}
\index{Inverse Jacobi Elliptic functions}
The following inverses of the 12 Jacobi elliptic functions are available:-
\hypertarget{INVELL}{}
\hypertarget{operator:ARCSN}{}
\hypertarget{operator:ARCCN}{}
\hypertarget{operator:ARCDN}{}
\hypertarget{operator:ARCCD}{}
\hypertarget{operator:ARCSD}{}
\hypertarget{operator:ARCND}{}
\hypertarget{operator:ARCDC}{}
\hypertarget{operator:ARCNC}{}
\hypertarget{operator:ARCSC}{}
\hypertarget{operator:ARCNS}{}
\hypertarget{operator:ARCDS}{}
\hypertarget{operator:ARCCS}{}
\ttindextype[ELLIPFN]{arcsn}{operator}\ttindextype[ELLIPFN]{arccn}{operator}
\ttindextype[ELLIPFN]{arcdn}{operator}\ttindextype[ELLIPFN]{arccd}{operator}
\ttindextype[ELLIPFN]{arcsd}{operator}\ttindextype[ELLIPFN]{arcnd}{operator}
\ttindextype[ELLIPFN]{arcdc}{operator}\ttindextype[ELLIPFN]{arcnc}{operator}
\ttindextype[ELLIPFN]{arcsc}{operator}\ttindextype[ELLIPFN]{arcns}{operator}
\ttindextype[ELLIPFN]{arcds}{operator}\ttindextype[ELLIPFN]{arccs}{operator}
\begin{itemize}
\item arcsn(u,k)
\item arcdn(u,k)
\item arccn(u,k)
\item arccd(u,k)
\item arcsd(u,k)
\item arcnd(u,k)
\item arcdc(u,k)
\item arcnc(u,k)
\item arcsc(u,k)
\item arcns(u,k)
\item arcds(u,k)
\item arccs(u,k)
\end{itemize}

Thus, for example,
\begin{verbatim}
   jacobisn(arcsn(x, k), k)   --> x
   jacobisc(arcsc(x, k), k)   --> x
\end{verbatim}

A rule list is provided to simplify these functions for special values of their
arguments such $x=0$, $k=0$ and $k=1$, to implement the inverse function
simplification formulae illustrated immediately above and for differentiation
of these functions with respect to their two arguments.

Note that $\mathrm{arccs}$ is not defined to be an odd
function of its first argument unlike $\mathrm{cs}$.
Instead it is taken to satisfy:
\[ \mathrm{arccs}(-x, k) = 2\mathrm{K}(k)-\mathrm{arccs}(x, k).\]
This is analogous to the situation in Reduce for $\mathrm{acot}$ where
\[ \mathrm{arctan}(-x) = -\mathrm{arctan}(x),\qquad
  \mathrm{arccot}(-x) = \pi -\mathrm{arccot}(x). \]

This choice means that the range of (real) principal values of
$\mathrm{arccs}$ is \emph{connected} -- it is the open set $(0, 2K(k))$.

When their arguments are \emph{numerical}, these functions will be
evaluated numerically if the \sw{rounded} and \sw{complex} switches are
both ON.  Note that in some cases the result may have an imaginary
part even if both arguments are real, hence the necessity of the switch
\sw{complex} being ON.

Note also that for $\mathrm{arcdn}$ and $\mathrm{arcnd}$ a zero value of
the modulus $k$ is excluded (since
$\mathrm{dn}(x,0) = \mathrm{nd}(x,0) = 1 \quad \forall x$).

As the Jacobi elliptic functions are doubly periodic, their inverse functions
are multi-valued. The numerical value returned is the principal value $v$ which
lies in the parallelogram in the complex plane whose vertices are given in the
table below. Other values of the inverse functions are indicated in the
fifth column of the table below where $m$ and $n$ are arbitrary integers.

\begin{tabular}{lllll}
  & Quarter&Periods&Principal&Other\\
Function&$p$&$q$&Parallelogram&Values\\
$\mathop{\mathrm{arcsn}}$&$\mathrm{K}$&$i\mathrm{K}'$&
$-(p+q)$, $-p+q$,&$2m p+2n q +(-1)^mv$\\
&&&$p+q$, $p-q$.&\\
$\mathop{\mathrm{arccn}}$&$\mathrm{K}$&$\mathrm{K}+i\mathrm{K}'$&
  $-q$, $q$, $2p+q$, $2p-q$.&$4m p +2n q \pm v$\\
$\mathop{\mathrm{arcdn}}$&$i\mathrm{K}'$&$\mathrm{K}$&
  $0$, $2p$, $2(p+q)$, $2q$.&$2m q+4n p \pm v$\\
$\mathop{\mathrm{arcns}}$&$\mathrm{K}$&$i\mathrm{K}'$&
$-(p+q)$, $-p+q$,&$2m p+2n q +(-1)^mv$\\
&&&$p+q$, $p-q$.&\\
$\mathop{\mathrm{arcnc}}$&$\mathrm{K}$&$\mathrm{K}+i\mathrm{K}'$&
  $-q$, $q$, $2p+q$, $2p-q$.&$4m p +2n q \pm v$\\
$\mathop{\mathrm{arcnd}}$&$i\mathrm{K}'$&$\mathrm{K}$&
  $0$, $2p$, $2(p+q)$, $2q$.&$2m q+4n p \pm v$\\
$\mathop{\mathrm{arccd}}$&$\mathrm{K}$&$i\mathrm{K}'$&
  $-q$, $q$, $2p+q$, $2p-q$.&$4m p +2n q \pm v$\\
$\mathop{\mathrm{arcdc}}$&$\mathrm{K}$&$i\mathrm{K}'$&
  $-q$, $q$, $2p+q$, $2p-q$.&$4m p +2n q \pm v$\\
$\mathop{\mathrm{arcsd}}$&$\mathrm{K}$&$\mathrm{K}+i\mathrm{K}'$&
$-(p+q)$, $-p+q$,&$2m p +2n q +(-1)^mv$\\
&&&$p+q$, $p-q$.&\\
$\mathop{\mathrm{arcds}}$&$\mathrm{K}$&$\mathrm{K}+i\mathrm{K}'$&
$-(p+q)$, $-p+q$,&$2m p +2n q +(-1)^mv$\\
&&&$p+q$, $p-q$.&\\
$\mathop{\mathrm{arcsc}}$&$i\mathrm{K}'$&$\mathrm{K}$&
$-(p+q)$, $-p+q$,&$2m q +2n q +(-1)^nv$\\
&&&$p+q$, $p-q$.&\\
$\mathop{\mathrm{arccs}}$&$i\mathrm{K}'$&$\mathrm{K}$&
  $-p$, $p$, $p+2q$, $-p+2q$&$2m q +2n q +(-1)^nv$\\
\end{tabular}

When both arguments are real and $|k|<=1$ and when there are certain
restrictions on the range of the first parameter $x$ (see the
table below), then the principal value of the inverse function is
real. It lies in the range  given in the third column of the table
below. (c.f. the inverse trigonometric functions). Other \emph{real}
values of the inverse functions are indicated in the fourth column
of the table below where $n$ is an arbitrary integer.

\begin{tabular}{llll}
  Fn & Domain & Principal Value $v$ & Other real values\\
$\mathop{\mathrm{arcsn}}$: & $ |x| <=1 $ &
  $-\mathrm{K}(k) <= v <= \mathrm{K}(k)$ &
  $2 n\mathrm{K}(k)+(-1)^nv$ \\
$\mathop{\mathrm{arccn}}$: &  $ |x| <=1 $ &
  $0 <= v <= 2\mathrm{K}(k)$ & 
  $4 n\mathrm{K}(k) \pm v$ \\
$\mathop{\mathrm{arccd}}$: & $ |x| <=1 $ &
  $0 <= v <= 2\mathrm{K}(k)$ &
  $4 n\mathrm{K}(k) \pm v$ \\
$\mathop{\mathrm{arcns}}$: & $ |x| >=1 $ &
  $-\mathrm{K}(k) <= v <= \mathrm{K}(k)$ \& $v \neq 0$ &
  $2 n\mathrm{K}(k)+(-1)^nv$ \\
$\mathop{\mathrm{arcnc}}$: & $ |x| >=1 $ &
  $0 <= v <= 2\mathrm{K}(k)$ \& $v \neq \mathrm{K}(k)$  &
  $4 n\mathrm{K}(k) \pm v$ \\
$\mathop{\mathrm{arcdc}}$: & $ |x| >=1 $ &
  $0 <= v <= 2\mathrm{K}(k)$ \& $v \neq \mathrm{K}(k)$ &
  $4 n\mathrm{K}(k) \pm v$ \\
$\mathop{\mathrm{arcdn}}$: & $ k' <= x <= 1$ &
  $0 <= v <= \mathrm{K}(k)$ &
  $2 n\mathrm{K}(k) \pm v$ \\  
$\mathop{\mathrm{arcnd}}$: & $ 1 <= x <= 1/k'$ &
  $0 <= v <= \mathrm{K}(k)$ &
  $2 n\mathrm{K}(k) \pm v$ \\   
$\mathop{\mathrm{arcds}}$: & $ |x| >= k'$ &
  $-\mathrm{K}(k) <= v <= \mathrm{K}(k)$ \& $v \neq 0$ &
  $2 n\mathrm{K}(k)+(-1)^nv $ \\  
$\mathop{\mathrm{arcsd}}$: & $  |x| <= 1/k'$ &
  $-\mathrm{K}(k) <= v <= \mathrm{K}(k)$ &
  $2 n\mathrm{K}(k)+(-1)^nv $ \\
$\mathop{\mathrm{arcsc}}$: & $ x \in \mathbb{R}$ &
  $-\mathrm{K}(k) < v < \mathrm{K}(k)$ &
  $2 n\mathrm{K}(k) + v$ \\  
$\mathop{\mathrm{arccs}}$: & $ x \in \mathbb{R}$ &
$0 < v < 2\mathrm{K}(k)$ \& $v \neq 0$ 
  $2 n\mathrm{K}(k) + v$ \\
\end{tabular}

The numerical values of the inverse functions are calculated by
expressing them in terms of the symmetric elliptic integral:
\[ R_F(x,y,z)=\int_0^\infty 1/\sqrt{(t-x)(t-y)(t-z)}\,\mathrm{d}t. \]
For more details see the DLMF website:
\href{https://dlmf.nist.gov/19.25#v}{Inverse Jacobian
Elliptic Functions}.

\subsection{Table of Elliptic Functions and Integrals}
\hypertarget{ELLIPFNTAB}{}

\fbox{
\begin{tabular}{r l}
\rule{0pt}{16pt}Function & Operator\\[5pt]
$\mathrm{am}(u,k)$ & \f{jacobiam(u,k)}\\
$\mathrm{sn}(u,k)$ & \f{jacobisn(u,k)}\\
$\mathrm{dn}(u,k)$ & \f{jacobidn(u,k)}\\
$\mathrm{cn}(u,k)$ & \f{jacobicn(u,k)}\\
$\mathrm{cd}(u,k)$ & \f{jacobicd(u,k)}\\
$\mathrm{sd}(u,k)$ & \f{jacobisd(u,k)}\\
$\mathrm{nd}(u,k)$ & \f{jacobind(u,k)}\\
$\mathrm{dc}(u,k)$ & \f{jacobidc(u,k)}\\
$\mathrm{nc}(u,k)$ & \f{jacobinc(u,k)}\\
$\mathrm{sc}(u,k)$ & \f{jacobisc(u,k)}\\
$\mathrm{ns}(u,k)$ & \f{jacobins(u,k)}\\
$\mathrm{ds}(u,k)$ & \f{jacobids(u,k)}\\
$\mathrm{cs}(u,k)$ & \f{jacobics(u,k)}\\
Inverse Functions of the above:\\
$\mathrm{arcsn}(u,k)$ & \f{arcsn(u,k)}\\
$\mathrm{arccn}(u,k)$ & \f{arccn(u,k)}\\
   ... \\
$\mathrm{arccs}(u,k)$ & \f{arccs(u,k)}\\

Complete Integral (1st kind) $\mathrm{K}(k)$ & \f{ellipticK(k)}\\
$\mathrm{K}'(k)$ & \f{ellipticK!'(k)}\\
Incomplete Integral (1st kind) $\mathrm{F}(\phi,k)$ & \f{ellipticF(phi,k)}\\
Complete Integral (2nd kind) $\mathrm{E}(k)$ & \f{ellipticE(k)}\\
$\mathrm{E}'(k)$ & \f{ellipticE!'(k)}\\
Legendre Incomplete Int (2nd kind) $\mathrm{E}(u,k)$ & \f{ellipticE(u,k)}\\
Alternative Incomplete Int (2nd kind) $\mathrm{D}(u,k)$ & \f{ellipticD(u,k)}\\
Jacobi Incomplete Int (2nd kind) $\mathcal{E}(u,k)$ & \f{jacobiE(u,k)}\\
Jacobi's Zeta $\mathrm{Z}(u,k)$ & \f{jacobiZeta(u,k)}\\
$\vartheta_1(u,\tau)$ & \f{elliptictheta1(u,tau)}\\
$\vartheta_2(u,\tau)$ & \f{elliptictheta2(u,tau)}\\
$\vartheta_3(u,\tau)$ & \f{elliptictheta3(u,tau)}\\
$\vartheta_4(u,\tau)$ & \f{elliptictheta4(u,tau)}\\
$\wp(u,\omega_1, \omega_3)$ & \f{weierstrass(u,omega1,omega3)}\\
$\zeta_w(u,\omega_1, \omega_3)$ & \f{weierstrassZeta(u,omega1,omega3)}\\
$\sigma(u,\omega_1, \omega_3)$ & \f{weierstrass\_sigma(u,omega1,omega3)}\\
$\sigma_1(u,\omega_1, \omega_3)$ & \f{weierstrass\_sigma1(u,omega1,omega3)}\\
$\sigma_2(u,\omega_1, \omega_3)$ & \f{weierstrass\_sigma2(u,omega1,omega3)}\\
$\sigma_3(u,\omega_1, \omega_3)$ & \f{weierstrass\_sigma3(u,omega1,omega3)}\\
$\wp(u \mid g_2, g_3)$ & \f{weierstrass1(u,g2,g3)}\\
$\zeta_w(u \mid g_2, g_3)$ & \f{weierstrassZeta1(u,g2,g3)}\\[5pt]
\end{tabular}}
