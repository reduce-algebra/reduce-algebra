% $Id:$
\section{Logo Turtle Graphics}
\indexpackage{LOGOTURTLE}
\index{Wright, Francis J.}\index{People!Wright, Francis J.}


\subsection{Introduction}

Logo Turtle Graphics\footnote{The Logo Turtle Graphics package was
written by Francis Wright.} (henceforth referred to as ``LogoTurtle'')
is a partial REDUCE emulation of
\href{https://en.wikipedia.org/wiki/Turtle_graphics}{traditional Logo
  turtle graphics} with one turtle, modelled on
\href{http://people.eecs.berkeley.edu/~bh/logo.html}{Berkeley Logo by
  Brian Harvey} and \href{https://fmslogo.sourceforge.io/}{FMSLogo by
  David Costanzo} (which is an updated version of George Mills'
MSWLogo, a multimedia-enhanced version for Microsoft Windows, which is
itself based on Berkeley Logo).  This manual section is derived
primarily from the Graphics chapter of the Berkeley Logo manual
available from
\href{https://github.com/jrincayc/ucblogo-code}{GitHub}.

This package is inspired by, and related to, the REDUCE
\hyperref[package:TURTLE]{Turtle package} by Caroline Cotter (ZIB,
Berlin, 1998), and the word ``Turtle'' below (with a capital T) will
refer specifically to that package.  Both packages are built on the
REDUCE \hyperref[package:GNUPLOT]{Gnuplot package}, which itself uses
\href{http://gnuplot.info/}{Gnuplot} to display plots.  This means
that plotting is not fully interactive as it would be in
\href{https://en.wikipedia.org/wiki/Logo_(programming_language)}{traditional
  Logo}; a plot is constructed invisibly and only displayed when
requested.  This package aims to be more efficient, more authentic,
more interactive and more complete than Turtle.

Note that LogoTurtle and Turtle cannot both be run in the same REDUCE
session because they define some procedures with the same names.


\subsection{Design}

LogoTurtle is entirely procedural.  It uses ``getters'' and
``setters'', and does not use any algebraic-mode variables (unlike
Turtle).  Most command names are as in Berkeley Logo and/or FMSLogo,
and their function is the same or similar.  (Identical behaviour is
not always possible.)  However, commands are REDUCE procedure calls
and so must be followed by parentheses unless they take a single
argument, in which case the parentheses are optional.  Note that
commands that take no argument \emph{must} be followed by empty
parentheses.

Getters (query procedures) return values that are accepted as input by
their matching setters (command procedures).  If more than one data
value is involved then a list is used.  For example, the getter
\hyperref[logoturtle:labelfont]{\texttt{LABELFONT}} returns a list of
the current label font face and size if both are set, which the
correspond setter
\hyperref[logoturtle:setlabelfont]{\texttt{SETLABELFONT}} accepts as
its argument.

LogoTurtle procedures other than queries return nothing (\texttt{nil})
and plotting is achieved via side effects, not via returned values
(unlike Turtle).  A plot is displayed by calling the (non-traditional)
\hyperref[logoturtle:draw]{\texttt{DRAW}} command, as for Turtle.
The plot displayed need not be complete; \texttt{DRAW} displays the
plot constructed so far, which allows an element of interactivity.

LogoTurtle makes essential use of commands to lower and raise the pen
(unlike Turtle; see \hyperref[logoturtle:PBC]{Pen and Background
  Control}).  Lowering the pen begins a ``curve'', namely a sequence
of points connected by straight lines, and raising the pen ends that
curve.  Each time the pen is lowered, the turtle moved and the pen
raised produces a distinct curve.

LogoTurtle uses Lisp floating-point numbers internally and does not
require any particular REDUCE number domain settings.  However, all
procedure arguments and list elements must (currently) evaluate to
explicit real numbers, which can be integers, rationals or floats.
This means that \emph{if any transcendental functions are used then
``on rounded'' is required}, and the default (or a little lower)
precision is appropriate.  Returned values and list elements will be
floats converted to the current number domain.  Note that LogoTurtle
lists are REDUCE algebraic-mode lists delimited by curly braces,
\texttt{\{~\}}, not the square brackets used in traditional Logo.


\subsection{User Interface}

To use LogoTurtle, execute the REDUCE command
\begin{verbatim}
     load_package logoturtle;
\end{verbatim}

LogoTurtle sets the scaling of the two axes to be the same so that the
aspect ratio is 1:1 and geometry is correct (although beware that
Gnuplot may not always honour this).  By default, LogoTurtle scales
the graphics window so that turtle coordinates $(-100,-100)$ and
$(100,100)$ fit, and the center of the graphics window is turtle
location $(0,0)$, i.e.\@ the origin or home position.  But this fixed
window can be turned off or the size changed (see
\hyperref[logoturtle:TWC]{Turtle and Window Control}), so that Gnuplot
automatically sizes the display to include the whole plot (as for
Turtle).  The position of the origin (the turtle home location
$(0,0)$) is then not fixed and will be different for different
pictures.

Positive $x$ is to the right; positive $y$ is up.  Headings (angles)
are measured in degrees clockwise from the positive $y$ axis.  (Note
that this differs from the common mathematical convention, also used
by Turtle, of measuring angles counterclockwise from the positive $x$
axis!)  Initially, the turtle is at the origin (Cartesian coordinates
$(0,0)$) facing up (heading 0 degrees) with the pen up.

The turtle is (optionally) represented as a black isoceles triangle;
the actual turtle position is at the midpoint of the base (the short
side).  However, the turtle is drawn one step behind its actual
position, so that the display of the base of the turtle's triangle
does not obscure a line drawn perpendicular to it (as would happen
after drawing a square).

LogoTurtle associates the following fixed colours with the colour
numbers 0--7:
\begin{center}
  \begin{tabular}{rlrlrlrl}
    0 & black  &  1 & blue    &  2 & green  &  3 & cyan  \\
    4 & red    &  5 & magenta &  6 & yellow &  7 & white
  \end{tabular}
\end{center}
It currently provides 8 additional user-settable colour numbers 8--15
with the following initial settings:
\begin{center}
  \begin{tabular}{rlrlrlrl}
     8 & brown  &  9 & tan     & 10 & forest & 11 & aqua  \\
    12 & salmon & 13 & purple  & 14 & orange & 15 & grey
  \end{tabular}
\end{center}
LogoTurtle uses by default a white background and pen colour chosen
automatically by Gnuplot.

Note that LogoTurtle command names are shown using upper case letters
in the descriptions after the example below to distinguish them
clearly from their arguments, but LogoTurtle is case insensitive, so
commands can be entered in either case.

When the switch \texttt{trlogoturtle} is on, LogoTurtle outputs
Cartesian coordinates corresponding to every move of the turtle and
\texttt{DRAW} outputs the list of curves that it is about to draw.
The switch \texttt{trplot} is also turned on, so that the commands
sent to Gnuplot (but not the actual plot data) are also output.


\subsection{A Simple Example}

This example assumes LogoTurtle has just been loaded but not yet used.
If this is not the case then first execute the commands
\begin{verbatim}
  clearscreen(); penup();
\end{verbatim}

The following code draws an equilateral triangle with side length 100
centred on the origin, with one vertex on the positive Y axis.  The
sides are coloured red, green and blue.  To make this example as
interactive as possible, the plot is displayed after each side is
drawn (but for this effect to be visible each line ending with
\texttt{draw();} must be executed separately; if you input all the
commands together then you will only see the complete triangle).
\begin{verbatim}
  on rounded;
  forward(100/sqrt 3); pendown();
  setpencolor red; right 150; forward 100; draw();
  setpencolor green; right 120; forward 100; draw();
  setpencolor blue; right 120; forward 100; draw();
\end{verbatim}

For more interesting examples, please see the file
\href{https://sourceforge.net/p/reduce-algebra/code/HEAD/tree/trunk/packages/plot/logoturtle.tst}
     {\texttt{logoturtle.tst}}, which can be found in the
     \texttt{packages/plot} directory in a standard REDUCE
     distribution.


\subsection{Displaying Logo Turtle Graphics}

\subsubsection*{Draw command}
\begin{verbatim}
     DRAW()
\end{verbatim}
\label{logoturtle:draw}
This command can be used at any time to display the current plot,
i.e.\@ the plot constructed so far (provided there is a plot to
display).  It initially opens a Gnuplot window and subsequently
updates it.

\begin{verbatim}
     DRAW(identifiers)
\end{verbatim}
The \texttt{draw} command accepts one or more optional arguments
consisting of identifiers used to save plots by the
\hyperref[logoturtle:savepict]{\texttt{SAVEPICT}} command and displays
superimposed the plot retrieved using each identifier, in the order
specified, followed by the current plot.  (The order is only
significant in that it determines the colour of each plot if it is set
automatically by Gnuplot and if plots overlap then later plots are on
top of earlier plots and so hide them.)


\subsection{Turtle Motion}

\subsubsection*{Forward command}
\begin{verbatim}
     FORWARD dist
\end{verbatim}
%% FD dist
moves the turtle forward, in the direction that it's facing, by the
specified distance (measured in turtle steps).

\subsubsection*{Back command}
\begin{verbatim}
     BACK dist
\end{verbatim}
%% BK dist
moves the turtle backward, i.e., exactly opposite to the direction
that it's facing, by the specified distance.  (The heading of the
turtle does not change.)

\subsubsection*{Left command}
\begin{verbatim}
     LEFT degrees
\end{verbatim}
%% LT degrees
turns the turtle counterclockwise by the specified angle, measured in
degrees (1/360 of a circle).

\subsubsection*{Right command}
\begin{verbatim}
     RIGHT degrees
\end{verbatim}
%% RT degrees
turns the turtle clockwise by the specified angle, measured in degrees
(1/360 of a circle).

\subsubsection*{Setpos command}
\begin{verbatim}
     SETPOS pos
\end{verbatim}
\label{logoturtle:setpos}
moves the turtle to an absolute position in the graphics window.  The
input is a list of two numbers, the $x$ and $y$ coordinates,
e.g.\ ``\texttt{SETPOS \{50, 50\}}''.

\subsubsection*{Setxy command}
\begin{verbatim}
     SETXY(xcor, ycor)
\end{verbatim}
moves the turtle to an absolute position in the graphics window.  The
two inputs are numbers, the $x$ and $y$ coordinates.

\subsubsection*{Setx command}
\begin{verbatim}
     SETX xcor
\end{verbatim}
moves the turtle horizontally from its old position to a new absolute
horizontal coordinate.  The input is the new $x$ coordinate.

\subsubsection*{Sety command}
\begin{verbatim}
     SETY ycor
\end{verbatim}
moves the turtle vertically from its old position to a new absolute
vertical coordinate.  The input is the new $y$ coordinate.

\subsubsection*{Setheading command}
\begin{verbatim}
     SETHEADING degrees
\end{verbatim}
\label{logoturtle:setheading}
%% SETH degrees
turns the turtle to a new absolute heading.  The input is a number,
the heading in degrees clockwise from the positive $y$ axis.

\subsubsection*{Home command}
\begin{verbatim}
     HOME()
\end{verbatim}
\label{logoturtle:home}
moves the turtle to its starting position (the origin) and orientation.
Equivalent to ``\texttt{\hyperref[logoturtle:setpos]{SETPOS}~\{0,~0\};
  \hyperref[logoturtle:setheading]{SETHEADING}~0}''.

\subsubsection*{Arc command}
\begin{verbatim}
     ARC(angle, radius)
\end{verbatim}
draws a circular arc centred on the turtle with the specified positive
radius, starting at the turtle's heading and extending clockwise
through the specified angle (counter-clockwise if angle is negative).
The turtle does not move and the arc is drawn as if the turtle mode is
\texttt{WINDOW} for all modes unless windowing is turned off.

\subsubsection*{Arc2 command}
\begin{verbatim}
     ARC2(angle, radius)
\end{verbatim}
moves the turtle around a circular arc that sweeps through the
specified angle with the specified positive radius.  The turtle always
moves forwards: if angle is positive, then the turtle moves forwards
in a clockwise direction; if angle is negative, then the turtle moves
forwards in a counter-clockwise direction.  At the end of the arc, the
turtle's heading is increased by angle.

\subsubsection*{Circle command}
\begin{verbatim}
     CIRCLE radius
\end{verbatim}
draws a circle centred on the turtle with the positive radius
specified.  The turtle does not move and the circle is drawn as if the
turtle mode is \texttt{WINDOW} for all modes unless windowing is
turned off.  Equivalent to \texttt{ARC(360, radius)}.

\subsubsection*{Circle2 command}
\begin{verbatim}
     CIRCLE2 radius
\end{verbatim}
moves the turtle clockwise around a circle with the specified positive
radius.  The turtle ends in the same position in which it starts.
Equivalent to \texttt{ARC2(360, radius)}.


\subsection{Turtle Motion Queries}

\subsubsection*{Pos query}
\begin{verbatim}
     POS()
\end{verbatim}
returns the turtle's current position, as a list of two numbers, the
$x$ and $y$ coordinates.

\subsubsection*{Xcor query}
\begin{verbatim}
     XCOR()
\end{verbatim}
returns a number, the turtle's $x$ coordinate.

\subsubsection*{Ycor query}
\begin{verbatim}
     YCOR()
\end{verbatim}
returns a number, the turtle's $y$ coordinate.

\subsubsection*{Heading query}
\begin{verbatim}
     HEADING()
\end{verbatim}
returns a number, the turtle's heading in degrees.

\subsubsection*{Towards query}
\begin{verbatim}
     TOWARDS pos
\end{verbatim}
returns a number, the heading at which the turtle should be facing so
that it would point from its current position to the position given as
the input in the form of a list of two numbers.


\subsection{Turtle and Window Control}
\label{logoturtle:TWC}

\subsubsection*{Showturtle command}
\begin{verbatim}
     SHOWTURTLE()
\end{verbatim}
\label{logoturtle:showturtle}
%% ST
makes the turtle visible (next time the picture is drawn).

\subsubsection*{Hideturtle command}
\begin{verbatim}
     HIDETURTLE()
\end{verbatim}
\label{logoturtle:hideturtle}
%% HT
makes the turtle invisible (next time the picture is drawn).

\subsubsection*{Clean command}
\begin{verbatim}
     CLEAN()
\end{verbatim}
erases all lines that the turtle has drawn on the graphics window.
The turtle's state (position, heading, pen mode, etc.) is not changed.

\subsubsection*{Clearscreen command}
\begin{verbatim}
     CLEARSCREEN()
\end{verbatim}
%% CS
erases (and closes) the graphics window and sends the turtle to its
initial position and heading.  Like
\texttt{\hyperref[logoturtle:home]{HOME}()} and \texttt{CLEAN()}
together.

\subsubsection*{Wrap command}
\begin{verbatim}
     WRAP()
\end{verbatim}
tells the turtle to enter wrap mode: From now on, if the turtle is
asked to move past the boundary of the graphics window, it will ``wrap
around'' and reappear at the opposite edge of the window.  The top
edge wraps to the bottom edge, while the left edge wraps to the right
edge.  (So the window is topologically equivalent to a torus.)  This
is the turtle's initial mode.  \texttt{WRAP} can also take one or two
arguments as described below.  Compare \texttt{WINDOW} and
\texttt{FENCE}.

\subsubsection*{Window command}
\begin{verbatim}
     WINDOW()
\end{verbatim}
tells the turtle to enter window mode: From now on, if the turtle is
asked to move past the boundary of the graphics window, it will move
offscreen.  The visible graphics window is considered as just part of
an infinite graphics plane; the turtle can be anywhere on the plane.
(If you lose the turtle, \texttt{HOME} will bring it back to the
center of the window.)  \texttt{WINDOW} can also take one or two
arguments as described below.  Compare \texttt{WRAP} and
\texttt{FENCE}.

\subsubsection*{Fence command}
\begin{verbatim}
     FENCE()
\end{verbatim}
tells the turtle to enter fence mode: From now on, if the turtle is
asked to move past the boundary of the graphics window, it will move
as far as it can and then stop at the edge with an ``out of bounds''
error message.  \texttt{FENCE} can also take one or two arguments as
described below.  Compare \texttt{WRAP} and \texttt{WINDOW}.

The commands \texttt{WRAP}, \texttt{WINDOW} and \texttt{FENCE} accept
zero, one or two arguments: if the single argument is \texttt{FALSE}
then no window mode is set, meaning that (like Turtle) LogoTurtle does
not use a fixed window size and there are no constraints on where the
turtle moves or draws; otherwise a windowing mode is set.  If there is
a single numerical argument $N$ then the size of the graphics window
is set so that $-|N| \le x,y \le |N|$; if there are two numerical
arguments $M, N$ then the size of the graphics window is set so that
$-|M| \le x \le |M|, -|N| \le y \le |N|$.

\subsubsection*{Fill command}
\begin{verbatim}
     FILL()
\end{verbatim}
fills the region of the graphics window bounded by the lines that have
just been drawn, i.e.\ the current curve if the pen is down or the
last curve if the pen is up (or the pen colour or size has been
changed).  The fill colour is the current pen colour and the pen size
is ignored.  The curve is implicitly closed but the turtle is not
moved.  For example, the following code draws a filled blue triangle
and a filled green circle:
\begin{verbatim}
  clearscreen();
  setpencolor(blue);
  pendown(); setxy(0, 20); setxy(20, 0); fill();
  penup(); setxy(50, 50);
  setpencolor(green);
  pendown(); circle(20); fill();
  draw();
\end{verbatim}
Note that filling may cause the \emph{default} pen (and hence fill)
colour to change, but if the pen colour has been set explicitly then
it will not change.

\subsubsection*{Filled command}
\begin{verbatim}
     FILLED(color, commands...)
\end{verbatim}
executes the commands in the order written, remembering all points
visited, and then draws the resulting curve, \emph{starting and
ending} with the turtle's initial position, filled with the specified
colour.  The pen size and whether the pen is up or down are ignored.
The first argument should specify a colour as a colour number, RGB
list, etc.\ (see
\hyperref[logoturtle:setpencolor]{\texttt{SETPENCOLOR}} for details)
or \texttt{FALSE} meaning that the current pen colour is used;
otherwise the pen colour is ignored.  Subsequent arguments should be
commands or lists of commands that move the turtle or draw curves.
For example, the following code draws the same filled blue triangle
and filled green circle as in the previous example:
\begin{verbatim}
  clearscreen(); penup();
  filled(blue, setxy(0, 20), setxy(20, 0));
  setxy(50, 50);
  filled(green, circle(20));
  draw();
\end{verbatim}
Note that the sequence of commands used by \texttt{FILLED} cannot be
generated using a loop construct such as \texttt{FOR} directly,
whereas with \texttt{FILL} it can.  However, the command arguments to
\texttt{FILLED} can be calls of procedures that can contain arbitrary
code, e.g.
\begin{verbatim}
  procedure shape;
     for i := 1 : 4 do << forward 80; arc2(-90, 40) >>;

  clearscreen(); penup();
  setxy(40, 80); setheading(-90);
  filled(false, shape());
  draw();
\end{verbatim}

\subsubsection*{Label command}
\begin{verbatim}
     LABEL text
\end{verbatim}
takes a printable item or list of printable items as input and prints
it on the graphics window, starting at the turtle's position.  The
items in a list are concatenated with no additional spacing.  Long
labels may fail!

\subsubsection*{Setlabelfont command}
\begin{verbatim}
     SETLABELFONT font
\end{verbatim}
\label{logoturtle:setlabelfont}
sets the face and/or size of the label font.  If the face is specified
then it should be the only or first input and must be an identifier or
string, e.g.\ \texttt{"Arial"}.  If the size is specified then it
should be the only or second input and must be a positive integer.  If
only one of the face and size is set then the other reverts to the
default, not the previous value set.  Alternatively, the single input
can be a list of the form \texttt{{face, size}}, or \texttt{false} to
revert to the default.  The inputs must specify a font in a way that
is accepted by Gnuplot but the details of font setting depend on the
Gnuplot terminal in use.  The defaults for the wxt terminal are face
Sans and size 10.  For the canvas terminal (and hence on Web REDUCE)
setting the label font face is ignored.

\subsubsection*{Setlabelcolor command}
\begin{verbatim}
     SETLABELCOLOR color
\end{verbatim}
sets the label foreground colour; see
\hyperref[logoturtle:setpencolor]{\texttt{SETPENCOLOR}} for details.


\subsection{Turtle and Window Queries}

\subsubsection*{Shownp query}
\begin{verbatim}
     SHOWNP()
\end{verbatim}
%% SHOWN?
returns \texttt{TRUE} if the turtle is shown (visible), \texttt{FALSE}
if the turtle is hidden.  See
\hyperref[logoturtle:showturtle]{\texttt{SHOWTURTLE}} and
\hyperref[logoturtle:hideturtle]{\texttt{HIDETURTLE}}.

Note that generally in LogoTurtle \texttt{TRUE}/\texttt{FALSE} values
can be used by writing code such as the following
\begin{verbatim}
     if shownp() = true then ...
\end{verbatim}
to facilitate programming LogoTurtle.

\subsubsection*{Turtlemode query}
\begin{verbatim}
     TURTLEMODE()
\end{verbatim}
returns the word \texttt{WRAP}, \texttt{FENCE}, \texttt{WINDOW} or
\texttt{FALSE} depending on the current turtle mode.

\subsubsection*{Labelfont query}
\begin{verbatim}
     LABELFONT()
\end{verbatim}
\label{logoturtle:labelfont}
returns a list of the current label font face and size if both are
set, or whichever of the face or size is set, or \texttt{false}
indicating that no label font information is set.  Unset font
information reverts to the Gnuplot default.

\subsubsection*{Labelcolor query}
\begin{verbatim}
     LABELCOLOR()
\end{verbatim}
returns the current label foreground colour; see
\hyperref[logoturtle:pencolor]{\texttt{PENCOLOR}} for details.


\subsection{Pen and Background Control}
\label{logoturtle:PBC}

The turtle carries a pen that can draw pictures.  At any time the pen
can be UP (in which case moving the turtle does not change what's on the
graphics screen) or DOWN (in which case the turtle leaves a trace).
Initially, the pen is UP.

\subsubsection*{Pendown command}
\begin{verbatim}
     PENDOWN()
\end{verbatim}
%% PD
sets the pen's position to \texttt{DOWN}, without changing its mode.

\subsubsection*{Penup command}
\begin{verbatim}
     PENUP()
\end{verbatim}
%% PU
sets the pen's position to \texttt{UP}, without changing its mode.

\subsubsection*{Setpencolor command}
\begin{verbatim}
     SETPENCOLOR color
\end{verbatim}
\label{logoturtle:setpencolor}
%% SETPC color
sets the pen colour to the given number, which must be an integer
between 0 and 15 inclusive.  The initial colour assignments are
\begin{center}
  \begin{tabular}{rlrlrlrl}
    0 & black  &  1 & blue    &  2 & green  &  3 & cyan  \\
    4 & red    &  5 & magenta &  6 & yellow &  7 & white \\
    8 & brown  &  9 & tan     & 10 & forest & 11 & aqua  \\
    12 & salmon & 13 & purple  & 14 & orange & 15 & grey
  \end{tabular}
\end{center}
but other colours can be assigned to numbers 8--15 by the
\texttt{SETPALETTE} command.  Alternatively, sets the pen colour to
the given RGB values (a list of three nonnegative numbers not greater
than 100 specifying the percent saturation of red, green, and blue in
the desired colour).

The argument can be a string or identifier representing a colour in
any way that is acceptable to Gnuplot, such as a colour name or
hexadecimal number, e.g.\ \texttt{red}, \texttt{"red"} or
\texttt{"\#FF0000"}.  Alternatively, it can be the identifier
\texttt{FALSE} meaning that no colour is set.  In this case, Gnuplot
uses its own automatic colour-choice algorithm.

\subsubsection*{Setpalette command}
\begin{verbatim}
     SETPALETTE(colornumber, rgblist)
\end{verbatim}
sets the actual colour corresponding to a given number.  The first
input must be an integer $N$ such that $8 \le N \le 15$. (LogoTurtle
keeps the first 8 colours constant.)  The second input is a list of
three nonnegative numbers not greater than 100 specifying the percent
saturation of red, green, and blue in the desired color.  The second
input can be a string or identifier representing a colour in any way
that is acceptable to Gnuplot, such as a colour name or hexadecimal
number, e.g.\ \texttt{red}, \texttt{"red"} or \texttt{"\#FF0000"}.

\subsubsection*{Setpensize command}
\begin{verbatim}
     SETPENSIZE size
\end{verbatim}
sets the thickness of the pen.  The input is a positive integer
representing a multiple of the default thickness, or false, meaning
unspecified, which is effectively equivalent to 1 but slightly less
efficient.

\subsubsection*{Setbackground command}
\begin{verbatim}
     SETBACKGROUND color
\end{verbatim}
%% SETBG color
sets the screen background color by slot number or RGB values, etc.
See \hyperref[logoturtle:setpencolor]{\texttt{SETPENCOLOR}} for
details.  Currently, however, this command requires the
\texttt{GNUTERM} environment variable to be set to the Gnuplot
terminal type.  This is because in Gnuplot the background is a
property of the terminal, so the terminal type is required as part of
the command to set the background.  Unless you already specify the
appropriate Gnuplot terminal type, you can find it by running Gnuplot
interactively, when it will report something like
\begin{verbatim}
     Terminal type set to 'wxt'
\end{verbatim}
In this case, the correct value to assign to the \texttt{GNUTERM}
environment variable would be \texttt{wxt} (without any quotes).


\subsection{Pen and Background Queries}

\subsubsection*{Pendownp query}
\begin{verbatim}
     PENDOWNP()
\end{verbatim}
%% PENDOWN?
returns the identifier \texttt{TRUE} if the pen is down,
\texttt{FALSE} if it's up.

\subsubsection*{Pencolor query}
\begin{verbatim}
     PENCOLOR()
\end{verbatim}
\label{logoturtle:pencolor}
%% PC
returns the pen colour as a string or identifier that represents a
colour in any way that is acceptable to Gnuplot, such as a colour name
or hexadecimal number.  Alternatively, it may output the identifier
\texttt{FALSE} meaning that no colour is set.

\subsubsection*{Palette query}
\begin{verbatim}
     PALETTE colornumber
\end{verbatim}
returns a string or identifier that represents the colour associated
with the given number in any way that is acceptable to Gnuplot, such
as a colour name or hexadecimal number.  Colornumber must be a
nonnegative integer not greater than 15.

\subsubsection*{Pensize query}
\begin{verbatim}
     PENSIZE()
\end{verbatim}
returns a positive integer specifying the thickness of the pen as a
multiple of the default thickness, or false, meaning unspecified,
which is effectively equivalent to 1 but slightly less efficient.

\subsubsection*{Background query}
\begin{verbatim}
     BACKGROUND()
\end{verbatim}
%% BG
returns the graphics background colour as a string or identifier that
represents a colour in any way that is acceptable to Gnuplot, such as
a colour name or hexadecimal number.  Alternatively, it may output the
identifier \texttt{FALSE} meaning that no colour is set.


\subsection{Saving and Loading Pictures}
%% \label{logoturtle:SLP}

\subsubsection*{Savepict command}
\begin{verbatim}
     SAVEPICT identifier
\end{verbatim}
\label{logoturtle:savepict}
saves the current plot to internal storage under the specified
identifier.  The current plot is not changed.  The saved plot can be
restored as the current plot using \texttt{LOADPICT} or displayed
using \hyperref[logoturtle:draw]{\texttt{DRAW}}.

\subsubsection*{Loadpict command}
\begin{verbatim}
     LOADPICT identifier
\end{verbatim}
retrieves the plot stored under the specified identifier, which must
have been saved by a \texttt{SAVEPICT} command, and makes it the
current plot.  The previous current plot is lost if not saved using
\texttt{SAVEPICT}.
