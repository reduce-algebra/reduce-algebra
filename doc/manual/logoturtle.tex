\section{Logo Turtle Graphics}
\indexpackage{LOGOTURTLE}
\index{Wright, Francis J.}\index{People!Wright, Francis J.}

\subsection{Introduction}

Logo Turtle Graphics\footnote{The Logo Turtle Graphics package was
written by Francis Wright.} (henceforth referred to as ``LogoTurtle'')
is a partial REDUCE emulation of
\href{https://en.wikipedia.org/wiki/Turtle_graphics}{traditional Logo
  turtle graphics} with one turtle, modelled on
\href{http://people.eecs.berkeley.edu/~bh/logo.html}{Berkeley Logo 6.2
  by Brian Harvey}.  This manual section is derived from the Graphics
chapter of the Berkeley Logo 6.2 manual available from
\href{https://github.com/jrincayc/ucblogo-code}{GitHub}.

%% Should use label references in the two hyperlinks below!
This package is inspired by, and related to, the REDUCE
\hyperref[package:TURTLE]{Turtle package} by Caroline Cotter (ZIB,
Berlin, 1998), and the word ``Turtle'' below (with a capital T) will
refer specifically to that package.  Both packages are built on the
REDUCE \hyperref[package:GNUPLOT]{Gnuplot package}, which itself uses
\href{http://gnuplot.info/}{Gnuplot} to display plots.  This means
that plotting is not fully interactive as it would be in
\href{https://en.wikipedia.org/wiki/Logo_(programming_language)}{traditional
  Logo}; a plot is constructed invisibly and only displayed when
requested.  This package aims to be more efficient, more authentic,
potentially more interactive and potentially more complete than
Turtle.

Note that LogoTurtle and Turtle cannot both be run in the same REDUCE
session because they define some procedures with the same names.

\subsection{Design}

LogoTurtle is entirely procedural.  It uses ``getters'' and
``setters'', and does not provide any algebraic-mode variables (unlike
Turtle, but see \hyperref[logoturtle:SLP]{Saving and Loading
  Pictures}).  Command names are as in Berkeley Logo 6.2, except that
they are REDUCE procedure calls and so must be followed by parentheses
unless they take a single argument, in which case the parentheses are
optional.  Note that commands that take no argument \emph{must} be
followed by empty parentheses.

LogoTurtle procedures other than queries return nothing (\texttt{nil})
and plotting is achieved via side effects, not via returned values
(unlike Turtle).  A plot is displayed by calling the (non-traditional)
\hyperref[logoturtle:draw]{\texttt{draw}} command, as for Turtle.
The plot displayed need not be complete; \texttt{draw} displays the
plot constructed so far, which allows an element of interactivity.

LogoTurtle makes essential use of commands to lower and raise the pen
(unlike Turtle; see \hyperref[logoturtle:PBC]{Pen and Background
  Control}).  Lowering the pen begins a ``curve'', namely a sequence
of points connected by straight lines, and raising the pen ends that
curve.  Each time the pen is lowered, the turtle moved and the pen
raised produces a distinct curve.

LogoTurtle uses Lisp floating-point numbers internally and does not
require any particular REDUCE number domain settings.  However, all
procedure arguments and list elements must (currently) evaluate to
explicit real numbers, which can be integers, rationals or floats.
This means that \emph{if any transcendental functions are used then
``on rounded'' is required}, and the default (or a little lower)
precision is appropriate.  Returned values and list elements will be
floats converted to the current number domain.  Note that LogoTurtle
lists are REDUCE algebraic-mode lists delimited by curly braces,
\texttt{\{~\}}, not the square brackets used in traditional Logo.

\subsection{User Interface}

To use LogoTurtle, execute the REDUCE command
\begin{verbatim}
     load_package logoturtle;
\end{verbatim}

LogoTurtle sets the scaling of the two axes to be the same so that
geometry is correct (although beware that Gnuplot does not always
honour this).  Apart from that, (currently) Gnuplot automatically
sizes the display so that it includes the whole plot.  The position of
the origin (the turtle home location \texttt{\{0,0\}}) is not fixed
and will be different for different plots.

Positive X is to the right; positive Y is up.  Headings (angles) are
measured in degrees clockwise from the positive Y axis.  (Note that
this differs from the common mathematical convention, also used by
Turtle, of measuring angles counterclockwise from the positive X
axis!)

The turtle itself is (currently) not displayed.

%% FJW: Commented out text describes facilities that I might implement
%% at some later date.

%% The turtle is represented as an isoceles triangle; the actual turtle
%% position is at the midpoint of the base (the short side).  However,
%% the turtle is drawn one step behind its actual position, so that the
%% display of the base of the turtle's triangle does not obscure a line
%% drawn perpendicular to it (as would happen after drawing a square).

%% Colors are, of course, hardware-dependent.  However, Logo provides partial
%% hardware independence by interpreting color numbers 0 through 7 uniformly
%% on all computers:

%% \begin{verbatim}
%% 0  black        1  blue         2  green        3  cyan
%% 4  red          5  magenta      6  yellow       7 white
%% \end{verbatim}
%% Where possible, Logo provides additional user-settable colors; how many
%% are available depends on the hardware and operating system environment.
%% If at least 16 colors are available, Logo tries to provide uniform
%% initial settings for the colors 8-15:

%% \begin{verbatim}
%%  8  brown        9  tan         10  forest      11  aqua
%% 12  salmon      13  purple      14  orange      15  grey
%% \end{verbatim}
LogoTurtle (currently) uses a white background and pen colour chosen
automatically by Gnuplot.

Initially, the turtle is at the origin (Cartesian coordinates (0,0))
facing up (heading 0 degrees) with the pen up.

Note that LogoTurtle command names are shown using upper case letters
in the descriptions after the example below to distinguish them
clearly from their arguments, but LogoTurtle is case insensitive, so
commands can be entered in either case.


\subsection{A Simple Example}

This example assumes LogoTurtle has just been loaded but not yet used.
If this is not the case then first execute the commands
\begin{verbatim}
    clearscreen(); penup();
\end{verbatim}

The following code draws an equilateral triangle with side length 100
centred on the origin, with one vertex on the positive Y axis.  To
make this example as interactive as possible, the plot is displayed
after each side is drawn.
\begin{verbatim}
    on rounded;
    forward(100/sqrt 3); pendown();
    right 150; forward 100; draw();
    right 120; forward 100; draw();
    right 120; forward 100; draw();
\end{verbatim}

For more interesting examples, please see the file
\href{https://sourceforge.net/p/reduce-algebra/code/HEAD/tree/trunk/packages/plot/logoturtle.tst}{\texttt{logoturtle.tst}},
which can be found in the \texttt{packages/plot} directory in a
standard REDUCE distribution.


\subsection{Displaying Logo Turtle Graphics}

\begin{verbatim}
     DRAW()
\end{verbatim}
\label{logoturtle:draw}
This command can be used at any time to display the current plot,
i.e.\@ the plot constructed so far (provided there is a plot to
display).  It initially opens a Gnuplot window and subsequently
updates it.

\begin{verbatim}
     DRAW(variables)
\end{verbatim}
The \texttt{draw} command accepts one or more optional arguments
consisting of variables assigned by the
\hyperref[logoturtle:savepict]{\texttt{savepict}} command and
displays superimposed the plot stored in each variable, in the order
specified, followed by the current plot.  (The order is only
significant in that it determines the colour of each plot set
automatically by Gnuplot.)


\subsection{Turtle Motion}

\begin{verbatim}
     FORWARD dist
\end{verbatim}
%% FD dist
moves the turtle forward, in the direction that it's facing, by the
specified distance (measured in turtle steps).

\begin{verbatim}
     BACK dist
\end{verbatim}
%% BK dist
moves the turtle backward, i.e., exactly opposite to the direction
that it's facing, by the specified distance.  (The heading of the
turtle does not change.)

\begin{verbatim}
     LEFT degrees
\end{verbatim}
%% LT degrees
turns the turtle counterclockwise by the specified angle, measured in
degrees (1/360 of a circle).

\begin{verbatim}
     RIGHT degrees
\end{verbatim}
%% RT degrees
turns the turtle clockwise by the specified angle, measured in degrees
(1/360 of a circle).

\begin{verbatim}
     SETPOS pos
\end{verbatim}
\label{logoturtle:setpos}
moves the turtle to an absolute position in the graphics window.  The
input is a list of two numbers, the X and Y coordinates.

\begin{verbatim}
     SETXY(xcor, ycor)
\end{verbatim}
moves the turtle to an absolute position in the graphics window.  The
two inputs are numbers, the X and Y coordinates.

\begin{verbatim}
     SETX xcor
\end{verbatim}
moves the turtle horizontally from its old position to a new absolute
horizontal coordinate.  The input is the new X coordinate.

\begin{verbatim}
     SETY ycor
\end{verbatim}
moves the turtle vertically from its old position to a new absolute
vertical coordinate.  The input is the new Y coordinate.

\begin{verbatim}
     SETHEADING degrees
\end{verbatim}
\label{logoturtle:setheading}
%% SETH degrees
turns the turtle to a new absolute heading.  The input is a number,
the heading in degrees clockwise from the positive Y axis.

\begin{verbatim}
     HOME()
\end{verbatim}
\label{logoturtle:home}
moves the turtle to its starting position (the origin) and orientation.
%% center of the screen.
Equivalent to ``\texttt{\hyperref[logoturtle:setpos]{SETPOS}~\{0,~0\};
  \hyperref[logoturtle:setheading]{SETHEADING}~0;}''.

%% \begin{verbatim}
%% ARC angle radius
%% \end{verbatim}

%% draws an arc of a circle, with the turtle at the center, with the
%% specified radius, starting at the turtle's heading and extending
%% clockwise through the specified angle.  The turtle does not move.


\subsection{Turtle Motion Queries}

\begin{verbatim}
     POS()
\end{verbatim}
returns the turtle's current position, as a list of two numbers, the X
and Y coordinates.

\begin{verbatim}
     XCOR()
\end{verbatim}
returns a number, the turtle's X coordinate.

\begin{verbatim}
     YCOR()
\end{verbatim}
returns a number, the turtle's Y coordinate.

\begin{verbatim}
     HEADING()
\end{verbatim}
returns a number, the turtle's heading in degrees.

\begin{verbatim}
     TOWARDS pos
\end{verbatim}
returns a number, the heading at which the turtle should be facing so
that it would point from its current position to the position given as
the input.

%% \begin{verbatim}
%% SCRUNCH
%% \end{verbatim}
%% returns a list containing two numbers, the X and Y scrunch factors, as used by
%% \texttt{SETSCRUNCH}.  (But note that \texttt{SETSCRUNCH} takes two numbers as inputs, not one
%% list of numbers.)

%% @xref{SETSCRUNCH} .


\subsection{Turtle and Window Control}

%% \begin{verbatim}
%% SHOWTURTLE
%% ST
%% \end{verbatim}
%% makes the turtle visible.

%% \begin{verbatim}
%% HIDETURTLE
%% HT
%% \end{verbatim}
%% makes the turtle invisible.  It's a good idea to do this while you're in
%% the middle of a complicated drawing, because hiding the turtle speeds up
%% the drawing substantially.

\begin{verbatim}
     CLEAN()
\end{verbatim}
erases all lines that the turtle has drawn on the graphics window.
The turtle's state (position, heading, pen mode, etc.) is not changed.

\begin{verbatim}
     CLEARSCREEN()
\end{verbatim}
%% CS
erases (and closes) the graphics window and sends the turtle to its
initial position and heading.  Like
\texttt{\hyperref[logoturtle:home]{HOME}()} and \texttt{CLEAN()}
together.

%% \begin{verbatim}
%% WRAP
%% \end{verbatim}
%% tells the turtle to enter wrap mode:  From now on, if the turtle is
%% asked to move past the boundary of the graphics window, it will ``wrap
%% around'' and reappear at the opposite edge of the window.  The top edge
%% wraps to the bottom edge, while the left edge wraps to the right edge.
%% (So the window is topologically equivalent to a torus.)  This is the
%% turtle's initial mode.  Compare \texttt{WINDOW} and \texttt{FENCE}.

%% @xref{FENCE} .

%% \begin{verbatim}
%% WINDOW
%% \end{verbatim}
%% tells the turtle to enter window mode:  From now on, if the turtle is
%% asked to move past the boundary of the graphics window, it will move
%% offscreen.  The visible graphics window is considered as just part of an
%% infinite graphics plane; the turtle can be anywhere on the plane.  (If
%% you lose the turtle, \texttt{HOME} will bring it back to the center of the
%% window.)  Compare \texttt{WRAP} and \texttt{FENCE}.

%% \begin{verbatim}
%% FENCE
%% \end{verbatim}
%% tells the turtle to enter fence mode:  From now on, if the turtle is
%% asked to move past the boundary of the graphics window, it will move as
%% far as it can and then stop at the edge with an ``out of bounds'' error
%% message.  Compare \texttt{WRAP} and \texttt{WINDOW}.

%% @xref{WRAP} .

%% \begin{verbatim}
%% FILL
%% \end{verbatim}
%% fills in a region of the graphics window containing the turtle and
%% bounded by lines that have been drawn earlier.  This is not portable; it
%% doesn't work for all machines, and may not work exactly the same way on
%% different machines.

%% \begin{verbatim}
%% FILLED color instructions
%% \end{verbatim}
%% runs the instructions, remembering all points visited by turtle
%% motion commands, starting @emph{and ending} with the turtle's initial
%% position.  Then draws (ignoring penmode) the resulting polygon,
%% in the current pen color, filling the polygon with the given color,
%% which can be a color number or an RGB list.  The instruction list
%% cannot include another FILLED invocation.  (wxWidgets only)

%% \begin{verbatim}
%% LABEL text
%% \end{verbatim}
%% takes a word or list as input, and prints the input on the graphics
%% window, starting at the turtle's position.

%% \begin{verbatim}
%% SETLABELHEIGHT height
%% \end{verbatim}
%% command (wxWidgets only).  Takes a positive integer argument and tries
%% to set the font size so that the character height (including
%% descenders) is that many turtle steps.  This will be different from
%% the number of screen pixels if \texttt{SETSCRUNCH} has been used.  Also, note
%% that \texttt{SETSCRUNCH} changes the font size to try to preserve this height
%% in turtle steps.  Note that the query operation corresponding to this
%% command is \texttt{LABELSIZE}, not \texttt{LABELHEIGHT}, because it tells you the width
%% as well as the height of characters in the current font.

%% \begin{verbatim}
%% SETSCRUNCH xscale yscale
%% \end{verbatim}
%% adjusts the aspect ratio and scaling of the graphics display.  After
%% this command is used, all further turtle motion will be adjusted by
%% multiplying the horizontal and vertical extent of the motion by the two
%% numbers given as inputs.  For example, after the instruction
%% @w{@t{SETSCRUNCH 2 1}} motion at a heading of 45 degrees will move twice
%% as far horizontally as vertically.  If your squares don't come out
%% square, try this.  (Alternatively, you can deliberately misadjust the
%% aspect ratio to draw an ellipse.)


%% \subsection{Turtle and Window Queries}

%% \begin{verbatim}
%% SHOWNP
%% SHOWN?
%% \end{verbatim}
%% returns \texttt{TRUE} if the turtle is shown (visible), \texttt{FALSE} if the turtle is
%% hidden.  See \texttt{SHOWTURTLE} and \texttt{HIDETURTLE}.

%% @xref{SHOWTURTLE} ,
%% @ref{HIDETURTLE} .

%% \begin{verbatim}
%% TURTLEMODE
%% \end{verbatim}
%% returns the word \texttt{WRAP}, \texttt{FENCE}, or \texttt{WINDOW} depending on the current
%% turtle mode.

%% \begin{verbatim}
%% LABELSIZE
%% \end{verbatim}
%% (wxWidgets only) returns a list of two positive integers, the width
%% and height of characters displayed by \texttt{LABEL} measured in turtle steps
%% (which will be different from screen pixels if \texttt{SETSCRUNCH} has been
%% used).  There is no \texttt{SETLABELSIZE} because the width and height of a
%% font are not separately controllable, so the inverse of this operation
%% is \texttt{SETLABELHEIGHT}, which takes just one number for the desired height.


\subsection{Pen and Background Control}
\label{logoturtle:PBC}

The turtle carries a pen that can draw pictures.  At any time the pen
can be UP (in which case moving the turtle does not change what's on the
graphics screen) or DOWN (in which case the turtle leaves a trace).
Initially, the pen is UP.
%% If the pen is down, it can operate in one of three modes: PAINT (so that it
%% draws lines when the turtle moves), ERASE (so that it erases any lines
%% that might have been drawn on or through that path earlier), or REVERSE
%% (so that it inverts the status of each point along the turtle's path).

\begin{verbatim}
     PENDOWN()
\end{verbatim}
%% PD
sets the pen's position to \texttt{DOWN}, without changing its mode.

\begin{verbatim}
     PENUP()
\end{verbatim}
%% PU
sets the pen's position to \texttt{UP}, without changing its mode.

%% \begin{verbatim}
%% PENPAINT
%% PPT
%% \end{verbatim}
%% sets the pen's position to \texttt{DOWN} and mode to \texttt{PAINT}.

%% \begin{verbatim}
%% PENERASE
%% PE
%% \end{verbatim}
%% sets the pen's position to \texttt{DOWN} and mode to \texttt{ERASE}.

%% \begin{verbatim}
%% PENREVERSE
%% PX
%% \end{verbatim}
%% sets the pen's position to \texttt{DOWN} and mode to \texttt{REVERSE}.  (This may interact
%% in system-dependent ways with use of color.)

%% \begin{verbatim}
%% SETPENCOLOR colornumber.or.rgblist
%% SETPC colornumber.or.rgblist
%% \end{verbatim}
%% sets the pen color to the given number, which must be a nonnegative
%% integer.  There are initial assignments for the first 16 colors:

%% \begin{verbatim}
%%  0  black	 1  blue	 2  green	 3  cyan
%%  4  red		 5  magenta	 6  yellow	 7 white
%%  8  brown	 9  tan		10  forest	11  aqua
%% 12  salmon	13  purple	14  orange	15  grey
%% \end{verbatim}
%% but other colors can be assigned to numbers by the \texttt{PALETTE} command.
%% Alternatively, sets the pen color to the given RGB values (a list of
%% three nonnegative numbers less than 100 specifying the percent saturation
%% of red, green, and blue in the desired color).

%% \begin{verbatim}
%% SETPALETTE colornumber rgblist
%% \end{verbatim}
%% sets the actual color corresponding to a given number, if allowed by the
%% hardware and operating system.  Colornumber must be an integer greater than or
%% equal to 8.  (Logo tries to keep the first 8 colors constant.)  The second
%% input is a list of three nonnegative numbers less than 100 specifying the
%% percent saturation of red, green, and blue in the desired color.

%% \begin{verbatim}
%% SETPENSIZE size
%% \end{verbatim}
%% sets the thickness of the pen.  The input is either a single positive
%% integer or a list of two positive integers (for horizontal and
%% vertical thickness).  Some versions pay no attention to the second
%% number, but always have a square pen.

%% \begin{verbatim}
%% SETPENPATTERN pattern
%% \end{verbatim}
%% sets hardware-dependent pen characteristics.  This command is not
%% guaranteed compatible between implementations on different machines.

%% \begin{verbatim}
%% SETPEN list
%% \end{verbatim}
%% sets the pen's position, mode, thickness, and hardware-dependent
%% characteristics according to the information in the input list, which should
%% be taken from an earlier invocation of \texttt{PEN}.

%% @xref{PEN} .

%% \begin{verbatim}
%% SETBACKGROUND colornumber.or.rgblist
%% SETBG colornumber.or.rgblist
%% \end{verbatim}
%% set the screen background color by slot number or RGB values.
%% See \texttt{SETPENCOLOR} for details.

%% @xref{SETPENCOLOR} .

\subsection{Pen Queries}

\begin{verbatim}
     PENDOWNP()
\end{verbatim}
%% PENDOWN?
returns the identifier \texttt{TRUE} if the pen is down,
\texttt{FALSE} if it's up.  This information can be used by writing
code such as the following:
\begin{verbatim}
     if pendownp() = true then ...
\end{verbatim}

%% \begin{verbatim}
%% PENMODE
%% \end{verbatim}
%% returns one of the words \texttt{PAINT}, \texttt{ERASE}, or \texttt{REVERSE} according to the
%% current pen mode.

%% \begin{verbatim}
%% PENCOLOR
%% PC
%% \end{verbatim}
%% returns a color number, a nonnegative integer that is associated with
%% a particular color, or a list of RGB values if such a list was used as
%% the most recent input to \texttt{SETPENCOLOR}.  There are initial assignments
%% for the first 16 colors:

%% \begin{verbatim}
%%  0  black        1  blue         2  green        3  cyan
%%  4  red          5  magenta      6  yellow       7 white
%%  8  brown        9  tan         10  forest      11  aqua
%% 12  salmon      13  purple      14  orange      15  grey
%% \end{verbatim}
%% but other colors can be assigned to numbers by the \texttt{PALETTE} command.

%% \begin{verbatim}
%% PALETTE colornumber
%% \end{verbatim}
%% returns a list of three nonnegative numbers less than 100 specifying the
%% percent saturation of red, green, and blue in the color associated with the
%% given number.

%% \begin{verbatim}
%% PENSIZE
%% \end{verbatim}
%% returns a list of two positive integers, specifying the horizontal
%% and vertical thickness of the turtle pen.  (In some implementations,
%% including wxWidgets, the two numbers are always equal.)

%% \begin{verbatim}
%% PENPATTERN
%% \end{verbatim}
%% returns system-specific pen information.

%% \begin{verbatim}
%% PEN
%% \end{verbatim}
%% returns a list containing the pen's position, mode, thickness, and
%% hardware-specific characteristics, for use by \texttt{SETPEN}.

%% @xref{SETPEN} .

%% \begin{verbatim}
%% BACKGROUND
%% BG
%% \end{verbatim}
%% returns the graphics background color, either as a slot number or
%% as an RGB list, whichever way it was set.  (See \texttt{PENCOLOR}.)


\subsection{Saving and Loading Pictures}
\label{logoturtle:SLP}

These two commands are different from the Berkeley Logo commands with
the same names in that they save to, and load from, REDUCE variables
instead of files.  The values of the variables are REDUCE
algebraic-mode lists, which can be manipulated as usual in REDUCE.

\begin{verbatim}
     SAVEPICT variable
\end{verbatim}
\label{logoturtle:savepict}
assigns the current plot to the specified REDUCE variable.  The
current plot is not changed.  The saved plot can be restored as the
current plot using \texttt{LOADPICT} or displayed using
\hyperref[logoturtle:draw]{\texttt{DRAW}}.

\begin{verbatim}
     LOADPICT variable
\end{verbatim}
reads a plot from the specified REDUCE variable, which must have been
assigned by a \texttt{SAVEPICT} command, and makes it the current
plot.  The previous current plot is lost if not saved using
\texttt{SAVEPICT}.
