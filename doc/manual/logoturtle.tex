% $Id$
\section{Logo Turtle Graphics}
\indexpackage{LOGOTURTLE}
\index{Wright, Francis J.}\index{People!Wright, Francis J.}


\subsection{Introduction}

Logo Turtle Graphics\footnote{The Logo Turtle Graphics package was
written by Francis Wright.} (henceforth referred to as ``LogoTurtle'')
is a REDUCE emulation of
\href{https://en.wikipedia.org/wiki/Turtle_graphics}{traditional Logo
  turtle graphics} with one turtle, modelled on
\href{http://people.eecs.berkeley.edu/~bh/logo.html}{Berkeley Logo by
  Brian Harvey} and \href{https://fmslogo.sourceforge.io/}{FMSLogo by
  David Costanzo} (which is an updated version of George Mills'
MSWLogo, a multimedia-enhanced version for Microsoft Windows, which is
itself based on Berkeley Logo).  This manual section is derived
primarily from the Graphics chapter of the Berkeley Logo manual
available from
\href{https://github.com/jrincayc/ucblogo-code}{GitHub}.

This package is inspired by, and related to, the REDUCE
\hyperref[package:TURTLE]{Turtle package} by Caroline Cotter (ZIB,
Berlin, 1998), and the word ``Turtle'' below (with a capital T) will
refer specifically to that package.  Both packages are built on the
REDUCE \hyperref[package:GNUPLOT]{Gnuplot package}, which itself uses
\href{http://gnuplot.info/}{Gnuplot} to display plots.  This means
that plotting is not fully interactive as it would be in
\href{https://en.wikipedia.org/wiki/Logo_(programming_language)}{traditional
  Logo}; a plot is constructed invisibly and only displayed when
requested.  However, turning on the \texttt{LOGOTURTLE\_AUTODRAW}
switch makes LogoTurtle as interactive as possible.  This package aims
to be more efficient, more authentic, more interactive and more
complete than Turtle.

Note that LogoTurtle and Turtle cannot both be run in the same REDUCE
session because they define some procedures with the same names.


\subsection{Design}

LogoTurtle is entirely functional.  It uses ``getters'' and
``setters'', and does not use any algebraic-mode variables (unlike
Turtle).  Most command names are as in Berkeley Logo and/or FMSLogo,
and their function is the same or similar.  (Identical behaviour is
not always possible.)  However, all commands are REDUCE procedure
calls.

Getters (query procedures) return values that are accepted as input by
their matching setters (command procedures).  If more than one data
value is involved then a list is used.  For example, the
\hyperref[logoturtle:labelfont]{\texttt{LABELFONT}} query returns a
list of the current label font face and size if both are set, which
the corresponding
\hyperref[logoturtle:setlabelfont]{\texttt{SETLABELFONT}} command
accepts as its argument.

LogoTurtle commands other than queries return nothing (\texttt{nil})
and plotting is achieved via side effects, not via returned values as
for Turtle.  A plot is displayed by calling the (non-traditional)
\hyperref[logoturtle:draw]{\texttt{DRAW}} command as for Turtle.  The
plot displayed need not be complete; \texttt{DRAW} displays the plot
constructed so far, which allows an element of interactivity.

LogoTurtle makes essential use of commands to lower and raise the pen
(unlike Turtle; see \hyperref[logoturtle:PBC]{Pen and Background
  Control}).  Lowering the pen begins a ``curve'', namely a sequence
of points connected by straight lines, and raising the pen ends that
curve.  Each time the pen is lowered, the turtle moved and the pen
raised, a distinct curve is produced.

LogoTurtle uses Lisp floating-point numbers internally and does not
require any particular REDUCE number domain settings.  However, all
command arguments and list elements relating to turtle position must
be expressions that can be evaluated to real numbers.  All returned
values and list elements relating to turtle position will be
floating-point numbers.  Note that LogoTurtle lists are REDUCE
algebraic-mode lists delimited by curly braces, \texttt{\{~\}}, not
the square brackets used in traditional Logo, and that REDUCE list
elements must be separated by commas.


\subsection{User Interface}

To use LogoTurtle, execute the REDUCE command
\begin{verbatim}
     load_package logoturtle;
\end{verbatim}

LogoTurtle sets the scaling of the two axes to be the same so that the
aspect ratio is 1:1 and geometry is correct (although beware that
Gnuplot may not always honour this).  By default, LogoTurtle scales
the graphics window so that turtle coordinates $(-100,-100)$ and
$(100,100)$ fit, and the center of the graphics window is turtle
location $(0,0)$, i.e.\ the origin or home position.  But this fixed
window can be turned off so that Gnuplot automatically sizes the
display to include the whole plot (as for Turtle).  The position of
the origin (the turtle home location $(0,0)$) is then not fixed and
may be different for different plots.  The window size can also be
changed; see \hyperref[logoturtle:TWC]{Turtle and Window Control} for
further details.

Positive $x$ is to the right; positive $y$ is up.  Headings (angles)
are measured in degrees clockwise from the positive $y$ axis.  (Note
that this differs from the common mathematical convention, also used
by Turtle, of measuring angles counterclockwise from the positive $x$
axis!)  Initially, the turtle is at the origin (Cartesian coordinates
$(0,0)$) facing up (heading 0 degrees) with the pen up.

LogoTurtle uses by default a white background, and pen, fill and label
colours chosen automatically by Gnuplot, but you can set all these to
any colour provided by Gnuplot.

Note that LogoTurtle command names are shown using upper case letters
in the descriptions after the example below to distinguish them
clearly from their arguments, but LogoTurtle is case insensitive, so
commands can be entered in either case.

Commands that never take any arguments use special syntax and
\emph{need not be followed by empty parentheses}.  Query commands that
take no arguments can be used like (read-only) variables.  For
example, the following command increments the $x$-coordinate of the
turtle by 10 steps:
\begin{verbatim}
     setx(xcor + 10);
\end{verbatim}
Multiple command arguments must be enclosed in parentheses; single or
no arguments may be enclosed in parentheses, although parentheses can
be, and usually are, omitted with a single argument or no arguments.
However, if a single argument is an expression involving infix
operators then it must be enclosed in parentheses.

When the switch \texttt{TRLOGOTURTLE} is on, LogoTurtle outputs
Cartesian coordinates corresponding to every move of the turtle and
\texttt{DRAW} outputs the list of curves that it is about to draw.
When the switch \texttt{TRPLOT} is on, the commands sent to Gnuplot
(but not the actual plot data) are also output.  Turning the switch
\texttt{TRLOGOTURTLE} on or off also turns the switch \texttt{TRPLOT}
on or off.  Both switches are off by default.


\subsection{A Simple Example}

This example assumes LogoTurtle has just been loaded but not yet used.
If this is not the case then first execute the commands
\begin{verbatim}
  clearscreen; penup;
\end{verbatim}

The following code draws an equilateral triangle with side length 100
centred on the origin, with one vertex on the positive Y axis.  The
sides are coloured red, green and blue.  To make this example as
interactive as possible, the plot is displayed after each side is
drawn (but for this effect to be visible each line ending with
\texttt{draw;} must be executed separately; if you input all the
commands together then you will only see the complete triangle).
\begin{verbatim}
  forward(100/sqrt 3); pendown;
  setpencolor red; right 150; forward 100; draw;
  setpencolor green; right 120; forward 100; draw;
  setpencolor blue; right 120; forward 100; draw;
\end{verbatim}

For more interesting examples, please see the files
\href{https://sourceforge.net/p/reduce-algebra/code/HEAD/tree/trunk/packages/plot/logoturtle.tst}
     {\texttt{logoturtle.tst}} and
\href{https://sourceforge.net/p/reduce-algebra/code/HEAD/tree/trunk/packages/plot/mondrian.tst}
     {\texttt{mondrian.tst}}, which can be found online or in the
     \texttt{packages/plot} directory in a standard REDUCE
     distribution.


\subsection{Turtle}

The turtle is optionally displayed as an unfilled isosceles triangle;
see \hyperref[logoturtle:TWC]{Turtle and Window Control}.  The turtle
is drawn using black default-thickness lines on top of the current
plot; the colour and line thickness cannot currently be changed.  The
turtle is never wrapped, although it is clipped if any turtle mode is
in effect, i.e.\ for drawing the turtle any turtle mode other than
\texttt{false} is treated as \texttt{window}.

The turtle looks like an arrow head pointing in the direction of the
turtle's heading.  The height of the isosceles triangle (i.e.\ the
length of the turtle) is $\ell (= 0.1)$ times the average of the
maximum $x$ and $y$ coordinates set by the window size (even if
windowing is not in effect) and the apex angle (at the head of the
turtle) is twice $\alpha (= 10^{\circ})$ (so that the two equal sides
are at angles of $\alpha$ to the turtle's heading).\footnote{The
turtle length relative to the window size $\ell$ and head half-angle
$\alpha$ are respectively the Lisp float values of the symbolic-mode
global variables \texttt{logoturtle!-rel!-len!*} and
\texttt{logoturtle!-angle!*}.  These values are used dynamically and
in principle you could change them after LogoTurtle has loaded, but
currently there is no algebraic-mode facility to do this.}

The nominal turtle position is at the midpoint of the base (the short
side).  However, the turtle is drawn one step behind its nominal
position, so that the display of the base of the turtle's triangle
does not obscure a line drawn perpendicular to it (as would happen
after drawing a square).


\subsection{Colours}
\label{logoturtle:Colours}

LogoTurtle offers both the traditional Berkeley Logo palette model and
some of the Gnuplot model, which is much more flexible and usually
more convenient.  Colours may be input and output in several different
formats, but colour numbers and RGB lists are used only for input and
only the formats accepted by Gnuplot are used for output, since these
are the only formats used internally.

\begin{description}
\item[A colour number (input only)] is an integer between 0 and 15
  inclusive.  The initial colour assignments are
  \begin{center}
    \begin{tabular}{rlrlrlrl}
       0 & black  &  1 & blue    &  2 & green  &  3 & cyan  \\
       4 & red    &  5 & magenta &  6 & yellow &  7 & white \\
       8 & brown  &  9 & tan     & 10 & forest & 11 & aqua  \\
      12 & salmon & 13 & purple  & 14 & orange & 15 & grey
    \end{tabular}
  \end{center}
  but other colours can be assigned to numbers 8--15 by the
  \hyperref[logoturtle:setpalette]{\texttt{SETPALETTE}} command.
  Colour numbers are useful for cycling programmatically through a
  range of colours.

\item[An RGB list (input only)] is a list of three nonnegative numbers
  not greater than 100 specifying the percent saturation of red,
  green, and blue in the desired colour.  RGB lists are also easy to
  use programmatically.

\item[An identifier or string (input and output)] can be used to
  represent a colour in any way that is acceptable to Gnuplot, such as
  a colour name or hexadecimal number, e.g.\ \texttt{red},
  \texttt{"red"} or \texttt{"\#FF0000"}.

\item[The identifier \texttt{FALSE} (input and output)] means that no
  colour is set.  In this case, Gnuplot uses its own automatic
  colour-choice algorithm.
\end{description}


\subsection{Displaying Logo Turtle Graphics}

\subsubsection*{Draw command}
\begin{verbatim}
     DRAW
\end{verbatim}
\label{logoturtle:draw}
can be used at any time to display the current plot, i.e.\ the plot
and/or labels constructed so far (provided there is something to
display).  It initially opens a Gnuplot window and subsequently
updates it.

\subsubsection*{Autodraw switch}
When the switch \texttt{LOGOTURTLE\_AUTODRAW} is on, making any
visible change to a plot causes it to be redrawn (or drawn)
automatically.  In this situation it can be useful to turn on display
of the turtle (using
\hyperref[logoturtle:showturtle]{\texttt{SHOWTURTLE}}), since this
makes the turtle's heading and position visible even when the pen is
up.

When using REDUCE programming constructs (e.g.\ group or loop
statements) to run multiple LogoTurtle commands, it is best to have
the \texttt{LOGOTURTLE\_AUTODRAW} switch off, and instead to execute
the \texttt{DRAW} command explicitly once the programming construct
has completed, because intermediate changes to the plot will not be
visible and repeatedly redrawing it is pointless.


\subsection{Turtle Motion}

\subsubsection*{Forward command}
\begin{verbatim}
     FORWARD dist
\end{verbatim}
%% FD dist
moves the turtle forward, in the direction that it's facing, by the
specified distance (measured in turtle steps).

\subsubsection*{Back command}
\begin{verbatim}
     BACK dist
\end{verbatim}
%% BK dist
moves the turtle backward, i.e., exactly opposite to the direction
that it's facing, by the specified distance.  (The heading of the
turtle does not change.)

\subsubsection*{Left command}
\begin{verbatim}
     LEFT degrees
\end{verbatim}
%% LT degrees
turns the turtle counterclockwise by the specified angle, measured in
degrees.  (A degree is 1/360 of a full circle.)

\subsubsection*{Right command}
\begin{verbatim}
     RIGHT degrees
\end{verbatim}
%% RT degrees
turns the turtle clockwise by the specified angle, measured in
degrees.

\subsubsection*{Setpos command}
\begin{verbatim}
     SETPOS position
\end{verbatim}
\label{logoturtle:setpos}
moves the turtle to an absolute position in the graphics window.  The
input is a list of two numbers, the $x$ and $y$ coordinates,
e.g.\ ``\texttt{SETPOS \{50, 50\}}''.

\subsubsection*{Setxy command}
\begin{verbatim}
     SETXY(xcor, ycor)
\end{verbatim}
moves the turtle to an absolute position in the graphics window.  The
two inputs are numbers, the $x$ and $y$ coordinates.

\subsubsection*{Setx command}
\begin{verbatim}
     SETX xcor
\end{verbatim}
moves the turtle horizontally from its old position to a new absolute
horizontal coordinate.  The input is the new $x$ coordinate.

\subsubsection*{Sety command}
\begin{verbatim}
     SETY ycor
\end{verbatim}
moves the turtle vertically from its old position to a new absolute
vertical coordinate.  The input is the new $y$ coordinate.

\subsubsection*{Setheading command}
\begin{verbatim}
     SETHEADING degrees
\end{verbatim}
\label{logoturtle:setheading}
%% SETH degrees
turns the turtle to a new absolute heading.  The input is a number,
the heading in degrees clockwise from the positive $y$ axis.

\subsubsection*{Home command}
\begin{verbatim}
     HOME
\end{verbatim}
\label{logoturtle:home}
moves the turtle to its starting position (the origin) and orientation.
Equivalent to ``\texttt{\hyperref[logoturtle:setpos]{SETPOS}~\{0,~0\};
  \hyperref[logoturtle:setheading]{SETHEADING}~0}''.

\subsubsection*{Arc command}
\begin{verbatim}
     ARC(angle, radius)
\end{verbatim}
draws a circular arc centred on the turtle with the specified positive
radius, starting at the turtle's heading and extending clockwise
through the specified angle (counter-clockwise if angle is negative).
The turtle does not move and the arc is drawn as if the turtle mode is
\texttt{WINDOW} for all modes unless windowing is turned off.

\subsubsection*{Circle command}
\begin{verbatim}
     CIRCLE radius
\end{verbatim}
draws a circle centred on the turtle with the positive radius
specified.  The turtle does not move and the circle is drawn as if the
turtle mode is \texttt{WINDOW} for all modes unless windowing is
turned off.  Equivalent to \texttt{ARC(360, radius)}.

\subsubsection*{Arc2 command}
\begin{verbatim}
     ARC2(angle, radius)
\end{verbatim}
moves the turtle around a circular arc that sweeps through the
specified angle with the specified positive radius.  The turtle always
moves forwards: if angle is positive, then the turtle moves forwards
in a clockwise direction; if angle is negative, then the turtle moves
forwards in a counter-clockwise direction.  At the end of the arc, the
turtle's heading is increased by angle.

\subsubsection*{Circle2 command}
\begin{verbatim}
     CIRCLE2 radius
\end{verbatim}
moves the turtle clockwise around a circle with the specified positive
radius.  The turtle ends in the same position in which it starts.
Equivalent to \texttt{ARC2(360, radius)}.

\subsubsection*{Ellipticarc command}
\begin{verbatim}
     ELLIPTICARC(range, crosswise, inline, start)
\end{verbatim}
draws an elliptic arc based on the turtle's position and heading.  The
turtle does not move.  The center-point of the ellipse is the turtle's
current position.  The size and shape of the ellipse are determined by
the specified positive crosswise and inline semi-axes.  The crosswise
semi-axis is the distance from the turtle to the ellipse in the
direction perpendicular to the turtle's current heading.  The inline
semi-axis is the distance from the turtle to the ellipse in the
direction in which the turtle is currently heading.  Hence the
turtle's heading determines the orientation of the ellipse.  The
elliptic arc starts at angle parameter start degrees and the angle
parameter sweeps through range degrees.  The elliptic arc is drawn
clockwise if range is positive and counter-clockwise if range is
negative.

\subsubsection*{Ellipse command}
\begin{verbatim}
     ELLIPSE(crosswise, inline)
\end{verbatim}
draws an ellipse based on the turtle's position and heading.  The
turtle does not move.  The center-point of the ellipse is the turtle's
current position.  The size and shape of the ellipse are determined by
the specified positive crosswise and inline semi-axes.  The crosswise
semi-axis is the distance from the turtle to the ellipse in the
direction perpendicular to the turtle's current heading.  The inline
semi-axis is the distance from the turtle to the ellipse in the
direction in which the turtle is currently heading.  Hence the
turtle's heading determines the orientation of the ellipse.
Equivalent to \texttt{ELLIPTICARC(360, crosswise, inline, 0)}.


\subsection{Turtle Motion Queries}

\subsubsection*{Pos query}
\begin{verbatim}
     POS
\end{verbatim}
returns the turtle's current position, as a list of two numbers, the
$x$ and $y$ coordinates.

\subsubsection*{Xcor query}
\begin{verbatim}
     XCOR
\end{verbatim}
returns a number, the turtle's $x$ coordinate.

\subsubsection*{Ycor query}
\begin{verbatim}
     YCOR
\end{verbatim}
returns a number, the turtle's $y$ coordinate.

\subsubsection*{Heading query}
\begin{verbatim}
     HEADING
\end{verbatim}
returns a number, the turtle's heading in degrees.

\subsubsection*{Towards query}
\begin{verbatim}
     TOWARDS position
\end{verbatim}
returns a number, the heading at which the turtle should be facing so
that it would point from its current position to the position given as
the input in the form of a list of two numbers, the $x$ and $y$
coordinates.

\subsubsection*{Distance query}
\begin{verbatim}
     DISTANCE position
\end{verbatim}
returns a number, the distance the turtle must travel along a straight
line to reach the position given as input in the form of a list of two
numbers, the $x$ and $y$ coordinates.

As an example of using \texttt{TOWARDS} and \texttt{DISTANCE}, here is
a somewhat convoluted way to angle the turtle towards, and then move
it to, the position with coordinates $(1,2)$:
\begin{verbatim}
     setheading towards {1, 2};
     forward distance {1, 2};
\end{verbatim}


\subsection{Turtle and Window Control}
\label{logoturtle:TWC}

\subsubsection*{Showturtle command}
\begin{verbatim}
     SHOWTURTLE
\end{verbatim}
\label{logoturtle:showturtle}
%% ST
makes the turtle visible (next time the picture is drawn).

\subsubsection*{Hideturtle command}
\begin{verbatim}
     HIDETURTLE
\end{verbatim}
\label{logoturtle:hideturtle}
%% HT
makes the turtle invisible (next time the picture is drawn).

\subsubsection*{Clean command}
\begin{verbatim}
     CLEAN
\end{verbatim}
erases the graphics window.  The turtle's state (position, heading,
pen mode, etc.) is not changed.

\subsubsection*{Clearscreen command}
\begin{verbatim}
     CLEARSCREEN
\end{verbatim}
%% CS
erases \emph{and closes} the graphics window, and sends the turtle to
its initial position and heading.  Like
\texttt{\hyperref[logoturtle:home]{HOME}} and \texttt{CLEAN} together.

\subsubsection*{Setwindowsize command}
\begin{verbatim}
     SETWINDOWSIZE n
     SETWINDOWSIZE(m, n)
     SETWINDOWSIZE {m, n}
\end{verbatim}
with a single numerical argument $n$ sets the size of the graphics
window so that $-|n| \le x,y \le |n|$; with two numerical arguments
$m, n$ or with a single list argument \texttt{\{m, n\}} in which $m$
and $n$ are numerical it sets the size of the graphics window so that
$-|m| \le x \le |m|, -|n| \le y \le |n|$.  The default window size is
$-|100| \le x,y \le |100|$.

\subsubsection*{Wrap command}
\begin{verbatim}
     WRAP
\end{verbatim}
tells the turtle to enter wrap mode.  From now on, if the turtle is
asked to move past the boundary of the graphics window, it will ``wrap
around'' and reappear at the opposite edge of the window.  The top
edge wraps to the bottom edge, while the left edge wraps to the right
edge.  (So the window is topologically equivalent to a torus.)  This
is the turtle's initial mode.  Compare \texttt{WINDOW} and
\texttt{FENCE}.

\subsubsection*{Window command}
\begin{verbatim}
     WINDOW
\end{verbatim}
tells the turtle to enter window mode.  From now on, if the turtle is
asked to move past the boundary of the graphics window, it will move
offscreen.  The visible graphics window is considered as just part of
an infinite graphics plane; the turtle can be anywhere on the plane.
(If you lose the turtle, \texttt{HOME} will bring it back to the
center of the window.)  Compare \texttt{WRAP} and \texttt{FENCE}.

\subsubsection*{Fence command}
\begin{verbatim}
     FENCE
\end{verbatim}
tells the turtle to enter fence mode.  From now on, if the turtle is
asked to move past the boundary of the graphics window, it will move
as far as it can and then stop at the edge with an ``out of bounds''
error message.  Compare \texttt{WRAP} and \texttt{WINDOW}.

\subsubsection*{Setturtlemode command}
\begin{verbatim}
     SETTURTLEMODE mode
\end{verbatim}
sets the turtle (windowing) mode to one of \texttt{WRAP},
\texttt{FENCE}, \texttt{WINDOW}, with meaning as above, or
\texttt{FALSE}, meaning that (like Turtle) there are no constraints on
where the turtle draws.

\subsubsection*{Fill command}
\begin{verbatim}
     FILL
\end{verbatim}
fills the region of the graphics window bounded by the lines that have
just been drawn, i.e.\ the current curve if the pen is down or the
last curve if the pen is up (or the pen colour or size has been
changed).  The fill colour is the current pen colour and the pen size
is ignored.  The curve is implicitly closed but the turtle is not
moved.  For example, the following code draws a filled blue triangle
and a filled green circle:
\begin{verbatim}
  clearscreen;
  setpencolor blue;
  pendown; setxy(0, 20); setxy(20, 0); fill;
  penup; setxy(50, 50);
  setpencolor green;
  pendown; circle(20); fill;
  draw;
\end{verbatim}
Note that filling may cause the \emph{default} pen (and hence fill)
colour to change, but if the pen colour has been set explicitly then
it will not change.

\subsubsection*{Filled command}
\begin{verbatim}
     FILLED(colour, commands...)
\end{verbatim}
executes the commands in the order written, remembering all points
visited, and then draws the resulting curve, \emph{starting and
ending} with the turtle's initial position, filled with the specified
colour; see \hyperref[logoturtle:Colours]{Colours}.  The pen size,
whether the pen is up or down, and the pen colour are all ignored.
The command arguments should be commands or lists of commands that
move the turtle or draw curves.  For example, the following code draws
the same filled blue triangle and filled green circle as in the
previous example:
\begin{verbatim}
  clearscreen; penup;
  filled(blue, setxy(0, 20), setxy(20, 0));
  setxy(50, 50);
  filled(green, circle(20));
  draw;
\end{verbatim}
Note that the sequence of commands used by \texttt{FILLED} cannot be
generated directly using a loop construct such as \texttt{FOR},
whereas with \texttt{FILL} it can.  However, the command arguments to
\texttt{FILLED} can be calls of procedures that can contain arbitrary
code, e.g.
\begin{verbatim}
  procedure shape;
     for i := 1 : 4 do << forward 80; arc2(-90, 40) >>;

  clearscreen; penup;
  setxy(40, 80); setheading(-90);
  filled(false, shape());
  draw;
\end{verbatim}

\subsubsection*{Label command}
\begin{verbatim}
     LABEL text
\end{verbatim}
takes a printable item or list of printable items as input and prints
it on the graphics window with the top left-hand corner of the label
at the turtle's position.  The items in a list are concatenated with
no additional spacing.  Beware that long labels will just fall off the
edge of the graphics window.  Multi-line labels can be produced by
including newline characters encoded as \texttt{\textbackslash n}
within the text, e.g.\ \texttt{"This is a\textbackslash nmulti-line
  label"}.  (The newline is recognised by Gnuplot, not by REDUCE.)

\subsubsection*{Setlabelfont command}
\begin{verbatim}
     SETLABELFONT face
     SETLABELFONT size
     SETLABELFONT(face, size)
     SETLABELFONT {face, size}
     SETLABELFONT false
\end{verbatim}
\label{logoturtle:setlabelfont}
sets the face and/or size of the label font.  If the face is specified
then it should be the only or first input and must be an identifier or
string, e.g.\ \texttt{"Arial"}.  If the size is specified then it
should be the only or second input and must be a positive integer.  If
only one of the face and size is set then the other reverts to the
default, not the previous value set.  Alternatively, the single input
can be a list of the form \texttt{\{face, size\}}, or \texttt{false}
to revert to the default.  The inputs must specify a font in a way
that is accepted by Gnuplot but the details of font setting depend on
the Gnuplot terminal in use.  The defaults for the wxt terminal are
face Sans and size 10.  For the canvas terminal (and hence on Web
REDUCE) setting the label font face is ignored.

\subsubsection*{Setlabelcolor command}
\begin{verbatim}
     SETLABELCOLOR colour
\end{verbatim}
sets the label foreground colour; see
\hyperref[logoturtle:Colours]{Colours}.


\subsection{Turtle and Window Queries}

\subsubsection*{Shownp query}
\begin{verbatim}
     SHOWNP
\end{verbatim}
%% SHOWN?
returns \texttt{TRUE} if the turtle is shown (visible), \texttt{FALSE}
if the turtle is hidden.  See
\hyperref[logoturtle:showturtle]{\texttt{SHOWTURTLE}} and
\hyperref[logoturtle:hideturtle]{\texttt{HIDETURTLE}}.

Note that generally in LogoTurtle \texttt{TRUE}/\texttt{FALSE} values
returned by query commands can be used to facilitate programming
LogoTurtle by writing code such as the following:
\begin{verbatim}
     if shownp = true then ...
\end{verbatim}

\subsubsection*{Windowsize query}
\begin{verbatim}
     WINDOWSIZE
\end{verbatim}
returns the current size of the graphics window as a list of the form
$\{x_{max},y_{max}\}$.

\subsubsection*{Turtlemode query}
\begin{verbatim}
     TURTLEMODE
\end{verbatim}
returns the word \texttt{WRAP}, \texttt{FENCE}, \texttt{WINDOW} or
\texttt{FALSE} depending on the current turtle mode.

\subsubsection*{Labelfont query}
\begin{verbatim}
     LABELFONT
\end{verbatim}
\label{logoturtle:labelfont}
returns a list of the current label font face and size if both are
set, or whichever of the face or size is set, or \texttt{false}
indicating that no label font information is set.  Unset font
information reverts to the Gnuplot default.

\subsubsection*{Labelcolor query}
\begin{verbatim}
     LABELCOLOR
\end{verbatim}
returns the current label foreground colour; see
\hyperref[logoturtle:Colours]{Colours}.


\subsection{Pen and Background Control}
\label{logoturtle:PBC}

The turtle carries a pen that can draw pictures.  At any time the pen
can be UP (in which case moving the turtle does not change what's on the
graphics screen) or DOWN (in which case the turtle leaves a trace).
Initially, the pen is UP.

\subsubsection*{Pendown command}
\begin{verbatim}
     PENDOWN
\end{verbatim}
%% PD
sets the pen's position to \texttt{DOWN}, i.e.\ lowers it so that the
turtle draws when it moves.

\subsubsection*{Penup command}
\begin{verbatim}
     PENUP
\end{verbatim}
%% PU
sets the pen's position to \texttt{UP}, i.e.\ raises it so that the
turtle moves without drawing.

\subsubsection*{Setpencolor command}
\begin{verbatim}
     SETPENCOLOR colour
\end{verbatim}
\label{logoturtle:setpencolor}
%% SETPC colour
sets the pen colour; see \hyperref[logoturtle:Colours]{Colours}.

\subsubsection*{Setpalette command}
\begin{verbatim}
     SETPALETTE(colournumber, colour)
\end{verbatim}
\label{logoturtle:setpalette}
sets the actual colour corresponding to a given colour number.  The
first argument must be an integer $n$ such that $8 \le n \le
15$. (LogoTurtle keeps the first 8 colours constant.)  The second
argument may be either an RGB list of three nonnegative numbers not
greater than 100 specifying the percent saturation of red, green, and
blue in the desired colour, or an identifier or string representing a
colour in any way that is acceptable to Gnuplot, such as a colour name
or hexadecimal number, e.g.\ \texttt{red}, \texttt{"red"} or
\texttt{"\#FF0000"}.  See \hyperref[logoturtle:Colours]{Colours} for
further details.

\subsubsection*{Setpensize command}
\begin{verbatim}
     SETPENSIZE size
\end{verbatim}
sets the thickness of the pen.  The input is a positive integer
representing a multiple of the default thickness, or \texttt{FALSE},
meaning unspecified, which is equivalent to 1 but slightly
more efficient.

\subsubsection*{Setbackground command}
\begin{verbatim}
     SETBACKGROUND colour
\end{verbatim}
%% SETBG colour
sets the screen background colour; see
\hyperref[logoturtle:Colours]{Colours}.  Currently, however, this
command requires the \texttt{GNUTERM} environment variable to be set
to the Gnuplot terminal type.  This is because in Gnuplot the
background is a property of the terminal, so the terminal type is
required as part of the command to set the background.  Unless you
already specify the appropriate Gnuplot terminal type, you can find it
by running Gnuplot interactively, when it will report something like
\begin{verbatim}
     Terminal type set to 'wxt'
\end{verbatim}
In this case, the correct value to assign to the \texttt{GNUTERM}
environment variable would be \texttt{wxt} (without any quotes).


\subsection{Pen and Background Queries}

\subsubsection*{Pendownp query}
\begin{verbatim}
     PENDOWNP
\end{verbatim}
%% PENDOWN?
returns the identifier \texttt{TRUE} if the pen is down,
\texttt{FALSE} if it's up.

\subsubsection*{Pencolor query}
\begin{verbatim}
     PENCOLOR
\end{verbatim}
\label{logoturtle:pencolor}
%% PC
returns the pen colour; see \hyperref[logoturtle:Colours]{Colours}.

\subsubsection*{Palette query}
\begin{verbatim}
     PALETTE colournumber
\end{verbatim}
returns the colour associated with the given number; see
\hyperref[logoturtle:Colours]{Colours}.  Colournumber must be a
nonnegative integer not greater than 15.

\subsubsection*{Pensize query}
\begin{verbatim}
     PENSIZE
\end{verbatim}
returns a positive integer specifying the thickness of the pen as a
multiple of the default thickness, or \texttt{false}, meaning
unspecified, which is equivalent to 1 but slightly more efficient.

\subsubsection*{Background query}
\begin{verbatim}
     BACKGROUND
\end{verbatim}
%% BG
returns the graphics background colour; see
\hyperref[logoturtle:Colours]{Colours}.


\subsection{Saving and Loading Pictures}
%% \label{logoturtle:SLP}

\subsubsection*{Savepict command}
\begin{verbatim}
     SAVEPICT identifier
\end{verbatim}
\label{logoturtle:savepict}
saves the current plot and labels to internal storage under the
specified identifier without changing them.  The saved data can be
restored as the current plot and labels using \texttt{LOADPICT}.

\subsubsection*{Loadpict command}
\begin{verbatim}
     LOADPICT(identifiers)
\end{verbatim}
retrieves the plots and labels saved under the specified identifiers,
which must have been saved by \texttt{SAVEPICT} commands, merges them
in the order specified, and makes the result current. (The order is
only significant in that it determines the colour of each plot if it
is set automatically by Gnuplot, and if plots or labels overlap then
later plots and labels overlay earlier ones and so hide them.)  The
previous current plot and labels are lost if not saved using
\texttt{SAVEPICT}.
