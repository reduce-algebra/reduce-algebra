\subsection{Introduction}

The MathML interface for REDUCE provides commands to evaluate and
output MathML\@.  The principal features of this package are as
follows.
\begin{itemize}
\item Evaluation of MathML code.  This allows REDUCE to parse MathML
  expressions and evaluate them.
\item Generation of MathML compliant code.  This provides output of
  REDUCE expressions in MathML code, which could be used directly in
  web page production.
\end{itemize}

\subsection{Loading}

To use the MathML-REDUCE Interface, the package must first be loaded
explicitly by executing the command
\begin{verbatim}
     load_package mathml;
\end{verbatim}

\subsection{Switches}

There are three switches which can be used together, which are all off
by default.
\begin{description}
\item[\sw{mathml}:] When on, all output will be displayed as MathML.
\item[\sw{both}:] When on, all output will be displayed as both MathML
  and normal REDUCE output (which is much easier to read than MathML).
\item[\sw{web}:] When on, all MathML output will be enclosed in an
  \verb|<embed>| element.  This was intended for use in HTML4 web
  pages but is now obsolete.
\end{description}

\subsection{Entering MathML}

The MathML-REDUCE Interface allows the user to provide MathML input
via a file containing MathML or by entering MathML interactively.

\subsubsection{Reading MathML from a File: \texttt{MML}}

When reading from a file the operator \texttt{mml} is used, which takes
as argument the name of the file containing the MathML.

\texttt{mml}(FILE:\textit{string}):\textit{expression}

\paragraph{Example:}
As long as the file given contains valid MathML, no errors should be
produced.
\begin{verbatim}
     mml "ex.mml";
\end{verbatim}

\subsubsection{Reading MathML interactively: \texttt{PARSEML}}

The operator \texttt{parseml} takes no arguments and expects you to
enter MathML markup starting with \verb|<mathml>| and ending with
\verb|</mathml>|.  It then outputs an expression resulting from
evaluating the input.

\paragraph{Example:}
Here is an extract of a REDUCE session where \texttt{parseml()} is
used:
\begin{verbatim}
     parseml();
     <math>
        <apply><plus/>
           <cn>3</cn>
           <cn>5</cn>
        </apply>
     </math>

     8
\end{verbatim}

\subsection{The Evaluation of MathML}

MathML is always evaluated by REDUCE in \texttt{algebraic} mode before
outputting any results.  Hence, undefined variables remain undefined,
and it is possible to use all normal REDUCE switches and packages.

\paragraph{Example:}
The following MathML input
\begin{verbatim}
     parseml();
     <math>
        <reln><gt/>
           <ci>x</ci>
           <ci>y</ci>
        </reln>
     </math>
\end{verbatim}
will evaluate to \verb|x>y|.  (Note that this expression cannot be
entered directly on its own as REDUCE input!)

Now suppose we execute the following:
\begin{verbatim}
     x:=3; y:=2;
\end{verbatim}

If we once again enter and evaluate the above MathML input we will
have as result \verb|3>2|.  This can be evaluated to \texttt{true}
or \texttt{false} as follows:
\begin{verbatim}
    if ws then true else false;
    true
\end{verbatim}

Of course, it is possible to set only one of the two variables
\texttt{x} or \texttt{y} used above, say \texttt{y:=4}.  If we once
again enter and evaluate the above MathML we will get the result
\verb|x>4|.

When one of the switches \sw{mathml} or \sw{both} is on, the MathML
output will be:
\begin{verbatim}
     <math>
        <reln><gt/>
           <ci>x</ci>
           <cn type="integer">4</cn>
        </reln>
     </math>
\end{verbatim}

\paragraph{Example:}
Let the file \texttt{ex.mml} contain the following MathML:
\begin{verbatim}
     <math>
        <apply><int/>
           <bvar>
              <ci>x</ci>
           </bvar>
           <apply><fn><ci>F</ci><fn>
              <ci>x</ci>
           </apply>
        </apply>
     </math>
\end{verbatim}
If we execute the command
\begin{verbatim}
     mml "ex.mml";
\end{verbatim}

we get \texttt{int(f(x),x)} or
\[ \int f(x) dx \]

It is clear that this has remained unevaluated.  We can now define the
function \texttt{f(x)} as follows:
\begin{verbatim}
     for all x let f(x) = 2*x^2;
\end{verbatim}
If we then enter \texttt{mml "ex.mml";} once again, we will have the
following result:
\[ \frac {2x^3}{3} \]

This shows that the MathML-REDUCE Interface allows normal REDUCE
interactions to manipulate the evaluation of MathML input without
needing to edit the MathML itself.

\subsection{Interpretation of Error Messages}

The MathML-REDUCE Interface has a set of error messages which aim to
help the user understand and correct any invalid MathML\@.  Because
there can exist many different causes of errors, such error messages
should be considered merely as advice.  Here are the most important
error messages.

\paragraph{Missing Tag:}
Many MathML tags are used in pairs, e.g.\ \verb|<apply>|
\verb|</apply>|, \verb|<reln>| \verb|</reln>|, \verb|<ci>|
\verb|</ci>|.  In the case where the ending tag is missed out, or
misspelled, it is very probable that an error of this type will be
thrown.

\paragraph{Ambiguous or Erroneous Use of \texttt{\textless apply\textgreater}:}
This error message is non-specific.  When it appears, it is not very
clear to the interface where exactly the error lies.  Probable causes
are misuse of the \verb|<apply>| \verb|</apply>| tags, or misspelling
of the tag preceding the \verb|<apply>| tag.  However, other types of
error may cause this error message to appear.

Tags following an \verb|<apply>| tag may be misspelled without causing
an error message, but they will be considered as operators by REDUCE
and therefore generate some unexpected expression.

\paragraph{Syntax Errors:}
It is possible that the input MathML is not syntactically correct.  In
such a situation, the error will be spotted, and in some cases a
resolution might be presented.  There are a variety of syntax error
messages, but relying on their advice might not always be helpful.

Despite the verbose nature of the error messages and their recommended
resolutions, in most situations reference to the MathML specification
is recommended.

\subsection{Limitations of the Interface}
\label{mathml:limitations}

Not all aspects of MathML have been perfectly fitted into the
interface.  There are still some unsolved problems in the present
version of the interface:
\begin{itemize}
\item MathML Presentation Markup is not supported.  The interface will
  treat every presentation tag as either an unknown tag or a REDUCE
  operator.
\item Certain MathML tags do not play an important role in the REDUCE
  environment.  Such tags will be parsed correctly, although their
  action will be ignored.  These tags are:
  \begin{itemize}
  \item \verb|<interval>| \verb|</interval>|
  \item \verb|<inverse>| \verb|</inverse>|
  \item \verb|<condition>| \verb|</condition>|
  \item \verb|<compose>| \verb|</compose>|
  \item \verb|<ident>| \verb|</ident>|
  \item \verb|<forall/>|
  \item \verb|<exists/>|
  \end{itemize}
  However, the tags \verb|<condition>| \verb|</condition>| and
  \verb|<interval>| \verb|</interval>| are supported when used within
  the following tags:
  \begin{itemize}
  \item \verb|<int/>|
  \item \verb|<limit/>|
  \item \verb|<sum/>|
  \item \verb|<product/>|
  \end{itemize}
\item The \verb|<declare>| construct takes one or two arguments.  It
  sets the first argument to the value of the second.  In the case
  where the second argument is a vector or a matrix, an obscure error
  message is produced.
\end{itemize}

\subsection{Including MathML in a Web Page}

We shall not get into the details of including MathML in the design of
a web page, but for completeness here is a quick overview.  One can
include MathML in two ways:
\begin{itemize}
\item directly within \verb|<math>| \verb|</math>| tags;
\item within \verb|<math>| \verb|</math>| tags within an
  \verb|<embed>| element, but this approach is obsolete.
\end{itemize}

When one has the \sw{mathml} switch on, MathML is generated inside
\verb|<math>| \verb|</math>| tags.  Modern mainstream browsers all
support this markup.

\paragraph{Example:}
\begin{verbatim}
<html>
<body>
<title>MathML example</title>
<h1>Example of MathML</h1>
<br/>
The following is MathML directly encoded within this
page:
<br/><br/>

<math>
   <apply><int>
      <bvar>
         <ci>x</ci>
      </bvar>
      <apply><power/>
         <ci>x</ci>
         <apply><power/>
            <ci>x</ci>
            <cn type="integer">2</cn>
         </apply>
      </apply>
   </apply>
</math>

<br/><br/>
Modern mainstream browsers should all render the
MathML correctly.
</html>
\end{verbatim} 

In the past, the only way to render MathML in many browsers was to use
an external plug-in application and include the MathML in the web page
within an \verb|<embed>| element.  This is easily done with the
REDUCE-MathML Interface by turning on the \sw{web} switch.  Here is an
example of embedded MathML:

\paragraph{Example}

\begin{verbatim}
<html>
<body>
<title>MathML example</title>
<h1>Example of MathML</h1>
<br>
The following is MathML directly encoded within this
page:
<br><br>

<EMBED TYPE="text/mathml" MMLDATA="<math><apply><int>
   <bvar><ci>x</ci></bvar><apply><power/><ci>x</ci>
   <apply><power/><ci>x</ci><cn type=&quot;integer&quot;>
   2</cn></apply></apply></apply></math>"
   HEIGHT=300 WIDTH=200>

<br><br>
On all browsers with a suitable plug-in, this is
rendered correctly.
</html>
\end{verbatim} 

Another way to embed MathML in a web page is to put it in a separate
file (usually with \texttt{.mml} extension) and embed it with the
following code:
\begin{verbatim}
<EMBED TYPE="text/mathml" SRC="mathml.mml"
   HEIGHT=300 WIDTH=200>
\end{verbatim}

\subsection{Examples}

\paragraph{Example 1:} Enter the following input interactively:

\begin{verbatim}
     on mathml;
     solve({z=x*a+1},{z,x});
\end{verbatim}

\paragraph{Example 2:}
Have a file \texttt{ex2.mml} containing the following MathML:
\begin{verbatim}
     <mathml>
        <apply><sum/>
           <bvar>
              <ci>x</ci>
           </bvar>
           <apply><fn><ci>F</ci><fn>
              <ci>x</ci>
           </apply>
        </apply>
     </mathml>
\end{verbatim}
and execute the following command:
\begin{verbatim}
     mml "ex2.mml";
\end{verbatim}

\paragraph{Example 3:}
This example illustrates how practical the switch \sw{both} can be for
interpreting verbose MathML.  Introduce the following MathML into a
file, say \texttt{ex3.mml}:
\begin{verbatim}
     <mathml>
        <apply><int/>
           <bvar>
              <ci>x</ci>
           </bvar>
           <apply><sin/>
              <apply><log/>
                 <ci>x</ci>
              </apply>
           </apply>
        </apply>
     </mathml>
\end{verbatim}
then execute the following:
\begin{verbatim}
     on both;
     mml "ml";
\end{verbatim}

\subsection{An Overview of How the Interface Works}

The interface is primarily built in two parts.  A first one which
parses and evaluates MathML, and a second one which parses REDUCE's
algebraic expressions and outputs them in MathML format.  Both parts
use Top-Down Recursive Descent parsing with one token look ahead.

The EBNF description of the MathML grammar is defined informally in
\href{https://www.w3.org/TR/REC-MathML/appendixE.html}{APPENDIX E} of
the original MathML specification, which was used to develop the
MathML parser.  The MathML parser evaluates all that is possible and
returns a valid REDUCE algebraic expression.  When either of the
switches \sw{mathml} or \sw{both} is on, this algebraic expression is
fed into the second part of the program, which parses these expressions
and transforms them back into MathML.

The MathML generator parses the algebraic expression produced by
either REDUCE itself or the MathML parser.  It works in a very similar
way to the MathML parser.  It is simpler, since no evaluation is
involved.  All the generated code is MathML compliant.  It is
important to note that the MathML code generator sometimes introduces
Presentation Markup tags, and other tags which are not understood by
the MathML parser of the interface.\footnote{The set of tags not
understood by the MathML parser is detailed in
\hyperref[mathml:limitations]{Limitations of the Interface}.}
